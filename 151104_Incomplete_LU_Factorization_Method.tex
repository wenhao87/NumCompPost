\documentclass[UTF8,nofonts]{ctexart}

\setCJKmainfont[BoldFont=STHeiti,ItalicFont=STKaiti]{STKaiti}
\setCJKsansfont[BoldFont=STHeiti]{STXihei}
\setCJKmonofont{STFangsong}

\usepackage[letterpaper,left=1in,top=1in]{geometry}
\usepackage{multirow}
\usepackage{amsmath}
\usepackage{amsfonts}
\usepackage{amssymb}
\usepackage[titlenotnumbered,lined,linesnumbered]{algorithm2e}

\begin{document}

% Title
\section*{不完全三角分解法}

\subsection*{稀疏模式及相关记号}

如果矩阵$A$的B非零元素较少,且按一定规则分布,即:非零元的[label]稀疏模式[/label],对其进行[label]LU分解[/label](也称为[label]完全LU分解[/label])得到的[label]单位下三角矩阵[/label]$L$和[label]上三角矩阵[/label]$U$B一般不能保持和$A$相同的稀疏模式。但若分解之后$LU$是$A$的一个B近似,则称为矩阵$A$的[lable]不完全分解[/label],其B逆矩阵$(LU)^{-1}$往往用于线性方程组$A\boldsymbol{x}=\boldsymbol{y}$的[label]预处理[/label]。

设$A=\left(a_{ij}\right)\in\mathbb{R}^{n \times n}$,如下定义两个集合和一个[label]指标集[/label]:

- 非对角元的B位置$ND$:$ND=\{(i,j)|1 \leq i \neq j \leq n\}$

- B非零的非对角元的位置$NZ$:$NZ=\{(i,j)\in ND|a_{ij}\neq 0\}$

- 指标集$\mathcal{F}$:满足$NZ \subseteq \mathcal{F} \subseteq ND$

若$A$有分解:$A=LU+R$,其中:

\begin{align}
\label{eq:l}
L=(l_{ij}),&\quad l_{ij}=0,\text{ if }i>j,(i,j)\notin\mathcal{F} \\
\label{eq:u}
U=(u_{ij}),&\quad u_{ij}=0,\text{ if }i<j,(i,j)\notin\mathcal{F} \\
\label{eq:r}
R=(r_{ij}),&\quad \begin{cases}r_{ij}=0,\text{ if }(i,j)\in\mathcal{F}\\r_{ii}=-\omega\displaystyle{\sum_{\substack{j=1\\j\neq i}}^n}r_{ij},\quad i=1,2,\cdots,n\end{cases}
\end{align}

indent 这便是$A$关于$\mathcal{F}$和$\omega$的[label]松弛不完全三角分解[/label],其中$\omega$为[lable]松弛参数[/label]:$0 \leq \omega \leq 1$。特别地,如果$\mathcal{F}=ND$,则$R=0$,即完全三角分解;如果$\mathcal{F}=NZ$,则式(\ref{eq:l}、\ref{eq:u})表明$L$和$U$具有和$A$B相同的零元分布。

如果把:

\[
ND/\mathcal{F}=\{(i,j)|i \leq i \neq j \leq n,(i,j)\notin\mathcal{F}\}
\]

indent 看作是$A$的[label]零模式[/label],则$L$和$U$B保持了这种零模式。特别地,若$\omega=0$,则称为[label]不完全LU分解[/label]([label]ILU[/label]);若$\omega=1$,则称为修正的不完全LU分解([label]MILU[/label])。

此外,定义[label]非零模式[/label]$\tilde{\mathcal{F}}=\mathcal{F}\cup\{(i,i):i=1,2,\cdots,n\}$,很显然,它是零模式$ND/\mathcal{F}$的补集。每次修正消元实际就是对非零模式$\tilde{\mathcal{F}}$的元素进行Gauss消元,同时将对角元做修正。

\subsection*{不完全LU分解(ILU)}

\subsubsection*{不完全LU分解算法实现}

对于不完全LU分解,其松弛参数$\omega=0$,算法只对B非零模式中的元素进行消元,所以很容易实现。对于给定的$A$和$\mathcal{F}$,不完全LU分解算法实现如下:

%[latex mode=1]
%[+preamble]
%\usepackage[titlenotnumbered,lined,linesnumbered]{algorithm2e}
%\let\oldnl\nl
%\newcommand{\nonl}{\renewcommand{\nl}{\let\nl\oldnl}}
%[/preamble]
%\quicklatex{size=14}
\TitleOfAlgo{Incomplete LU Factorization}
\begin{algorithm}[H]
\For{$k\gets 1$ \KwTo $n-1$}{
	\For{$i\gets k+1$ \KwTo $n,(i,k)\in\tilde{\mathcal{F}}$} {
		$a_{ik} \gets a_{ik}/a_{kk}$\;
		\For{$j\gets k+1$ \KwTo $n,(i,j)\in\tilde{\mathcal{F}}$} {
			$a_{ij} \gets a_{ij}-a_{ik}a_{kj}$\;
		}
	}
}
\end{algorithm}
%[/latex]

\subsubsection*{不完全LU分解算法实例}

以一个$3 \times 3$阶矩阵$A$为例,设:

\[
A=\begin{pmatrix}1 & 1 & 0 \\ 0 & 2 & 1 \\ 2 & 0 & 1\end{pmatrix},\quad
\mathcal{F}=NZ=\Big\{(1,2),(2,3),(3,1)\Big\},\quad\omega=0
\]

indent 显然,非零模式$\tilde{\mathcal{F}}=\Big\{(1,1),(1,2),(2,2),(2,3),(3,1),(3,3)\Big\}$,不完全LU分解就是对非零模式$\tilde{\mathcal{F}}$的Gauss消元。这里,第$k=1$步时,只有$a_{31}\in\tilde{\mathcal{F}}$所在的第$i=3$行需要消元,其中,又只有第$j=3$列的$a_{33}$需要消元。所以,第$1$步消元后,得到:

\[
A^{(1)}=\begin{pmatrix}1 & 1 & 0 \\ 0 & 2 & 1 \\ \boldsymbol{0} & 0 & 1\end{pmatrix},\quad
L^{(1)}=\begin{pmatrix}1 & 0 & 0 \\ 0 & 1 & 0 \\ \boldsymbol{2} & 0 & 1\end{pmatrix},\quad
U^{(1)}=\begin{pmatrix}1 & 1 & 0 \\ 0 & 2 & 1 \\ \boldsymbol{0} & 0 & 1\end{pmatrix},\quad
\]

indent 此时,$L^{(1)}$和$U^{(1)}$已经是单位下三角矩阵和上三角矩阵了,且它们具有和$A$B相同的非零元分布。

\subsection*{修正的不完全LU分解(MILU)}

\subsubsection*{修正的不完全LU分解算法实现}

修正的不完全LU分解(MILU)和不完全LU分解(ILU)类似,不同之处在于前者会对[label]对角元[/label]进行[label]修正[/label]。修正的B原理是:对B零模式$\mathcal{F}$中的元素进行消元,将消减的量乘以松弛参数$\omega$后,B加到对角元上,最后将原零模式位置下的元素置$0$,B保持其零模式。因此,对于给定的$A$、$\mathcal{F}$和$\omega$,修正的不完全LU分解算法实现如下:

%[latex mode=1]
%[+preamble]
%\usepackage[titlenotnumbered,lined,linesnumbered]{algorithm2e}
%\let\oldnl\nl
%\newcommand{\nonl}{\renewcommand{\nl}{\let\nl\oldnl}}
%[/preamble]
%\quicklatex{size=14}
\TitleOfAlgo{Modified Incomplete LU Factorization}
\begin{algorithm}[H]
\For{$k\gets 1$ \KwTo $n-1$}{
	\For{$i\gets k+1$ \KwTo $n,(i,k)\in\tilde{\mathcal{F}}$} {
		$a_{ik} \gets a_{ik}/a_{kk}$\;
		\For{$j\gets k+1$ \KwTo $n,(i,j)\in\tilde{\mathcal{F}}$} {
			$a_{ij} \gets a_{ij}-a_{ik}a_{kj}$\;
		}
		\For{$j\gets k+1$ \KwTo $n,(i,j)\notin\tilde{\mathcal{F}}$} {
			$a_{ij} \gets a_{ij}-a_{ik}a_{kj}=-a_{ik}a_{kj}$\;
		}
		\For{$j\gets k+1$ \KwTo $n,(i,j)\notin\tilde{\mathcal{F}}$} {
			$a_{ii} \gets a_{ii}+\omega a_{ij}$\;
			$a_{ij}\gets 0$\;
		}
	}
}
\end{algorithm}
%[/latex]

\subsubsection*{修正的不完全LU分解实例}

设$A$为$4 \times 4$阶矩阵:

\[
A=\begin{pmatrix}1 & 2 & 0 & 4 \\ 0 & 3 & 1 & 0 \\ 2 & 0 & 1 & 0 \\ 4 & 0 & 5 & 1\end{pmatrix},\quad
\mathcal{F}=NZ,\quad\omega=1
\]

第$k=1$步时,需要对第$i=3,4$行消元。以第$3$行消元为例:

\begin{align*}
a'_{31}&=a_{31}/a_{11}=2 \\
a'_{32}&=-a'_{31}a_{12}=-4,\quad a'_{34}=-a'_{31}a_{14}=-8 \\
a'_{33}&=a_{33}+\omega a'_{32} + \omega a'_{34}=-11 \\
a'_{32}&=a'_{34}=0
\end{align*}

[alert type="info"]实际过程中,一般很少用到$R$的信息,所以一般不保存。但如果需要计算$R=(r_{ij})$的话,可以先令其为零矩阵,并在上述算法第$11,12$行之间加入\[r_{ij}\gets r_{ij}+a_{ij},\quad r_{ii}\gets r_{ii}-\omega a_{ij}\][/alert]

% H2, H3
\subsection*{dsfsdt}
%[latex mode=1]
%[+preamble]
%\usepackage[titlenotnumbered,lined,linesnumbered]{algorithm2e}
\let\oldnl\nl
\newcommand{\nonl}{\renewcommand{\nl}{\let\nl\oldnl}}
%[/preamble]
%\quicklatex{size=14}
\TitleOfAlgo{BLAS-3 Gaussian Elimination with Partial Pivoting}
\begin{algorithm}[H]
\For{$l\gets 1$ \KwTo $n-1$}{
$k\gets (l-1)b+1$\;
factorize $PA^{(l)}=LU$ using BLAS-2\;
store $L$ and $U$ in $A$\;
left multiply $A(1:n,k+b:n)$ by $P$\;
$A(k:k+b-1,k+b:n)\gets T^{-1}A(k:k+b-1,k+b:n)$\;
\nonl\Indp where $T$ is the unit-lower-trian of $A(k:k+b-1,k:k+b-1)$\;
\DontPrintSemicolon\Indm$A(k+b:n,k+b:n)\gets A(k+b:n,k+b:n)$\;
\PrintSemicolon\nonl\Indp$-A(k+b:n,k:k+b-1)A(k:k+b-1,k+b:n)$\;
}
\end{algorithm}
%[/latex]

% H4
% \subsubsection*{}

\end{document}
