\documentclass[UTF8,nofonts]{ctexart}

\setCJKmainfont[BoldFont=STHeiti,ItalicFont=STKaiti]{STKaiti}
\setCJKsansfont[BoldFont=STHeiti]{STXihei}
\setCJKmonofont{STFangsong}

\usepackage[letterpaper,left=1in,top=1in]{geometry}
\usepackage{multirow}
\usepackage{amsmath}
\usepackage{amsfonts}
\usepackage{amssymb}
\usepackage[titlenotnumbered,lined,linesnumbered]{algorithm2e}

\begin{document}

% Title
\section*{稀疏矩阵的图表示和行列重排}

[label]图论[/label]有助于发掘[label]稀疏矩阵[/label]中的[label]Gauss消去法[/label]和[label]预处理[/label]技术的并行性,另一方面,对于大型稀疏矩阵,不论是[label]直接解法[/label]还是[label]迭代法[/label],在它们的并行实现中,都会用到[label]行列重排[/label]技术。本文主要就这两方面进行详细讨论。

\subsection*{图表示法}

对于一个矩阵$A=(a_{ij})\in\mathbb{R}^{n \times n}$,其图表示法可以使用两个集合来定义:

- 顶点的集合($V$):($V=\{v_1,v_2,\cdots,v_n\}$)

- 边的集合($E$):由顶点对$(v_i,v_j)$构成,且$E\subseteq V\times V$

indent 稀疏矩阵的[label]邻接图[/label]定义为$G=(V,E)$,其中$V$中的$n$个顶点代表了$n$个未知量;若$a_{ij} \neq 0$,则表示两个顶点$i$和$j$有边相连。特别地,对于[label]对称矩阵[/label],即满足$a_{ij}$和$a_{ji}$同时不为$0$时,其邻接图是一个[label]无向图[/label],图中的边由[label]无向边[/label]所表示。邻接图表示法对于和下文的行列重排技术有着紧密的联系。

\subsection*{行列重排}

\subsubsection*{矩阵的行列重排}

对于一个$n \times n$阶矩阵$A=(a_{ij})$,用$a_{i*}$表示$A$的第$i$行;用$a_{*j}$表示$A$的第$j$行。定义一个排列$\pi=\{k_1,k_2,\cdots,k_n\}$,它是集合$\{1,2,\cdots,n\}$的一个排列,并将:

\begin{eqnarray*}
& A_{\pi,\ast}&=\{a_{\pi(i),j}\}_{i=1,\cdots,n;j=1,\cdots,n} \\
& A_{\ast,\pi}&=\{a_{i,\pi(j)}\}_{i=1,\cdots,n;j=1,\cdots,n} \\
\end{eqnarray*}

indent 分别定义为矩阵$A$的[label]行-$\pi$重排[/label]和[label]列-$\pi$重排[/label]。此外,定义[label]排列矩阵[/label]$X_{ij}$,它是交换[label]单位矩阵[/label]第$i$行和第$j$行后的结果,而为了实现交换矩阵$A$的第$i$行和第$j$行,只需将其左乘$X_{ij}$即可。进而,排列$\pi=\{k_1,k_2,\cdots,k_n\}$相当于连续$n$次[label]行交换[/label]$\sigma(i_k,j_k),k=1,\cdots,n$后相乘的结果;类似地,矩阵$A$右乘$X_{ij}$可以实现第$i$列和第$j$列的[label]列交换[/label]。于是,行-$\pi$重排和列-$\pi$重排也可以表示为如下形式:

\begin{eqnarray}
& A_{\pi,\ast} = P_\pi A,\quad P_\pi=X_{i_n,j_n}X_{i_{n-1},j_{n-1}}\cdots X_{i_1,j_1} \label{eq:row}\\
& A_{\ast,\pi} = AQ_\pi,\quad Q_\pi=X_{i_1,j_1}X_{i_2,j_2}\cdots X_{i_n,j_n} \label{eq:col}
\end{eqnarray}

indent 其中,行交换矩阵的乘积$P_\pi$和列交换矩阵的乘积$Q_\pi$统称为[label]排列矩阵[/label]。需要注意的是,式(\ref{eq:row})和式(\ref{eq:col})对矩阵$A$的行列交换使用了B同一种重排,所以仔细观察$X_{ij}$的话可以发现:$X^2_{ij}=I$,也就是说一个行交换或者列交换矩阵的[label]逆矩阵[/label]是它自己B本身;另一方面,$X_{ij}$本身是B对称的,于是可以推得:

\begin{eqnarray*}
& P_\pi Q_\pi = X_{i_n,j_n}X_{i_{n-1},j_{n-1}}\cdots X_{i_1,j_1} \times X_{i_1,j_1}X_{i_2,j_2}\cdots X_{i_n,j_n}=I \\
& P^T_{\pi}=X^T_{i_1,j_1}X^T_{i_2,j_2}\cdots X^T_{i_n,j_n}=X_{i_1,j_1}X_{i_2,j_2}\cdots X_{i_n,j_n}=Q_\pi \\
& \Longrightarrow Q_\pi=P^{-1}_\pi=P^T_\pi
\end{eqnarray*}

indent 也就是说,排列矩阵是[lable]酉矩阵[/label],其转置矩阵和逆矩阵相等(见数值计算中的重要矩阵概念一文)。所以说,如果使用同一种重排对矩阵$A$进行行交换和列交换,相当于执行了一次[label]相似变换[/label]($P_\pi AP_\pi^T$)。

\subsubsection*{线性方程组的行列重排}

如果矩阵$A$处于[label]线性方程组[/label]$A\boldsymbol{x}=\boldsymbol{b}$中,则对矩阵$A$的行-$\pi$重排需要同样作用于[label]常数向量[/label]$\boldsymbol{b}$之上;而对矩阵$A$的列-$\pi$重排则需要同样作用于[label]未知数向量[/label]$\boldsymbol{x}$之上,以保证线性方程组的B一致性。以一个$4 \times 4$阶矩阵$A$为例,设排列$\pi=\{1,3,2,4\}$,若其作为行-$\pi$重排时,常数向量$\boldsymbol{b}$也重排为$(b_1,b_3,b_2,b_4)^T$;若其作为列-$\pi$重排时,未知数向量$\boldsymbol{x}$需重排为$(x_1,x_3,x_2,x_4)^T$:

\begin{equation}
\label{eq:axb}
A\boldsymbol{x}=\boldsymbol{b} \Longleftrightarrow
\begin{pmatrix}
a_{11} & 0 & a_{13} & 0 \\
0 & a_{22} & a_{23} & a_{24} \\
a_{31} & a_{32} & a_{33} & 0 \\
0 & a_{42} & 0 & a_{44}
\end{pmatrix}
\begin{pmatrix}x_1 \\ x_2 \\ x_3 \\ x_4\end{pmatrix}=
\begin{pmatrix}b_1 \\ b_2 \\ b_3 \\ b_4\end{pmatrix}
\end{equation}
\begin{equation*}
P_\pi A\boldsymbol{x}=
\begin{pmatrix}
1 & 0 & 0 & 0 \\
0 & 0 & 1 & 0 \\
0 & 1 & 0 & 0 \\
0 & 0 & 0 & 1 \\
\end{pmatrix}
\begin{pmatrix}
a_{11} & 0 & a_{13} & 0 \\
0 & a_{22} & a_{23} & a_{24} \\
a_{31} & a_{32} & a_{33} & 0 \\
0 & a_{42} & 0 & a_{44}
\end{pmatrix}
\begin{pmatrix}x_1 \\ x_2 \\ x_3 \\ x_4\end{pmatrix}=
\begin{pmatrix}
a_{11} & 0 & a_{13} & 0 \\
a_{31} & a_{32} & a_{33} & 0 \\
0 & a_{22} & a_{23} & a_{24} \\
0 & a_{42} & 0 & a_{44}
\end{pmatrix}
\begin{pmatrix}x_1 \\ x_2 \\ x_3 \\ x_4\end{pmatrix}=
\begin{pmatrix}b_1 \\ b_3 \\ b_2 \\ b_4\end{pmatrix}
\end{equation*}

更常见的情况是既有行重排,也有列重排,也称之为[label]双边重排[/label],则未知数向量和常数向量都要做相应地重排。如果像式(\ref{eq:row})和式(\ref{eq:col})那样,行重排$P_\pi$和列重排$Q_\pi$使用同一种排列$\pi$,则称之为[label]对称重排[/label],并记为$A_{\pi,\pi}$,且其满足:

\[A_{\pi,\pi}=P_\pi A P^T_\pi\]

\subsection*{行列重排和邻接图的关系}

线性方程组的行列重排相当于重新标记其邻接图的顶点编号,而不改变边,即:若$(i,j)$代表原矩阵$A$邻接图中的一条边,其行列重排后的矩阵为$A'$,则有$a'_{ij}=a_{\pi(i),\pi(j)}$,其中$(i,j)$是$A'$邻接图中的一条边。换言之,就是用新标号$i$来代替原标号$\pi(i)$。以式(\ref{eq:axb})中的线性方程组为例,令排列$\pi=\{1,3,2,4\}$,则经过行列重排后得到矩阵$A'$:

\[
A'=PAP^T=
\begin{pmatrix}
a_{11} & a_{13} & 0 & 0 \\
a_{31} & a_{33} & a_{32} & 0 \\
0 & a_{23} & a_{22} & a_{24} \\
0 & 0 & a_{42} & a_{44}
\end{pmatrix}
\]

indent 令$i=3,j=4$,可以验证:$a'_{ij}=a'_{34}=a_{\pi(i),\pi(j)}=a_{\pi(3),\pi(4)}=a_{24}$。因此,矩阵行列重排后其邻接图并没有改变,改变的只是顶点的编号。

% H2, H3
\subsection*{dsfsdt}
%[latex mode=1]
%[+preamble]
%\usepackage[titlenotnumbered,lined,linesnumbered]{algorithm2e}
\let\oldnl\nl
\newcommand{\nonl}{\renewcommand{\nl}{\let\nl\oldnl}}
%[/preamble]
%\quicklatex{size=14}
\TitleOfAlgo{BLAS-3 Gaussian Elimination with Partial Pivoting}
\begin{algorithm}[H]
\For{$l\gets 1$ \KwTo $n-1$}{
$k\gets (l-1)b+1$\;
factorize $PA^{(l)}=LU$ using BLAS-2\;
store $L$ and $U$ in $A$\;
left multiply $A(1:n,k+b:n)$ by $P$\;
$A(k:k+b-1,k+b:n)\gets T^{-1}A(k:k+b-1,k+b:n)$\;
\nonl\Indp where $T$ is the unit-lower-trian of $A(k:k+b-1,k:k+b-1)$\;
\DontPrintSemicolon\Indm$A(k+b:n,k+b:n)\gets A(k+b:n,k+b:n)$\;
\PrintSemicolon\nonl\Indp$-A(k+b:n,k:k+b-1)A(k:k+b-1,k+b:n)$\;
}
\end{algorithm}
%[/latex]

% H4
% \subsubsection*{}

\end{document}
