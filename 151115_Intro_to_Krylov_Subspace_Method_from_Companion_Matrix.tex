\documentclass[UTF8,nofonts]{ctexart}

\setCJKmainfont[BoldFont=STHeiti,ItalicFont=STKaiti]{STKaiti}
\setCJKsansfont[BoldFont=STHeiti]{STXihei}
\setCJKmonofont{STFangsong}

\usepackage[letterpaper,left=1in,top=1in]{geometry}
\usepackage{multirow}
\usepackage{amsmath}
\usepackage{amsfonts}
\usepackage{amssymb}
\usepackage[titlenotnumbered,lined,linesnumbered]{algorithm2e}

\begin{document}

% Title

Krylov子空间法是透过数个Krylov序列,建立一[label]相似变换[/label],使得$A$相似于一个上Hessenberg矩阵。Krylov序列的生成机制和Power迭代法相同,$A^k\boldsymbol{v}$终将收敛至$A$一个特征向量,但当$k$增大时,Krylov序列生成的向量方向逐渐趋于一致,[label]Krylov矩阵[/label]$K$跟着倾向[label]病态[/label]。好在[label]QR分解[/label]的出现克服了这种病态问题,因为具有标准正交向量的$Q$是良态的。由Krylov子空间法衍生的线性方程组求解算法,将另文详述。

\subsection*{Krylov子空间}

令$A$是一个$n \times n$阶[label]复矩阵[/label],$\boldsymbol{v}$为一个$n$维非零向量,定义向量序列$\{\boldsymbol{v},A\boldsymbol{v},\cdots,A^{r-1}\boldsymbol{v}\}$为[label]Krylov序列[/label]。此序列可以用来计算$A$的[label]特征多项式[/label],并且Krylov序列的[label]扩张[/label]称为[label]Krylov子空间[/label](或[label]循环子空间[/label]),记为:

\[
\mathcal{K}_r(A,\boldsymbol{v})=\text{span}\{\boldsymbol{v},A\boldsymbol{v},\cdots,A^{r-1}\boldsymbol{v}\}
\]

indent显然,$\mathcal{K}_r(A,\boldsymbol{v})$是$\mathbb{C}^n$的一个子空间,则必定存在一个最小正整数$k$,使得$A^k\boldsymbol{v}$可以表示为$\{\boldsymbol{v},A\boldsymbol{v},\cdots,A^{k-1}\boldsymbol{v}\}$的[label]线性组合[/label],即:

\[
A^k\boldsymbol{v}=a_0\boldsymbol{v}+a_1A\boldsymbol{v}+\cdots+a_{k-1}A^{k-1}\boldsymbol{v}
\]

indent为此,定义一个$k$次多项式:

\[
f(t)=t^k-a_{k_1}t^{k-1}-\cdots-a_1t-a_0
\]

indent因为有$f(A)\boldsymbol{v}=0$,所以$f(t)$是$\boldsymbol{v}$B相对于$A$的[label]消减多项式[/label]。另一方面,对于[label]矩阵向量对[/label]$(A,\boldsymbol{v})$,次数最小的首一消减多项式唯一存在,即最小多项式,记为$m(t)$(见矩阵的最小多项式及其应用一文)。所以,可以推知:$m(t)=f(t)$。

\subsubsection*{最小公倍式定理}

上述的讨论针对于某个向量$\boldsymbol{v}$,如果$\{\boldsymbol{v}_1,\boldsymbol{v}_2,\cdots,\boldsymbol{v}_n\}$构成$\mathbb{C}^n$的一组[label]基底[/label],那么$m_i(t)$就是$\boldsymbol{v}_i$相对于矩阵$A$的最小多项式。而[label]最小公倍式定理[/label]指出:矩阵$A$的最小多项式$m_A(t)$是$m_1(t),\cdots,m_n(t)$乘积的[label]最小公倍式[/label]。

\subsubsection*{相伴矩阵}

在详解相伴矩阵和循环子空间及其应用一文中指出:给定一个[label]首一多项式[/lable]:

\[p(t)=t^n+a_{n-1}t^{n-1}+\cdots+a_1t+a_0\]

indent总能找到一个$n \times n$阶矩阵,使其最小多项式恰好为该多项式$p(t)$,这个方阵便称为$p(t)$的相伴矩阵(companion matrix),它具有如下形式:

\[
C=\begin{pmatrix}
0&0&\cdots&0&-a_0\\
1&0&\cdots&0&-a_1\\
\vdots&\ddots&\ddots&\vdots&\vdots\\
0&\cdots&1&0&-a_{n-2}\\
0&0&\cdots&1&-a_{n-1}
\end{pmatrix}
\]

indent文章同时证明了该多项式$p(t)$同时也是相伴矩阵$C$的特征多项式和最小多项式,即:$p_C(t)=m_C(t)=p(t)$。

\subsection*{Krylov子空间法}

上面的最小公倍式定理指出,非零向量$\boldsymbol{v}_i$相对于$n \times n$阶矩阵$A$的最小多项式$m_i(t)$整除$A$的最小多项式$m_A(t)$,又根据最小多项式和特征多项式的关系(见矩阵的最小多项式及其应用一文),有:$A$最小多项式$m_A(t)$整除$A$的特征多项式$p_A(t)$,于是,Krylov子空间法指出:$A$的特征多项式$p_A(t)$也可以表示为$\boldsymbol{v}_i$相对于$A$的最小多项式$m_i(t)$的乘积。

设非零向量$\boldsymbol{v}_1$为Krylov序列的[label]种子向量[/label],则$\boldsymbol{v}_1$相对于$A$的最小多项式为$m_1(t)=t^k-\sum_{i=0}^{k-1}c_it^i$。将线性无关的Krylov序列$\{\boldsymbol{v}_1,A\boldsymbol{v}_1,\cdots,A^{k-1}\boldsymbol{v}_1\}$合并为一个$n \times k$的矩阵$K_1=\begin{pmatrix}\boldsymbol{v}_1&A\boldsymbol{v}_1&\cdots&A^{k-1}\boldsymbol{v}_1\end{pmatrix}$,该矩阵也称为[label]Krylov矩阵[/label],且$\text{dim}K_1=k$。令$C_1$为对应多项式$m_1(t)$的$k \times k$阶相伴矩阵,则:

\[
\begin{aligned}
K_1C_1&=\begin{pmatrix}
& & & & \\
& & & & \\
\boldsymbol{v}_1 & A\boldsymbol{v}_1 & \cdots & A^{k-2}\boldsymbol{v}_1 & A^{k-1}\boldsymbol{v}_1 \\
& & & & \\
& & & & \\
\end{pmatrix}
\begin{pmatrix}
0 & 0 & \cdots & 0 & a_0 \\
1 & 0 & \cdots & 0 & a_1 \\
\vdots & \ddots & \ddots & \vdots & \vdots \\
0 & \cdots & 1 & 0 & a_{k-2} \\
0 & 0 & \cdots & 1 & a_{k-1}
\end{pmatrix} \\
&=\begin{pmatrix}
& & & & \\
A\boldsymbol{v}_1 & A^2\boldsymbol{v}_1 & \cdots & A^{k-1}\boldsymbol{v}_1 & \displaystyle{\sum_{i=0}^{k-1}a_iA^i\boldsymbol{v}_1} \\
& & & & 
\end{pmatrix}=\begin{pmatrix}
& & & & \\
A\boldsymbol{v}_1 & A^2\boldsymbol{v}_1 & \cdots & A^{k-1}\boldsymbol{v}_1 & A^k\boldsymbol{v}_1 \\
& & & & 
\end{pmatrix} \\
&=AK_1
\end{aligned}
\]

indent如果有$k=n$,则$K_1^{-1}AK_1=C_1$,也就是说$A$相似于$C_1$。由于[label]相似矩阵[/lable]具有相同特征多项式,而$m_1(t)$同时是$C_1$的特征多项式和最小多项式,因此它也是$A$的特征多项式,即:$p_A(t)=m_1(t)$。

另一方面,如果$k<n$,首先可以根据[label]Steinitz替换原则[/label],将$K_1$扩充为一个$n \times n$阶可逆矩阵$\tilde{K}_1=\big(K_1,K'_1\big)$,其中$K'_1$为$n \times (n-k)$阶矩阵,于是有:

\[
\begin{aligned}
A\tilde{K}_1=&
\begin{pmatrix}AK_1&AK'_1\end{pmatrix}=
\begin{pmatrix}K_1C_1&A{K}'_1\end{pmatrix} \\
=&\begin{pmatrix}K_1&{K}'_1\end{pmatrix}\begin{pmatrix}C_1&X\\0&A_2\end{pmatrix} \\
=&\tilde{K}_1\begin{pmatrix}C_1&X\\0&A_2\end{pmatrix}
\end{aligned}
\]

indent进一步可以推得:

\[
\tilde{K}_1^{-1}A\tilde{K}_1=\begin{pmatrix}C_1&X\\0&A_2\end{pmatrix}
\]

indent也就是说,$A$相似于上式右端,因此:$p_A(t)=m_1(t)\det{(tI_{n-k}-A_2)}$。继续对$(n-k)\times(n-k)$阶矩阵$A_2$重复该步骤,便可得到最小多项式$m_2(t)$以及对应的相伴矩阵$C_2$,并有:

\begin{eqnarray*}
& K_2^{-1}A_2K_2=C_2 \\
& p_A(t)=m_1(t)m_2(t)
\end{eqnarray*}

假设经过$q$个步骤,最后$A$相似于一个分块上三角矩阵,称为[label]Frobenius矩阵[/label],其形式为:

\[
K^{-1}AK=\begin{pmatrix}
C_1 & \cdots & \ast \\
& \ddots & \vdots \\
& & C_q
\end{pmatrix}=H
\]

indent其中,$C_i$是相伴矩阵,对应特征多项式为$m_i(t)$,而矩阵$A$的特征多项式则为:

\[p_A(t)=m_1(t) \cdots m_q(t)\]

值得注意的是,所有的相伴矩阵$C_i$都是[label]上Hessenberg矩阵[/lable],显然包含主对角分块$C_i$的分块上三角矩阵$H$也是一个上Hessenberg矩阵。




% H2, H3
\subsection*{dsfsdt}
%[latex mode=1]
%[+preamble]
%\usepackage[titlenotnumbered,lined,linesnumbered]{algorithm2e}
\let\oldnl\nl
\newcommand{\nonl}{\renewcommand{\nl}{\let\nl\oldnl}}
%[/preamble]
%\quicklatex{size=14}
\TitleOfAlgo{BLAS-3 Gaussian Elimination with Partial Pivoting}
\begin{algorithm}[H]
\For{$l\gets 1$ \KwTo $n-1$}{
$k\gets (l-1)b+1$\;
factorize $PA^{(l)}=LU$ using BLAS-2\;
store $L$ and $U$ in $A$\;
left multiply $A(1:n,k+b:n)$ by $P$\;
$A(k:k+b-1,k+b:n)\gets T^{-1}A(k:k+b-1,k+b:n)$\;
\nonl\Indp where $T$ is the unit-lower-trian of $A(k:k+b-1,k:k+b-1)$\;
\DontPrintSemicolon\Indm$A(k+b:n,k+b:n)\gets A(k+b:n,k+b:n)$\;
\PrintSemicolon\nonl\Indp$-A(k+b:n,k:k+b-1)A(k:k+b-1,k+b:n)$\;
}
\end{algorithm}
%[/latex]

% H4
% \subsubsection*{}

\end{document}
