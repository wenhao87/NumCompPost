\documentclass[UTF8,nofonts]{ctexart}

\setCJKmainfont[BoldFont=STHeiti,ItalicFont=STKaiti]{STKaiti}
\setCJKsansfont[BoldFont=STHeiti]{STXihei}
\setCJKmonofont{STFangsong}

\usepackage[letterpaper,left=1in,top=1in]{geometry}
\usepackage{multirow}
\usepackage{amsmath}
\usepackage{amsfonts}
\usepackage{amssymb}
\usepackage[titlenotnumbered,lined,linesnumbered]{algorithm2e}

\begin{document}

% Title

上文借由[label]相伴矩阵[/label]引入了[label]Krylov子空间[/label],它是由一组具有$p(A)\boldsymbol{v}$形式的向量所[label]张成[/label]的子空间,其中$p(A)$是一个关于矩阵$A$的多项式。事实上,求解B大型线性方程组的绝大部分迭代法,都是在Krylov子空间上进行[label]正交投影[/label]或者[label]斜投影[label],其核心思想就是用$p(A)\boldsymbol{b}$来近似$A^{-1}\boldsymbol{b}$。

在迭代法中投影过程的基本框架一文中提到,求解线性方程组$A\boldsymbol{x}=\boldsymbol{b}$的一种通用的[label]投影方法[/label]是:在$m$维[label]仿射空间[/label]$\boldsymbol{x}^{(0)}+\mathcal{K}_m$上搜索近似解$\boldsymbol{x}^\ast$,并要求其满足[label]Petrov-Galerkin条件[/label]:$\boldsymbol{b}-A\boldsymbol{x}^\ast\perp\mathcal{L}_m$,其中$\mathcal{L}_m$是另外一个$m$维子空间,$\boldsymbol{x}^{(0)}$为任意初始向量。[label]Krylov子空间法[/label]就是令子空间$\mathcal{K}_m$为Krylov子空间,即:

\[
\mathcal{K}_m(A,\boldsymbol{r}^{(0)})=\text{span}\{\boldsymbol{r}^{(0)},A\boldsymbol{r}^{(0)},A^2\boldsymbol{r}^{(0)},\cdots,A^{m-1}\boldsymbol{r}^{(0)}\}
\]

indent其中,$\boldsymbol{r}^{(0)}=\boldsymbol{b}-A\boldsymbol{x}^{(0)}$,且除非特别说明,一般就用$\mathcal{K}_m$代表$\mathcal{K}_m(A,\boldsymbol{r}^{(0)})$。对于不同的Krylov子空间法,它们的区别在于子空间$\mathcal{L}_m$的选取上,以及对线性方程组的[label]预处理[/label]上。$\mathcal{L}_m$的选取大致可以分为如下两类:

- $\mathcal{L}_m=\mathcal{K}_m$和$\mathcal{L}_m=A\mathcal{K}_m$

- $\mathcal{L}_m=\mathcal{K}_m(A^T,\boldsymbol{r}^{(0)})$(这里$\mathcal{L}_m$是一个关于$A^T$的Krylov子空间)

\subsection*{Krylov子空间}

令子空间$\mathcal{K}_m(A,\boldsymbol{v})$具有如下形式:

\begin{equation}
\label{eq:kmv}
\mathcal{K}_m(A,\boldsymbol{v})\equiv\text{span}\{\boldsymbol{v},A\boldsymbol{v},A^2\boldsymbol{v},\cdots,A^{m-1}\boldsymbol{v}\}
\end{equation}

indent则$\mathcal{K}_m$是$\mathbb{R}^n$的一个子空间,其中的任意向量$\boldsymbol{x}$都可以表示为:

\begin{align*}
\boldsymbol{x}&=a_0\boldsymbol{v}+a_1A\boldsymbol{v}+a_2A^2\boldsymbol{v}+\cdots+a_{m-1}A^{m-1}\boldsymbol{v} \\
&=(a_0+a_1A+a_2A^2+\cdots+a_{m-1}A^{m-1})\boldsymbol{v} \\
&=p(A)\boldsymbol{v}
\end{align*}

indent其中,$p$是一个最高次数不超过$m-1$的多项式。另一方面,向量$\boldsymbol{v}$的[label]最小多项式[/label]定义为满足$p(A)\boldsymbol{v}=0$且其中B次数最小的首一多项式$p$,而这个$\boldsymbol{v}$关于$A$的最小多项式的次数,便称为$\boldsymbol{v}$关于$A$的[label]阶数[/label](grade),也可以直接称为$\boldsymbol{v}$的阶数。根据[label]Cayley-Hamilton定理[/label],$\boldsymbol{v}$的阶数不超过$n$(见矩阵的特征多项式和Cayley-Hamilton定理一文)。所以,如果$\mu$是$\boldsymbol{v}$的阶数,则$\mathcal{K}_\mu$是$A$的不变子空间,且对于任意$m\geq\mu$,都有$\mathcal{K}_m=\mathcal{\mu}$,理由是$\mathcal{K}_m$的维数不会再增加。据此,可以得到另一个重要结论:Krylov子空间$\mathcal{K}_m$的维数为$m$,当且仅当$\boldsymbol{v}$关于$A$的阶数$\mu$满足:$\text{grade}(\mu)\geq m$,即:

\begin{eqnarray*}
& \dim(\mathcal{K}_m)=m\quad\Longleftrightarrow\quad\text{grade}(\boldsymbol{v})\geq m \\
& \dim(\mathcal{K}_m)=\min\{m,\text{grade}(\boldsymbol{v})\}
\end{eqnarray*}

在详解不变子空间和限定算子表示矩阵一文中提到,如果将$A$限定在一个子空间$\mathcal{X}$上,则可以用限定算子$A_{/\mathcal{X}}$来表示。令$Q$为$\mathcal{X}$上的一个[label]投影算子[/label],则$A$在$\mathcal{X}$上的投影部分可以表示为$QA_{/\mathcal{X}}$,那么如果$\mathcal{X}$是一个Krylov子空间$\mathcal{K}_m$,则有:$A_m=Q_mA_{/\mathcal{K}_m}$,其中,$Q_m$是$\mathcal{K}_m$上的投影算子,$A_m$是$A$在$\mathcal{K}_m$上的投影部分。

\subsection*{基本Arnoldi算法}

有两种方法对Krylov子空间$\mathcal{K}_m$进行正交化,一种是[label]Gram-Schmidt正交化[/label],另一种是[label]Householder正交化[/label]。本文将简单介绍基于Gram-Schmidt正交化的基本[label]Arnoldi算法[/label]。基于[label]改进的Gram-Schmidt正交化[/label]和Householder正交化的Arnoldi算法将另文详述。

\subsubsection*{Gram-Schmidt正交化思想}

在线性无关向量组的Gram-Schmidt正交化一文中,给出了经典的Gram-Schmidt正交化方法。其核心思想是:将$m$个线性无关的$n$维向量组$\{\boldsymbol{x}_j\}_{j=1}^m$,使用Gram-Schmidt正交化产生相互正交的单位向量组$\{\boldsymbol{q}_j\}_{j=1}^m$,即:

\[
\text{span}\{\boldsymbol{x}_1,\boldsymbol{x}_2,\cdots,\boldsymbol{x}_j\}=\text{span}\{\boldsymbol{q}_1,\boldsymbol{q}_2,\cdots,\boldsymbol{q}_j\},\quad j=1,2,\cdots,m
\]

indent其中,$\boldsymbol{q}_j$满足如下关系式:

\begin{align*}
\boldsymbol{q}_j^T\boldsymbol{x}_j\boldsymbol{q}_j&=\boldsymbol{x}_j-\left(\boldsymbol{q}_1^T\boldsymbol{x}_j\boldsymbol{q}_1+\boldsymbol{q}_2^T\boldsymbol{x}_j\boldsymbol{q}_2+\cdots+\boldsymbol{q}_{j-1}^T\boldsymbol{x}_j\boldsymbol{q}_{j-1}\right) \\
&=\boldsymbol{x}_j-\sum_{i=1}^{j-1}\boldsymbol{q}_i^T\boldsymbol{x}_j\boldsymbol{q}_i\\
\Longleftrightarrow\quad r_{jj}\boldsymbol{q}_j&=\boldsymbol{x}_j-\sum_{i=1}^{j-1}r_{ij}\boldsymbol{q}_i=\tilde{\boldsymbol{q}}_j \\
\text{where }&
\begin{cases}
r_{ij}=\boldsymbol{q}_i^T\boldsymbol{x}_j=(\boldsymbol{x}_j,\boldsymbol{q}_i),\quad i=1,\cdots,j-1 \\
r_{jj}=\|\tilde{\boldsymbol{q}}_j\|
\end{cases}
\end{align*}

类比该思想,Krylov子空间$\mathcal{K}_m$的Gram-Schmidt正交化,可以表示为:

\[
\begin{array}{rrrrrrrr}
\text{span}\{ & \boldsymbol{v}_1, & A\boldsymbol{v}_1, & \cdots, & A^{j-1}\boldsymbol{v}_1, & A^j\boldsymbol{v}_1, & \cdots, & A^{m-1}\boldsymbol{v}_1\} \\
=\:\text{span}\{ & \boldsymbol{v}_1, & \boldsymbol{v}_2, & \cdots, & \boldsymbol{v}_j, & \boldsymbol{v}_{j+1}, & \cdots, & \boldsymbol{v}_m\}
\end{array}
\]

indent也就是说,$\boldsymbol{v}_{j+1}$是$\boldsymbol{v}_1,\cdots,\boldsymbol{v}_j,A^j\boldsymbol{v}_1$的线性组合,但其中的$A^j\boldsymbol{v}_1$直接计算的话,工作量太大。仔细观察可以发现,$\boldsymbol{v}_j$是$\boldsymbol{v}_1,\cdots,\boldsymbol{v}_{j-1},A^{j-1}\boldsymbol{v}_1$的线性组合,那么同乘以$A$的话,则有$A\boldsymbol{v}_j$是$A\boldsymbol{v}_1,\cdots,A\boldsymbol{v}_{j-1},A^{j}\boldsymbol{v}_1$的线性组合,也就是说$A\boldsymbol{v}_j$包含有成分$A^j\boldsymbol{v}_1$。因此,$\boldsymbol{v}_{j+1}$可以表示成$\boldsymbol{v}_1,\cdots,\boldsymbol{v}_j,A\boldsymbol{v}_j$的线性组合,从而大大减少了计算量,并有如下关系式:

\begin{align*}
\boldsymbol{v}_{j+1}^T(A\boldsymbol{v}_j)\boldsymbol{v}_{j+1}&=
A\boldsymbol{v}_j-\Big(\boldsymbol{v}_1^T(A\boldsymbol{v}_j)\boldsymbol{v}_1+\boldsymbol{v}_2^T(A\boldsymbol{v}_j)\boldsymbol{v}_2+\cdots+\boldsymbol{v}_j^T(A\boldsymbol{v}_j)\boldsymbol{v}_j\Big)\\
&=A\boldsymbol{v}_j-\sum_{i=1}^j\boldsymbol{v}_i^T(A\boldsymbol{v}_j)\boldsymbol{v}_i \\
h_{j+1,j}\boldsymbol{v}_{j+1}&=A\boldsymbol{v}_j-\sum_{i=1}^jh_{ij}\boldsymbol{v}_i=\boldsymbol{w}_j,\quad\begin{cases}
h_{ij}=\boldsymbol{v}_i^T(A\boldsymbol{v}_j)=\left(A\boldsymbol{v}_j,\boldsymbol{v}_i\right),\quad i=1,\cdots,j \\
h_{j+1,j}=\|\boldsymbol{w}_j\|_2
\end{cases} \\
\boldsymbol{v}_{j+1}&=\boldsymbol{w}_j/\|\boldsymbol{w}_j\|_2=\boldsymbol{w}_j/h_{j+1,j}
\end{align*}

\subsubsection*{基本Arnoldi算法实现}

上述基于Gram-Schmidt正交化方法的Krylov子空间$\mathcal{K}_m$的正交化,便是[label]Arnoldi方法[/label]。在算法的第$j$步,将[label]Arnoldi向量[/label]$\boldsymbol{v}_j$(已正交化)左乘$A$,并按经典Gram-Schmidt正交化过程计算$\boldsymbol{w}_j$,使其与之前的所有$\boldsymbol{v}_i$都正交,并得到新的正交化向量$\boldsymbol{v}_{j+1}$。给定$n \times n$阶矩阵$A$以及正整数$m$,Arnoldi算法实现如下:

%[latex mode=1]
%[+preamble]
%\usepackage[titlenotnumbered,lined,linesnumbered]{algorithm2e}
%[/preamble]
\TitleOfAlgo{Arnoldi}
\begin{algorithm}[H]
choose a vector $\boldsymbol{v}_1$, such that $\|\boldsymbol{v}_1\|_2=1$\;
\For{$j\gets 1$ \KwTo $m$}{
	$h_{ij} \gets (A\boldsymbol{v}_j,\boldsymbol{v}_i),\:i=1,\cdots,j$\;
	$\boldsymbol{w}_j=A\boldsymbol{v}_j-\sum_{i=1}^jh_{ij}\boldsymbol{v}_i$\;
	$h_{j+1,j}=\|\boldsymbol{w}_j\|_2$\;
	\If{$h_{j+1,j}=0$}{Stop\;}
	$\boldsymbol{v}_{j+1}=\boldsymbol{w}_{j}/h_{j+1,j}$\;
}
\end{algorithm}
%[/latex]

\subsubsection*{基本Arnoldi算法分析}

如果上述算法在第$j(j \leq m)$步终止,也就是说:

\begin{eqnarray*}
& h_{i+1,i} \neq 0,\quad i=1,\cdots,j-1 \\
& h_{j+1,j}=0
\end{eqnarray*}

indent其含义等价于$\{\boldsymbol{v}_1,\cdots,\boldsymbol{v}_j\}$构成了$\mathcal{K}_m$的[label]标准正交基[/label],此时$\dim(\mathcal{K}_m)=\text{grad}  (A,\boldsymbol{v}_1)=j$,换言之,$\boldsymbol{v}_1$关于$A$的最小多项式的阶数为即为$j$。

如果上述算法在第$m$次循环之前,都没有结束,换言之,$h_{j+1,j} \neq 0$($j=1,\cdots,m-1$),则说明算法产生了$\boldsymbol{v}_1,\cdots,\boldsymbol{v}_m$,并且它们构成了$\mathcal{K}_m$的标准正交基,即:

\[\mathcal{K}_m=\text{span}\{\boldsymbol{v}_1,\boldsymbol{v}_2,\cdots,\boldsymbol{v}_m\}\]

indent另一方面,根据Gram-Schmidt正交化方法,有:

\begin{eqnarray*}
& \displaystyle\boldsymbol{v}_{j+1}^T(A\boldsymbol{v}_j)\boldsymbol{v}_{j+1}=A\boldsymbol{v}_j-\sum_{i=1}^j\boldsymbol{v}_i^T(A\boldsymbol{v}_j)\boldsymbol{v}_i \\
& \displaystyle h_{j+1,j}\boldsymbol{v}_{j+1}=A\boldsymbol{v}_j-\sum_{i=1
}^jh_{ij}\boldsymbol{v}_i\quad\Longleftrightarrow\quad A\boldsymbol{v}_j=\sum_{i=1}^{j+1}h_{ij}\boldsymbol{v}_i
\end{eqnarray*}

indent如果展开上面的最后一个等式,可以得到:

\begin{align}
\label{eq:avhv}
\begin{split}
A\boldsymbol{v}_1 &= h_{11}\boldsymbol{v}_1+h_{21}\boldsymbol{v}_2 \\
A\boldsymbol{v}_2 &= h_{12}\boldsymbol{v}_1+h_{22}\boldsymbol{v}_2+h_{32}\boldsymbol{v}_3 \\
&\cdots \\
A\boldsymbol{v}_{m-1} &= h_{1,m-1}\boldsymbol{v}_1+\cdots+h_{m-1,m-1}\boldsymbol{v}_{m-1}+h_{m,m-1}\boldsymbol{v}_m \\
A\boldsymbol{v}_{m} &= h_{1m}\boldsymbol{v}_1+\cdots+h_{mm}\boldsymbol{v}_m+\big(h_{m+1,m}\boldsymbol{v}_{m+1}\big)
\end{split}
\end{align}

indent如果将式(\ref{eq:avhv})用矩阵形式表示的话,有:

\begin{align}
\label{eq:avvhbar}
\begin{split}
A\begin{pmatrix}\boldsymbol{v}_1 & \cdots & \boldsymbol{v}_m \end{pmatrix} &=
\begin{pmatrix}\begin{array}{ccc|c}\boldsymbol{v}_1 & \cdots & \boldsymbol{v}_{m} & \boldsymbol{v}_{m+1}\end{array}\end{pmatrix}
\begin{pmatrix}
h_{11} & h_{21} & \cdots & \cdots & \cdots \\
h_{21} & h_{22} & \cdots & \cdots & \cdots \\
& h_{32} & \ddots &  & \vdots \\
& & \ddots & \ddots & \vdots \\
& & & h_{m,m-1} & h_{mm} \\
\cline{1-5}
& & & & h_{m+1,m}
\end{pmatrix} \\
\Longleftrightarrow\quad AV_m&=V_{m+1}\bar{H}_m
\end{split}
\end{align}

indent其中,$V_m$是列向量$\boldsymbol{v}_1,\cdots,\boldsymbol{v}_m$构成的$n \times m$阶矩阵,$V_{m+1}=\left(V_m,\boldsymbol{v}_{m+1}\right)$的阶数则为$n \times (m+1)$,$\bar{H}_m$是$(m+1) \times m$阶[label]上Hessenberg矩阵[/label]。观察式(\ref{eq:avhv}),其最后一项$h_{m+1,m}\boldsymbol{v}_{m+1}$即为$\boldsymbol{w}_j$,则$A\boldsymbol{v}_m$可以表示为:

\[A\boldsymbol{v}_m=\begin{pmatrix}\boldsymbol{v}_1&\cdots&\boldsymbol{v}_m\end{pmatrix}\begin{pmatrix}h_{1m}\\\cdots\\h_{mm}\end{pmatrix}+\boldsymbol{w}_j\]

indent因此,式(\ref{eq:avvhbar})可以进一步改写为如下形式:

\begin{equation}
\label{eq:avvh}
AV_m=V_mH_m+\boldsymbol{w}_j\boldsymbol{e}_m^T
\end{equation}

indent其中,$H_m$是$\bar{H}_m$去掉最后一行得到的上Hessenberg矩阵。如果将式(\ref{eq:avvh})两边同时左乘$V_m^T$,由其正交化性质,可以推知:$V^T_mV_m=I$,$V^T_m\boldsymbol{w}_j\boldsymbol{e}_m^T=0$,所以最后有:

\[
V^T_mAV_m=H_m
\]

% H2, H3
\subsection*{dsfsdt}
%[latex mode=1]
%[+preamble]
%\usepackage[titlenotnumbered,lined,linesnumbered]{algorithm2e}
\let\oldnl\nl
\newcommand{\nonl}{\renewcommand{\nl}{\let\nl\oldnl}}
%[/preamble]
%\quicklatex{size=14}
\TitleOfAlgo{BLAS-3 Gaussian Elimination with Partial Pivoting}
\begin{algorithm}[H]
\For{$l\gets 1$ \KwTo $n-1$}{
$k\gets (l-1)b+1$\;
factorize $PA^{(l)}=LU$ using BLAS-2\;
store $L$ and $U$ in $A$\;
left multiply $A(1:n,k+b:n)$ by $P$\;
$A(k:k+b-1,k+b:n)\gets T^{-1}A(k:k+b-1,k+b:n)$\;
\nonl\Indp where $T$ is the unit-lower-trian of $A(k:k+b-1,k:k+b-1)$\;
\DontPrintSemicolon\Indm$A(k+b:n,k+b:n)\gets A(k+b:n,k+b:n)$\;
\PrintSemicolon\nonl\Indp$-A(k+b:n,k:k+b-1)A(k:k+b-1,k+b:n)$\;
}
\end{algorithm}
%[/latex]

% H4
% \subsubsection*{}

\end{document}
