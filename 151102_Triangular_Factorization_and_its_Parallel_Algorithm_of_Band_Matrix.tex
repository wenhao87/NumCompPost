\documentclass[UTF8,nofonts]{ctexart}

\setCJKmainfont[BoldFont=STHeiti,ItalicFont=STKaiti]{STKaiti}
\setCJKsansfont[BoldFont=STHeiti]{STXihei}
\setCJKmonofont{STFangsong}

\usepackage[letterpaper,left=1in,top=1in]{geometry}
\usepackage{multirow}
\usepackage{amsmath}
\usepackage{amsfonts}
\usepackage{amssymb}
\usepackage[titlenotnumbered,lined,linesnumbered]{algorithm2e}

\begin{document}

% Title
\section*{带状矩阵的直接三角分解}

上文(见对称(正定)矩阵的直接三角分解一文)讨论了一类特殊矩阵——[label]对称(正定)矩阵[/label]的[label]直接三角分解法[/label],本文将针对另一类特殊矩阵——[label]带状矩阵[/label],给出其直接三角分解法。

\subsection*{带状矩阵的定义和重要定理}

如果矩阵的非零元集中在对角线附近,则称为[label]带状矩阵[/label]。对于$n \times n$阶矩阵$A$,若$a_{ij}=0,j>i+r$,则称$A$的[label]上带宽[/label]为$r$;若$a_{ij}=0,i>j+s$,则称$A$的[label]下带宽[/label]为$s$;若以上两条同时满足,则称$r+s+1$为$A$的[label]带宽[/label]。

[alert type="info"]对于一般$m \times n$阶矩阵$A$,称$w=\displaystyle{\max_{1\leq i \leq m}\{j-k+1:a_{ij}a_{ik}\neq 0\}}$为$A$的带宽。[/alert]

带状矩阵具有以下两个重要性质:

若$A,B\in\mathbb{R}^{n \times n}$分别有下(上)带宽$r$和下(上)带宽$s$,则它们的乘积$AB$有下(上)带宽$r+s$。(证略)

设$A\in\mathbb{R}^{n \times n}$为上带宽$r$、下带宽$s$的带状矩阵,对$A$进行LU分解得到$A=LU$,其中$L$为单位下三角矩阵,$U$为上三角矩阵,则$U$的上带宽$r$,$L$的下带宽为$s$。(证略)

\begin{align*}
a_{ij}=\sum_{k=1}^nl_{ik}u_{kj}&=\sum_{k=1}^{i-1}l_{ik}u_{kj}+u_{ij},\quad i \leq j\\
&=\sum_{k=1}^{j}l_{ik}u_{kj},\quad i \geq j
\end{align*}

\[
\begin{pmatrix}
1 & & & & & & & & \\
\vdots & \ddots & & & & & & & \\
\vdots & & \ddots & & & & & & \\
\cline{1-4}
l_{i1} & & & \multicolumn{1}{|c|}{1} & & & & & \\
\cline{1-4}
\vdots & & & & \ddots & & & & \\
l_{s1} & & & & & 1 & & & \\
& \ddots & & & & & \ddots & \\
& & \ddots & & & & & \ddots & \\
& & & l_{n,n-s+1} & \cdots & l_{n,n-i+1} & \cdots & \cdots & 1 \\
\end{pmatrix}
\begin{pmatrix}
\cline{4-4}
u_{11} & \cdots & \cdots & \multicolumn{1}{|c|}{u_{1j}} & \cdots & u_{1r} & & & \\
& \ddots & & \multicolumn{1}{|c|}{} & & & \ddots & & \\
& & \ddots & \multicolumn{1}{|c|}{} & & & & \ddots &\\
\cline{4-4}
& & & \multicolumn{1}{|c|}{u_{jj}} & & & & & u_{n-r+1,n} \\
\cline{4-4}
& & & & \ddots & & & & \vdots \\
& & & & & u_{rr} & & & u_{n-j+1,n}\\
& & & & & & \ddots & & \vdots \\
& & & & & & & \ddots & \vdots \\
& & &  & & & & & u_{nn} \\
\end{pmatrix}
\]

\subsection*{特殊带状矩阵的逆矩阵直接分解法}

有两类特殊的带状矩阵:[label]Hessenberg矩阵[/label]和[label]对称三对角矩阵[/label]。若$n \times n$阶矩阵$H$满足:$h_{ij}=0$,其中$1 \leq j<i-1 \leq n$,则称其为[label]上Hessenberg矩阵[/label],类似的也有[label]下Hessenberg矩阵[/label](见Hessenberg矩阵详解一文)。

[alert type="error"]注意:带状矩阵的逆矩阵一般不是带状矩阵。[/alert]

若$H$满足$h_{j+1,j} \neq 0$,其中$j=1,2,\cdots,n-1$,则$H$[label]非奇异[/label],且存在两个$n$维向量$\boldsymbol{p}=(p_i)\in\mathbb{R}^{n}$和$\boldsymbol{q}=(q_i)\in\mathbb{R}^{n}$,使得其逆矩阵满足:

\[
(H^{-1})_{ij}=p_iq_j,\quad i \geq j
\]

更进一步,若$H$为[label]次对角线[/label]非零的[label]对称三对角矩阵[/label],由于$H^{-1}$也是对称矩阵,则有:

\[
(H^{-1})_{ij}=
\begin{cases}
p_iq_j,&\quad i \geq j\\
p_jq_i,&\quad i < j\\
\end{cases}
\]

写成矩阵形式如下:

\[
H=
\begin{pmatrix}
\alpha_1 & \beta_2 & & & &\\
\cline{1-3}
\beta_2 & \alpha_2 & \multicolumn{1}{c|}{\multirow{1}{*}{$\beta_3$}} & & &\\
\cline{1-3}
& \beta_3 & \alpha_3 & \beta_4 & & \\
& & \ddots & \ddots & \ddots & \\
& & & \beta_{n-1} & \alpha_{n-1} & \beta_n \\
& & & & \beta_n & \alpha_n
\end{pmatrix},\quad
H^{-1}=
\begin{pmatrix}
\cline{3-3}
p_1q_1 & p_2q_1 & \multicolumn{1}{|c|}{\multirow{1}{*}{$p_3q_1$}} & \cdots & \cdots & p_nq_1 \\
\vdots & p_2q_2 & \multicolumn{1}{|c|}{\multirow{1}{*}{$p_3q_2$}} & \cdots & \cdots & p_{n}q_2 \\
\vdots & \vdots & \multicolumn{1}{|c|}{\multirow{1}{*}{$p_3q_3$}} & \cdots & \cdots & p_{n}q_3 \\
\cline{3-3}
\vdots & \vdots & \vdots & \ddots & \vdots & \vdots \\
\vdots & \vdots & \cdots & \cdots & \ddots & \vdots \\
\vdots & \cdots & \cdots & \cdots & \cdots & p_nq_n
\end{pmatrix}
\]

利用$HH^{-1}=I$及其次对角线为$0$的特点,从上式可以推得:

\begin{align*}
\begin{pmatrix}\beta_i & \alpha_i & \beta_{i+1} \end{pmatrix}
\begin{pmatrix}p_{i+1}q_{i-1} \\ p_{i+1}q_{i} \\ p_{i+1}q_{i+1}\end{pmatrix}=0
&\Longrightarrow \beta_iq_{i-1}+\alpha_iq_i+\beta_{i+1}q_{i+1}=0 \\
&\Longrightarrow q_{i+1}=-\dfrac{1}{\beta_{i+1}}(\beta_iq_{i-1}+\alpha_iq_i),\quad i=2,3,\cdots,n-1
\end{align*}

indent 这里令$q_1=1$,显然有$q_2=-\alpha_1/\beta_2$,则利用上式,可以迭代计算得到$q_i$。同样,利用$H^{-1}H=I$及其次对角线为$0$的特点,可以逆序计算求得$p_i$:

\begin{align*}
\begin{pmatrix}p_iq_i & p_{i+1}q_i & p_{i+2}q_i \end{pmatrix}
\begin{pmatrix}\beta_{i+1} \\ \alpha_{i+1} \\ \beta_{i+2}\end{pmatrix}=0
&\Longrightarrow \beta_{i+1}p_i+\alpha_{i+1}p_{i+1}+\beta_{i+2}p_{i+2}=0 \\
&\Longrightarrow p_{i}=-\dfrac{1}{\beta_{i+1}}(\alpha_{i+1}p_{i+1}+\beta_{i+2}p_{i+2}),\quad i=n-2,n-3,\cdots,1
\end{align*}

indent 其中$p_n=1/(\beta_nq_{n-1}+\alpha_nq_n)$,$p_{n-1}=-\alpha_np_n/\beta_n$。


\subsection*{带状矩阵的Gauss消去法}

\subsubsection*{Gauss消去法算法实现}

与BLAS三个级别的列主元Gauss消去法一文中提出的算法类似,只需更改一下循环上界,即可得到带状矩阵的Gauss消去法。设$A\in\mathbb{R}^{n \times n}$为上带宽$r$、下带宽$s$的带状矩阵,其Gauss消去法算法实现如下:

%[latex mode=1]
%[+preamble]
%\usepackage[titlenotnumbered,lined,linesnumbered]{algorithm2e}
%\let\oldnl\nl
%\newcommand{\nonl}{\renewcommand{\nl}{\let\nl\oldnl}}
%[/preamble]
\TitleOfAlgo{Gaussian Elimination on Banded Matrix}
\begin{algorithm}[H]
\For{$k\gets 1$ \KwTo $n-1$}{
	\For{$i\gets k+1$ \KwTo $\min\{k+s,n\}$}{
		$a_{ik}\gets a_{ik}/a_{kk}$\;
		\For{$j\gets k+1$ \KwTo $\min\{k+r,n\}$}{
			$a_{ij}\gets a_{ij}-a_{ik}a_{kj}$\;
		}
	}
}
%$k\gets (l-1)b+1$\;
%factorize $PA^{(l)}=LU$ using BLAS-2\;
%store $L$ and $U$ in $A$\;
%left multiply $A(1:n,k+b:n)$ by $P$\;
%$A(k:k+b-1,k+b:n)\gets T^{-1}A(k:k+b-1,k+b:n)$\;
%\nonl\Indp where $T$ is the unit-lower-trian of $A(k:k+b-1,k:k+b-1)$\;
%\DontPrintSemicolon\Indm$A(k+b:n,k+b:n)\gets A(k+b:n,k+b:n)$\;
%\PrintSemicolon\nonl\Indp$-A(k+b:n,k:k+b-1)A(k:k+b-1,k+b:n)$\;
%}
\end{algorithm}
%[/latex]

\subsubsection*{算法计算量分析}

分析上述算法,可以计算出其所需要的乘法次数为:

\begin{align*}
\label{eq:sum}
\sum_{k=1}^{n-1}\sum_{i=k+1}^{\min\{k+s,n\}}\sum_{j=k+1}^{\min\{k+r,n\}}1&=
\sum_{k=1}^{n-r-1}\sum_{i=k+1}^{\min\{k+s,n\}}\sum_{j=k+1}^{\min\{k+r,n\}}1+
\sum_{k=n-r}^{n-1}\sum_{i=k+1}^{\min\{k+s,n\}}\sum_{j=k+1}^{\min\{k+r,n\}}1 \\
&=\sum_{k=1}^{n-r-1}\sum_{i=k+1}^{\min\{k+s,n\}}\sum_{j=k+1}^{k+r}1+
\sum_{k=n-r}^{n-1}\sum_{i=k+1}^{\min\{k+s,n\}}\sum_{j=k+1}^{n}1 \\
&=\sum_{k=1}^{n-r-1}\sum_{i=k+1}^{\min\{k+s,n\}}r+
\sum_{k=n-r}^{n-1}\sum_{i=k+1}^{\min\{k+s,n\}}(n-k) \\
\end{align*}

若$r \leq s$:

\begin{align*}
\sum_{k=1}^{n-r-1}\sum_{i=k+1}^{\min\{k+s,n\}}r&=\sum_{k=1}^{n-s-1}\sum_{i=k+1}^{k+s}r+\sum_{k=n-s}^{n-r-1}\sum_{i=k+1}^{n}r=r\left[s(n-s-1)+\dfrac{(s-r)(s+r+1)}{2}\right]\\
\sum_{k=n-r}^{n-1}\sum_{i=k+1}^{\min\{k+s,n\}}(n-k)&=
\sum_{k=n-r}^{n-1}\sum_{i=k+1}^{n}(n-k)=\sum_{k=n-r}^{n-1}(n-k)^2=
\dfrac{r(r+1)(2r+1)}{6}
\end{align*}

上两式相加,即得总的乘法次数为:

\begin{equation}
\label{eq:rleqs}
r\left[s(n-s-1)+\dfrac{(s-r)(s+r+1)}{2}\right]+\dfrac{r(r+1)(2r+1)}{6}=
r\left[sn-\dfrac{s^2}{2}-\dfrac{r^2}{2}-\dfrac{s}{2}+\dfrac{1}{6}\right]
\end{equation}

类似地,当$s\leq r$时,其总的乘法次数为:

\begin{equation}
\label{eq:sleqr}
s\left[r(n-r-1)+\dfrac{(r+s)(r-s+1)}{2}\right]+\dfrac{s(s-1)(2s-1)}{6}=
s\left[rn-\dfrac{r^2}{2}-\dfrac{s^2}{2}-\dfrac{r}{2}+\dfrac{1}{6}\right]
\end{equation}

可以发现,式(\ref{eq:sleqr})正好是式(\ref{eq:rleqs})中的$r$和$s$互换得到的。特别地,其所需乘法次数为:

\[
\begin{cases}
r\left[rn-\dfrac{2r^2}{3}-\dfrac{s}{2}+\dfrac{1}{6}\right],&r=s\\
\dfrac{n(n-1)(2n-1)}{6},&r=s=n-1
\end{cases}
\]

其中,$r=s=n-1$的情况,就是满矩阵Gauss消元法的乘法次数。


[alert type="info"]对于带状矩阵,如果使用[label]不选主元的Gauss消去法[/label](见不选主元的Gauss消去法一文),所产生的下三角矩阵和上三角矩阵的非零元也在对角线附近。换言之,Gauss消去法具有[label]保带宽性[/label]。[/alert]

\subsection*{带状系数矩阵线性方程组的并行求解}

对于线性方程组$A\boldsymbol{x}=\boldsymbol{b}$,若[label]系数矩阵[/label]$A$为带状矩阵,则可以使用如下的并行求解方法。假设有$2$个进程,$n \times n$阶矩阵$A$的带宽为$w$,则可以先对系数矩阵$A$进行如下分块:

【图片】

indent 其中,$A_1\in\mathbb{R}^{m_1\times n_1}$,$B_1\in\mathbb{R}^{m_1\times w}$,$A_2\in\mathbb{R}^{m_2\times n_2}$,$C_2\in\mathbb{R}^{m_2\times w}$,且$n=n_1+w+n_2$。于是线性方程组$A\boldsymbol{x}=\boldsymbol{b}$可以表示为:

\begin{equation}
\label{eq:b12}
\begin{pmatrix}
A_1 & B_1 & \\
& C_2 & A_2
\end{pmatrix}
\begin{pmatrix}
\boldsymbol{x}' \\
\boldsymbol{\xi} \\
\boldsymbol{x}''
\end{pmatrix}=
\begin{pmatrix}
\boldsymbol{b}_1 \\
\boldsymbol{b}_2
\end{pmatrix},\quad
\boldsymbol{x}'\in\mathbb{R}^{n_1},\quad\boldsymbol{\xi}\in\mathbb{R}^{w},\quad\boldsymbol{x}''\in\mathbb{R}^{n_2}
\end{equation}

\begin{equation}
\label{eq:b1b2}
\begin{pmatrix}
A_1 & B_1
\end{pmatrix}
\begin{pmatrix}
\boldsymbol{x}' \\
\boldsymbol{\xi}
\end{pmatrix}=\boldsymbol{b}_1,\quad
\begin{pmatrix}
C_2 & A_2
\end{pmatrix}
\begin{pmatrix}
\boldsymbol{\tilde{\xi}} \\
\boldsymbol{x}''
\end{pmatrix}=\boldsymbol{b}_2
\end{equation}

indent 所以,当$\boldsymbol{\xi}=\boldsymbol{\tilde{\xi}}$时,式(\ref{eq:b1b2})的解就是方程组(\ref{eq:b12})的解。如果令$E_1=\left(A_1,B_2\right)$,$E_2=\left(C_2,A_2\right)$,则方程组(\ref{eq:b1b2})的解可以表示为一个[label]特解[/label]与扩张成[label]零空间[/label]$N(E_i)$($E_{i}\boldsymbol{x}_i=0$)的一组[label]标准正交基[/label]之和,即$\boldsymbol{x}_i=\boldsymbol{p}_{i}+Q_i\boldsymbol{y}_i$,其中,特解$\boldsymbol{p}_i\in\mathbb{R}^{n_i+w}$($i=1,2$),$Q_i\in\mathbb{R}^{(n_i+w)\times(n_i+w-m_i)}$,$\boldsymbol{y}_i\in\mathbb{R}^{n_i+w-m_i}$。对$\boldsymbol{x}_i$继续分块,可以得到:

\begin{equation}
\label{eq:y1y2}
\begin{pmatrix}
\multicolumn{1}{c}{\multirow{2}{*}{$\boldsymbol{x}_1$}}\\
\\
\end{pmatrix}=
\begin{pmatrix}
\boldsymbol{x}'\\
\boldsymbol{\xi}
\end{pmatrix}=
\begin{pmatrix}
\boldsymbol{p}_{1,1}\\
\boldsymbol{p}_{1,2}
\end{pmatrix}+
\begin{pmatrix}
\boldsymbol{Q}_{1,1}\\
\boldsymbol{Q}_{1,2}
\end{pmatrix}\boldsymbol{y}_1,\quad
\begin{pmatrix}
\multicolumn{1}{c}{\multirow{2}{*}{$\boldsymbol{x}_2$}}\\
\\
\end{pmatrix}=
\begin{pmatrix}
\boldsymbol{\tilde{\xi}}\\
\boldsymbol{x}''
\end{pmatrix}=
\begin{pmatrix}
\boldsymbol{p}_{2,1}\\
\boldsymbol{p}_{2,2}
\end{pmatrix}+
\begin{pmatrix}
\boldsymbol{Q}_{2,1}\\
\boldsymbol{Q}_{2,2}
\end{pmatrix}\boldsymbol{y}_2
\end{equation}

indent 其中,$\boldsymbol{y}_i\in\mathbb{R}^{n_i+w-m_i}$,$\boldsymbol{p}_i=(\boldsymbol{p}_{i,1}^T,\boldsymbol{p}_{i,2}^T)$,$Q_i=(Q_{i,1}^T,Q_{i,2}^T)$($i=1,2$)。为了满足$\boldsymbol{\xi}=\boldsymbol{\tilde{\xi}}$,必须有:

\begin{align}
\label{eq:myg}
&\quad\boldsymbol{p}_{1,2}+Q_{1,2}\boldsymbol{y}_1=\boldsymbol{p}_{2,1}+Q_{2,1}\boldsymbol{y}_2 \nonumber\\
\Longleftrightarrow &\quad Q_{1,2}\boldsymbol{y}_1-Q_{2,1}\boldsymbol{y}_2=\boldsymbol{p}_{2,1}-\boldsymbol{p}_{1,2} \nonumber\\
\Longleftrightarrow &\quad
\begin{pmatrix}Q_{1,2}&-Q_{2,1}\end{pmatrix}\begin{pmatrix}\boldsymbol{y}_1\nonumber\\\boldsymbol{y}_2\end{pmatrix}=\boldsymbol{p}_{2,1}-\boldsymbol{p}_{1,2} \nonumber\\
\Longleftrightarrow &\quad M\boldsymbol{y}=\boldsymbol{g}
\end{align}

indent 其中,$M=(Q_{1,2},-Q_{2,1})\in\mathbb{R}^{w \times w}$,$\boldsymbol{y}=(\boldsymbol{y}_1^T,\boldsymbol{y}_2^T)^T\in\mathbb{R}^{w}$,$\boldsymbol{g}=\boldsymbol{p}_{2,1}-\boldsymbol{p}_{1,2}\in\mathbb{R}^{w}$。

\subsubsection*{带状系数矩阵的并行求解算法}

给定线性方程组$A\boldsymbol{x}=\boldsymbol{b}$,其中$A$为带状矩阵,其并行求解算法实现如下(假设进程数为$2$):

%[latex mode=1]
%[+preamble]
%\usepackage[titlenotnumbered,lined,linesnumbered]{algorithm2e}
%\let\oldnl\nl
%\newcommand{\nonl}{\renewcommand{\nl}{\let\nl\oldnl}}
%[/preamble]
\TitleOfAlgo{Solve Banded Linear System in Paralle}
\begin{algorithm}[H]
solve eq.(\ref{eq:b1b2}) to get $\boldsymbol{p}_i,Q_i$, where $i=1,2$\;
solve eq.(\ref{eq:myg}) to get $\boldsymbol{y}$\;
find $\boldsymbol{x}_1,\boldsymbol{x}_2$ and $\boldsymbol{\xi}=\boldsymbol{\tilde{\xi}}$ from eq.(\ref{eq:y1y2})\;
%\For{$l\gets 1$ \KwTo $n-1$}{
%$k\gets (l-1)b+1$\;
%factorize $PA^{(l)}=LU$ using BLAS-2\;
%store $L$ and $U$ in $A$\;
%left multiply $A(1:n,k+b:n)$ by $P$\;
%$A(k:k+b-1,k+b:n)\gets T^{-1}A(k:k+b-1,k+b:n)$\;
%\nonl\Indp where $T$ is the unit-lower-trian of $A(k:k+b-1,k:k+b-1)$\;
%\DontPrintSemicolon\Indm$A(k+b:n,k+b:n)\gets A(k+b:n,k+b:n)$\;
%\PrintSemicolon\nonl\Indp$-A(k+b:n,k:k+b-1)A(k:k+b-1,k+b:n)$\;
%}
\end{algorithm}
%[/latex]

显然,算法的第一步各个进程可以并行执行;第二步求解$w \times w$阶线性方程组,可用穿行方法求解,其计算量远比$n \times n$阶线性方程组的求解要少;第三步也是可以并行计算的。最后值得一提的是,这种方法也可以推广到$p$个进程的情况。

% H2, H3
\subsection*{}
%[latex mode=1]
%[+preamble]
%\usepackage[titlenotnumbered,lined,linesnumbered]{algorithm2e}
%\let\oldnl\nl
%\newcommand{\nonl}{\renewcommand{\nl}{\let\nl\oldnl}}
%[/preamble]
%\quicklatex{size=14}
%\TitleOfAlgo{BLAS-3 Gaussian Elimination with Partial Pivoting}
%\begin{algorithm}[H]
%\For{$l\gets 1$ \KwTo $n-1$}{
%$k\gets (l-1)b+1$\;
%factorize $PA^{(l)}=LU$ using BLAS-2\;
%store $L$ and $U$ in $A$\;
%left multiply $A(1:n,k+b:n)$ by $P$\;
%$A(k:k+b-1,k+b:n)\gets T^{-1}A(k:k+b-1,k+b:n)$\;
%\nonl\Indp where $T$ is the unit-lower-trian of $A(k:k+b-1,k:k+b-1)$\;
%\DontPrintSemicolon\Indm$A(k+b:n,k+b:n)\gets A(k+b:n,k+b:n)$\;
%\PrintSemicolon\nonl\Indp$-A(k+b:n,k:k+b-1)A(k:k+b-1,k+b:n)$\;
%}
%\end{algorithm}
%[/latex]

% H4
% \subsubsection*{}

\end{document}
