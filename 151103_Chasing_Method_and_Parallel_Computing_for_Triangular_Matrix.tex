\documentclass[UTF8,nofonts]{ctexart}

\setCJKmainfont[BoldFont=STHeiti,ItalicFont=STKaiti]{STKaiti}
\setCJKsansfont[BoldFont=STHeiti]{STXihei}
\setCJKmonofont{STFangsong}

\usepackage[letterpaper,left=1in,top=1in]{geometry}
\usepackage{multirow}
\usepackage{amsmath}
\usepackage{amsfonts}
\usepackage{amssymb}
\usepackage[titlenotnumbered,lined,linesnumbered]{algorithm2e}

\begin{document}

% Title
\section*{三对角占优矩阵的追赶法分解和三对角方程组的并行求解}

\subsection*{三对角占优矩阵的定义}

考虑[label]带宽[/label]为$3$的[label]$n \times n$阶系数矩阵[/label](即[label]三对角矩阵[label])所对应的[label]线性方程组[/label]:

\begin{equation}
\label{eq:trimat}
\begin{pmatrix}
b_1 & c_1 & & & & & \\
a_2 & b_2 & c_2 & & & & \\
 & \ddots & \ddots & \ddots & & & \\
& & a_i & b_i & c_i & & \\
& & & \ddots & \ddots & \ddots & \\
& & & & a_{n-1} & b_{n-1} & c_{n-1} \\
& & & & & b_{n} & c_{n} \\
\end{pmatrix}
\begin{pmatrix}
x_1 \\ x_2 \\ \vdots \\ x_i \\ \vdots \\ x_{n-1} \\ x_n
\end{pmatrix}=
\begin{pmatrix}
f_1 \\ f_2 \\ \vdots \\ f_i \\ \vdots \\ f_{n-1} \\ f_n
\end{pmatrix}
\end{equation}

indent 其中,若系数矩阵$A$满足:

\[
A_{ij}=0,\quad|i-j|>0,\quad|b_1|>|c_1|>0,\quad|b_n|>|c_n|>0
\]

\[
|b_i|\geq|a_i|+|c_i|,\quad a_i,c_i \neq 0,\quad i=2,3,\cdots,n-1
\]

indent 则称系数矩阵$A$为[label]三对角占优矩阵[/label](也可参考矩阵相关的几个基本概念一文)。

\subsection*{追赶法:三对角占优矩阵的三角分解}

式(\ref{eq:trimat})中的三对角占优矩阵可以[label]三角分解[/label]为以下形式:

\begin{align*}
&
\begin{pmatrix}
b_1 & c_1 & & & & & \\
a_2 & b_2 & c_2 & & & & \\
& \ddots & \ddots & \ddots & & & \\
\cline{4-4}
& & a_i & \multicolumn{1}{|c|}{\multirow{1}{*}{$b_i$}} & c_i & & \\
\cline{4-4}
& & & \ddots & \ddots & \ddots & \\
& & & & a_{n-1} & b_{n-1} & c_{n-1} \\
& & & & & a_{n} & b_{n}
\end{pmatrix} \\
=&
\begin{pmatrix}
\alpha_1 & & & & & & \\
r_2 & \alpha_2 & & & & & \\
& \ddots & \ddots & & & & \\
\cline{4-4}
& & r_i & \multicolumn{1}{|c|}{\multirow{1}{*}{$\alpha_i$}} & & & \\
\cline{4-4}
& & & \ddots & \ddots & & \\
& & & & r_{n-1} & \alpha_{n-1} & \\
& & & & & r_{n} & \alpha_{n}
\end{pmatrix}
\begin{pmatrix}
1 & \beta_1 & & & & & \\
& 1 & \beta_2 & & & & \\
& & \ddots & \ddots & & & \\
\cline{4-4}
& & & \multicolumn{1}{|c|}{\multirow{1}{*}{$1$}} & \beta_i & & & \\
\cline{4-4}
& & & & \ddots & \ddots & & \\
& & & & & 1 & \beta_{n-1} & \\
& & & & & & 1
\end{pmatrix}
\end{align*}

indent 所以,根据矩阵运算可以得到:

\[
\begin{cases}
b_1=\alpha_1,\quad c_1=\alpha_1\beta_1 \\
a_i=r_i,\quad b_i=r_i\beta_{i-1}+\alpha_i,\quad i=2,3,\cdots,n \\
c_i=\alpha_i\beta_i,\quad i=2,3,\cdots,n-1
\end{cases}
\]

indent 而上式的计算次序为:

\[
\alpha_1\rightarrow\beta_1\rightarrow\alpha_2\rightarrow\beta_2\rightarrow\cdots\rightarrow\alpha_i\rightarrow\beta_i\rightarrow\cdots\rightarrow\alpha_{n-1}\rightarrow\beta_{n-1}\rightarrow\alpha_n
\]

indent 需要注意的是,$\beta_i=c_i/\alpha_i$用到了B除法运算,但因为$|\alpha_1|=|b_1|>|c_1|$,可知$0<|\beta_1|<1$,再利用[label]数学归纳法[/label]可以证明$|\alpha_i|>|c_i|>0$,其中$i=1,2,\cdots,n-1$,则有:$|\beta_i|<1$。所以说,这个算法是B稳定的,并称其为[label]追赶法[/label]。

[alert type="error"]注意:也可以对$A$进行[label]LU分解[/label]:$A=LU$,其中$L$为[label]单位下三角矩阵[/label],$U$为[label]上三角矩阵[/label],但可能会产生较大[label]舍入误差[/label],所以三对角占优矩阵的LU分解法B不可靠。[/alert]

\subsection*{三对角矩阵方程组的并行求解}

系数矩阵为三对角矩阵的线性方程组,利用[label]循环约化方法[/label],可以实现B并行计算。这里,假定系数矩阵的阶数为B奇数,即:$n=2^p-1$,$p$为正整数。类似式(\ref{eq:trimat}),三对角方程组可以表示为:

\begin{align}
\label{eq:trisys}
\begin{split}
a_{i-1}x_{i-2}+b_{i-1}x_{i-1}+c_{i-1}x_{i} &= f_{i-1} \\
a_ix_{i-1}+b_ix_i+c_ix_{i+1} &= f_i \\
a_{i+1}x_{i}+b_{i+1}x_{i+1}+c_{i+1}x_{i+2} &= f_{i+1}
\end{split}
\end{align}

indent 其中,$1 \leq i \leq n$,且$a_1=c_n=0$。而对式(\ref{eq:trisys})消元($x_{i-1}$和$x_{i+1}$)可以得到:

\begin{eqnarray*}
& a_i\dfrac{f_{i-1}-a_{i-1}x_{i-2}-c_{i-1}x_i}{b_{i-1}}+b_ix_i+c_i\dfrac{f_{i+1}-a_{i+1}x_i-c_{i+1}x_{i+2}}{b_{i+1}}=f_i \\
& a'_ix_{i-2}+b'_ix_i+c'_ix_{i+2}=f'_i
\end{eqnarray*}

indent 其中:

\begin{align*}
a'_i &= -a_i\dfrac{a_{i-1}}{b_{i-1}} \\
b'_i &= b_i-a_i\dfrac{c_{i-1}}{b_{i-1}}-c_i\dfrac{a_{i+1}}{b_{i+1}} \\
c'_i &= -c_i\dfrac{c_{i+1}}{b_{i+1}} \\
f'_i &= f_i-a_i\dfrac{f_{i-1}}{b_{i-1}}-f_{i+1}\dfrac{c_i}{b_{i+1}}
\end{align*}

indent 至此,可以得到关于$x_2,x_4,\cdots,x_{n-3},x_{n-1}$的线性方程组:

\begin{equation}
\label{eq:even}
\begin{pmatrix}
b'_2 & c'_2 & & & \\
a'_4 & b'_4 & c'_4 & & \\
& \ddots & \ddots & \ddots & \\
& & a'_{n-3} & b'_{n-3} & c'_{n-3} \\
& & & b'_{n-1} & c'_{n-1}
\end{pmatrix}
\begin{pmatrix}
x_2 \\ x_4 \\ \vdots \\ x_{n-3} \\ x_{n-1}
\end{pmatrix}=
\begin{pmatrix}
f'_2 \\ f'_4 \\ \vdots \\ f'_{n-3} \\ f'_{n-1}
\end{pmatrix}
\end{equation}

indent 而对于$x_1,x_3,\cdots,x_{n-2},x_n$来说,因为有$a_ix_{i-1}+b_ix_i+c_ix_{i+1}= f_i$,所以可以将其表示为:

\begin{equation}
\label{eq:odd}
\begin{pmatrix}
b_1 & & & & \\
& b_3 & & & \\
& & \ddots & & \\
& & & b_{n-2} & \\
& & & & b_n
\end{pmatrix}
\begin{pmatrix}
x_1 \\ x_3 \\ \vdots \\ x_{n-2} \\ x_{n}
\end{pmatrix}=
\begin{pmatrix}
f_1 \\ f_3 \\ \vdots \\ f_{n-2} \\ f_n
\end{pmatrix}-
\begin{pmatrix}
c_1 & & & & \\
a_3 & c_3 & & & \\
& \ddots & \ddots & & \\
& & & a_{n-2} & c_{n-2} \\
& & & & a_n
\end{pmatrix}
\begin{pmatrix}
x_2 \\ x_4 \\ \vdots \\ x_{n-3} \\ x_{n-1}
\end{pmatrix}
\end{equation}

所以,式(\ref{eq:trimat})被分为了两个$(n/2)\times(n/2)$阶方程组(式\ref{eq:even}、\ref{eq:odd}),而式(\ref{eq:even})又可以B继续分解为两个更小的$(n/4)\times(n/4)$阶方程组,在式(\ref{eq:even})的基础之上,式(\ref{eq:odd})可以B并行求解。这种技术也称为[label]奇偶分开技术[/label]。


% H2, H3
%\subsection*{dsfsdt}
%[latex mode=1]
%[+preamble]
%\usepackage[titlenotnumbered,lined,linesnumbered]{algorithm2e}
%\let\oldnl\nl
%\newcommand{\nonl}{\renewcommand{\nl}{\let\nl\oldnl}}
%[/preamble]
%\quicklatex{size=14}
%\TitleOfAlgo{BLAS-3 Gaussian Elimination with Partial Pivoting}
%\begin{algorithm}[H]
%\For{$l\gets 1$ \KwTo $n-1$}{
%$k\gets (l-1)b+1$\;
%factorize $PA^{(l)}=LU$ using BLAS-2\;
%store $L$ and $U$ in $A$\;
%left multiply $A(1:n,k+b:n)$ by $P$\;
%$A(k:k+b-1,k+b:n)\gets T^{-1}A(k:k+b-1,k+b:n)$\;
%\nonl\Indp where $T$ is the unit-lower-trian of $A(k:k+b-1,k:k+b-1)$\;
%\DontPrintSemicolon\Indm$A(k+b:n,k+b:n)\gets A(k+b:n,k+b:n)$\;
%\PrintSemicolon\nonl\Indp$-A(k+b:n,k:k+b-1)A(k:k+b-1,k+b:n)$\;
%}
%\end{algorithm}
%[/latex]

% H4
% \subsubsection*{}

\end{document}
