\documentclass[UTF8,nofonts]{ctexart}

\setCJKmainfont[BoldFont=STHeiti,ItalicFont=STKaiti]{STKaiti}
\setCJKsansfont[BoldFont=STHeiti]{STXihei}
\setCJKmonofont{STFangsong}

\usepackage[letterpaper,left=1in,top=1in]{geometry}
\usepackage{multirow}
\usepackage{amsmath}
\usepackage{amsfonts}
\usepackage{amssymb}
\usepackage[titlenotnumbered,lined,linesnumbered]{algorithm2e}

\begin{document}

% Title
\section*{一维子空间的投影过程}

给定线性方程组$A\boldsymbol{x}=\boldsymbol{b}$,几乎所有的基本[label]迭代法[/label]求解都用到了[label]投影方法[/label]。上文(见迭代法中投影过程的基本框架一文)给出了[label]投影过程[/label]的B基本框架和步骤,其B核心思想是在[label]搜索子空间[/label]$\mathcal{K}$中寻找[label]近似解[/label],并使其对应的[label]残差向量[/label]B正交于另一个子空间$\mathcal{L}$。本文主要讨论最基本的[label]一维子空间[/label]上的投影过程。

\subsection*{一维子空间}

一维子空间$\mathcal{K}$和$\mathcal{L}$可以表示为:

\[\mathcal{K}=\text{span}\{\boldsymbol{v}\},\quad\mathcal{L}=\text{span}\{\boldsymbol{w}\}\]

indent其中,$\boldsymbol{v}$和$\boldsymbol{w}$是两个向量,且向量$\boldsymbol{r}$为对应于当前近似解$\boldsymbol{x}$的残差向量,即:$\boldsymbol{r}=\boldsymbol{b}-A\boldsymbol{x}$。在上文迭代法中投影过程的基本框架中,给出了如下的B基本投影步骤:

\begin{align*}
\begin{split}
\boldsymbol{x}\gets\boldsymbol{x}+\boldsymbol{\delta},&\quad\boldsymbol{\delta}\in\mathcal{K} \\
\boldsymbol{w}^T(\boldsymbol{r}-A\boldsymbol{\delta})=0,&\quad\forall\boldsymbol{w}\in\mathcal{L} \end{split} \end{align*}

indent而在一维情况下,$\boldsymbol{\delta}$就可以表示为$\alpha\boldsymbol{v}$,那么对于当前近似解$\boldsymbol{x}$,结合Petrov-Galerkin条件:$\boldsymbol{r}-A\boldsymbol{\delta}\perp\boldsymbol{w}$,便可以推得:

\[
\boldsymbol{w}^T\boldsymbol{r}=\boldsymbol{w}^TA\boldsymbol{\delta}=\alpha\boldsymbol{w}^TA\boldsymbol{v}\quad\Longrightarrow\quad\alpha=\dfrac{\boldsymbol{w}^T\boldsymbol{r}}{\boldsymbol{w}^TA\boldsymbol{v}}=\dfrac{(\boldsymbol{r},\boldsymbol{w})}{(A\boldsymbol{v},\boldsymbol{w})}
\]

indent于是,问题的关键转换为了向量$\boldsymbol{v}$和$\boldsymbol{w}$的选取上。下文给出了三种常用的选取方法,及其相关算法和收敛性分析。

\subsection*{最速下降法}

[label]最速下降法[/label](steepest descent),也称为[label]梯度下降法[/label],它要求矩阵$A$为[label]对称正定矩阵[/label],并在每一步投影过程中,向量$\boldsymbol{v}$和$\boldsymbol{w}$都取值为$\boldsymbol{r}$,即:$\boldsymbol{v}=\boldsymbol{r}$,$\boldsymbol{w}=\boldsymbol{r}$。其迭代投影过程可以描述为:

\begin{align}
\label{eq:sditer}
\begin{split}
\boldsymbol{r} &\gets \boldsymbol{b}-A\boldsymbol{x} \\
\alpha &\gets \dfrac{(\boldsymbol{r},\boldsymbol{w})}{(A\boldsymbol{v},\boldsymbol{w})}=\dfrac{(\boldsymbol{r},\boldsymbol{r})}{(A\boldsymbol{r},\boldsymbol{r})} \\
\boldsymbol{x} &\gets \boldsymbol{x}+\alpha\boldsymbol{v}=\boldsymbol{x}+\alpha\boldsymbol{r}
\end{split}
\end{align}

indent 注意到,该计算过程中包含有两个[label]矩阵向量乘法[/label]($A\boldsymbol{x}$和$A\boldsymbol{r}$),稍作修改的话,可以降为只需一次矩阵向量乘($A\boldsymbol{r}$)。事实上,第$k+1$步($k=0,1,\cdots$)的$\boldsymbol{r}^{(k+1)}$只需要使用上一步的$\boldsymbol{r}^{(k)}$即可计算得到。即:

\begin{equation}
\label{eq:sdred}
\boldsymbol{r}^{(k+1)}=\boldsymbol{b}-A\boldsymbol{x}^{(k+1)}=\boldsymbol{b}-A\Big(\boldsymbol{x}^{(k)}+\alpha\boldsymbol{r}^{(k)}\Big)=\boldsymbol{r}^{(k)}-\alpha A\boldsymbol{r}^{(k)}
\end{equation}

\subsubsection*{最速下降法算法实现}

有了式(\ref{eq:sditer})和式(\ref{eq:sdred}),便可以得到如下的最速下降法算法实现:

%[latex mode=1]
%[+preamble]
%\usepackage[titlenotnumbered,lined,linesnumbered]{algorithm2e}
%[/preamble]
\TitleOfAlgo{Steepest Descent Algorithm}
\begin{algorithm}[H]
compute $\boldsymbol{r}=b-A\boldsymbol{x}$ and $\boldsymbol{p}=A\boldsymbol{r}$\;
\Repeat{convergence}{
	$\alpha\gets(\boldsymbol{r},\boldsymbol{r})/(\boldsymbol{p},\boldsymbol{r})$\;
	$\boldsymbol{x}\gets\boldsymbol{x}+\alpha\boldsymbol{r}$\;
	$\boldsymbol{r}\gets\boldsymbol{r}-\alpha\boldsymbol{p}$\;
	$\boldsymbol{p}\gets A\boldsymbol{r}$\;
}
\end{algorithm}
%[/latex]

indent事实上,对于所有具有形式$\boldsymbol{x}+\alpha\boldsymbol{d}$的向量,上述算法的每一个迭代步都极小化了函数:

\[f(\boldsymbol{x})=\|\boldsymbol{x}-\tilde{\boldsymbol{x}}\|_A^2=\Big(A(\boldsymbol{x}-\tilde{\boldsymbol{x}}),(\boldsymbol{x}-\tilde{\boldsymbol{x}})\Big)\]

indent其中,$\tilde{\boldsymbol{x}}$为精确解,$\boldsymbol{d}$是[label]负梯度方向[/label]$-\nabla f$,这个负梯度方向决定了每次迭代B新的搜索方向,且沿着负梯度方向能B最快达到极小点。

\subsubsection*{最速下降法的收敛性}

最速下降法要求$A$是[label]对称正定矩阵[/label],令$\lambda_{max}$和$\lambda_{min}$分别为$A$的最大和最小[label]特征值[/label],[label]Kantorovich不等式[/label]指出(证略):

\begin{equation}
\label{lm:kant}
\dfrac{(A\boldsymbol{x},\boldsymbol{x})(A^{-1}\boldsymbol{x},\boldsymbol{x})}{(\boldsymbol{x},\boldsymbol{x})^2} \leq 
\dfrac{(\lambda_{max}+\lambda_{min})^2}{4\lambda_{max}\lambda_{min}},\quad\forall\boldsymbol{x} \neq 0
\end{equation}

indent在引理(\ref{lm:kant})的基础之上,可以推得最速下降法中,第$k$步迭代计算得到的[label]误差[/label]向量$\boldsymbol{d}^{(k)}=\tilde{\boldsymbol{x}}-\boldsymbol{x}^{(k)}$,其[label]$A$-范数[/label]满足:

\begin{equation}
\label{lm:err}
\|\boldsymbol{\epsilon}^{(k+1)}\|_A\leq\dfrac{\lambda_{max}-\lambda_{min}}{\lambda_{max}+\lambda_{min}}\|\boldsymbol{\epsilon}^{(k)}\|_A
\end{equation}

indent且最速下降法对任意初始$\boldsymbol{x}^{0}$都收敛(证略)。

\subsection*{极小残差迭代}

极小残差(minimal residul, MR)迭代法中,要求矩阵$A$为B正定的,但不需要是对称的($A+A^T$是对称正定的)。在每一步迭代过程中,取$\boldsymbol{v}=\boldsymbol{r}$,$\boldsymbol{w}=A\boldsymbol{r}$,则其迭代投影过程可以描述为:

\begin{align}
\label{eq:mriter}
\begin{split}
\boldsymbol{r} &\gets \boldsymbol{b}-A\boldsymbol{x} \\
\alpha &\gets \dfrac{(\boldsymbol{r},\boldsymbol{w})}{(A\boldsymbol{v},\boldsymbol{w})}=\dfrac{(\boldsymbol{r},A\boldsymbol{r})}{(A\boldsymbol{r},A\boldsymbol{r})} \\
\boldsymbol{x} &\gets \boldsymbol{x}+\alpha\boldsymbol{v}=\boldsymbol{x}+\alpha\boldsymbol{r}
\end{split}
\end{align}

indent同最速下降法类似,上述极小残差迭代过程也可以修改为只需要执行一次矩阵向量乘($A\boldsymbol{r}$)即可。这样便能得到如下的极小残差算法实现。

\subsubsection*{极小残差迭代算法实现}

%[latex mode=1]
%[+preamble]
%\usepackage[titlenotnumbered,lined,linesnumbered]{algorithm2e}
%[/preamble]
\TitleOfAlgo{Minimal Residual Iteration}
\begin{algorithm}[H]
compute $\boldsymbol{r}=b-A\boldsymbol{x}$ and $\boldsymbol{p}=A\boldsymbol{r}$\;
\Repeat{convergence}{
	$\alpha\gets(\boldsymbol{r},\boldsymbol{p})/(\boldsymbol{p},\boldsymbol{p})$\;
	$\boldsymbol{x}\gets\boldsymbol{x}+\alpha\boldsymbol{r}$\;
	$\boldsymbol{r}\gets\boldsymbol{r}-\alpha\boldsymbol{p}$\;
	$\boldsymbol{p}\gets A\boldsymbol{r}$\;
}
\end{algorithm}
%[/latex]

indent事实上,上述算法的每一个迭代步都沿着$\boldsymbol{r}$方向,极小化了函数$f(\boldsymbol{x})=\|\boldsymbol{b}-A\boldsymbol{x}\|_2^2$。

\subsubsection*{极小残差迭代的收敛性}

设$A$是[label]正定矩阵[/label],令:

\[\mu=\lambda_{min}(A+A^T)/2,\quad\sigma=\|A\|_2\]

indent则上述极小残差迭代算法中计算得到的残差向量$\boldsymbol{r}$满足如下关系式:

\[
\|\boldsymbol{r}^{(k+1)}\|_2\leq\left(1-\dfrac{\mu^2}{\sigma^2}\right)^{1/2}\|\boldsymbol{r}^{(k)}\|_2
\]

indent且极小残差迭代对任意初始向量$\boldsymbol{x}^{(0)}$都收敛(证略)。

\subsection*{残差范数最速下降法}

在残差范数最速下降(residual norm steepest descent)迭代法中,进一步放宽了对矩阵$A$的限制,只要求其是[label]非奇异[/label]的。并且,在每一步迭代过程中,取$\boldsymbol{v}=A^T\boldsymbol{r}$,$\boldsymbol{w}=A\boldsymbol{v}$,则其迭代投影过程可以描述为:

\begin{align}
\label{eq:rnsditer}
\begin{split}
\boldsymbol{r} &\gets \boldsymbol{b}-A\boldsymbol{x},\quad\boldsymbol{v}=A^T\boldsymbol{r}\\
\alpha &\gets \dfrac{(\boldsymbol{r},\boldsymbol{w})}{(A\boldsymbol{v},\boldsymbol{w})}=\dfrac{(\boldsymbol{r},A\boldsymbol{v})}{(A\boldsymbol{v},A\boldsymbol{v})}=\dfrac{\boldsymbol{v}^T(A^T\boldsymbol{r})}{\|A\boldsymbol{v}\|_2^2}=\dfrac{\|\boldsymbol{v}\|^2_2}{\|A\boldsymbol{v}\|_2^2}\\
\boldsymbol{x} &\gets \boldsymbol{x}+\alpha\boldsymbol{v}
\end{split}
\end{align}

indent注意到,该计算过程中包含了三个[label]矩阵向量乘法[/label]($A\boldsymbol{x}$、$A^T\boldsymbol{r}$、$A\boldsymbol{v}$)。如果稍作修改的话,可以降为只需两次矩阵向量乘($A^T\boldsymbol{r}$和$A\boldsymbol{v}$),原理则同前两个算法一样。事实上,第$k+1$步($k=0,1,\cdots$)的$\boldsymbol{r}^{(k+1)}$只需要使用上一步的$\boldsymbol{r}^{(k)}$和$A\boldsymbol{v}^{(k)}$即可计算得到。即:

\begin{equation}
\label{eq:rnsdred}
\boldsymbol{r}^{(k+1)}=\boldsymbol{b}-A\boldsymbol{x}^{(k+1)}=\boldsymbol{b}-A\Big(\boldsymbol{x}^{(k)}+\alpha\boldsymbol{v}^{(k)}\Big)=\boldsymbol{r}^{(k)}-\alpha A\boldsymbol{v}^{(k)}
\end{equation}

\subsubsection*{残差范数最速下降算法实现}

有了式(\ref{eq:rnsditer})和式(\ref{eq:rnsdred}),便可以得到如下残差范数最速下降法的算法实现:

%[latex mode=1]
%[+preamble]
%\usepackage[titlenotnumbered,lined,linesnumbered]{algorithm2e}
%[/preamble]
\TitleOfAlgo{Residual Norm Steepest Descent}
\begin{algorithm}[H]
compute $\boldsymbol{r}=b-A\boldsymbol{x}$\;
\Repeat{convergence}{
	$\boldsymbol{v} \gets A^T\boldsymbol{r}$\;
	$\boldsymbol{p} \gets A\boldsymbol{v}$\;
	$\alpha \gets \|\boldsymbol{v}\|^2_2/\|\boldsymbol{p}\|^2_2$\;
	$\boldsymbol{x}\gets\boldsymbol{x}+\alpha\boldsymbol{v}$\;
	$\boldsymbol{r}\gets\boldsymbol{r}-\alpha\boldsymbol{p}$\;
}
\end{algorithm}
%[/latex]

indent事实上,残差范数最速下降算法的每一个迭代步,都沿着$-\nabla f$方向,极小化了函数$f(\boldsymbol{x})=\|\boldsymbol{b}-A\boldsymbol{x}\|^2_2$。

\subsubsection*{残差范数最速下降的收敛性}


事实上,残差范数最速下降法等价于对[label]正规方程[/label]$A^TA\boldsymbol{x}=A^T\boldsymbol{b}$使用最速下降法。如果矩阵$A$非奇异,则$A^TA$对称正定,由引例(\ref{lm:kant}、\ref{lm:err})可推知,残差范数最速下降法对任意初始向量$\boldsymbol{x}^{(0)}$都收敛。

\subsection*{总结}



% H2, H3
\subsection*{dsfsdt}
%[latex mode=1]
%[+preamble]
%\usepackage[titlenotnumbered,lined,linesnumbered]{algorithm2e}
\let\oldnl\nl
\newcommand{\nonl}{\renewcommand{\nl}{\let\nl\oldnl}}
%[/preamble]
%\quicklatex{size=14}
\TitleOfAlgo{BLAS-3 Gaussian Elimination with Partial Pivoting}
\begin{algorithm}[H]
\For{$l\gets 1$ \KwTo $n-1$}{
$k\gets (l-1)b+1$\;
factorize $PA^{(l)}=LU$ using BLAS-2\;
store $L$ and $U$ in $A$\;
left multiply $A(1:n,k+b:n)$ by $P$\;
$A(k:k+b-1,k+b:n)\gets T^{-1}A(k:k+b-1,k+b:n)$\;
\nonl\Indp where $T$ is the unit-lower-trian of $A(k:k+b-1,k:k+b-1)$\;
\DontPrintSemicolon\Indm$A(k+b:n,k+b:n)\gets A(k+b:n,k+b:n)$\;
\PrintSemicolon\nonl\Indp$-A(k+b:n,k:k+b-1)A(k:k+b-1,k+b:n)$\;
}
\end{algorithm}
%[/latex]

% H4
% \subsubsection*{}

\end{document}
