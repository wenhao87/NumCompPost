\documentclass[12pt, UTF8, nofonts]{ctexart}

\setCJKmainfont[BoldFont=STHeiti,ItalicFont=STKaiti]{STKaiti}
\setCJKsansfont[BoldFont=STHeiti]{STXihei}
\setCJKmonofont{STFangsong}

\usepackage[letterpaper,left=.5in,right=.5in,top=1in,bottom=.5in]{geometry}
\usepackage{afterpage, color, multirow}
\usepackage{amsmath}
\usepackage{amsfonts}
\usepackage{amssymb}
\usepackage[titlenotnumbered,lined,linesnumbered]{algorithm2e}

\definecolor{bgcolor}{RGB}{13, 12, 12}
\definecolor{ttcolor}{RGB}{0, 255, 0}
\pagecolor{bgcolor}\afterpage{\nopagecolor}
\makeatletter
\newcommand{\globalcolor}[1]{\color{#1}\global\let\default@color\current@color}
\makeatother
\AtBeginDocument{\globalcolor{ttcolor}}

\begin{document}

%[aside_content]

\section*{详解基于矩阵分裂的迭代格式的收敛性}

在基本迭代法的收敛性分析一文中,简单介绍了迭代格式的收敛性。本文作为补充,将详细分析基于矩阵分裂的迭代算法和格式的收敛性,及其相关定义和性质。

\subsection*{矩阵分裂}

[label]矩阵分裂[/label](Matrix Splitting)指:设矩阵$\boldsymbol{A}\in\mathbb{R}^{n \times n}$非奇异,称$\boldsymbol{A}=\boldsymbol{M}-\boldsymbol{N}$为$\boldsymbol{A}$的一个矩阵分裂,其中$\boldsymbol{M}$B非奇异。此时,线性方程组$\boldsymbol{Ax}=\boldsymbol{b}$等价于$\boldsymbol{Mx}=\boldsymbol{Nx}+\boldsymbol{b}$。于是,便能构造如下[label]迭代格式[/label]:

\begin{equation}
    \label{eq:iterformat}
    \boldsymbol{x}^{(k+1)} = \boldsymbol{M}^{-1}\boldsymbol{N}\boldsymbol{x}^{(k)}+\boldsymbol{M}^{-1}\boldsymbol{b} \triangleq \boldsymbol{G}\boldsymbol{x}^{(k)}+\boldsymbol{g} \;,\quad k=0,1,\ldots,
\end{equation}

indent其中,$\boldsymbol{G}=\boldsymbol{M}^{-1}\boldsymbol{N}$称为该迭代格式的[lable]迭代矩阵[/label],因此选取不同的$\boldsymbol{M}$,便能构造不同的迭代算法。

另一方面,观察式(\ref{eq:iterformat})可知:$\boldsymbol{x}^{(0)}$表示的是B迭代初始值,通过迭代格式便能生成一个[label]迭代序列[/label]$\{\boldsymbol{x}^{(k)}\}_{k=0}^{\infty}$,并记该序列的B极限为$\boldsymbol{x}^{\ast}$,即:

\[
    \lim_{k\to\infty}\boldsymbol{x}^{(k)} = \boldsymbol{x}^{\ast} \triangleq \boldsymbol{A}^{-1}\boldsymbol{b}
\]

\subsection*{收敛性定义}

\subsubsection*{序列的收敛定义}

对于[label]向量序列[/label]$\{\boldsymbol{x}^{(k)}\}_{k=0}^{\infty}\subset\mathbb{R}^n$,如果存在$\boldsymbol{x}=(x_1,x_2,\ldots,x_n)^T\in\mathbb{R}^{n}$,满足:

\[
    \lim_{k\to\infty}x_i^{(k)}=x_i \;,\quad i=1,2,\ldots,n \quad \Rightarrow \quad \lim_{k\to\infty}\boldsymbol{x}^{(k)}=\boldsymbol{x}
\]

noindent则称$\boldsymbol{x}^{(k)}$收敛到$\boldsymbol{x}$,即:$\boldsymbol{x}$是$\boldsymbol{x}^{(k)}$的极限,其中$x_i^{(k)}$表示$x^{(k)}$的第$i$个分量。

类似地,对于[label]矩阵序列[label]$\left\{\boldsymbol{A}^{(k)}=\left(a_{ij}^{(k)}\right)\right\}_{k=0}^{\infty}\subset\mathbb{R}^{n \times n}$,如果存在矩阵$A=\left(a_{ij}\right)\in\mathbb{R}^{n \times n}$,满足:

\[
    \lim_{k\to\infty}a_{ij}^{(k)}=a_{ij} \;,\quad i,j=1,2,\ldots,n \quad \Rightarrow \quad \lim_{k\to\infty}\boldsymbol{A}^{(k)} = \boldsymbol{A}
\]

noindent则称$\boldsymbol{A}^{(k)}$收敛到$\boldsymbol{A}$,即:$\boldsymbol{A}$是$\boldsymbol{A}^{(k)}$的极限。

[alert type="info"]如果对于任意初始向量$\boldsymbol{x}^{(0)}$,都有$\lim_{k\to\infty}\boldsymbol{x}^{(k)}=\boldsymbol{x}$,则称式(\ref{eq:iterformat})的迭代格式是[label]收敛的[/label],反之则是[label]发散的[/label]。[/alert]

\subsubsection*{矩阵的谱半径和范数}

事实上,基于矩阵分裂的迭代算法,其收敛性B取决于迭代矩阵的[label]谱半径[/label]。对于矩阵$\boldsymbol{A}\in\mathbb{R}^{n \times n}$,其谱半径定义为:

\[
    \rho(\boldsymbol{A}) \triangleq \max_{\lambda\in\sigma(\boldsymbol{A})}|\lambda|
\]

noindent其中,$\sigma(\boldsymbol{A})$表示$\boldsymbol{A}$的所有[label]特征值[/label]所组成的集合。而矩阵的谱半径和[label]矩阵范数[/label]具有如下重要关系(设$\boldsymbol{G}\in\mathbb{R}^{n \times n}$):对于任意算子范数,都有$\rho(\boldsymbol{G})\leq\|\boldsymbol{G}\|$。证明如下:

\[
    \boldsymbol{Gx}=\lambda\boldsymbol{x} \quad \Rightarrow \quad
    |\lambda|\cdot\|\boldsymbol{x}\| = \|\lambda\boldsymbol{x}\| = \|\boldsymbol{Gx}\| \leq \|\boldsymbol{G}\| \cdot \|\boldsymbol{x}\|
    \quad \Rightarrow \quad |\lambda| \leq \|\boldsymbol{G}\|
\]

此外,矩阵的谱半径和矩阵范数之间还存在以下几个重要性质(设$\boldsymbol{G}\in\mathbb{R}^{n \times n}$):

- $\lim_{k\to\infty}\boldsymbol{G}^{k}=0$当且仅当$\rho(G)<1$;

- 对于任意算子范数$\|\cdot\|$,有$\lim_{k\to\infty}\|\boldsymbol{G}^k\|^{\frac{1}{k}}=\rho(\boldsymbol{G})$;

\subsection*{迭代算法的收敛性}

对于基于矩阵分离的迭代算法,如下分别给出迭代算法收敛的B充分条件和B充要条件。

[alert type="info"]B充分条件:若存在算子范数$\|\cdot\|$,使得$\|\boldsymbol{G}\|\leq1$,则式(\ref{eq:iterformat})的迭代格式收敛。[/alert]

证明如下:

\[
    \left.\begin{aligned}
        \boldsymbol{x}^{(k+1)} &= \boldsymbol{Gx}^{(k)} + \boldsymbol{g} \\
        \boldsymbol{x}^{\ast} &= \boldsymbol{Gx}^{\ast} + \boldsymbol{g} \\
    \end{aligned} \;\right\} \;\Rightarrow\;
    \begin{aligned}
        \boldsymbol{x}^{(k+1)}-\boldsymbol{x}^{\ast} &=
        \boldsymbol{G}(\boldsymbol{x}^{(k)}-\boldsymbol{x}^{\ast}) \\
        &= \boldsymbol{G}^2(\boldsymbol{x}^{k-1}-\boldsymbol{x}^{\ast}) \\
        &= \cdots \\
        &= \boldsymbol{G}^{(k+1)}(\boldsymbol{x}^{(0)}-\boldsymbol{x}^{\ast})
    \end{aligned} \;\overset{\|\boldsymbol{G}\|<1}{\Longrightarrow}\;
    \|\boldsymbol{x}^{(k+1)}-\boldsymbol{x}^{\ast}\|\to0
\]

[alert type="info"]B充要条件(亦即[label]收敛性定理[/label]):对于任意迭代初始向量$\boldsymbol{x}^{(0)}$,式(\ref{eq:iterformat})的迭代格式收敛始终收敛的充要条件是$\rho(\boldsymbol{G})<1$。[/alert]

此外,对于迭代矩阵$\boldsymbol{G}$,式(\ref{eq:iterformat})迭代算法的b收敛速度定义为:

- [label]平均收敛速度[/label]:$R_k(\boldsymbol{G})\triangleq-\ln\|\boldsymbol{G}^k\|^{\frac{1}{k}}$(它与迭代步数和所用的范数有关);

- [label]渐进收敛速度[/label]:$R(\boldsymbol{G})\triangleq\lim_{k\to\infty}R_k(\boldsymbol{G})=-\ln\rho(\boldsymbol{G})$(仅依赖于迭代矩阵的谱半径)

最后给出如下几个重要关系式及其证明。对于式(\ref{eq:iterformat})的迭代算法,如果存在某个算子范数$\|\cdot\|$使得$\|\boldsymbol{G}\|=q<1$,则有:

- $\|\boldsymbol{x}^{(k)}-\boldsymbol{x}^{\ast}\|\leq q^k\;\|\boldsymbol{x}^{(0)}-\boldsymbol{x}^{\ast}\|$

\[
    \begin{aligned}
        \left.\begin{aligned}
            \boldsymbol{x}^{(k+1)} &= \boldsymbol{Gx}^{(k)}+\boldsymbol{g} \\
            \boldsymbol{x}^{\ast} &= \boldsymbol{Gx}^{\ast}+\boldsymbol{g} \\
        \end{aligned} \;\right\}\; &\Rightarrow \;
        \boldsymbol{x}^{(k+1)}-\boldsymbol{x}^{\ast} = \boldsymbol{G}(\boldsymbol{x}^{(k)}-\boldsymbol{x}^{\ast}) \\
        &\Rightarrow \;
        \|\boldsymbol{x}^{(k+1)}-\boldsymbol{x}^{\ast}\| \leq q\; \|(\boldsymbol{x}^{(k)}-\boldsymbol{x}^{\ast})\| \\
        &\Rightarrow \;
        \|\boldsymbol{x}^{(k)}-\boldsymbol{x}^{\ast}\| \leq q^k\; \|(\boldsymbol{x}^{(0)}-\boldsymbol{x}^{\ast})\| \\
    \end{aligned}
\]

- $\|\boldsymbol{x}^{(k)}-\boldsymbol{x}^{\ast}\|\leq\dfrac{q}{1-q}\;\|\boldsymbol{x}^{k}-\boldsymbol{x}^{(k-1)}\|$

\begin{eqnarray*}
    & \left.\begin{aligned}
        \boldsymbol{x}^{(k+1)} &= \boldsymbol{Gx}^{(k)}+\boldsymbol{g} \\
        \boldsymbol{x}^{(k)} &= \boldsymbol{Gx}^{(k-1)}+\boldsymbol{g} \\
    \end{aligned} \;\right\}\; \Rightarrow \;
    \|\boldsymbol{x}^{(k+1)}-\boldsymbol{x}^{(k)}\| \leq q \; \|\boldsymbol{x}^{(k)}-\boldsymbol{x}^{(k-1)}\| \\
    & \left.\begin{aligned}
        \|\boldsymbol{x}^{(k+1)}-\boldsymbol{x}^{(k)}\| &=
        \|(\boldsymbol{x}^{\ast}-\boldsymbol{x}^{(k)})-(\boldsymbol{x}^{\ast}-\boldsymbol{x}^{(k+1)})\| \\
        &\geq \|\boldsymbol{x}^{\ast}-\boldsymbol{x}^{(k)}\| -
        \underbrace{\|\boldsymbol{x}^{\ast}-\boldsymbol{x}^{(k+1)}\|}_{\leq q\;\|\boldsymbol{x}^{(k)}-\boldsymbol{x}^{\ast}\|} \\
        &\geq (1-q)\;\|\boldsymbol{x}^{(k)}-\boldsymbol{x}^{\ast}\|
    \end{aligned} \;\right\}\; \Rightarrow \;
    \begin{aligned}
        \|\boldsymbol{x}^{(k)}-\boldsymbol{x}^{\ast}\| &\leq
        \dfrac{1}{1-q} \; \|\boldsymbol{x}^{(k+1)}-\boldsymbol{x}^{(k)}\| \\
        &\leq \dfrac{q}{1-q} \; \|\boldsymbol{x}^{(k)}-\boldsymbol{x}^{(k-1)}\|
    \end{aligned}
\end{eqnarray*}

- $\|\boldsymbol{x}^{(k)}-\boldsymbol{x}^{\ast}\|\leq\dfrac{q^k}{1-q}\;\|\boldsymbol{x}^{(1)}-\boldsymbol{x}^{(0)}\|$

\[
    \left.\begin{aligned}
        \|\boldsymbol{x}^{(k+1)}-\boldsymbol{x}^{(k)}\| &\leq q \;
        \|\boldsymbol{x}^{(k)}-\boldsymbol{x}^{(k-1)}\| \leq q^k \;
        \|\boldsymbol{x}^{(1)}-\boldsymbol{x}^{(0)}\| \\
        \|\boldsymbol{x}^{(k)}-\boldsymbol{x}^{\ast}\| &\leq
        \dfrac{1}{1-q} \; \|\boldsymbol{x}^{(k+1)}-\boldsymbol{x}^{(k)}\|
    \end{aligned} \;\right\} \;\Rightarrow\;
    \|\boldsymbol{x}^{(k)}-\boldsymbol{x}^{\ast}\| \leq \dfrac{q^k}{1-q} \;
    \|\boldsymbol{x}^{(1)}-\boldsymbol{x}^{(0)}\|
\]

\end{document}
