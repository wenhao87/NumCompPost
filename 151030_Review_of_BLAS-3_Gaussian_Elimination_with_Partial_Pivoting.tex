\documentclass[UTF8,nofonts]{ctexart}

\setCJKmainfont[BoldFont=STHeiti,ItalicFont=STKaiti]{STKaiti}
\setCJKsansfont[BoldFont=STHeiti]{STXihei}
\setCJKmonofont{STFangsong}

\usepackage[letterpaper,left=1in,top=1in]{geometry}
\usepackage{multirow}
\usepackage{amsmath}
\usepackage{amsfonts}
\usepackage{amssymb}
\usepackage{algorithm2e}

\begin{document}

\section*{BLAS-3的列主元Gauss消去法详解}

\subsection*{列主元消去法的并行实现}

\subsubsection*{负载不均衡实现}

设有$p$个进程,将$n \times n$阶矩阵$A$按行分为$p$分块$m \times n$阶子矩阵,其中$m=n/p$,进程$i(i=0,\cdots,p-1)$存放数据块$A_i$并负责相应数据块的更新,则有:

\[
A=\begin{pmatrix}A_0\\A_1\\\cdots\\A_{p-1}\end{pmatrix},\quad
A_i\in\mathbb{R}^{m \times n},i=0,1,\cdots,p-1
\]

第$k=1$步消元时,先要选取第$1$列中绝对值最大的元素,记为$a_{l0}$,这个过程需要所有$p$个进程B共同参与,然后交换进程$0$的第$1$行和$a_{l0}$所在进程的相应行;然后进程$0$把新的$a_{00},a_{01},\cdots,a_{0,n-1}$发送给其他进程;其他进程在接受到这些元素后,计算各自的$l_{10},l_{20},\cdots,l_{n-1,0}$,并更新$A$的右下角$(n-1)\times(n-1)$阶子矩阵。这里的计算和更新可以B并行进行。

然而,该算法的问题在于当进行了$k=m$步消元以后,进程$0$负责的数据域,即子矩阵$A_0$,不会再被修改,后续的消元过程将不需要进程$0$的参与,因此这种算法会导致[label]负载不均衡[/label]现象,即:进程$0$的工作量比进程$p-1$B远远要少。

\subsubsection*{负载均衡的一维卷帘存储}

[label]一维卷帘存储方式[/label]的核心思想在于:依旧让$p$个进程各自负责$m$行,但所负责的两行之间B相距$p$行,也就是说,进程$0$负责第$0,p,2p,\cdots,(m-1)p$行,进程$i$负责第$i,p+i,2p+i,\cdots,(m-1)p+i$行,以此类推。以一个$n \times b$阶矩阵$A$为例,使用$p$个进程的话,每个进程负责的数据域示意图如下:

\[
\begin{matrix}0\\\vdots\\p-1\\\vdots\\0\\\vdots\\p-1\end{matrix}\quad
\begin{pmatrix}
	\ast&\cdots&\cdots&\cdots&\cdots&\ast \\
	\vdots&&&&&\vdots \\
	\ast&\cdots&\cdots&\cdots&\cdots&\ast \\
	\vdots&&&&&\vdots \\
	\ast&\cdots&\cdots&\cdots&\cdots&\ast \\
	\vdots&&&&&\vdots \\
	\ast&\cdots&\cdots&\cdots&\cdots&\ast \\
\end{pmatrix}
\]

考虑第$k$次消元($0 \leq k \leq n-1$):消元由进程$k \mod p$发起,在选取主元并进行了行交换后,进程$k \mod p$将新的$a_{kk},a_{k+1,k},\cdots,a_{n,k}$发送给其他进程,然后并行更新相应数据。

[alert type="info"]通过一维卷帘模式存储的矩阵$A$,最后消元得到的$L$和$U$,也是按行一维卷帘存储的。[/alert]

$A\boldsymbol{x}=\boldsymbol{b}$的并行求解

假设已经使用一维卷帘存储方式实现了$PA=LU$,则有$PA\boldsymbol{x}=LU\boldsymbol{x}=P\boldsymbol{b}$。所以,首先需要得到$P\boldsymbol{b}$,同时也将其按一维卷帘存储方式存放在每个进程中,即进程$i$需要$(P\boldsymbol{b})_{i+sp}$,其中$i=0,\cdots,p-1$,$s=0,\cdots,m-1$。

[alert type="error"]注意:此时每个进程都需要保存$P$,且为了计算$(P\boldsymbol{b})_{i+sp}$,也需要整个向量$\boldsymbol{b}$。[/alert]

但是,如果事先知道求解方程$A\boldsymbol{x}=\boldsymbol{b}$的话,可以直接对[lable]增广矩阵[/label]$(A,\boldsymbol{b})$使用基于一维卷帘存储的并行列主元消去法。此时返回的$P\boldsymbol{b}$已经按一维卷帘存储方式存放于每个进程之中了,并且$\boldsymbol{b}'$直接使用原来的$\boldsymbol{b}$的存储空间即可。

对于线性方程组$LU\boldsymbol{x}=\boldsymbol{b}'$,首先可以并行计算$L\boldsymbol{y}=\boldsymbol{b}'$,这里$L$和$\boldsymbol{b}$都按一维卷帘方式存放在各个进程中。考虑以下实例($n=6,p=3$):

\[
\begin{matrix}0\\1\\2\\0\\1\\2\end{matrix}\quad
\begin{pmatrix}
	1&&&&&\\
	l_{21}&1&&&&\\
	l_{31}&l_{32}&1&&&\\
	l_{41}&l_{42}&l_{43}&1&&\\
	l_{51}&l_{52}&l_{53}&l_{54}&1&\\
	l_{61}&l_{62}&l_{63}&l_{64}&l_{65}&l_{66}\\
\end{pmatrix}
\begin{pmatrix}y_1\\y_2\\y_3\\y_4\\y_5\\y_6\end{pmatrix}=
\begin{pmatrix}b_1\\b_2\\b_3\\b_4\\b_5\\b_6\end{pmatrix}
\]

进程$0$发送$y_1=b_1$给其他进程,所有进程同时执行$b_i\gets b_i-l_{i1}y_1,i=2,3,\cdots,n$;然后进程$1$把$y_2=b_2$发送给其他进程,其他进程同时执行$b_i\gets b_i-l_{i2}y_2,i=3,4,\cdots,n$,以此类推,直至求出$\boldsymbol{y}$。

[alert type="error"]注意:上述过程通信和计算相分离,每次都需要一个数的通信和两个浮点操作,而通信的代价远比浮点计算来得大。[/alert]

更高效的做法是使用[label]流水线[/label]的方式:即一旦某个进程完成了计算,便将结果发送给其他进程,而B不用等待它们完成先前的计算。例如,进程$0$将$y_1$发送给了其他进程,相比其他进程,进程$1$会较早收到数据,它立刻完成$y_2$的计算,并在其他进程还在计算$b_i\gets b_i-l_{i1}y_1,i=3,4,\cdots,n$的同时,将$y_2$发给它们,从而实现了计算和通信的B重叠。最后,使用类似的方法,即可完成$U\boldsymbol{x}=\boldsymbol{y}$的求解。

\end{document}