\documentclass[UTF8,nofonts]{ctexart}

\setCJKmainfont[BoldFont=STHeiti,ItalicFont=STKaiti]{STKaiti}
\setCJKsansfont[BoldFont=STHeiti]{STXihei}
\setCJKmonofont{STFangsong}

\usepackage[letterpaper,left=1in,top=1in]{geometry}
\usepackage{multirow}
\usepackage{amsmath}
\usepackage{amsfonts}
\usepackage{amssymb}
\usepackage[titlenotnumbered,lined,linesnumbered]{algorithm2e}

\begin{document}

% Title

\section*{FOM}

%[aside_content]

上文介绍了[label]线性方程组[/label]$A\boldsymbol{x}=\boldsymbol{b}$的[label]Arnoldi方法[/label](见线性方程组的Arnoldi方法之:FOM一文),其核心思想是以初始向量$\boldsymbol{x}_0$所对应的初始[label]残差[/label]$\boldsymbol{r}_0$为[label]Krylov子空间[/label]$\mathcal{K}_m$的[label]种子向量[/label],然后以[label]Galerkin条件[/label]($\boldsymbol{b}-A\boldsymbol{x}_m\perp\mathcal{K}_m$)为限制,利用[label]改进的Gram-Schmidt正交化方法[/label]对该Krylov子空间进行正交化,得到该子空间的一组[label]基底向量[/label]$V_m$,由此便可以计算得到线性方程组的近似解$\boldsymbol{x}_m$。这种方法也称为[label]完全正交化方法[/lable](Full Orthogonalization Method, [label]FOM[/label])。

上文文末也提到,完全正交化方法(FOM)在$m$的选取上受计算量和存储量的限制。所以,一般有两种改进的方法。一种是当一次FOM完成后,如果$\boldsymbol{x}_m$的残差过大,则重新开始新一轮的FOM过程,这种方法也称为[label]重新开始的FOM方法[/label](Restarted FOM);另一种方法则是“缩减”正交化的工作量,也就是说在Krylov子空间$\mathcal{K}_m$的正交化过程中,向量$\boldsymbol{v}_i$只需与一部分(通常为前$k$个)向量正交,这种方法便是本文所要介绍的[label]不完全正交化方法[/label]([label]IOM[/label]),它有一种变形称为[label]直接不完全正交分解方法[/label]([label]DIOM[/label])。

\subsection*{不完全正交化方法(IOM)}

\subsubsection*{不完全正交化方法(IOM)算法实现}

Arnoldi方法的核心内容是Krylov子空间$\mathcal{K}_m$的正交化,通常使用的是基于改进的Gram-Schmidt正交化方法。从上面的描述中可以看到,不完全正交化方法(IOM)只是在完全正交化方法(FOM)的基础上缩减了正交化的工作量(只与前$k$个向量正交),因此不完全正交化方法(IOM)的核心正交化方法其实就是不完全正交化的Arnoldi方法(见改进的和不完全正交化的Arnoldi算法一文)。因而很容易得到不完全正交化方法(IOM)的算法实现。

对于给定的$A\in\mathbb{R}^{n \times n}$,$\boldsymbol{b},\boldsymbol{x}_0\in\mathbb{R}^n$,以及正整数$k$和$m$,不完全正交化方法(IOM)的算法实现如下:

%[latex mode=1]
%[+preamble]
%\usepackage[titlenotnumbered,lined,linesnumbered]{algorithm2e}
%[/preamble]
\TitleOfAlgo{Incomplete Orthogonalization Method}
\begin{algorithm}[H]
compute $\boldsymbol{r}_0\gets\boldsymbol{b}-A\boldsymbol{x}_0$, $\beta\gets\|\boldsymbol{r}_0\|_2$, $\boldsymbol{v}_1\gets\boldsymbol{r}_0/\beta$\;
define $H_m=\{h_{ij}\}_{i,j=1,\cdots,m}$; set $H_m=0$\;
\For{$j\gets 1$ \KwTo $m$}{
	$\boldsymbol{w}_j\gets A\boldsymbol{v}_j$\;
	\For{$i\gets\max\{1,j-k+1\}$ \KwTo $j$}{
		$h_{ij}\gets(\boldsymbol{w}_j,\boldsymbol{v}_i)$\;
		$\boldsymbol{w}_j\gets\boldsymbol{w}_j-h_{ij}\boldsymbol{v}_i$\;
	}
	$h_{j+1,j}\gets\|\boldsymbol{w}_j\|_2$\;
	\If{$h_{j+1,j}=0$}{
		set $m\gets j$; Goto 15\;
	}
	$\boldsymbol{v}_{j+1}\gets\boldsymbol{w}_j/h_{j+1,j}$\;
}
$\boldsymbol{y}_m\gets H_m^{-1}(\beta\boldsymbol{e}_1)$; $\boldsymbol{x}_m\gets\boldsymbol{x}_0+V_m\boldsymbol{y}_m$\;
\eIf{$h_{m+1,m}$ is small enough}{
	Stop\;
	}{
	$\boldsymbol{x}_0\gets\boldsymbol{x}_m$ and Goto 1\;
}
\end{algorithm}
%[/latex]

\subsubsection*{不完全正交化方法(IOM)算法说明}

- 上述算法的第$3\thicksim14$行便是不完全正交的Arnoldi方法(见改进的和不完全正交化的Arnoldi算法一文),主要体现在第$5$行内循环的起始条件。也就是说,第$j$步中计算的正交化向量$\boldsymbol{v}_{j+1}$只和前$k$个向量$\{\boldsymbol{v}_{j-k+1},\cdots,\boldsymbol{v}_j\}$正交,从而缩减了正交化的计算量;

- 虽然上面的不完全正交的Arnoldi方法只计算并保存了$k+1$个正交化向量,但观察算法的第$15$行,$\boldsymbol{x}_m$的计算却需要完整的$m$个正交化向量。一种解决方式是在算法最后重新计算$m$个正交化向量$\{\boldsymbol{v}_1,\cdots,\boldsymbol{v}_m\}$,但并不高效;另一种解决方法则是使用下面所要介绍的直接不完全正交化方法(DIOM);

- 算法的第$16\thicksim20$引入了重启机制,当$\boldsymbol{x}_m$的残差过大时(通常以$h_{m+1,m}$的值为判断依据),重新开始新一轮的不完全正交化方法(IOM);

求解[label]线性方程组[/label]$A\boldsymbol{x}=\boldsymbol{b}$的[label]现代迭代法[/label]都是在[label]Krylov子空间[/label]框架下进行推导的(见由相伴矩阵引入Krylov子空间法一文)。在$\mathbb{R}^n$中,关于$A$和$\boldsymbol{r}_0$的Krylov子空间定义为:

\subsection*{直接不完全正交化方法(DIOM)}

\subsection*{直接不完全正交化方法(DIOM)算法原理}

不完全正交化方法(IOM)的不足主要在于算法第$15$行处的计算,即:

\begin{equation}
\label{eq:line15}
\boldsymbol{y}_m=H_m^{-1}(\beta\boldsymbol{e}_1),\quad\boldsymbol{x}_m=\boldsymbol{x}_0+V_m\boldsymbol{y}_m
\end{equation}

indent其中$\boldsymbol{y}_m$的计算需要用到$H_m^{-1}$,而$\boldsymbol{x}_m$的计算又需要用到$V_m$和$\boldsymbol{y}_m$,但由于IOM方法只保存了最近$k+1$个正交化向量,因此无法高效计算$m$维的$H_m^{-1}$和$V_m$。而直接不完全正交化方法(DIOM)就是要改变原算法第$15$行处$\boldsymbol{x}_m$的计算方式,其思想是用上一步$\boldsymbol{x}_{m-1}$来表示$\boldsymbol{x}_m$。

观察式(\ref{eq:line15})中$\boldsymbol{y}_m$的计算:它需要用到上Hessenberg矩阵$H_m$的逆矩阵。对于给定的$k$,通过不完全正交化方法(IOM)得到的$H_m$是一个带宽为$k+1$(上带宽为$k-1$)的矩阵(见改进的和不完全正交化的Arnoldi算法一文)。如果对$H_m$进行[label]LU分解[/label]的话,可以得到$H_m=L_mU_m$。根据上Hessenberg矩阵的结构特性可知,$L_m=(l_{ij})$是下带宽为$1$的[label]单位下三角矩阵[/label];而$U_m$是上带宽为$k-1$的上三角矩阵。于是,可以略去式(\ref{eq:line15})中$\boldsymbol{y}_m$的计算,并将其改写为:

\begin{align}
\label{eq:pz}
\begin{split}
\boldsymbol{x}_m&=\boldsymbol{x}_0+V_m(L_mU_m)^{-1}(\beta\boldsymbol{e}_1) \\
&=\boldsymbol{x}_0+\Big(V_mU_m^{-1}\Big)\Big(L_m^{-1}(\beta\boldsymbol{e}_1)\Big)=\boldsymbol{x}_0+P_m\boldsymbol{z}_m
\end{split}
\end{align}

考虑$P_m=V_mU_m^{-1}$,即:$P_mU_m=V_m$,其矩阵形式可以表示为:

\[
\begin{pmatrix}
~ & ~ & ~ & ~ & \\
~ & ~ & ~ & ~ & \\
\boldsymbol{p}_1 & \cdots & \boldsymbol{p}_{m-k+1} & \cdots & \boldsymbol{p}_m \\
~ & ~ & ~ & ~ & \\
~ & ~ & ~ & ~ & \\
\end{pmatrix}
\begin{pmatrix}
u_{11} & \cdots & u_{1k} &  & \\
~ & \ddots & & \ddots & \\
& & \ddots & & u_{m-k+1,m} \\
& & & \ddots & \vdots \\
& & & & u_{mm} \\
\end{pmatrix}=
\begin{pmatrix}
~ & ~ & ~ \\
~ & ~ & ~ \\
\boldsymbol{v}_1 & \cdots & \boldsymbol{v}_m \\
~ & ~ & ~ \\
~ & ~ & ~ \\
\end{pmatrix}
\]

indent因此,$\boldsymbol{v}_m$可以表示为$\boldsymbol{p}_{m-k+1},\cdots,\boldsymbol{p}_m$的线性组合,即:

\begin{equation}
\label{eq:pm}
\boldsymbol{v}_m=\sum_{i=m-k+1}^{m}u_{im}\boldsymbol{p}_i\quad\Longrightarrow\quad
\boldsymbol{p}_m=\dfrac{1}{u_{mm}}\left(\boldsymbol{v}_m-\sum_{i=m-k+1}^{m-1}u_{im}\boldsymbol{p}_i\right)
\end{equation}

indent式(\ref{eq:pm})说明$\boldsymbol{p}_m$可以借由此前的$\boldsymbol{p}_i$和$\boldsymbol{v}_i$计算得到。进而考虑$\boldsymbol{z}_m=L_m^{-1}(\beta\boldsymbol{e}_1)$,即:$L_m\boldsymbol{z}_m=\beta\boldsymbol{e}_1$,同样其矩阵形式可以表示为:

\[
\begin{pmatrix}
1 & & & & \\
l_{21} & 1 & & & \\
& l_{32} & \ddots & & \\
& & \ddots & 1 & \\
& & & l_{m,m-1} & 1 \\
\end{pmatrix}
\begin{pmatrix}
(\boldsymbol{z}_m)_1 \\ (\boldsymbol{z}_m)_2 \\ \vdots \\ (\boldsymbol{z}_m)_{m-1} \\ (\boldsymbol{z}_m)_m
\end{pmatrix}=
\begin{pmatrix}
\beta \\ 0 \\ \vdots \\ \vdots \\ 0
\end{pmatrix}
\]

indent因此,有如下关系式:

\begin{eqnarray}
& (\boldsymbol{z}_m)_1=\beta,\quad l_{i,i-1}(\boldsymbol{z}_m)_{i-1} + (\boldsymbol{z}_m)_i=0\quad\Longrightarrow\quad(\boldsymbol{z}_m)_i=-l_{i,i-1}(\boldsymbol{z}_m)_{i-1} \nonumber \\
& \boldsymbol{z}_m=\begin{pmatrix}\boldsymbol{z}_{m-1}\\\xi_m\end{pmatrix}\quad\Longrightarrow\quad
\begin{aligned}\xi_{m}&=(\boldsymbol{z}_m)_m=-l_{m,m-1}(\boldsymbol{z}_m)_{m-1}\\
&=-l_{m,m-1}(\boldsymbol{z}_{m-1})_{m-1} \\
&=-l_{m,m-1}\xi_{m-1}
\end{aligned} \label{eq:xim}
\end{eqnarray}

结合式(\ref{eq:pz}、\ref{eq:pm}、\ref{eq:xim}),则$\boldsymbol{x}_m$可以改写为如下形式:

\begin{align*}
\boldsymbol{x}_m&=\boldsymbol{x}_0+P_m\boldsymbol{z}_m=\boldsymbol{x}_0+\begin{pmatrix}P_{m-1} & \boldsymbol{p}_m\end{pmatrix}\begin{pmatrix}\boldsymbol{z}_{m-1}\\\xi_m\end{pmatrix} \\
&= (\boldsymbol{x}_0+P_{m-1}\boldsymbol{z}_{m-1})+\xi_m\boldsymbol{p}_m=\boldsymbol{x}_{m-1}+\xi_m\boldsymbol{p}_m
\end{align*}

indent其中,$\boldsymbol{p}_m$按式(\ref{eq:pm})由前$k-1$个$\{\boldsymbol{p}_i\}_{i=m-k+1}^{m-1}$计算得到,而$\xi_m$则按式(\ref{eq:xim})由$\xi_{m-1}$计算得到,这样就得到了$\boldsymbol{x}_m$的直接计算方法,而这也就是直接不完全正交法(DIOM)的根本原理。

\subsection*{直接不完全正交化方法(DIOM)算法实现}

给定矩阵$A\in\mathbb{R}^{n \times n}$,$\boldsymbol{b},\boldsymbol{x}_0\in\mathbb{R}^{n}$,以及正整数$k$,直接不完全正交化(DIOM)算法实现如下:

%[latex mode=1]
%[+preamble]
%\usepackage[titlenotnumbered,lined,linesnumbered]{algorithm2e}
%[/preamble]
\TitleOfAlgo{Direct Incomplete Orthogonalization Method}
\begin{algorithm}[H]
compute $\boldsymbol{r}_0\gets\boldsymbol{b}-A\boldsymbol{x}_0$, $\beta\gets\|\boldsymbol{r}_0\|_2$, $\boldsymbol{v}_1\gets\boldsymbol{r}_0/\beta$\;
\For{$m\gets 1,2,\cdots$}{
	$\boldsymbol{w}_m\gets A\boldsymbol{v}_m$\;
	\For{$i\gets\max\{1,m-k+1\}$ \KwTo $m$}{
		$h_{im}\gets(\boldsymbol{w}_m,\boldsymbol{v}_i)$\;
		$\boldsymbol{w}_m\gets\boldsymbol{w}_m-h_{im}\boldsymbol{v}_i$\;
	}
	$h_{m+1,m}\gets\|\boldsymbol{w}_m\|_2$\;
	$\boldsymbol{v}_{m+1}\gets\boldsymbol{w}_m/h_{m+1,m}$\;
	update LU factorization of $H_m$, i.e., $l_{m,m-1}$, $\{u_{im}\}_{i=\max\{1,m-k+1\}}^{m}$\;
	\eIf{$m=1$}{
		$\xi_m\gets\beta$\;
	}{
		$\xi_m\gets-l_{m,m-1}\xi_{m-1}$\;
	}
	$\boldsymbol{p}_m\gets u^{-1}_{mm}\left(\boldsymbol{v}_m-\sum_{i=\max\{1,m-k+1\}}^{m-1}u_{im}\boldsymbol{p}_i\right)$\;
	$\boldsymbol{x}_m=\boldsymbol{x}_{m-1}+\xi_m\boldsymbol{p}_m$\;
	\If{residual of $\boldsymbol{x}_m$ is small enough}{
		Stop\;
	}
}
\end{algorithm}
%[/latex]

\subsubsection*{直接不完全正交化方法(DIOM)算法分析}

上述直接不完全正交化方法(DIOM)有以下几点需要特别说明:

- 首先以$m$作为算法最外层循环(第$2$行)的条件,每一次迭代的结果$\boldsymbol{x}_m$都要用到上一次的$\boldsymbol{x}_{m-1}$;

- 第$3\thicksim9$行实际就是不完全正交化方法(IOM),主要为了得到上Hessenberg矩阵$H_m$;特别需要注意的是,第$9$行计算得到的$\boldsymbol{v}_{m+1}$在后面$\boldsymbol{x}_m$并没有用到,因此可以省略;

- 第$10$行对新得到的上Hessenberg矩阵$H_m$进行LU分解。事实上,上一步迭代后已经有了$H_{m-1}=L_{m-1}U_{m-1}$,因此$L_m$中只需计算$l_{m,m-1}$,而$U_m$中只需计算最后一列的后$k$个元素$\{u_{im}\}_{i=\max\{1,m-k+1\}}^{m}$。一般使用Gauss消去法即可实现$H_m$的更新,原理如下(假定$m=6,k=3$):

\begin{eqnarray*}
& \left(\begin{array}{ccccc|c}
\ast & \ast & \ast & ~ & ~ & ~ \\
\ast & \ast & \ast & \ast & ~ & ~ \\
~ & \ast & \ast & \ast & \ast & ~ \\
~ & ~ & \ast & \ast & \ast & h_{46} \\
~ & ~ & ~ & \ast & \ast & h_{56} \\
\cline{1-6}
~ & ~ & ~ & ~ & h_{65} & h_{66} \\
\end{array}\right)=
\left(\begin{array}{ccccc|c}
1 & ~ & ~ & ~ & ~ & ~ \\
l_{21} & 1 & ~ & ~ & ~ & ~ \\
~ & l_{32} & 1 & ~ & ~ & ~ \\
~ & ~ & h_{43} & 1 & ~ & ~ \\
~ & ~ & ~ & l_{54} & 1 & ~ \\
\cline{1-6}
& & & & \boldsymbol{l_{65}} & 1 \\
\end{array}\right)
\left(\begin{array}{ccccc|c}
\ast & \ast & \ast & ~ & ~ & ~ \\
~ & \ast & \ast & \ast & ~ & ~ \\
~ & ~ & \ast & \ast & \ast & ~ \\
~ & ~ & ~ & \ast & \ast & \boldsymbol{u_{46}} \\
~ & ~ & ~ & ~ & u_{55} & \boldsymbol{u_{56}} \\
\cline{1-6}
~ & ~ & ~ & ~ & ~ & \boldsymbol{u_{66}} \\
\end{array}\right) \\
& l_{65}=h_{65}/u_{55},\quad\begin{aligned}
u_{46}&=h_{46} \\
u_{56}&=h_{56}-l_{54}u_{46} \\
u_{66}&=h_{66}-l_{65}u_{56}
\end{aligned}
\end{eqnarray*}

[alert type="error"]注意:一般通常会采用不选主元的Gauss消去法(见不选主元的Gauss消去法一文),所以一旦出现$u_{mm}=0$的情况,程序则会提前结束。此时可以改用部分主元的Gauss消去法,其原理见BLAS三个级别的列主元Gauss消去法一文。[/alert]

- 算法的第$11\thicksim15$行是利用上面更新计算得到的$l_{m,m-1}$以及上一步的$\xi_{m-1}$,计算求得当前步的$\xi_m$;

- 算法的第$16$行是利用上面更新计算得到的$u_{mm}$,$\{u_{im}\}_{i=\max\{1,m-k+1\}}^{m-1}$以及此前得到的$\{\boldsymbol{p}_i\}_{i=\max\{1,m-k+1\}}^{m-1}$,计算求得当前步的$\boldsymbol{p}_m$;

- 最后,算法第$17$行利用当前步求得的$\xi$,$\boldsymbol{p}_m$,以及上一步的$\boldsymbol{x}_{m-1}$,计算求得当前步的$\boldsymbol{x}_m$;并在算法第$18\thicksim20$行判断$\boldsymbol{x}_m$的残差是否满足精度要求。

[alert type="info"]事实上,正如如完全正交化方法(FOM)的迭代终止条件一样(见线性方程组的Arnoldi方法之:FOM一文),直接不完全正交化方法(DIOM)也可以以$h_{m+1,m}$是否足够小为判断依据。理由见式(\ref{eq:hmm})[/alert]

\begin{align}
\label{eq:hmm}
\begin{split}
\|\boldsymbol{b}-A\boldsymbol{x}_m\|_2&=h_{m+1,m}\left|\boldsymbol{e}_m^T\boldsymbol{y}_m\right| \\
&=h_{m+1,m}\left|\boldsymbol{e}_m^T\left[(H_m)^{-1}(\beta\boldsymbol{e}_1)\right]\right|=h_{m+1,m}\left|\boldsymbol{e}_m^T\left[U_m^{-1}(L_m^{-1}\beta\boldsymbol{e}_1)\right]\right| \\
&=h_{m+1,m}\left|\boldsymbol{e}_m^T\left(U_m^{-1}\boldsymbol{z}_m\right)\right| \\
&=h_{m+1,m}\left|
\begin{pmatrix}0&\cdots&0&1\end{pmatrix}\begin{pmatrix}\ast&\cdots&\cdots&\ast\\&\ddots&\ddots&\vdots\\&&\ddots&\vdots\\&&&u^{-1}_{mm}\end{pmatrix}\begin{pmatrix}\xi_1\\\vdots\\\vdots\\\xi_m\end{pmatrix}\right| \\
&=h_{m+1,m}\left|\dfrac{\xi_m}{u_{mm}}\right|
\end{split}
\end{align}

%[/aside_content]


% H2, H3
\subsection*{dsfsdt}
%[latex mode=1]
%[+preamble]
%\usepackage[titlenotnumbered,lined,linesnumbered]{algorithm2e}
\let\oldnl\nl
\newcommand{\nonl}{\renewcommand{\nl}{\let\nl\oldnl}}
%[/preamble]
%\quicklatex{size=14}
\TitleOfAlgo{BLAS-3 Gaussian Elimination with Partial Pivoting}
\begin{algorithm}[H]
\For{$l\gets 1$ \KwTo $n-1$}{
$k\gets (l-1)b+1$\;
factorize $PA^{(l)}=LU$ using BLAS-2\;
store $L$ and $U$ in $A$\;
left multiply $A(1:n,k+b:n)$ by $P$\;
$A(k:k+b-1,k+b:n)\gets T^{-1}A(k:k+b-1,k+b:n)$\;
\nonl\Indp where $T$ is the unit-lower-trian of $A(k:k+b-1,k:k+b-1)$\;
\DontPrintSemicolon\Indm$A(k+b:n,k+b:n)\gets A(k+b:n,k+b:n)$\;
\PrintSemicolon\nonl\Indp$-A(k+b:n,k:k+b-1)A(k:k+b-1,k+b:n)$\;
}
\end{algorithm}
%[/latex]

% H4
% \subsubsection*{}

\end{document}
