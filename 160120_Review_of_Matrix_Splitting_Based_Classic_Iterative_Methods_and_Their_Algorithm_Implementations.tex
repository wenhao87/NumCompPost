\documentclass[12pt, UTF8, nofonts]{ctexart}

\setCJKmainfont[BoldFont=STHeiti,ItalicFont=STKaiti]{STKaiti}
\setCJKsansfont[BoldFont=STHeiti]{STXihei}
\setCJKmonofont{STFangsong}

\usepackage[letterpaper,left=.5in,right=.5in,top=1in,bottom=.5in]{geometry}
\usepackage{afterpage, color, multirow}
\usepackage{amsmath}
\usepackage{amsfonts}
\usepackage{amssymb}
\usepackage[titlenotnumbered,lined,linesnumbered]{algorithm2e}

\definecolor{bgcolor}{RGB}{13, 12, 12}
\definecolor{ttcolor}{RGB}{0, 255, 0}
\pagecolor{bgcolor}\afterpage{\nopagecolor}
\makeatletter
\newcommand{\globalcolor}[1]{\color{#1}\global\let\default@color\current@color}
\makeatother
\AtBeginDocument{\globalcolor{ttcolor}}

\begin{document}

%[aside_content]

\section*{详解基于矩阵分裂的经典迭代算法及算法实现}

上文详细讨论了基于矩阵分裂的迭代格式的收敛性(见详解基于矩阵分裂的迭代格式的收敛性一文),此外在求解线性方程组的基本迭代法一文中,简单介绍了基本迭代法原理以及几个经典迭代算法。本文作为补充,将详细介绍经典迭代算法的相关特性和算法实现,并且将着重讨论经典迭代法在[label]二维Poisson问题[/label]模型上(见详解一维和二维离散Poisson方程一文)的算法实现。

常见的基于矩阵分裂的古典迭代算法(Classic Iterative Methods)包括有:1)Jacobi算法、2)Gauss-Seidel算法、3)SOR(Successive Over-Relaxation)算法、4)SSOR(Symmetric SSOR)算法。此外,还有AOR(Accelerated Over-Relaxation)算法、Richardson算法等。

\subsection*{Jacobi迭代}

[label]Jacobi迭代[/label]将矩阵$\boldsymbol{A}$分裂为$\boldsymbol{A}=\boldsymbol{D}-\boldsymbol{L}-\boldsymbol{U}$的形式,其中$\boldsymbol{D}$为$\boldsymbol{A}$的B对角线部分,$-\boldsymbol{L}$和$-\boldsymbol{U}$分别为$\boldsymbol{A}$的B严格下三角部分和B严格上三角部分。根据[label]矩阵分裂[/label]形式$\boldsymbol{A}=\boldsymbol{M}-\boldsymbol{N}$,取$\boldsymbol{M}=\boldsymbol{D}$,$\boldsymbol{N}=\boldsymbol{L}+\boldsymbol{U}$,即可得到Jacobi迭代算法,即:

\begin{eqnarray}
    \label{eq:jacobi}
    & \boldsymbol{x}^{(k+1)} = \boldsymbol{D}^{-1}(\boldsymbol{L}+\boldsymbol{U})\boldsymbol{x}^{(k)}+\boldsymbol{D}^{-1}\boldsymbol{b} \;,\quad k=0,1,2,\ldots \\
    & \boldsymbol{G}_{\mathrm{J}}=\boldsymbol{D}^{-1}(\boldsymbol{L}+\boldsymbol{U})
\end{eqnarray}

noindent其中$\boldsymbol{G}_{\mathrm{J}}$为Jacobi迭代算法的迭代矩阵。式(\ref{eq:jacobi})的分量形式可以表示为:

\begin{equation}
    \label{eq:jacobicomp}
    x_{i}^{(k+1)}=\dfrac{1}{a_{ii}}\left(b_i-\sum_{j=1,j \neq i}^na_{ij}x_j^{(k)}\right) \;,\quad i=1,2,\cdots,n
\end{equation}

[alert type="info"]观察式(\ref{eq:jacobicomp}),Jacobi迭代中分量$x_i^{(k+1)}$的更新顺序与$i$无关,换言之,Jacobi迭代B非常适合并行计算。[/alert]

%[latex mode=1]
%[+preamble]
%\usepackage[titlenotnumbered,lined,linesnumbered]{algorithm2e}
%[/preamble]
\TitleOfAlgo{Jacobi Iterative Method for Linear Systems}
\begin{algorithm}[H]
    choose an initial guess $\boldsymbol{x}^{(0)}$ \;
    \While{not converge}{
        \For{$i \gets 1$ \KwTo $n$}{
            $\displaystyle x_i^{(k+1)}=\left(b_i-\sum_{j=1,j \neq i}^na_{ij}x_j^{(k)}\right)\Big/a_{ii}$ \;
        }
    }
\end{algorithm}
%[/latex]

式(\ref{eq:jacobi})的迭代格式也可以改写为如下形式,其中$\boldsymbol{r}_k\triangleq \boldsymbol{b}-A\boldsymbol{x}^{(k)}$是第$k$次迭代后的[label]残差[/label]:

\[
    \begin{aligned}
        \boldsymbol{x}^{(k+1)} &= \boldsymbol{D}^{-1}(\boldsymbol{D}-\boldsymbol{A})\boldsymbol{x}^{(k)}+\boldsymbol{D}^{-1}\boldsymbol{b} \\
        &= \boldsymbol{x}^{(k)}+\boldsymbol{D}^{-1}(\boldsymbol{b}-\boldsymbol{Ax}^{(k)}) =
        \boldsymbol{x}^{(k)}+\boldsymbol{D}^{-1}\boldsymbol{r}_k \\
    \end{aligned} \;,\quad k=0,1,2,\ldots
\]

在详解一维和二维离散Poisson方程一文中提到,通过五点差分离散二维Poisson方程后得到的离散方程为:

\begin{eqnarray*}
    & 4u_{ij}-u_{i-1,j}-u_{i+1,j}-u_{i,j-1}-u_{i,j+1} = h^2f_{ij} \\
    & \boldsymbol{T}\boldsymbol{u} = h^2\boldsymbol{f} \\
\end{eqnarray*}

结合Jacobi迭代算法,便能得到如下的求解二维Poisson方程的Jacobi迭代算法:

%[latex mode=1]
%[+preamble]
%\usepackage[titlenotnumbered,lined,linesnumbered]{algorithm2e}
%[/preamble]
\TitleOfAlgo{Jacobi Iterative Method for 2D Poisson}
\begin{algorithm}[H]
    choose an initial guess $\boldsymbol{v}^{(0)}$ \;
    \While{not converge}{
        \For{$i \gets 1$ \KwTo $n$}{
            \For{$j \gets 1$ \KwTo $n$}{
                $u_{ij}^{(k+1)}=\left(h^2f_{ij}+u^{(k)}_{i-1,j}+u^{(k)}_{i+1,j}+u^{(k)}_{i,j-1}+u^{(k)}_{i,j+1}\right)\Big/4$ \;
            }
        }
    }
\end{algorithm}
%[/latex]

\subsection*{Gauss-Seidel迭代}

对于矩阵分裂$\boldsymbol{A}=\boldsymbol{M}-\boldsymbol{N}=\boldsymbol{D}-\boldsymbol{L}-\boldsymbol{U}$,[label]Gauss-Seidel迭代[/label]取$\boldsymbol{M}=\boldsymbol{D}-\boldsymbol{L}$,$\boldsymbol{N}=\boldsymbol{U}$,即:

\begin{eqnarray}
    \label{eq:gs}
    & \boldsymbol{x}^{(k+1)}=(\boldsymbol{D}-\boldsymbol{L})^{-1}\boldsymbol{U}\boldsymbol{x}^{(k)}+(\boldsymbol{D}-\boldsymbol{L})^{-1}\boldsymbol{b} \\
    & \boldsymbol{G}_{\mathrm{GS}}=(\boldsymbol{D}-\boldsymbol{L})^{-1}\boldsymbol{U}
\end{eqnarray}

noindent其中$\boldsymbol{G}_{\mathrm{GS}}$为Gauss-Seidel迭代算法的迭代矩阵。对式(\ref{eq:gs})进行改写后便能得到式(\ref{eq:gscomp})的分量形式,即:

\begin{eqnarray}
    \label{eq:gscomp}
    & (\boldsymbol{D}-\boldsymbol{L})\boldsymbol{x}^{(k+1)} = \boldsymbol{Ux}^{(k)}+\boldsymbol{b} \;\Rightarrow\;
    \boldsymbol{D}\boldsymbol{x}^{(k+1)}=\boldsymbol{L}\boldsymbol{x}^{(k+1)} +\boldsymbol{Ux}^{(k)}+\boldsymbol{b} \\
    & \displaystyle x_{i}^{(k+1)} = \dfrac{1}{a_{ii}}\left(b_i-\sum_{j=1}^{i-1}a_{ij}x_j^{(k+1)}-\sum_{j=i+1}^{n}a_{ij}x_{j}^{(k)}\right) \;,\quad i=1,2,\ldots,n
\end{eqnarray}

[alert type="info"]观察式(\ref{eq:gscomp})可以发现,Gauss-Seidel迭代算法充分利用了已经获得的最新数据,因此它通常比Jacobi算法B更快速。但是需要注意的是,Gauss-Seidel迭代算法需要B按顺序跟新分量,因此b不适合并行计算。[/alert]

%[latex mode=1]
%[+preamble]
%\usepackage[titlenotnumbered,lined,linesnumbered]{algorithm2e}
%[/preamble]
\TitleOfAlgo{Gauss-Seidel Iterative Method for Linear Systems}
\begin{algorithm}[H]
    choose an initial guess $\boldsymbol{x}^{(0)}$ \;
    \While{not converge}{
        \For{$i \gets 1$ \KwTo $n$}{
            $\displaystyle x_i^{(k+1)}=\left(b_i-\sum_{j=1}^{i-1}a_{ij}x_j^{(k+1)}-\sum_{j=i+1}^na_{ij}x_j^{(k)}\right)\Big/a_{ii}$ \;
        }
    }
\end{algorithm}
%[/latex]

对于二维离散Poisson方程,可以利用[label]红黑排序[/label]来提升算法的并行性。其原理是将二维网格的点依次做红黑标记,而在更新未知量的过程中,先只更新红色节点,此时只需要用到黑色节点的数据,然后再使用红色节点来更新黑色节点,从而实现了并行计算。结合Gauss-Seidel迭代算法和红黑排序方法,便能得到如下的求解二维Poisson方程的红黑排序Gauss-Seidel迭代算法:

%[latex mode=1]
%[+preamble]
%\usepackage[titlenotnumbered,lined,linesnumbered]{algorithm2e}
%[/preamble]
\TitleOfAlgo{B/R G-S Iterative Method for 2D Poisson}
\begin{algorithm}[H]
    choose an initial guess $\boldsymbol{v}^{(0)}$ \;
    \While{not converge}{
        \For{$(i,j)$ is red}{
            $u_{ij}^{(k+1)}=\left(h^2f_{ij}+u^{(k)}_{i-1,j}+u^{(k)}_{i+1,j}+u^{(k)}_{i,j-1}+u^{(k)}_{i,j+1}\right)\Big/4$ \;
        }
        \For{$(i,j)$ is black}{
            $u_{ij}^{(k+1)}=\left(h^2f_{ij}+u^{(k+1)}_{i-1,j}+u^{(k+1)}_{i+1,j}+u^{(k+1)}_{i,j-1}+u^{(k+1)}_{i,j+1}\right)\Big/4$ \;
        }
    }
\end{algorithm}
%[/latex]

\subsection*{SOR迭代}

在Gauss-Seidel迭代算法的基础之上,通过引入一个[label]松弛参数[/label]$\omega$,可以加速收敛速度,所得到的便是[label]SOR[/label](Successive Overrelaxation)算法。该算法的B核心思想是将式(\ref{eq:gscomp})Gauss-Seidel迭代算法中第$k+1$步的近似解$\boldsymbol{x}^{(k+1)}$与第$k$步的近似解做一个B加权平均,从而得到一个新的近似解。因此,SOR迭代算法的迭代格式为:

\begin{eqnarray}
    \label{eq:sor}
    & \begin{aligned}
        & \boldsymbol{x}^{(k+1)} = (1-\omega)\boldsymbol{x}^{(k)} + \omega\overbrace{\left(\boldsymbol{D}^{-1}(\boldsymbol{L}\boldsymbol{x}^{(k+1)} + \boldsymbol{U}\boldsymbol{x}^{(k)}) + \boldsymbol{D}^{-1}\boldsymbol{b}\right)}^{\boldsymbol{x}^{(k+1)}} \\
        \Rightarrow \;& \boldsymbol{D}\boldsymbol{x}^{(k+1)} = (1-\omega)\boldsymbol{D}\boldsymbol{x}^{(k)} + \omega\boldsymbol{L}\boldsymbol{x}^{(k+1)} + \omega\boldsymbol{U}\boldsymbol{x}^{(k)} + \omega\boldsymbol{b} \\
        \Rightarrow \;& \boldsymbol{x}^{(k+1)} = (\boldsymbol{D}-\omega\boldsymbol{L})^{-1}\Big((1-\omega)\boldsymbol{D}+\omega\boldsymbol{U}\Big)\boldsymbol{x}^{(k)} + \omega(\boldsymbol{D}-\omega\boldsymbol{L})^{-1}\boldsymbol{b}
    \end{aligned} \\
    & \boldsymbol{G}_{\mathrm{SOR}} = (\boldsymbol{D}-\omega\boldsymbol{L})^{-1}\Big((1-\omega)\boldsymbol{D}+\omega\boldsymbol{U}\Big) \quad
    \left\{\;\begin{aligned}
        \boldsymbol{M} &=
        \dfrac{1}{\omega}\boldsymbol{D}-\boldsymbol{L} \\
        \boldsymbol{N} &=
        \dfrac{1-\omega}{\omega}\boldsymbol{D}+\boldsymbol{U} \\
    \end{aligned}\right.
\end{eqnarray}

noindent其中$\boldsymbol{G}_{\mathrm{SOR}}$为SOR迭代算法的迭代矩阵,$\omega$为松弛参数。特别地,当$\omega=1$时,SOR迭代算法即为Gauss-Seidel迭代算法。一般而言,当$\omega<1$时,称为[label]低松弛算法[/label],而$\omega>1$时则称为[label]超松弛算法[/label]。

对式(\ref{eq:sor})进行改写即可得到如下式(\ref{eq:sorcomp})的分量形式,即:

\begin{equation}
    \label{eq:sorcomp}
    x_{i}^{(k+1)} = (1-\omega)x_{i}^{(k)} + \dfrac{\omega}{a_{ii}} \left(b_i-\sum_{j=1}^{i-1}a_{ij}x_j^{(k+1)}-\sum_{j=i+1}^na_{ij}x_{j}^{(k)}\right)
\end{equation}

%[latex mode=1]
%[+preamble]
%\usepackage[titlenotnumbered,lined,linesnumbered]{algorithm2e}
%[/preamble]
\TitleOfAlgo{SOR Iterative Method for Linear Systems}
\begin{algorithm}[H]
    choose an initial guess $\boldsymbol{x}^{(0)}$ and parameter $\omega$ \;
    \While{not converge}{
        \For{$i \gets 1$ \KwTo $n$}{
            $\displaystyle x_i^{(k+1)} = (1-\omega)x_i^{(k)} + \dfrac{w}{a_{ii}}\left(b_i-\sum_{j=1}^{i-1}a_{ij}x_j^{(k+1)}-\sum_{j=i+1}^na_{ij}x_j^{(k)}\right)$ \;
        }
    }
\end{algorithm}
%[/latex]

[alert type="info"]SOR迭代算法在很长一段时间内是求解线性方程组的首选方法,并且在大多是情况下,当$\omega>1$时会取得较好的收敛效果。SOR迭代算法的优点在于通过选取适当的$\omega$可以大大提高算法的收敛速度,但其难点也在于如何选取B最优的参数。[/alert]

同样,SOR迭代算法也可以通过结合红黑排序算法来求解二维离散Poisson方程,其算法描述如下:

%[latex mode=1]
%[+preamble]
%\usepackage[titlenotnumbered,lined,linesnumbered]{algorithm2e}
%[/preamble]
\TitleOfAlgo{B/R G-S Iterative Method for 2D Poisson}
\begin{algorithm}[H]
    choose an initial guess $\boldsymbol{v}^{(0)}$ and parameter $\omega$ \;
    \While{not converge}{
        \For{$(i,j)$ is red}{
            $u_{ij}^{(k+1)} = (1-\omega)v_{ij}^{(k)} + \omega \left( h^2f_{ij} + u^{(k)}_{i-1,j} + u^{(k)}_{i+1,j} + u^{(k)}_{i,j-1} + u^{(k)}_{i,j+1} \right)\Big/4$ \;
        }
        \For{$(i,j)$ is black}{
            $u_{ij}^{(k+1)} = (1-\omega)v_{ij}^{k} + \omega \left( h^2f_{ij} + u^{(k+1)}_{i-1,j} + u^{(k+1)}_{i+1,j} + u^{(k+1)}_{i,j-1} + u^{(k+1)}_{i,j+1} \right)\Big/4$ \;
        }
    }
\end{algorithm}
%[/latex]

\subsection*{SSOR迭代算法}

如果将式(\ref{eq:sor})SOR迭代算法中的$\boldsymbol{L}$和$\boldsymbol{U}$互换,并与原SOR算法的迭代格式相结合,便能得到如下式(\ref{eq:ssor})的两步迭代算法,称之为[label]SSOR[/label](Symmetric SOR,对称超松弛)算法,即:

\begin{equation}
    \label{eq:ssor}
    \begin{aligned}
        & \left\{ \; \begin{aligned}
            \boldsymbol{x}^{(k+\frac{1}{2})} &= (1-\omega) \boldsymbol{x}^{(k)} + \omega \left( \boldsymbol{D}^{-1} (\boldsymbol{L}\boldsymbol{x}^{(k+\frac{1}{2})} + \boldsymbol{U}\boldsymbol{x}^{(k)}) + \boldsymbol{D}^{-1}\boldsymbol{b} \right) \\
            \boldsymbol{x}^{(k+1)} &= (1-\omega) \boldsymbol{x}^{(k+\frac{1}{2})} + \omega \left( \boldsymbol{D}^{-1} (\boldsymbol{U}\boldsymbol{x}^{(k+1)} + \boldsymbol{L}\boldsymbol{x}^{(k+\frac{1}{2})}) + \boldsymbol{D}^{-1}\boldsymbol{b}\right) \\
        \end{aligned} \right. \\
        \Longrightarrow \; &\left\{ \; \begin{aligned}
            \boldsymbol{x}^{(k+\frac{1}{2})} &= (\boldsymbol{D}-\omega\boldsymbol{L})^{-1}\Big((1-\omega)\boldsymbol{D}+\omega\boldsymbol{U}\Big)\boldsymbol{x}^{(k)} + \omega(\boldsymbol{D}-\omega\boldsymbol{L})^{-1}\boldsymbol{b} \\
            \boldsymbol{x}^{(k+1)} &= (\boldsymbol{D}-\omega\boldsymbol{U})^{-1}\Big((1-\omega)\boldsymbol{D}+\omega\boldsymbol{L}\Big)\boldsymbol{x}^{(k+\frac{1}{2})} + \omega(\boldsymbol{D}-\omega\boldsymbol{U})^{-1}\boldsymbol{b}
        \end{aligned} \right. \\
        \overset{elim.\;\boldsymbol{x}^{(k+\frac{1}{2})}}{\Longrightarrow} \; &\left\{ \; \begin{aligned}
            \boldsymbol{x}^{(k+1)} &= \boldsymbol{G}_{\mathrm{SSOR}}\boldsymbol{x}^{(k)} + \boldsymbol{g} \\
            \boldsymbol{G}_{\mathrm{SSOR}} &= (\boldsymbol{D}-\omega\boldsymbol{U})^{-1}\Big((1-\omega)\boldsymbol{D}+\omega\boldsymbol{L}\Big)(\boldsymbol{D}-\omega\boldsymbol{L})^{-1}\Big((1-\omega)\boldsymbol{D}+\omega\boldsymbol{U}\Big) \\
            \boldsymbol{M} &= \dfrac{1}{\omega(2-\omega)}(\boldsymbol{D}-\omega\boldsymbol{L})\boldsymbol{D}^{-1}(\boldsymbol{D}-\omega\boldsymbol{U}) \\
            \boldsymbol{N} &= \dfrac{1}{\omega(2-\omega)}\Big((1-\omega)\boldsymbol{D}+\omega\boldsymbol{L}\Big)\boldsymbol{D}^{-1}\Big((1-\omega)\boldsymbol{D}+\omega\boldsymbol{U}\Big)
        \end{aligned} \right. \\
    \end{aligned}
\end{equation}

[alert type="info"]一般来说,SOR迭代算法的渐进收敛速度(见详解基于矩阵分裂的迭代格式的收敛性一文)对参数$\omega$较为敏感,但SSOR迭代算法对参数$\omega$则不太敏感。[/alert]

%[latex mode=1]
%[+preamble]
%\usepackage[titlenotnumbered,lined,linesnumbered]{algorithm2e}
%[/preamble]
\TitleOfAlgo{SSOR Iterative Method for Linear Systems}
\begin{algorithm}[H]
    choose an initial guess $\boldsymbol{x}^{(0)}$ and parameter $\omega$ \;
    \While{not converge}{
        \For{$i \gets 1$ \KwTo $n$}{
            $\displaystyle x_i^{(k+\frac{1}{2})} = (1-\omega)x_i^{(k)} + \dfrac{w}{a_{ii}}\left(b_i-\sum_{j=1}^{i-1}a_{ij}x_j^{(k+\frac{1}{2})}-\sum_{j=i+1}^na_{ij}x_j^{(k)}\right)$ \;
        }
        \For(\tcp*[f]{in reversed order}){$i \gets n$ \KwTo $1$}{
            $\displaystyle x_i^{(k+1)} = (1-\omega)x_i^{(k+\frac{1}{2})} + \dfrac{w}{a_{ii}}\left(b_i-\sum_{j=1}^{i-1}a_{ij}x_j^{(k+\frac{1}{2})}-\sum_{j=i+1}^na_{ij}x_j^{(k+1)}\right)$ \;
        }
    }
\end{algorithm}
%[/latex]


\end{document}
