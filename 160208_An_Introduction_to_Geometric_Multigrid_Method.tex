\documentclass[12pt, UTF8, nofonts]{ctexart}

\setCJKmainfont[BoldFont=STHeiti,ItalicFont=STKaiti]{STKaiti}
\setCJKsansfont[BoldFont=STHeiti]{STXihei}
\setCJKmonofont{STFangsong}

\usepackage[letterpaper,left=.5in,right=.5in,top=1in,bottom=.5in]{geometry}
\usepackage{afterpage, color, multirow}
\usepackage{amsmath}
\usepackage{amsfonts}
\usepackage{amssymb}
\usepackage[titlenotnumbered,lined,linesnumbered]{algorithm2e}

\definecolor{bgcolor}{RGB}{13, 12, 12}
\definecolor{ttcolor}{RGB}{0, 255, 0}
\pagecolor{bgcolor}\afterpage{\nopagecolor}
\makeatletter
\newcommand{\globalcolor}[1]{\color{#1}\global\let\default@color\current@color}
\makeatother
\AtBeginDocument{\globalcolor{ttcolor}}

\begin{document}

%[aside_content]

\section*{几何多重网格方法概述}

此前详细介绍了三种用于多重网格方法中的[label]光滑子[/label](smoother),分别是:[label]Richardson 迭代[/label]光滑子、[label]带权重的 Jacobi 迭代[/label]光滑子、以及 [label]Gauss-Seidel 迭代[/label]光滑子(见 Richardson 迭代和带权重的 Jacobi 迭代光滑子和 Gauss-Seidel 迭代光滑子等文)。通过若干次迭代后,误差的高频部分很快衰减,因此都是很好的光滑子;但另一方面,低频部分却很难消减,从而有了多重网格方法,即:将残差方程限制在更粗的网格上,以此来减少低频误差(见由 Richardson 迭代引入多重网格方法和多重网格方法相关的矩阵理论等文)。本文将仅作为几何多重网格方法的概述,所使用的记号等与此后详细介绍的文章略有不同,但它们的核心思想都是一致的。

[alert type="success"]实用的多重网格技术,最初由 Brandt 在 1970 年引入。它可以经过$\mathcal{O}(N)$的操作求解 $N$ 个网格点离散后的[label]椭圆方程[/label]。[/alert]

\subsection*{几何多重网格}

\subsubsection*{问题模型的离散化}

考虑二维规则区域上的椭圆问题:$\mathcal{L}\boldsymbol{u}=\boldsymbol{f}$,其中$\mathcal{L}$为线性椭圆微分算子;$\boldsymbol{f}$为方程的右端项,并假设$\boldsymbol{u}$在求解区域$\Omega$的边界上为$0$。换言之,$\mathcal{L}\boldsymbol{u}$可以表示为$-(\boldsymbol{u}_{xx}+\boldsymbol{u}_{yy})$。那么,用[label]五点差分[/label]格式离散后,便能得到如下的线性方程组(其中$h=1/N$为网格大小,这里用$N$代表偶数,等价于此前的$n+1$,且$\boldsymbol{A}_h$的阶数为$N^2 \times N^2$):

\begin{equation}
  \label{eq:5ptls}
  \boldsymbol{A}_h\boldsymbol{u}_h = \boldsymbol{f}_h
\end{equation}

由此前的讨论可知,Gauss-Seidel 迭代法是一个很好的光滑子(见 Gauss-Seidel 迭代光滑子一文),也就是说,用 Gauss-Seidel 迭代若干次后,误差的[label]高频部分[/label]B很快衰减,但[label]低频部分[/label]却B很难消减。于是便有了[label]多重网格[/label]思想,即将迭代若干次后的[label]误差方程[/label]B限制在更粗的网格上来求解,从而能够大大减小低频误差。下面给出几何多重网格方法的核心思想。

针对式(\ref{eq:5ptls})得到的离散方程,其中$\boldsymbol{u}_h$为该线性方程组的B真解。设$\bar{\boldsymbol{u}}_h$为B初值,$\tilde{\boldsymbol{u}}_h$则为以该初值并用 Gauss-Seidel 迭代 $\nu_1$ 次后得到的B近似解,那么[label]误差[/label]$\boldsymbol{v}_h$和[label]残差[/label]$\boldsymbol{d}_h$可以定义为:

\begin{equation}
  \label{eq:errandres}
  \begin{aligned}
    \boldsymbol{v}_h &= \boldsymbol{u}_h - \tilde{\boldsymbol{u}}_h \\
    \boldsymbol{d}_h &= \boldsymbol{A}_h\tilde{\boldsymbol{u}}_h - \boldsymbol{f}_h
  \end{aligned}
\end{equation}

[alert type="info"]上述过程称之为[label]前光滑[/label](pre-smooth)处理。[/alert]

由此可以推得如下的[label]残差方程[/label]:

\begin{equation}
  \label{eq:reseqn}
  \left.\begin{aligned}
    \boldsymbol{d}_h &= \boldsymbol{A}_h(\boldsymbol{u}_h - \boldsymbol{v}_h) - \boldsymbol{f}_h \\
    &= - \boldsymbol{A}_h\boldsymbol{v}_h
  \end{aligned} \;\right\} \;\Rightarrow\;
  \boldsymbol{A}_h\boldsymbol{v}_h = -\boldsymbol{d}_h
\end{equation}

\subsubsection*{几何多重网格方法的通用框架}

B残差方程的重要性在于,近似解$\tilde{\boldsymbol{u}}_h$和残差$\boldsymbol{d}_h$是可以计算得到的,那么如果能通过式(\ref{eq:reseqn})的残差方程计算得到误差的精确解$\boldsymbol{v}_h$的话,便能代回式(\ref{eq:errandres})得到原线性方程组的真解$\boldsymbol{u}_h$。但是,式(\ref{eq:reseqn})的残差方程能得到的也只是误差的B近似解$\tilde{\boldsymbol{v}}_h$,而多重网格方法的目标便是得到$\boldsymbol{v}_h$的一个更好的近似。它首先将式(\ref{eq:reseqn})的[label]细网格[/label]上的残差方程[label]限制[/label]到[label]粗网格[/label]上,即:取粗网格的网格大小为$H=2h=2/N$,并记原椭圆问题在空间步长为$H$的情况下离散后得到的系数矩阵为$\boldsymbol{A}_H$,显然$\boldsymbol{A}_H$和$\boldsymbol{A}_h$具有相同的结构,只是前者的阶数为$(N/2-1)^2\times(N/2-1)^2$。然后,多重网格方法的一个核心步骤是通过一个细网格到粗网格上的[label]限制算子[/label]$\boldsymbol{R}$,将细网格上的残差$\boldsymbol{d}_h$限制到粗网格上的$\boldsymbol{d}_H$,即:

\begin{equation}
  \label{eq:resrestrict}
  \boldsymbol{d}_H = \boldsymbol{R}\boldsymbol{d}_h
\end{equation}

于是,结合粗网格上的系数矩阵$\boldsymbol{A}_H$和残差$\boldsymbol{d}_H$,便有了粗网格上的残差方程,即:

\begin{equation}
  \label{eq:reseqncrs}
  \boldsymbol{A}_H\boldsymbol{v}_H = -\boldsymbol{d}_H
\end{equation}

如果式(\ref{eq:reseqncrs})的粗网格上的残差方程可以通过某种方法(如 Gauss 直接消去法)得到B精确解$\tilde{\boldsymbol{v}}_H$,那么就可以通过限制操作(restriction)的逆操作:[label]延拓[/label]操作(prolongation),将粗网格上的误差$\tilde{\boldsymbol{v}}_H$延拓为细网格上的误差$\tilde{\boldsymbol{v}}_h$(类似地,便有了[label]延拓算子[/label]$\boldsymbol{P}$),即:

\begin{equation}
  \label{eq:errprol}
  \tilde{\boldsymbol{v}}_h \gets P\tilde{v}_H
\end{equation}

至此,由式(\ref{eq:errprol})得到的$\tilde{\boldsymbol{v}}_h$便可以看作是式(\ref{eq:reseqn})残差方程的近似解。最后,再回到式(\ref{eq:errandres}),将其累加到原近似解$\tilde{\boldsymbol{u}}_h$上,即:

\begin{equation}
  \label{eq:uhnew}
  \tilde{\boldsymbol{u}}_h^{\mathrm{new}} = \tilde{\boldsymbol{u}}_h + \tilde{\boldsymbol{v}}_h
\end{equation}

相比原近似解$\tilde{\boldsymbol{u}}_h$,式(\ref{eq:uhnew})得到的$\tilde{\boldsymbol{u}}_h^{\mathrm{new}}$是原式(\ref{eq:5ptls})的离散线性方程组一个B更好的近似解。最后,再以$\tilde{\boldsymbol{u}}_h^{\mathrm{new}}$作为初值,对原方程式(\ref{eq:5ptls})再次使用 Gauss-Seidel 迭代$\nu_2$次后,得到一个比$\tilde{\boldsymbol{u}}_h^{\mathrm{new}}$B更好的近似解$\bar{\boldsymbol{u}}_h^{\mathrm{new}}$。

[alert type="info"]类似地,这个过程也称之为[label]后光滑[/label](post-smooth)处理。[/alert]

[alert type="success"]上述过程就是几何多重网格方法的一个通用框架。前光滑和后光滑处理过程中间的操作,统称为[label]粗网格修正[/label],它涉及残差$\boldsymbol{d}_h$计算、粗网格化、误差$\tilde{\boldsymbol{v}}_H$求解、细网格化、和近似解$\tilde{\boldsymbol{u}}_h$修正等过程。由于此过程涉及了两个网格(一个细网格和一个粗网格),因此也称为[label]$2$-层网格[/label]。由此还可以引申出$V$-型和$W$-型的多重网格(层数$>2$)。[/alert]

\subsubsection*{几何多重网格方法的算法描述}

根据上述的分析,便可以给出一个几何多重网格方法通用框架的算法描述。这里以$2$-层网格为例,对于式(\ref{eq:5ptls})给定的细网格上的方程,以及初值$\bar{\boldsymbol{u}}_h$和两个代表迭代次数的正整数$\nu_1$(前光滑次数)、$\nu_2$(后光滑次数),算法目标是得到一个比$\bar{\boldsymbol{u}}_h$更好的近似解$\bar{\boldsymbol{u}}_h^{\mathrm{new}}$。那么基于 Gauss-Seidel 迭代光滑子的几何多重网格方法算法$G(\bar{\boldsymbol{u}}_h,\nu_1,\nu_2,\bar{\boldsymbol{u}}_h^{\mathrm{new}})$可以描述如下:

%[latex mode=1]
%[+preamble]
%\usepackage[titlenotnumbered,lined,linesnumbered]{algorithm2e}
%[/preamble]
\TitleOfAlgo{Two-Grid Cycle}
\begin{algorithm}[H]
  \tcc{Pre-smooth}
  $\tilde{\boldsymbol{u}}_h \gets \textbf{\texttt{smooth}}^{\nu_1}(\boldsymbol{A}_h,\bar{\boldsymbol{u}}_h,\boldsymbol{f}_h)$ \;
  \tcc{Correction on Coarse Grid}
  $\boldsymbol{d}_h \gets \boldsymbol{A}_h\tilde{\boldsymbol{u}}_h - \boldsymbol{f}_h$
  \tcp*[r]{calc. residual}
  $\boldsymbol{d}_H \gets \boldsymbol{R}_{h}^{H}\boldsymbol{d}_h$
  \tcp*[r]{coarsen $\to \boldsymbol{A}_H$, $\boldsymbol{d}_H$}
  $\boldsymbol{A}_H\boldsymbol{v}_H = -\boldsymbol{d}_H$
  \tcp*{solve accurate $\tilde{\boldsymbol{v}}_H$}
  $\tilde{\boldsymbol{v}}_h \gets \boldsymbol{P}_{H}^{h}\tilde{\boldsymbol{v}}_H$
  \tcp*[r]{prolong $\to \boldsymbol{v}_h$}
  $\tilde{\boldsymbol{u}}_h^{\mathrm{new}} \gets \tilde{\boldsymbol{u}}_h + \tilde{\boldsymbol{v}}_h$
  \tcp*[r]{correct $\tilde{\boldsymbol{u}}_h$}
  \tcc{Post-smooth}
  $\bar{\boldsymbol{u}}_h^{\mathrm{new}} \gets \textbf{\texttt{smooth}}^{\nu_2}(\boldsymbol{A},\tilde{\boldsymbol{u}}_h^{\mathrm{new}},\boldsymbol{f}_h)$ \;
\end{algorithm}
%[/latex]

上述算法即为$2$-层网格的算法描述。进一步分析该算法,在算法第$4$行,如果此时在粗网格上无法得到精确解$\tilde{\boldsymbol{v}}_H$,那么可以在算法第$5$行处B递归调用$2$-层网格算法。更具体来说,就是使用$2$-层网格方法来求解线性方程组$\boldsymbol{A}_H\boldsymbol{v}_H=-\boldsymbol{d}_H$,而这个过程中需要使用到更粗一层($2H$)的网格,于是就得到了$3$-层网格,其中递归部分调用的算法为$G(0,\nu_1,\nu_2,\tilde{\boldsymbol{v}}_H)$。

[alert type="info"]如果重复调用$2$-层网格算法,就得到了[lable]多重网格[/label]方法,其中最粗一层的网格对应的方程往往有B很低的阶数,可以用 [label]Gauss 消去法[/label],甚至是[label]直接法[/label]求得。特别地,这种方法产生的网格也称为 [label]V-型多重网格[/label]。[/alert]

\subsubsection*{V-型和 W-型多重网格}

【Figure】

上图给出了$2\thicksim5$重的 V-型多重网格示意图。另一种类型的多重网格称之为 [label]W-型多重网格[label]。和 V-型多重网格方法的区别在于,W-型多重网格在每一次递归过程中调用$2$次多重网格方法。下图给出了$2\thicksim4$重的 W-型多重网格示意图。

【Figure】

以上图中间的$3$重 W-型多重网格为例进行分析:第$1$层即为最细的网格($h$),算法过程中需要调用$2$次$2$-层多重网格方法(对应节点$3\thicksim4$和节点$4\thicksim5$),而节点$6$和$7$对应最粗的$4h$网格。类似地,上图最后是$4$-层 W-型多重网格,也就是在最细一层网格的节点$1$和节点$2$之间,调用$2$次$3$-层 W-性多重网格方法(对应节点$3\thicksim4$和节点$4\thicksim5$),其中节点$6\thicksim11$对应$4h$网格,节点$12\thicksim15$则对应最粗的$8h$网格,且它们对应的都是B精确解。

至此,本文简单概述了几何多重网格方法的通用框架,并由此得到了两类(V-型和 W-型)多重网格。但整个过程中,还有两个问题并没有解决:

1. 初值$\bar{\boldsymbol{u}}_h$的选取;

2. 限制算子$\boldsymbol{R}_{h}^{H}$和延拓算子$\boldsymbol{P}_{H}^{h}$的计算;

事实上,第一个问题较好解决,通常可以取$\bar{\boldsymbol{u}}_h=0$;但还有一种方法是使用由粗网格相应的近似解经过延拓后得到的值,这种方法和多重网格方法结合以后,所得到的方法称之为[label]完全多重网格方法[/label],此方法将另文详述。

针对第二个问题,如上文所述,限制操作和延拓操作是多重网格方法的两个核心步骤。关于这两个算子,也将另文详述。

%[/aside_content]

\end{document}
