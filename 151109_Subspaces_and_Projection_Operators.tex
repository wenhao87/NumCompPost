\documentclass[UTF8,nofonts]{ctexart}

\setCJKmainfont[BoldFont=STHeiti,ItalicFont=STKaiti]{STKaiti}
\setCJKsansfont[BoldFont=STHeiti]{STXihei}
\setCJKmonofont{STFangsong}

\usepackage[letterpaper,left=1in,top=1in]{geometry}
\usepackage{multirow}
\usepackage{amsmath}
\usepackage{amsfonts}
\usepackage{amssymb}
\usepackage[titlenotnumbered,lined,linesnumbered]{algorithm2e}

\begin{document}

% Title
\section*{子空间与投影算子}

投影算子(Projection operators)在数值线性代数中,尤其是求解线性方程组时使用的[label]迭代法[/label]中,有着极其重要的作用。本文主要通过引入子空间的概念,并从纯代数的角度出发,讨论投影算子及其相关性质。

\subsection*{子空间与直和}

\subsubsection*{子空间和}

设$\mathcal{X}$和$\mathcal{Y}$是定义于向量空间$\mathcal{V}$中的两个[label]子空间/label],表示为$\mathcal{X}\subseteq\mathcal{V},\mathcal{X}\subseteq\mathcal{V}$。定义[label]子空间交集[/label]为:$\mathcal{X}\cap\mathcal{Y}$,它满足向量加法和标量乘法的封闭性,因此子空间交集是$\mathcal{V}$的一个子空间;定义$\mathcal{X}\cup\mathcal{Y}$为包含所有$\mathcal{X}$和$\mathcal{Y}$的向量集合,但它B不满足向量加法封闭性,因此$\mathcal{X}\cup\mathcal{Y}$并不是一个子空间。但另一方面,任何向量集的[label]扩张[/lable]都是子空间,所以$\mathcal{X}\cup\mathcal{Y}$的扩张是$\mathcal{V}$的一个子空间,因此可以定义$\mathcal{X}$和$\mathcal{Y}$的[label]子空间和[/label]为:

\[
\mathcal{X}+\mathcal{Y}\overset{\underset{\mathrm{def}}{}}{=}\mathrm{span}(\mathcal{X}\cup\mathcal{Y})=\big\{\boldsymbol{x}+\boldsymbol{y}|\boldsymbol{x}\in\mathcal{X},\boldsymbol{y}\in\mathcal{Y}\big\}
\]

indent 并且有等式$\text{dim}\mathcal{X}+\text{dim}\mathcal{Y}=\text{dim}(\mathcal{X}+\mathcal{Y})+\text{dim}(\mathcal{X}\cap\mathcal{Y})$,该关系式也称为子空间的[label]容斥定理[/label]。

\subsubsection*{补子空间}

设$\mathcal{X}$为向量空间$\mathcal{V}$的子空间,如果$\mathcal{X}\cap\mathcal{Y}=\mathcal{O}$且$\mathcal{X}+\mathcal{Y}=\mathcal{V}$,则称$\mathcal{Y}$是$\mathcal{X}$的[label]补子空间[/label](或简称[label]补空间[/label])。此时,$\text{dim}(\mathcal{X}\cap\mathcal{Y})=0$且$\text{dim}(\mathcal{X}+\mathcal{Y})=\text{dim}\mathcal{V}$。补子空间的意义在于:$\text{dim}\mathcal{X}+\text{dim}\mathcal{Y}=\text{dim}\mathcal{V}$。

\subsubsection*{直和}

如果$\mathcal{X}$和$\mathcal{Y}$是向量空间$\mathcal{V}$的子空间,且$\mathcal{Y}$是$\mathcal{X}$的补子空间,则称$\mathcal{V}$是$\mathcal{X}$与$\mathcal{Y}$的[label]直和[/label]](direct sum),记为:$\mathcal{V}=\mathcal{X}\oplus\mathcal{Y}$,并且有$\text{dim}(\mathcal{X}\oplus\mathcal{Y})=\text{dim}\mathcal{X}+\text{dim}\mathcal{Y}$。从B几何观点来说,要将向量空间$\mathcal{V}$中的向量$\boldsymbol{z}$分解为$\mathcal{X}$-成分的$\boldsymbol{x}$和$\mathcal{Y}$-成分的$\boldsymbol{y}$,其关键技术便在于[label]投影[/label]。

\subsection*{投影}

[label]投影[/lable]是剖析子空间直和的重要工具。设$\mathcal{V}=\mathcal{X}\oplus\mathcal{Y}$,对于$\boldsymbol{z}\in\mathcal{V}$,有唯一$\boldsymbol{x}\in\mathcal{X},\boldsymbol{y}\in\mathcal{Y}$,使得$\boldsymbol{z}=\boldsymbol{x}+\boldsymbol{y}$,其中$\boldsymbol{x}$为向量$\boldsymbol{z}$沿着$\mathcal{Y}$至$\mathcal{X}$的投影;而$\boldsymbol{y}$为向量$\boldsymbol{z}$沿着$\mathcal{X}$至$\mathcal{Y}$的投影。特别地,如果子空间$\mathcal{X}$[label]正交[/label]于$\mathcal{Y}$,则称之为[label]正交投影[/label],否则为[label]斜投影[/label]。以子空间$\mathcal{V}=\mathbb{R}^3=\mathcal{P}\oplus\mathcal{L}$为例,其中$\mathcal{P}$为一平面,$\mathcal{L}$为平面外一直线,如下图所示,$\boldsymbol{x}\in\mathcal{P}$就是将向量$\boldsymbol{z}$沿着与$\mathcal{L}$平行的直线投影至平面$\mathcal{P}$(斜投影)。

\subsubsection*{投影算子}

如果$\mathbb{R}^n=\mathcal{X}\oplus\mathcal{Y}$,对于任意$\boldsymbol{z}\in\mathbb{R}^n$,要计算$\boldsymbol{z}$沿着$\mathcal{Y}$至$\mathcal{X}$的[label]投影分量[/label]$\boldsymbol{x}$,需设计一个$n \times n$阶矩阵$P$,称为[label]投影算子[/label](projection operator),$P\boldsymbol{z}=\boldsymbol{x}$便是$\boldsymbol{z}$沿着$\mathcal{Y}$至$\mathcal{X}$的投影。

另一方面,由$\boldsymbol{z}=\boldsymbol{x}+\boldsymbol{y}=P\boldsymbol{z}+\boldsymbol{y}$可以推得:$\boldsymbol{y}=(I-P)\boldsymbol{z}$,其中$I-P$称为$P$的[label]补投影算子[/label]。

子空间$\mathcal{X}$和$\mathcal{Y}$与投影算子$P$和$I-P$的[label]列空间[/label]和[label]零空间[/lable]有着如下重要关系式:

\begin{eqnarray*}
& \mathcal{X}=C(P)=N(I-P) \\
& \mathcal{Y}=C(I-P)=N(P) \\
\end{eqnarray*}

\subsubsection*{正交投影算子}

在复向量空间$\mathbb{C}^n$中,[label]正交分解定理[/label]指的是任意向量$boldsymbol{z}\in\mathbb{C}^n$可唯一分解为$\boldsymbol{z}=\boldsymbol{x}+\boldsymbol{y}$,其中$\boldsymbol{x}\perp\boldsymbol{y},\boldsymbol{x}\in\mathcal{X},\boldsymbol{y}\in\mathcal{X}^{\perp}$,这里的$\mathcal{X}^{\perp}$是子空间$\mathcal{X}$的正交补集,且$\mathbb{C}^n=\mathcal{X}\oplus\mathcal{X}^\perp$。[label]正交投影矩阵[/label]$P$便是将$\mathbb{C}^n$的所有向量投影至$\mathcal{X}$:$\mathcal{X}=\{P\boldsymbol{x}|\boldsymbol{x}\in\mathbb{C}^n\}=C(P)$,且$\mathcal{X}^\perp=\{\boldsymbol{x}\in\mathbb{C}^n|P\boldsymbol{x}=0\}=N(P)$。

令$\mathcal{X}$是$\mathbb{C}^n$的一个子空间,且$P$是子空间$\mathcal{X}$的正交投影矩阵(正交投影算子),则$P$满足如下性质:

- 对于所有$\boldsymbol{x}\in\mathcal{X}$,有:$P\boldsymbol{x}=\boldsymbol{x}$;

- 对于所有$\boldsymbol{x}\in\mathbb{C}^n$,有:$(\boldsymbol{x}-P\boldsymbol{x})\in\mathcal{X}^\perp$;

indent 上述两个性质的意义在于:1)子空间$\mathcal{X}$中任何向量的投影为其本身;2)正交投影后的[label]残差[/label]正交于投影子空间。

最后,给出正交投影矩阵的主要界定条件:

- 若$P$为一正交投影矩阵,则$P=P^2=P^*$,反之亦然;

- 若$P$为一正交投影矩阵,则对于任意$\boldsymbol{x}\in\mathbb{C}^n$,有:$\|P\boldsymbol{x}\|_2\leq\|\boldsymbol{x}\|_2$。


%\begin{eqnarray*}
%& M=\dfrac{}
%\end{eqnarray*}

% H2, H3
\subsection*{dsfsdt}
%[latex mode=1]
%[+preamble]
%\usepackage[titlenotnumbered,lined,linesnumbered]{algorithm2e}
\let\oldnl\nl
\newcommand{\nonl}{\renewcommand{\nl}{\let\nl\oldnl}}
%[/preamble]
%\quicklatex{size=14}
\TitleOfAlgo{BLAS-3 Gaussian Elimination with Partial Pivoting}
\begin{algorithm}[H]
\For{$l\gets 1$ \KwTo $n-1$}{
$k\gets (l-1)b+1$\;
factorize $PA^{(l)}=LU$ using BLAS-2\;
store $L$ and $U$ in $A$\;
left multiply $A(1:n,k+b:n)$ by $P$\;
$A(k:k+b-1,k+b:n)\gets T^{-1}A(k:k+b-1,k+b:n)$\;
\nonl\Indp where $T$ is the unit-lower-trian of $A(k:k+b-1,k:k+b-1)$\;
\DontPrintSemicolon\Indm$A(k+b:n,k+b:n)\gets A(k+b:n,k+b:n)$\;
\PrintSemicolon\nonl\Indp$-A(k+b:n,k:k+b-1)A(k:k+b-1,k+b:n)$\;
}
\end{algorithm}
%[/latex]

% H4
% \subsubsection*{}

\end{document}
