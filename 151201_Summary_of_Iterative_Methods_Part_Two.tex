\documentclass[UTF8,nofonts]{ctexart}

\setCJKmainfont[BoldFont=STHeiti,ItalicFont=STKaiti]{STKaiti}
\setCJKsansfont[BoldFont=STHeiti]{STXihei}
\setCJKmonofont{STFangsong}

\usepackage[letterpaper,left=1in,top=1in]{geometry}
\usepackage{multirow}
\usepackage{amsmath}
\usepackage{amsfonts}
\usepackage{amssymb}
\usepackage[titlenotnumbered,lined,linesnumbered]{algorithm2e}

\begin{document}

% Title

\section*{迭代法整理(二):最小二乘问题和基于Givens变换的QR分解}

上文提到(见数值线性代数中的迭代法整理(一)一文),[label]Krylov方法[/label]的每一步都是在第$k$个Krylov子空间内寻找最优的$\boldsymbol{x}^k$,而其本质便是求解一个[label]最小二乘问题[/label]。通常,将此类Krylov子空间法称为广义极小残差(GMRES)方法。事实上,广义极小残差(GMRES)方法是在共轭梯度(CG)方法之后提出的。根据Arnoldi迭代方法的特性,最小二乘问题的[label]QR分解[/label]可以使用Givens平面旋转变换实现。上文最后提到,Arnoldi迭代方法中的搜索方向$\boldsymbol{q}_i$满足如下递推表达式:

\[
A\boldsymbol{q}_k=\sum_{i=1}^{k+1}h_{ik}\boldsymbol{q}_i\quad k=1,\cdots,n-1
\]

indent它也等价于:$AQ_k=Q_{k+1}H_k$,其中$H_k$为$(k+1) \times k$阶的[label]上Hessenberg矩阵[/label]。利用上Hessenberg矩阵的特殊结构可以在第$k$步迭代时,实现$\mathcal{O}(k)$量级的QR分解,其矩阵表示形式为:

\[
H_k=
\begin{pmatrix}
h_{11} & h_{12} & \cdots & \cdots & h_{1k} \\
h_{21} & h_{22} & \cdots & \cdots & \vdots \\
& h_{32} & \ddots & & \vdots \\
& & \ddots & \ddots & \vdots \\
& & & h_{k,k-1} & h_{kk} \\
& & & & h_{k+1,k}
\end{pmatrix}
\]

\subsection*{QR分解和最小二乘问题}

\subsubsection*{最小二乘问题}

最小二乘问题通常描述为:

\begin{equation}
\label{eq:lsq}
\boldsymbol{x}=\text{argmin}_{\boldsymbol{y}}\|\boldsymbol{b}-A\boldsymbol{y}\|_2
\end{equation}

indent其中,$A\in\mathbb{R}^{m \times n}$,且$m \geq n$。式(\ref{eq:lsq})的求解有很多种方法,其中较为常用的一种便是QR分解,即:$A=Q\hat{R}$,其中$QQ^T=I$,而$\hat{R}$是一个上三角矩阵,它们的矩阵表示形式为:

\begin{equation}
	\label{eq:qrmat}
	\begin{aligned}
		Q&=
		\begin{pmatrix}
			\hat{\boldsymbol{q}}_1 & \hat{\boldsymbol{q}}_2 & \cdots & \hat{\boldsymbol{q}}_m
		\end{pmatrix}
		\in\mathbb{R}^{m \times m} \\
		\hat{R}&=\begin{pmatrix}R \\ 0\end{pmatrix}\in\mathbb{R}^{m \times n}
	\end{aligned} \quad
	R=
	\begin{pmatrix}
		r_{11} & r_{12} & \cdots & r_{1n} \\
		& r_{22} & \cdots &	r_{2n} \\
		& & \ddots & \vdots \\
		& & & r_{nn} \\
	\end{pmatrix}\in\mathbb{R}^{n \times n}
\end{equation}

indent于是,利用正交变换的[label]$2$-范数[/label]不变的性质,结合式(\ref{eq:qrmat}),式(\ref{eq:lsq})可以改写为:

\begin{equation}
\label{eq:lsqprocd}
\begin{aligned}
\boldsymbol{x} &=
\text{argmin}_{\boldsymbol{y}}\|Q^T(\boldsymbol{b}-A\boldsymbol{y})\|_2=
\text{argmin}_{\boldsymbol{y}}\|Q^T\boldsymbol{b}-Q^TA\boldsymbol{y}\|_2 \\
&= \text{argmin}_{\boldsymbol{y}}\|Q^T\boldsymbol{b}-\hat{R}\boldsymbol{y}\|_2 \\
\|Q^T\boldsymbol{b}-\hat{R}\boldsymbol{y}\|_2 &=
\left\|
\begin{pmatrix}
	\hat{\boldsymbol{q}}_1^T\boldsymbol{b} \\ \vdots \\
	\hat{\boldsymbol{q}}_n^T\boldsymbol{b} \\ \cline{1-1}
	\hat{\boldsymbol{q}}_{n+1}^T\boldsymbol{b} \\ \vdots \\
	\hat{\boldsymbol{q}}_m^T\boldsymbol{b} \\
\end{pmatrix}-
\begin{pmatrix}
	~ \\ R\boldsymbol{y} \\ ~ \\ \cline{1-1}
	~ \\ 0 \\ ~
\end{pmatrix}
\right\|_2=
\sqrt{
	\left\|
	\begin{pmatrix}
		\hat{\boldsymbol{q}}_{n+1}^T\boldsymbol{b} \\
		\vdots \\
		\hat{\boldsymbol{q}}_{m}^T\boldsymbol{b} \\
	\end{pmatrix}
	\right\|_2^2 +
	\left\|
		\begin{pmatrix}
			\hat{\boldsymbol{q}}_1^T\boldsymbol{b} \\
			\vdots \\
			\hat{\boldsymbol{q}}_n^T\boldsymbol{b} \\
		\end{pmatrix}-R\boldsymbol{y}
	\right\|_2^2
}
\end{aligned}
\end{equation}

indent因此,如果令$\hat{\boldsymbol{b}}=\left(\hat{\boldsymbol{q}}_1^T\boldsymbol{b},\hat{\boldsymbol{q}}_2^T\boldsymbol{b},\cdots,\hat{\boldsymbol{q}}_n^T\boldsymbol{b}\right)^T$,则式(\ref{eq:lsq})的最小二乘问题也就等价于:

\begin{equation}
\label{eq:lsqnew}
\boldsymbol{x}=\text{argmin}_{\boldsymbol{y}}\|\hat{\boldsymbol{b}}-R\boldsymbol{y}\|_2
\end{equation}

indent而显然,式(\ref{eq:lsqnew})的最小二乘解即为$\boldsymbol{y}=R^{-1}\hat{\boldsymbol{b}}$,又由于$R$是上三角矩阵,因此很容易通过回代法求得最小二乘解$\boldsymbol{y}$。尽管如此,计算一个给定矩阵的QR分解代价还是很大的,而通过Arnoldi迭代法则可以极大程度上地减少QR分解的计算量。

\subsubsection*{Arnoldi方法的QR分解}

上文提到(见数值线性代数中的迭代法整理(一)一文),Krylov方法的第$k$步迭代的解$\boldsymbol{x}_k$所对应的便是求解以下最小二乘问题:

\begin{equation}
\label{eq:wk}
\boldsymbol{x}_k=Q_k\boldsymbol{w}_k,\quad\text{where}\quad\boldsymbol{w}_k=\text{argmin}_{\boldsymbol{w}}\|\boldsymbol{b}-AQ_k\boldsymbol{w}\|_2
\end{equation}

indent而结合Arnoldi迭代方法的特性,即:$AQ_k=Q_{k+1}H_{k}$,则式(\ref{eq:wk})可以改写为:

\begin{equation}
\label{eq:wknew}
\boldsymbol{w}_k=\text{argmin}_{\boldsymbol{w}}\left\|\boldsymbol{b}-Q_{k+1}H_k\boldsymbol{w}\right\|_2
\end{equation}

indent于是,仿照式(\ref{eq:lsqprocd})的过程,再次利用正交变换$2$-范数的不变性,并令$Q_n=\left(Q_{k+1},\boldsymbol{q}_{k+2},\cdots,\boldsymbol{q}_n\right)$,则有:

\[
\begin{aligned}
\left\|\boldsymbol{b}-Q_{k+1}H_k\boldsymbol{w}\right\|_2 &=
\left\|Q_n^T(\boldsymbol{b}-Q_{k+1}H_k\boldsymbol{w})\right\|_2=
\left\|Q_n^T\boldsymbol{b}-Q_n^TQ_{k+1}H_k\boldsymbol{w}\right\|_2 \\ &=
\left\|Q_n^T\boldsymbol{b}-\begin{pmatrix}I_{k+1}\\0\end{pmatrix}H_k\boldsymbol{w}\right\|_2=
\left\|Q_n^T\boldsymbol{b}-\begin{pmatrix}H_k\boldsymbol{w}\\0\end{pmatrix}\right\|_2
\end{aligned}
\]

indent所以,式(\ref{eq:wknew})的最小二乘问题也就等价于:

\begin{equation}
\label{eq:wkfinal}
\boldsymbol{w}_k=\text{argmin}\left\|Q_{k+1}\boldsymbol{b}-H_k\boldsymbol{w}\right\|_2
\end{equation}

indent显然,式(\ref{eq:wkfinal})的最小二乘问题比原始的式(\ref{eq:wk})的最小二乘问题要简单许多。一来,$(k+1) \times k$阶上Hessenberg矩阵$H_k$大大降低了矩阵的维数;二来,上Hessenberg十分接近于上三角矩阵,所以很容易通过Householder或者Givens变换来计算其QR分解形式;最后更重要的是,观察式(\ref{eq:hk1}),在$H_k$的最后新增一列后便得到了$H_{k+1}$,所以两者具有极其相似的结构,因此它们的QR分解形式也十分相似。

\begin{equation}
\label{eq:hk1}
H_{k+1}=
\begin{pmatrix}
	~ & ~ & ~ & h_{1,k+1} \\
	~ & H_k & ~ & \vdots \\
	~ & ~ & ~ & h_{k+1,k+1} \\
	~ & 0 & ~ & h_{k+2,k+1} \\
\end{pmatrix}
\end{equation}

\subsection*{基于Givens平面旋转变换的QR分解}

[label]Givens平面旋转变换[/label]是一种有效的矩阵清零方法,例如在矩阵$A$的QR分解中将其转换为上三角矩阵$R$。Givens矩阵$G(i,j,\theta)$的矩阵形式如下,其作用是将向量$\boldsymbol{x}$中的第$j$个分量置$0$(具体见详解基本GMRES方法及其变形一文):

\[
G(i,j,\theta)\boldsymbol{x}=
\begin{pmatrix}
	1&&&&&&\\&\ddots&&&&&\\
	&&c&\cdots&s&&\\
	&&\vdots&\ddots&\vdots&&&\\
	&&-s&\cdots&c&&\\
	&&&&&\ddots&\\&&&&&&1
\end{pmatrix}
\begin{pmatrix}
	x_1\\\vdots\\x_i\\\vdots\\x_j\\\vdots\\x_n
\end{pmatrix}=
\begin{pmatrix}
	x_1\\\vdots\\cx_i+sx_j\\\vdots\\-sx_i+cx_j\\\vdots\\x_n
\end{pmatrix}=
\begin{pmatrix}
	x_1\\\vdots\\x'_i\\\vdots\\0\\\vdots\\x_n
\end{pmatrix}
\]

indent其中,$c=\cos\theta$,$s=\sin\theta$。那么根据下面的关系式,便可得到$c$和$s$以及$x'_i$的值:

\begin{equation*}
\left.
	\begin{aligned} c^2+s^2&=1 \\ -sx_i+cx_j&=0 \end{aligned} \quad
\right\}\quad\Longrightarrow\quad
\begin{aligned}
	c&=\dfrac{x_i}{\sqrt{x_i^2+x_j^2}} \\
	s&=\dfrac{x_j}{\sqrt{x_i^2+x_j^2}} \\
\end{aligned}\quad,\quad
x'_i=\sqrt{x_i^2+x_j^2}
\end{equation*}

前面提到,由于$H_{k}$和$H_{k+1}$在结构上具有相似性,因此,在第$k$步$H_k$的QR分解形式上,很容易根据Givens平面旋转变换得到下一步迭代$H_{k+1}$的QR分解形式。具体分析如下:考虑如下$H_k$和$H_{k+1}$的矩阵表示形式:

\begin{equation}
\label{eq:hkhk1}
H_{k}=
\begin{pmatrix}
	h_{11} & \cdots & \cdots & h_{1k} \\
	h_{21} & \ddots & & \vdots \\
	& \ddots & \ddots & \vdots \\
	& & \ddots & h_{kk} \\
	& & & h_{k+1,k}
\end{pmatrix}\quad
H_{k+1}=
\begin{pmatrix}
	h_{11} & \cdots & \cdots & h_{1k} & h_{1,k+1}\\
	h_{21} & \ddots & & \vdots & \vdots \\
	& \ddots & \ddots & \vdots & \vdots \\
	& & \ddots & h_{kk} & h_{k,k+1} \\
	& & & h_{k+1,k} & h_{k+1,k+1} \\
	& & & & h_{k+2,k+1} \\
\end{pmatrix}
\end{equation}

假设第$k$步迭代时已通过连续$k$个Givens变换将$H_{k}$转换为了上三角形式,即:

\[
G(k+1,k,\theta_k) \cdots G(3,2,\theta_2) \cdot G(2,1,\theta_1) \cdot H_{k}=
\begin{pmatrix}
	h_{11}^{(k)} & \cdots & \cdots & h_{1k}^{(k)} \\
	0 & \ddots & & \vdots \\
	& \ddots & \ddots & \vdots \\
	& & \ddots & h_{kk}^{(k)} \\
	& & & 0
\end{pmatrix}
\]

indent而在第$k+1$步时,得到了新的一列$(h_{1,k+1},\cdots,h_{k+2,k+1})^T$,因此需要先将第$k$步的$k$个Givens矩阵作用于该列上,得到:

\begin{equation}
\label{eq:gk1}
G(k+1,k,\theta_k) \cdots G(3,2,\theta_2) \cdot G(2,1,\theta_1) \cdot
\begin{pmatrix}
	h_{1,k+1} \\ h_{2,k+1} \\ \vdots \\ h_{k+1,k+1} \\ h_{k+2,k+1}
\end{pmatrix} =
\begin{pmatrix}
	h'_{1,k+1} \\ h'_{2,k+1} \\ \vdots \\ h'_{k+1,k+1} \\ h_{k+2,k+1}
\end{pmatrix}
\end{equation}

indent然后构造$G(k+2,k+1,\theta_{k+1})$,使得式(\ref{eq:gk1})中的末元$h_{k+2,k+1}$置$0$,即:

\begin{equation}
\label{eq:gk2}
\begin{aligned}
& G(k+2,k+1,\theta_{k+1}) \cdot
\begin{pmatrix}
	h'_{1,k+1} \\ h'_{2,k+1} \\ \vdots \\ h'_{k+1,k+1} \\ h_{k+2,k+1}
\end{pmatrix} \\ =&
\begin{pmatrix}
	1 & & & & \\
	& \ddots & & & \\
	& & \ddots & & \\
	& & & c & s \\
	& & & -s & c \\
\end{pmatrix}
\begin{pmatrix}
	h'_{1,k+1} \\ h'_{2,k+1} \\ \vdots \\ h'_{k+1,k+1} \\ h_{k+2,k+1}
\end{pmatrix} =
\begin{pmatrix}
	h'_{1,k+1} \\ h'_{2,k+1} \\ \vdots \\ h''_{k+1,k+1} \\ 0
\end{pmatrix} =
\begin{pmatrix}
	h^{(k+1)}_{1,k+1} \\ h^{(k+1)}_{2,k+1} \\ \vdots \\ h^{(k+1)}_{k+1,k+1} \\ 0
\end{pmatrix}
\end{aligned}
\end{equation}

indent因此归纳来说,假设第$k$步迭代时已经得到$k$个能将$H_k$转换为上三角形式的Givens矩阵,那么当第$k+1$步迭代时,得到了$H_{k+1}$,而它只是比$H_k$多了最末一列,所以先要将上一步的$k$个Givens矩阵作用于$H_{k+1}$的最末一列,然后构造第$k+1$个Givens矩阵用来消减$H_{k+1}$最后一列的最末元,并且由于第$k+1$个Givens矩阵不影响$H_k$中已经上三角化的$k$列,因此末元消减后的结果也就是$H_{k+1}$的QR分解结果。

[alert type="success"]式(\ref{eq:gk1})和式(\ref{eq:gk2})说明第$k$步迭代时$H_{k}$的QR分解只需要针对其最后一列的$k$个元素进行更新即可,其计算复杂度为$\mathcal{O}(k)$。因此也可以说,$H_k$的QR分解复杂度也就是$\mathcal{O}(k)$。[/alert]

%[/aside_content]


% H2, H3
% \subsection*{dsfsdt}
%[latex mode=1]
%[+preamble]
%\usepackage[titlenotnumbered,lined,linesnumbered]{algorithm2e}
\let\oldnl\nl
\newcommand{\nonl}{\renewcommand{\nl}{\let\nl\oldnl}}
%[/preamble]
%\quicklatex{size=14}
\TitleOfAlgo{Jacobi Method}
\begin{algorithm}[H]
\For{$k\gets1,\cdots$}{
	$\boldsymbol{r}\gets\boldsymbol{b}-A\boldsymbol{x}$\;
	\For{$i\gets 1$ \KwTo $n$}{
		$\delta\gets r_i/A_{ii}$\;
		$x_i\gets x_i+\delta$\;
	}
}
\end{algorithm}
%[/latex]

% H4
% \subsubsection*{}

\end{document}
