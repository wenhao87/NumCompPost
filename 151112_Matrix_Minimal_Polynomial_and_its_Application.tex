\documentclass[UTF8,nofonts]{ctexart}

\setCJKmainfont[BoldFont=STHeiti,ItalicFont=STKaiti]{STKaiti}
\setCJKsansfont[BoldFont=STHeiti]{STXihei}
\setCJKmonofont{STFangsong}

\usepackage[letterpaper,left=1in,top=1in]{geometry}
\usepackage{multirow}
\usepackage{amsmath}
\usepackage{amsfonts}
\usepackage{amssymb}
\usepackage[titlenotnumbered,lined,linesnumbered]{algorithm2e}

\begin{document}

% Title

\subsection*{最小多项式}

上文(矩阵的特征多项式和Cayley-Hamilton定理)中提到了矩阵$A$的两个特殊多项式:[label]特征多项式[/label](characteristic polynomial)和[label]最小多项式[/label](minimal polynomial)。前者特征多项式$p_A(t)=\det(tI-A)$可以用来B消减矩阵,即:$p_A(A)=0$,该结论也称为[label]Cayley-Hamilton定理[/label]。本文将重点讨论后者:最小多项式。事实上,它也可以用于消减矩阵。

\subsection*{最小多项式}

对于$n \times n$阶矩阵$A$,在所有满足$m(A)=0$的多项式$m(t)$中,其中[label]次数[/label](均指[label]最高项次数[/label])最小的那个,便称为[label]最小多项式[/label]。显然,其次数不大于$n$,且对于任意$c$,都有$cm(A)=0$,则完全可以限制该多项式最高次项的系数为$1$,称为[label]首一多项式[/label](monic polynomial)。

\subsubsection*{最小多项式的唯一性}

因为$m(t)$的次数不大于$p_A(t)$的次数,则根据[label]欧几里得算法[/label],有多项式$h(t)$和$r(t)$满足:

\begin{equation}
\label{eq:ecld}
p_A(t)=m(t)h(t)+r(t)
\end{equation}

indent其中[label]余式[/label]$r(t)$的次数小于$m(t)$。由于$p_A(t)$和$m(t)$都是$A$的消减多项式,那么代入$A$后,要使等式成立,则必须有$r(A)=0$。如果$r(t)$不是[label]零多项式[/label](所有系数为零)的话,也就意味着存在一个次数比$m(t)$还小的多项式,这与$m(t)$次数最小相矛盾。所以,可以推知$r(t)$是零多项式,也就是说$m(t)$整除$p_A(t)$。

另外,如果存在两个次数相同的首一多项式都可以消减$A$,则它们可以彼此整除,可推知两者完全相同。这也就证明了最小次数的首一消减多项式B唯一存在,也称为最小多项式,记作$m_A(t)$。

[alert type="info"]相似矩阵具有相同特征值,也就有相同的特征多项式,以及相同的最小多项式。[/alert]

\subsubsection*{最小多项式和特征多项式的关系}

根据上面的讨论,有:最小多项式$m_A(t)$整除特征多项式$p_A(t)$,那么显然,$m_A(t)$的根同时也是$p_A(t)$的根。下面将证明,反过来$p_A(t)$的根也是$m_A(t)$的根。

设矩阵$A$有特征值$\lambda$,对应的[label]特征向量[/label]为$\boldsymbol{x}$,其首一最小多项式为:$m_A(t)=t^k+a_{k-1}t^{k-1}+\cdots+a_1t+a_0$。由性质$A^r\boldsymbol{x}=\lambda^r\boldsymbol{x}$($r \geq 1$),可以推得:

\begin{align*}
m_A(A)\boldsymbol{x}
&=(A^{k}+a_{k-1}A^{k-1}+\cdots+a_1A+a_0I)\boldsymbol{x} \\
&=(\lambda^k+a_{k-1}\lambda^{k-1}+\cdots+a_1\lambda+a_0)\boldsymbol{x} \\
&=m_A(\lambda)\boldsymbol{x}
\end{align*}

indent由于$m_A(A)=0$,则可以推得$m_A(\lambda)=0$,其含义便是矩阵$A$的特征多项式的根(即:特征值$\lambda$),同时也是其最小多项式的根。那么,如上文所述(见矩阵的特征多项式和Cayley-Hamilton定理一文),根据代数定理,$A$的特征多项式可以表示为:

\begin{equation}
\label{eq:chpoly}
p_A(t)=(t-\lambda_1)^{\beta_1}\cdots(t-\lambda_m)^{\beta_m}
\end{equation}

indent其中$m$为相异的特征值个数,且$\lambda_i$的[label]相重数[/label]为$\beta_i$($i=1,\cdots,m$,$\beta_1+\cdots+\beta_m=n$)。而$A$的最小多项式也必定具有以下形态:

\begin{equation}
\label{eq:mmpoly}
m_A(t)=(t-\lambda_1)^{r_1}\cdots(t-\lambda_m)^{r_m},\quad
1 \leq r_i \leq \beta_i
\end{equation}

\subsubsection*{最小多项式实例}

以下面的$4 \times 4$阶矩阵$A$为例:

\[
A=\begin{bmatrix}2&1&0&0\\0&2&0&0\\0&0&1&1\\0&0&-2&4\end{bmatrix}
\]

indent其特征多项式为$p_A(t)=(t-3)(t-2)^3$,则其最小多项式$m_A(t)=(t-3)^{r_1}(t-2)^{r_2}$有最小次数$r_1+r_2$使得$m_A(A)=0$。因此,$(r_1,r_2)$可能组合有$(1,1),(1,2),(1,3)$,分别代入后发现$m_A(A)=(A-3I)^{1}(A-2I)^{1} \neq 0$,而$m_A(A)=(A-3I)^{1}(A-2I)^{2}=0$,所以$A$的最小多项式为$m_A(t)=(t-3)(t-2)^2$。

[alert type="error"]注意:上面的方法需要实现求得矩阵的特征多项式,一般只适用于低阶矩阵。而通过[label]循环子空间[/label],则可以不用求特征多项式。基于循环子空间的最小多项式计算将另文介绍。[/alert]

\subsection*{最小多项式与Jordan典型形式}

事实上,矩阵的最小多项式和[label]Jordan典型形式[/label]有着密切的关系。简单来说,根据最小多项式能判断矩Jordan形式的结构,而Jordan形式反过来也可以计算最小多项式。

上文提到,两个相似矩阵具有相同的特征多项式和最小多项式,因此,考虑$n \times n$阶矩阵$A$的Jordan形式:$A=MJM^{-1}$,其中$J$为[label]Jordan矩阵[/label],可以推得:$p_A{t}=p_J(t)$,$m_A{t}=m_J(t)$。

在矩阵的三种标准分解一文中,给出了Jordan矩阵$J$的一般形式。设$J$有$m$个相异特征值$\lambda_1,\cdots,\lambda_m$,其中$\lambda_i$的[label]代数重数[/label]为$\beta_i$,[label]指标[/label]为$r_i$。Jordan的典型形式是:对应于特征值$\lambda_i$的所有[/label]基本Jordan分块[/label]的[label]中,最大的分块阶数为$r_i$,它们的直和[/label]为一个超级Jordan分块,且其阶数为$\beta_i$。$J$的特征多项式和最小多项式可以表示为:

\begin{align*}
p_J(t)&=p_A(t)=(t-\lambda_1)^{\beta_1}\cdots(t-\lambda_i)^{\beta_i}\cdots(t-\lambda_m)^{\beta_m} \\
m_J(t)&=m_A(t)=(t-\lambda_1)^{r_1}\cdots(t-\lambda_i)^{r_i}\cdots(t-\lambda_m)^{r_m}
\end{align*}

以下面的Jordan矩阵$J$为例:

\[J=
\begin{bmatrix}7&1&0&&&&&\\0&7&1&&&&&\\0&0&7&&&&&\\&&&7&1&&&\\&&&0&7&&&\\&&&&&5&1&\\&&&&&0&5&\\&&&&&&&5\end{bmatrix}=
\begin{bmatrix}7&1&0\\0&7&1\\0&0&7\end{bmatrix}\oplus
\begin{bmatrix}7&1\\0&7\end{bmatrix}\oplus
\begin{bmatrix}5\end{bmatrix}\oplus
\begin{bmatrix}3&1\\0&3\end{bmatrix}\oplus
\begin{bmatrix}3&1\\0&3\end{bmatrix}
\]

indent其中,特征值$7$对应的代数重数为$5$,指标为$3$,对应的特征多项式和最小多项式分别为:

\begin{align*}
p_J(t)&=p_A(t)=(t-7)^{5}(t-5)(t-3)^{4} \\
m_J(t)&=m_A(t)=(t-7)^{3}(t-5)(t-3)^{2}
\end{align*}

可以看到,经由矩阵$A$的Jordan形式$J$可得到最小多项式,但Jordan形式的计算往往比最小多项式的计算更困难。较之更关键的是,从最小多项式的形式,可以判定矩阵$A$是否为可对角化矩阵。如果每一个Jordan分块的阶数都为$1$,即$\lambda_i$的指标$r_i=1$,其等价于每个特征值的[label]几何重数[/label]和代数重数相等,于是$A$有完整的$n$个线性无关特征向量,可以对角化。

% H2, H3
\subsection*{dsfsdt}
%[latex mode=1]
%[+preamble]
%\usepackage[titlenotnumbered,lined,linesnumbered]{algorithm2e}
\let\oldnl\nl
\newcommand{\nonl}{\renewcommand{\nl}{\let\nl\oldnl}}
%[/preamble]
%\quicklatex{size=14}
\TitleOfAlgo{BLAS-3 Gaussian Elimination with Partial Pivoting}
\begin{algorithm}[H]
\For{$l\gets 1$ \KwTo $n-1$}{
$k\gets (l-1)b+1$\;
factorize $PA^{(l)}=LU$ using BLAS-2\;
store $L$ and $U$ in $A$\;
left multiply $A(1:n,k+b:n)$ by $P$\;
$A(k:k+b-1,k+b:n)\gets T^{-1}A(k:k+b-1,k+b:n)$\;
\nonl\Indp where $T$ is the unit-lower-trian of $A(k:k+b-1,k:k+b-1)$\;
\DontPrintSemicolon\Indm$A(k+b:n,k+b:n)\gets A(k+b:n,k+b:n)$\;
\PrintSemicolon\nonl\Indp$-A(k+b:n,k:k+b-1)A(k:k+b-1,k+b:n)$\;
}
\end{algorithm}
%[/latex]

% H4
% \subsubsection*{}

\end{document}
