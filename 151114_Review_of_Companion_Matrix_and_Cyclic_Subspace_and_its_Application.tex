\documentclass[UTF8,nofonts]{ctexart}

\setCJKmainfont[BoldFont=STHeiti,ItalicFont=STKaiti]{STKaiti}
\setCJKsansfont[BoldFont=STHeiti]{STXihei}
\setCJKmonofont{STFangsong}

\usepackage[letterpaper,left=1in,top=1in]{geometry}
\usepackage{multirow}
\usepackage{amsmath}
\usepackage{amsfonts}
\usepackage{amssymb}
\usepackage[titlenotnumbered,lined,linesnumbered]{algorithm2e}

\begin{document}

% Title

\section*{详解循环子空间及其相关特性}

在上文(见详解不变子空间和限定算子表示矩阵一文)中,介绍了一种特殊的[label]不变子空间[/label]:[label]循环子空间[/label],并且在文末给出了基于[label]循环向量[/label]$\boldsymbol{v}$的循环子空间$\mathcal{X}$,在[lable]限定算子[/label]$T_{/\mathcal{X}_{\boldsymbol{v}}}$上关于基底向量$\mathfrak{B}_{\boldsymbol{v}}$的表示矩阵:$\big(T_{/\mathcal{X}_{\boldsymbol{v}}}\big)_{\mathfrak{B}_{\boldsymbol{v}}}$,而它其实也是一个多项式的[label]相伴矩阵[/label]。本文将从相伴矩阵切入,详细讨论循环子空间,并利用循环子空间证明[lable]Cayley-Hamilton定理[/label]。

\subsection*{相伴矩阵}

给定一个$n \times n$阶矩阵$A$,在矩阵的最小多项式及其应用一文中指出,[label]最小多项式[/label]$m_A(t)$为次数(指最高次项)最小的[label]首一消减多项式[/label](最高此项系数为$1$),即:$m_A(A)=0$。反过来,给定任意首一多项式$p(t)$:

\[
p(A)=t^n+a_{n-1}t^{n-1}+\cdots+a_1t+a_0
\]

indent事实上总能找到一个方阵,使其最小多项式恰好为该多项式$p(t)$,这个方阵便称为[lable]相伴矩阵[/label](companion matrix),且其形式为:

\begin{equation}
\label{eq:cmat}
C=\begin{pmatrix}
0&0&\cdots&0&-a_0\\
1&0&\cdots&0&-a_1\\
\vdots&\ddots&\ddots&\vdots&\vdots\\
0&\cdots&1&0&-a_{n-2}\\
0&0&\cdots&1&-a_{n-1}
\end{pmatrix}
\end{equation}

\subsubsection*{消减多项式的证明}

如果$p(t)$是$C$的[label]消减多项式[/label],即:$p(C)=C^n+a_{n-1}C^{n-1}+\cdots+a_1C+a_)=0$,其证明思路是:证明对$\boldsymbol{e}_i,i=1,\cdots,n$,有$p(C)\boldsymbol{e}_i=0$,其中$\boldsymbol{e}_i$是第$i$元为$1$的[label]标准单位向量[/label]。利用$C$的结构性质,可以得到:

\begin{align}
\label{eq:cei}
\begin{split}
C^{0}\boldsymbol{e}_1 &= \boldsymbol{e}_1,\quad C^{1}\boldsymbol{e}_1=\boldsymbol{e}_2,\quad C^{2}\boldsymbol{e}_1=C\boldsymbol{e}_2=\boldsymbol{e}_3,\quad C^{3}\boldsymbol{e}_1= C\boldsymbol{e}_3=\boldsymbol{e}_4\\
C^{n-1}\boldsymbol{e}_1 &= C\boldsymbol{e}_{n-1}=\boldsymbol{e}_n,\quad C^{n}\boldsymbol{e}_1=C\boldsymbol{e}_n=
-a_0\boldsymbol{e}_1-a_1\boldsymbol{e}_2-\cdots-a_{n-1}\boldsymbol{e}_n
\end{split}
\end{align}

indent也就是说,存在关系式$C^{i-1}\boldsymbol{e}_1=C\boldsymbol{e}_{i}$。结合式(\ref{eq:cmat}),则式(\ref{eq:cei})的最末式右端可以改写为如下形式:

\[C^{n}\boldsymbol{e}_1=-a_0\boldsymbol{e}_1-a_1C\boldsymbol{e}_1-\cdots-a_{n-1}C^{n-1}\boldsymbol{e}_1=\big(C^n-p(C)\big)\boldsymbol{e}_1\]

indent因此,可以推知$p(C)\boldsymbol{e}_1=0$。进一步,对于所有$i=1,\cdots,n$,有:

\[p(C)\boldsymbol{e}_i=p(C)C^{i-1}\boldsymbol{e}_1=C^{i-1}\big(p(C)\boldsymbol{e}_1\big)=0\]

\subsubsection*{最小多项式的证明}

可以利用反证法进行证明,其思路是:如果存在次数$m<n$的首一多项式,即:

\[q(t)=t^m+b_{m-1}t^{m-1}+\cdots+b_1t+b_0,\quad m<n\]

indent也能消减$C$,则$q(C)=0$。观察$q(C)\boldsymbol{e}_1$,并利用关系式$C^{i-1}\boldsymbol{e}_1=\boldsymbol{e}_{i}$,可以得到:

\begin{align*}
\label{eq:qc}
q(C)\boldsymbol{e}_1 &=
C^m\boldsymbol{e}_1+b_{m-1}C^{m-1}\boldsymbol{e}_1+\cdots+b_1C\boldsymbol{e}_1+b_0\boldsymbol{e}_1 \\
&= \boldsymbol{e}_{m+1}+b_{m-1}\boldsymbol{e}_m+\cdots+b_1\boldsymbol{e}_2+b_0\boldsymbol{e}_1=0
\end{align*}

indent式(\ref{eq:qc})的含义是$\{\boldsymbol{e}_1,\cdots,\boldsymbol{e}_m,\boldsymbol{e}_{m+1}\}$是一个[label]线性相关[/label]向量集,而对于所有的$\boldsymbol{e}_i\in\mathbb{C}^n$,是不可能线性相关的,所以说$p(t)$便是相伴矩阵$C$的最小多项式。另一方面,$C$的特征多项式是$n$次的消减多项式,所以$p(t)$同时也是$C$的特征多项式。

\subsection*{循环子空间}

上文(见详解不变子空间和限定算子表示矩阵一文)中提出了不变子空间的概念,令$\mathcal{V}$为一[label]向量空间[/label],$T:\mathcal{V}\to\mathcal{V}$是一个线性变换([label]线性算子[/label])。如果$T$将$\mathcal{V}$的一个[label]子空间[/label]$\mathcal{X}\subseteq\mathcal{V}$映射至另一个子空间$T(\mathcal{X})=\{T(\boldsymbol{x})|\boldsymbol{x}\in\mathcal{X}\}$,且满足$T(\mathcal{X})\subseteq\mathcal{X}$,也可以描述为$\forall\boldsymbol{x},T(\boldsymbol{x})\in\mathcal{X}$,则称$\mathcal{X}$是$T$的[label]不变子空间[/label](invariant subspace),标记为:$T_{/\mathcal{X}}:\mathcal{X}\to\mathcal{X}$。正如上文所提到的,不变子空间可以用来简化线性算子[label]表示矩阵[/label]。

上文还介绍了一种特殊的[label]不变子空间[/label]:[label]循环子空间[/label]。设非零向量$\boldsymbol{x}$,在线性算子$T$下张成如下的循环子空间具有如下形式:

\begin{equation}
\label{eq:cysub}
\mathcal{X}=\text{span}\{\boldsymbol{x},T\boldsymbol{x},T^2\boldsymbol{x},\cdots\}
\end{equation}

indent特别地,如果$\mathcal{X}=\mathcal{V}$,则称$\boldsymbol{x}$为[label]循环向量[/label](cyclic vector)。设$k$是使向量集$\boldsymbol{\beta}_{\mathcal{X}}=\{\boldsymbol{x},T\boldsymbol{x},\cdots,T^{k-1}\boldsymbol{x}\}$为[label]线性无关[/label]向量集的B最小整数,则存在$a_0,a_1,\cdots,a_{k-1}$,满足:

\begin{equation}
\label{eq:atk}
a_0\boldsymbol{x}+a_1T\boldsymbol{x}+\cdots+a_{k-1}T^{k-1}\boldsymbol{x}+T^k\boldsymbol{x}=0
\end{equation}

indent并且,上式连续左乘$T$得到的$T^m\boldsymbol{x}$(m>k),都能写成$\boldsymbol{x},T\boldsymbol{x},\cdots,T^{k-1}\boldsymbol{x}$的[label]线性组合[/label],所以$\boldsymbol{\beta}_{\mathcal{X}}$便是循环子空间$\mathcal{X}$的一组[label]基底[/label]。于是有:

\[
\begin{cases}
T^i\boldsymbol{x} &= 1 \cdot T^i\boldsymbol{x},\quad i=0,1,\cdots,k-1\\
T^k\boldsymbol{x} &= -a_0\boldsymbol{x}-a_1T\boldsymbol{x}-\cdots-a_{k-1}T^{k-1}\boldsymbol{x}\\
\end{cases}
\]

indent则限定算子$T_{/\mathcal{X}}$关于基底$\boldsymbol{\beta}_{\mathcal{X}}$的表示矩阵具有如下形式:

\[
\Big(T_{/\mathcal{X}}\Big)_{\boldsymbol{\beta}_{\mathcal{X}}}=
\begin{bmatrix}
0&0&\cdots&0&-a_0\\
1&0&\cdots&0&-a_1\\
\vdots&\ddots&\ddots&\vdots&\vdots\\
0&\cdots&1&0&-a_{k-2}\\
0&0&\cdots&1&-a_{k-1}
\end{bmatrix}
\]

indent可以看到,它具有和式(\ref{eq:cmat})的相伴矩阵一样的形式,只不过阶数变成了$k times k$阶,也就是说,限定算子表示矩阵是多项式$p(t)=t^k+a_{k-1}t^{k-1}+\cdots+a_1t+a_0$的相伴矩阵。

\begin{equation}
\label{eq:cmatpoly}
p(t)=t^k+a_{k-1}t^{k-1}+\cdots+a_1t+a_0
\end{equation}

indent更重要的是,该多项式也是限定算子$T_{/\mathcal{X}}$的[label]特征多项式[/label]。

\subsection*{利用循环子空间证明Cayley-Hamilton定理}

在矩阵的特征多项式和Cayley-Hamilton定理一文中给出了[label]Cayley-Hamilton定理[/label]。它可以表述为:如果$p(t)$为矩阵A的特征多项式,则矩阵$A$替换$t$所得到的[label]矩阵多项式[label]满足$p(A)=0$。这里,利用循环子空间,可以证明Cayley-Hamilton定理。证明的思路是,对于任意向量$\boldsymbol{x}\in\mathcal{V}$,证明式(\ref{eq:cmatpoly})的矩阵多项式$p(T)$满足$p(T)\boldsymbol{x}=0$,换言之,$p(T)$的[label]零空间[/label]$N(p(T))$充满整个向量空间$\mathcal{V}$。

设$\mathcal{X}$为非零向量$\boldsymbol{x}$在线性算子$T$下张成的循环子空间,且$\text{dim}\mathcal{X}=k$,向量集$\mathfrak{B}_{\mathcal{X}}=\{\boldsymbol{x},T\boldsymbol{x},\cdots,T^{k-1}\boldsymbol{x}\}$为循环子空间$\mathcal{X}$的基底。如果将$\mathfrak{B}_{\mathcal{X}}$的$k$个基底向量扩充为整个向量空间$\mathcal{V}$的基底,则有:

\[
\mathfrak{B}=\{\boldsymbol{x},T\boldsymbol{x},T^{k-1}\boldsymbol{x},\boldsymbol{y}_1,\boldsymbol{y}_2,\cdots,\boldsymbol{y}_{n-k}\}
\]

indent则$T$关于$\mathfrak{B}$的表示矩阵具有如下分块上三角形式(推导见详解不变子空间和限定算子表示矩阵一文):

\[
\Big(T\Big)_{\mathfrak{B}}=
\begin{pmatrix}B & C \\ 0 & D\end{pmatrix}
\]

indent其中,分块$B$为限定算子表示矩阵$\big(T_{/\mathcal{X}}\big)_{\mathfrak{B}_{\mathcal{X]}}}$。根据特征多项式的定义,线性算子$T$的特征多项式即为主对角分块$B$和$D$的特征多项式的乘积:$p(t)|_{\mathcal{V}}=p(t)|_{B}p(t)|_D=p(t)|_D \cdot p(t)|_{B}$,其中:

\begin{align*}
p(t)|_B&=p(t)|_{T_{/\mathcal{X}}}=t^k+a_{k-1}t^{k-1}+\cdots+a_1t+a_0 \\
p(T)|_B\cdot\boldsymbol{x}&=T^k\boldsymbol{x}+a_{k-1}T^{k-1}\boldsymbol{x}+\cdots+a_1T\boldsymbol{x}+a_0\boldsymbol{x}=0
\end{align*}

indent所以,最后可以推得:$p(T)\boldsymbol{x}=p(T)|_D \cdot p(T)|_B\cdot\boldsymbol{x}=p(T)|_D \cdot 0=0$。














% H2, H3
\subsection*{dsfsdt}
%[latex mode=1]
%[+preamble]
%\usepackage[titlenotnumbered,lined,linesnumbered]{algorithm2e}
\let\oldnl\nl
\newcommand{\nonl}{\renewcommand{\nl}{\let\nl\oldnl}}
%[/preamble]
%\quicklatex{size=14}
\TitleOfAlgo{BLAS-3 Gaussian Elimination with Partial Pivoting}
\begin{algorithm}[H]
\For{$l\gets 1$ \KwTo $n-1$}{
$k\gets (l-1)b+1$\;
factorize $PA^{(l)}=LU$ using BLAS-2\;
store $L$ and $U$ in $A$\;
left multiply $A(1:n,k+b:n)$ by $P$\;
$A(k:k+b-1,k+b:n)\gets T^{-1}A(k:k+b-1,k+b:n)$\;
\nonl\Indp where $T$ is the unit-lower-trian of $A(k:k+b-1,k:k+b-1)$\;
\DontPrintSemicolon\Indm$A(k+b:n,k+b:n)\gets A(k+b:n,k+b:n)$\;
\PrintSemicolon\nonl\Indp$-A(k+b:n,k:k+b-1)A(k:k+b-1,k+b:n)$\;
}
\end{algorithm}
%[/latex]

% H4
% \subsubsection*{}

\end{document}
