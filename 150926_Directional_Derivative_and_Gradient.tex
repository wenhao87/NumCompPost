\documentclass[UTF8,nofonts]{ctexart}

\setCJKmainfont[BoldFont=STHeiti,ItalicFont=STKaiti]{STKaiti}
\setCJKsansfont[BoldFont=STHeiti]{STXihei}
\setCJKmonofont{STFangsong}

\usepackage[letterpaper,left=1in,top=1in]{geometry}
\usepackage{multirow}
\usepackage{amsmath}
\usepackage{amsfonts}
\usepackage{amssymb}
\usepackage[titlenotnumbered,lined,linesnumbered]{algorithm2e}

\begin{document}

% Title

\section*{共轭残差(CR)方法}

\subsection*{方向导数}

[label]多元函数[label]的[label]偏导数[/label]反映了函数值沿B坐标轴方向的变化率,但许多实际问题中往往还需要掌握函数在某点处沿某一B特定方向的变化率,这就引出了[label]方向导数[label]的概念。为了便于刻画方向,需要先引入方向余弦的概念。设$\boldsymbol{l}$是一个$n$维非零向量,令$\boldsymbol{l}_0=\boldsymbol{l}/\|\boldsymbol{l}\|$,则$\boldsymbol{l}_0$是与$\boldsymbol{l}$同向的[label]单位向量[/label]。取$0\leq\alpha_i\leq\pi$,使$\boldsymbol{l}_0=(\cos\alpha_1,\cdots,\cos\alpha_n)$满足$\cos^2\alpha_1+\cdots+\cos^2\alpha_n=1$,则称$\cos\alpha_1,\cdots,\cos\alpha_n$为向量$\boldsymbol{l}$的方向余弦。

这里以二元函数为例,事实上它可以推广到多元函数。设$z=f(x,y)$为一个曲面,$P(x_0,y_0)$为定义域$f$内的一个点,$\boldsymbol{l}$是$xOy$平面上以$P(x_0,y_0)$为始点的一条射线,单位向量$\boldsymbol{l}_0=(\cos\alpha,\cos\beta)$与$\boldsymbol{l}$同方向,其中$\cos\alpha$和$\cos\beta$为$\boldsymbol{l}_0$的方向余弦。考虑$\boldsymbol{l}$方向的斜率。作一个通过点$P$,且平行于$\boldsymbol{l}$方向的垂直平面,该垂直平面与曲面$z=f(x,y)$相交于曲线$C$。而曲面在点$\big(x_0,y_0,f(x_0,y_0)\big)$上沿着$\boldsymbol{l}$方向的斜率就是曲线$C$在点$\big(x_0,y_0,f(x_0,y_0)\big)$的斜率。连接点$P(x_0,y_0)$与点$Q(x,y)$所得到的直线$L$可以用如下的参数式表示:

\begin{equation}
\label{eq:xy}
\begin{cases}
	& x  = x_0+t\cos\alpha \\
	& y  = y_0+t\cos\beta \\
\end{cases}
\end{equation}

indent其中,$t$为任意常数,$Q(x,y)$则是直线$L$上的任意一点。另外,点$P$和点$Q$之间的距离可以表示为:

\begin{equation}
\label{eq:pqdist}
\sqrt{(x-x_0)^2+(y-y_0)^2}=\sqrt{(t\cos\alpha)^2+(t\cos\beta)^2}=|t|
\end{equation}

indent结合式(\ref{eq:xy})和式(\ref{eq:pqdist}),则连接点$P$和点$Q$的直线$L$在曲面$C$上的切线斜率可以表示为:

\begin{equation}
\label{eq:tan}
\dfrac{f(x,y)-f(x_0,y_0)}{t}=
\dfrac{f(x_0+t\cos\alpha,y_0+t\cos\beta)-f(x_0,y_0)}{t}
\end{equation}

indent而当式(\ref{eq:tan})中的$t$趋近于$0$时,便能得到如下的[label]方向导数[/label](directional derivative)的定义:设$f(x,y)$为一个二元函数,$\boldsymbol{l}_0=(\cos\alpha,\cos\beta)$为一单位向量,如果式(\ref{eq:tan})的极限值存在,则$f$在点$(x_0,y_0)$处沿着$\boldsymbol{l}$方向的方向导数即为该极限值,记为:

\begin{equation}
\label{eq:dddef}
\begin{aligned}
\left.D_{\boldsymbol{l}}f\right|_{(x_0,y_0)}=\left.\dfrac{\partial f}{\partial \boldsymbol{l}}\right|_{(x_0,y_0)} &= \lim_{t\to0}\dfrac{f(x_0+t\cos\alpha,y_0+t\cos\beta)-f(x_0,y_0)}{t} \\
&= \lim_{\substack{\Delta x\to0\\\Delta y\to0}}\dfrac{f(x_0+\Delta x,y_0+\Delta y)-f(x_0,y_0)}{\sqrt{\Delta x^2+\Delta y^2}}
\end{aligned}
\end{equation}

[alert type="info"]事实上,式(\ref{eq:tan})便是方向导数的B几何意义:割线(平面与曲面的交线)在曲面对应点上关于$\boldsymbol{l}$方向的切线(若存在)的斜率,即为方向导数:$\tan\varphi=\partial f/\partial \boldsymbol{l}$。[/alert]

最后,关于方向导数有如下一个B重要定理:如果函数$z=f(x,y)$在点$P(x_0,y_0)$处[label]可微[/label],则函数在该点沿任意方向$\boldsymbol{l}_0$的方向导数都存在,且有:

\begin{equation}
\label{eq:ddfinal}
\left.\dfrac{\partial f}{\partial \boldsymbol{l}}\right|_{(x_0,y_0)}=f_x(x_0,y_0)\cos\alpha+f_y(x_0,y_0)\cos\beta
\end{equation}

indent其中,$\cos\alpha$和$\cos\beta$为单位向量$\boldsymbol{l}_0$的方向余弦。理由如下:

\begin{equation*}
\begin{aligned}
\Delta f &= f_x(x_0,y_0)\Delta x + f_y(x_0,y_0)\Delta y+o(t) \\
&= t\cdot\big[f_x(x_0,y_0)\cos\alpha+f_y(x_0,y_0)\cos\beta\big]+o(t) \\
\quad\Longrightarrow\quad
&\left.\dfrac{\partial f}{\partial\boldsymbol{l}}\right|_{(x_0,y_0)}=\lim_{t\to0}\dfrac{\Delta f}{t}=f_x(x_0,y_0)\cos\alpha+f_y(x_0,y_0)\cos\beta
\end{aligned}
\end{equation*}

\subsection*{梯度}

设二元函数$z=f(x,y)$在点$P(x_0,y_0)$具有[label]偏导数[/label],则称向量:

\begin{equation}
\big\{f_x(x_0,y_0),f_y(x_0,y_0)\big\}
\end{equation}

indent为函数$z=f(x,y)$在点$P$处的梯度(gradient),记作:

\begin{equation}
\label{eq:graddef}
\nabla f=\text{grad}f=\left.\left(\dfrac{\partial f}{\partial x}\boldsymbol{i}+\dfrac{\partial f}{\partial y}\boldsymbol{j}\right)\right|_{(x_0,y_0)}=\left.\left(\dfrac{\partial f}{\partial x},\dfrac{\partial f}{\partial y}\right)\right|_{(x_0,y_0)}
\end{equation}

indent显然,结合式(\ref{eq:ddfinal})和式(\ref{eq:graddef}),便可以得到方向导数和梯度的关系式,即:

\begin{equation}
D_{\boldsymbol{l}}f(x,y)=\nabla f(x,y)\cdot \boldsymbol{l}_0=\|\nabla f(x,y)\|\cdot\|\boldsymbol{l}_0\|\cdot\cos\theta
\end{equation}

indent其中$\theta$为梯度$\nabla f(x,y)$和单位向量$\boldsymbol{l}_0$之间的夹角。显然,当$\cos\theta=1$时($\theta=0$),$D_{\boldsymbol{l}}f(x,y)$取到最大值,此时$\boldsymbol{l}_0$和梯度$\nabla f(x,y)$方向相同;而当$\cos\theta=-1$时($\theta=\pi$),$D_{\boldsymbol{l}}f(x,y)$取到最小值,此时$\boldsymbol{l}_0$和梯度$\nabla f(x,y)$方向相反,即负梯度方向。所以,梯度的几何意义在于:就方向而言,它朝着使得函数$f(x,y)$在$P(x_0,y_0)$处有B最大方向导数的那个方向,此时函数值增加最快;就长度而言,其长度为那个最大方向导数的值。

\subsection*{小结}

\subsubsection*{方向导数计算公式}

- 二元函数$f(x,y)$在点$P(x,y)$沿方向$\boldsymbol{l}$(方向角为$\alpha,\beta$)的方向导数为:

\[
\dfrac{\partial f}{\partial \boldsymbol{l}}=\dfrac{\partial f}{\partial x}\cos\alpha+\dfrac{\partial f}{\partial y}\cos\beta
\]

- 三元函数$f(x,y,z)$在点$P(x,y,z)$沿方向$\boldsymbol{l}$(方向角为$\alpha,\beta,\gamma$)的方向导数为:

\[
\dfrac{\partial f}{\partial \boldsymbol{l}}=\dfrac{\partial f}{\partial x}\cos\alpha+\dfrac{\partial f}{\partial y}\cos\beta+\dfrac{\partial f}{\partial z}\cos\gamma
\]

\subsubsection*{梯度}

- 二元函数$f(x,y)$在点$P(x,y)$处的梯度为

\[\nabla f=\text{grad}f=\left(\dfrac{\partial f}{\partial x},\dfrac{\partial f}{\partial y}\right)\]

- 三元函数$f(x,y,z)$在点$P(x,y,z)$处的梯度为

\[\nabla f=\text{grad}f=\left(\dfrac{\partial f}{\partial x},\dfrac{\partial f}{\partial y},\dfrac{\partial f}{\partial z}\right)\]

\subsubsection*{关系}

\[\dfrac{\partial f}{\partial \boldsymbol{l}}=\nabla f \cdot\boldsymbol{l}_0\]


%[aside_content]


% H2, H3
\subsection*{dsfsdt}
%[latex mode=1]
%[+preamble]
%\usepackage[titlenotnumbered,lined,linesnumbered]{algorithm2e}
\let\oldnl\nl
\newcommand{\nonl}{\renewcommand{\nl}{\let\nl\oldnl}}
%[/preamble]
%\quicklatex{size=14}
\TitleOfAlgo{BLAS-3 Gaussian Elimination with Partial Pivoting}
\begin{algorithm}[H]
\For{$l\gets 1$ \KwTo $n-1$}{
$k\gets (l-1)b+1$\;
factorize $PA^{(l)}=LU$ using BLAS-2\;
store $L$ and $U$ in $A$\;
left multiply $A(1:n,k+b:n)$ by $P$\;
$A(k:k+b-1,k+b:n)\gets T^{-1}A(k:k+b-1,k+b:n)$\;
\nonl\Indp where $T$ is the unit-lower-trian of $A(k:k+b-1,k:k+b-1)$\;
\DontPrintSemicolon\Indm$A(k+b:n,k+b:n)\gets A(k+b:n,k+b:n)$\;
\PrintSemicolon\nonl\Indp$-A(k+b:n,k:k+b-1)A(k:k+b-1,k+b:n)$\;
}
\end{algorithm}
%[/latex]

% H4
% \subsubsection*{}

\end{document}
