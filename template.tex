\documentclass{article}

\usepackage[letterpaper,left=1in,top=1in]{geometry}

\usepackage{verbatim}
\usepackage{multirow}
\usepackage{amsmath}
\usepackage{amsfonts}
\usepackage{amssymb}
\usepackage[titlenotnumbered,lined,linesnumbered]{algorithm2e}

\begin{document}

\tableofcontents

\begin{align}
a_0&=b_0\,, &b_0&=c_0\,, &c_0&=d_0\,,\\ a_1&=b_1\,, &b_1&=c_1\,, &c_1&=d_1\,,\\ a_2&=b_2\,, &b_2&=c_2\,, &c_2&=d_2\,.
\end{align}

\[ n! =
 \left\{
   \begin{aligned}
      & 1               && \text{if $n \leq 1$}\,, & 12\\
      & n \times (n-1)! && \text{otherwise}\,. & 123123
   \end{aligned}
 \right. \]

\[\binom{100}{safas}\]

... Using matrices the linear transformation $\langle\,x,y\,\rangle
\mapsto
\langle\,2 x + y, 3 y\,\rangle$
is written as follows:
$\bigl[\begin{smallmatrix} 2&1 \\ 0&3 \end{smallmatrix}
\bigr]
\bigl[\begin{smallmatrix} x \\ y \end{smallmatrix}
\bigr]$.
You probably knew this already. ...

% array stretch

\begin{equation}
\renewcommand\arraystretch{1.5}
\begin{array}{lcl}
\nu_k=\dfrac{(\boldsymbol{r}_k,\boldsymbol{r}_k)}{(A\boldsymbol{d}_k,\boldsymbol{d}_k)} & \to &
\boldsymbol{x}_{k+1}=\boldsymbol{x}_k+\nu_k\boldsymbol{d}_k \\
\omega_k=\dfrac{(\boldsymbol{r}_{k+1},\boldsymbol{r}_{k+1})}{(\boldsymbol{r}_k,\boldsymbol{r}_k)} & \to &
\boldsymbol{d}_{k+1}=\boldsymbol{r}_{k+1}+\omega_k\boldsymbol{d}_k
\end{array}
\end{equation}

\begin{eqnarray*}
    &x + (y - 2)^4 = w^2 \\
    &(x  - 9)^2 + y_2^2 = w^2 \\
    &x + (y - 2)^4 = w^2 = (x  - 9)^2 + y_2^2 \\
\end{eqnarray*}

\begin{align}
\label{eq:trisys}
\begin{split}
a_{i-1}x_{i-2}+b_{i-1}x_{i-1}+c_{i-1}x_{i} &= f_{i-1} \\
a_ix_{i-1}+b_ix_i+c_ix_{i+1} &= f_i \\
a_{i+1}x_{i}+b_{i+1}x_{i+1}+c_{i+1}x_{i+2} &= f_{i+1}
\end{split}
\end{align}

% \hline, \multicolumn, \multirow, \xrightarrow
\addcontentsline{toc}{subsection}{Use of \textbackslash hline, \textbackslash multicol, \textbackslash multirow, \textbackslash xrightarrow }
\subsection*{Use of \textbackslash hline, \textbackslash multicol, \textbackslash multirow, \textbackslash xrightarrow }

\begin{align*}
A=\left(\begin{array}{cc|cc|cc}
	\ast&\ast&\ast&\ast&\ast&\ast\\
	\ast&\ast&\ast&\ast&\ast&\ast\\
	\ast&\ast&\ast&\ast&\ast&\ast\\
	\ast&\ast&\ast&\ast&\ast&\ast\\
	\ast&\ast&\ast&\ast&\ast&\ast\\
	\ast&\ast&\ast&\ast&\ast&\ast
\end{array}\right)
=&\left(\begin{array}{c|c|c}
	\multirow{6}{*}{$A_1$} & \multirow{6}{*}{$A_2$} & \multirow{6}{*}{$A_3$} \\
	&&\\&&\\&&\\&&\\&&
\end{array}\right)\\
\xrightarrow[\text{get }L_1,U_1\text{ with }P_1A=L_1U_1]{\text{find }P_1\text{ with }b=2\text{ times permt.}}
&\left(\begin{array}{cc|cc|cc}
	\ast'&\ast'&\ast'&\ast'&\ast'&\ast' \\
	l_{21}&\ast'&\ast'&\ast'&\ast'&\ast' \\ \hline
	l_{31}&l_{32}&\ast'&\ast'&\ast'&\ast' \\
	l_{41}&l_{42}&\ast'&\ast'&\ast'&\ast' \\
	l_{51}&l_{52}&\ast'&\ast'&\ast'&\ast' \\
	l_{61}&l_{62}&\ast'&\ast'&\ast'&\ast'
\end{array}\right)\\
=&\left(\begin{array}{cc|cccc}
	&U_1&\ast'&\ast'&\ast'&\ast'\\
	L_{11}&&\ast'&\ast'&\ast'&\ast'\\ \hline
	\multicolumn{2}{c|}{\multirow{4}{*}{$L_{21}$}}&\ast'&\ast'&\ast'&\ast'\\
	\multicolumn{2}{c|}{}&\ast'&\ast'&\ast'&\ast'\\
	\multicolumn{2}{c|}{}&\ast'&\ast'&\ast'&\ast'\\
	\multicolumn{2}{c|}{}&\ast'&\ast'&\ast'&\ast'\\
\end{array}\right)
=\left(\begin{array}{cc|cccc}
	&U_1&\multicolumn{4}{c}{\multirow{2}{*}{$A_{1s1}$}}\\
	L_{11}&&\quad&\quad&\quad&\quad\\ \hline
	\multicolumn{2}{c|}{\multirow{4}{*}{$L_{21}$}}&\multicolumn{4}{c}{\multirow{4}{*}{$A_{1s2}$}}\\
	\multicolumn{2}{c|}{}&\multicolumn{4}{c}{}\\
	\multicolumn{2}{c|}{}&\multicolumn{4}{c}{}\\
	\multicolumn{2}{c|}{}&\multicolumn{4}{c}{}\\
\end{array}\right)\\
\xrightarrow[\text{to get }A^{(1)}]{\text{update }A_{1s1},A_{1s2}}&
\left(\begin{array}{cc|cccc}
	&U_1&\multicolumn{4}{c}{\multirow{2}{*}{$A'_{1s1}\gets L_{11}^{-1}A_{1s1}$}}\\
	L_{11}&&\quad&\quad&\quad&\quad\\ \hline
	\multicolumn{2}{c|}{\multirow{4}{*}{$L_{21}$}}&\multicolumn{4}{c}{\multirow{4}{*}{$A'_{1s2}\gets A_{1s2}-L_{21}A'_{1s1}$}}\\
	\multicolumn{2}{c|}{}&\multicolumn{4}{c}{}\\
	\multicolumn{2}{c|}{}&\multicolumn{4}{c}{}\\
	\multicolumn{2}{c|}{}&\multicolumn{4}{c}{}\\
\end{array}\right)=A^{(1)}
\end{align*}

% simple algorithm
\addcontentsline{toc}{subsection}{Simple algorithm with \texttt{algorithm2e}}
\subsection*{Simple algorithm with \texttt{algorithm2e}}
%[latex mode=1]
%[+preamble]
%\usepackage[titlenotnumbered,lined,linesnumbered]{algorithm2e}
\let\oldnl\nl
\newcommand{\nonl}{\renewcommand{\nl}{\let\nl\oldnl}}
%[/preamble]
\TitleOfAlgo{BLAS-3 Gaussian Elimination with Partial Pivoting}
\begin{algorithm}[H]
\For{$l\gets 1$ \KwTo $n-1$}{
$k\gets (l-1)b+1$\;
factorize $PA^{(l)}=LU$ using BLAS-2\;
store $L$ and $U$ in $A$\;
left multiply $A(1:n,k+b:n)$ by $P$\;
$A(k:k+b-1,k+b:n)\gets T^{-1}A(k:k+b-1,k+b:n)$\;
\nonl\Indp where $T$ is the unit-lower-trian of $A(k:k+b-1,k:k+b-1)$\;
\DontPrintSemicolon\Indm$A(k+b:n,k+b:n)\gets A(k+b:n,k+b:n)$\;
\PrintSemicolon\nonl\Indp$-A(k+b:n,k:k+b-1)A(k:k+b-1,k+b:n)$\;
}
\end{algorithm}
%[/latex]


% algorithm
\addcontentsline{toc}{subsection}{Complex algorithm with \texttt{algorithm2e}}
\subsection*{Complex algorithm with \texttt{algorithm2e}}

\IncMargin{1em}
\TitleOfAlgo{How to write algorithms}
\begin{algorithm}[H]
\KwData{this text}
\KwResult{how to write algorithm with \LaTeX2e }
\SetKwData{Left}{left}\SetKwData{This}{this}\SetKwData{Up}{up}
\SetKwFunction{Union}{Union}\SetKwFunction{FindCompress}{FindCompress}
\SetKwInOut{Input}{input}\SetKwInOut{Output}{output}
\Input{A bitmap $Im$ of size $w\times l$}
\Output{A partition of the bitmap}
\BlankLine
\emph{special treatment of the first line}\;
\For{$i\leftarrow 2$ \KwTo $l$}{
	\emph{special treatment of the first element of line $i$}\;
	\For{$j\leftarrow 2$ \KwTo $w$}{
		\label{forins}
		\Left$\leftarrow$ \FindCompress{$Im[i,j-1]$}\;
		\Up$\leftarrow$ \FindCompress{$Im[i-1,]$}\;
		\This$\leftarrow$ \FindCompress{$Im[i,j]$}\;
		\If(\tcp*[h]{O(\Left,\This)==1}){\Left compatible with \This}{
			\label{lt}
        	\lIf{\Left $<$ \This}{\Union{\Left,\This}}\;
			\lElse{\Union{\This,\Left}\;}
		}
		\If(\tcp*[f]{O(\Up,\This)==1}){\Up compatible with \This}{
			\label{ut}
			\lIf{\Up $<$ \This}{\Union{\Up,\This}}\;
			\tcp{\This is put under \Up to keep tree as flat as possible}
			\label{cmt}
			\lElse{\Union{\This,\Up}}\tcp*[r]{\This linked to \Up}
			\label{lelse}
		}
	}
	\lForEach{element $e$ of the line $i$}{\FindCompress{p}}
}
\label{algo_disjdecomp}
\end{algorithm}
\DecMargin{1em}

\end{document}
