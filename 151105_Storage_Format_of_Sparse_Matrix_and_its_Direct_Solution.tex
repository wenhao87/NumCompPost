\documentclass[UTF8,nofonts]{ctexart}

\setCJKmainfont[BoldFont=STHeiti,ItalicFont=STKaiti]{STKaiti}
\setCJKsansfont[BoldFont=STHeiti]{STXihei}
\setCJKmonofont{STFangsong}

\usepackage[letterpaper,left=1in,top=1in]{geometry}
\usepackage{multirow}
\usepackage{amsmath}
\usepackage{amsfonts}
\usepackage{amssymb}
\usepackage[titlenotnumbered,lined,linesnumbered]{algorithm2e}

\begin{document}

% Title
\section*{稀疏矩阵的存储格式和方程组的直接解法}

若矩阵$A\in\mathbb{R}^{n \times n}$中很少一部分元素非零(如非零元素个数为$O(n)$),则称其为[label]稀疏矩阵[/label],而其对应的线性方程组称为[label]稀疏线性方程组[/label]。本文主要介绍稀疏矩阵的[label]存储格式[/label],以及稀疏矩阵方程组的[label]直接解法[/label]。后续将重点讨论大型稀疏矩阵的[label]LU分解[/label]和[label]QR分解[/label]。

\subsection*{稀疏矩阵的存储格式}

考虑$5 \times  5$阶稀疏矩阵$A$:

\[
A=
\begin{pmatrix}
a_{11} & 0 & a_{13} & 0 & 0 \\
a_{21} & a_{22} & 0 & a_{24} & 0 \\
0 & a_{32} & a_{33} & 0 & a_{35} \\
0 & a_{42} & 0 & a_{44} & 0 \\
0 & 0 & 0 & a_{54} & a_{55}
\end{pmatrix}
\]

indent 使用数组$\boldsymbol{a}$存放[label]非零元[/label];数组$\boldsymbol{ia}$存放每一行B第一个非零元在数组$\boldsymbol{a}$中的位置,且其最后一个元素减$1$为非零元的B个数;数组$\boldsymbol{ja}$存放每个非零元的[label]列标[/label]。因此,对于上述矩阵$A$,其存储格式为:

\begin{align*}
\boldsymbol{a} &= [a_{11},a_{13},a_{21},a_{22},a_{24},a_{32},a_{33},a_{35},a_{42},a_{44},a_{54},a_{55}] \\
\boldsymbol{ia} &= [1,3,6,9,11,13] \\
\boldsymbol{ja} &= [1,3,1,2,4,2,3,5,2,4,4,5]
\end{align*}

indent 另外,为了避免因为添加非零元(消元过程中产生的“[label]填入[/label]”的非零元)带来的B额外数组操作,通常先估计稀疏矩阵每行B最多非零元个数,并以此生成矩阵$\boldsymbol{a}$和$\boldsymbol{ja}$,并在插入完所有非零元后,再对它们进行B压缩。因此,上面的例子中矩阵$\boldsymbol{a}$和$\boldsymbol{ja}$最初的大小为$3 \times 5$。

另一方面,对于这种形式的存储方式,不同的访问方式效率也是不同的。按行访问的话,效率很高,因为$A$的第$i$行即为$\boldsymbol{a}[\boldsymbol{ia}[i]]$和$\boldsymbol{a}[\boldsymbol{ia}[i+1]-1]$之间的元素;反之,按列访问的话,效率则会很低,所以通常会先计算矩阵$A$的[label]转置矩阵[/label]的存储格式。

给定矩阵$n \times n$阶矩阵$A=(a_{ij})$,将$A$的第$i$行第一个非零元的[label]列指标[/label]和第$j$列的第一个非零元的行指标分别记为$f_i$和$l_i$:

\begin{align*}
f_i=f_i(A)=&\min\{j|a_{ij} \neq 0\} \\
l_i=l_i(A)=&\min\{i|a_{ij} \neq 0\}
\end{align*}

indent 以及$A$的[label]包[/label]定义为[label]指标集合[/label]:

\[
Env(A)=\Big\{(i,j)|f_i \leq j \leq i\text{ or }l_j \leq i < j\Big\}
\]

indent 关于包的一个重要性质是:若$A$可以分解为$A=LU$,其中$L$为单位下三角矩阵,$U$为上三角矩阵,则有:

\[Env(L+U)=Env(A)\]

\subsection*{稀疏方程组的直接解法}

[label]Gauss消去法[/label]和[label]直接三角分解法[/label]是解大型稀疏线性方程组的有效方法。需要注意的是,消去过程中的每一步都有可能产生B新的非零元,称之为“填入”(Fill-in),为此需要尽可能B最小化填入量,并把每一步填入的量定义为[label]局部填入量[/label]。

\subsubsection*{预处理法}


考察如下两个矩阵$M$和$N$的第一步消去:

\[
M^{(1)}=\begin{pmatrix}
\ast & \ast & \ast & \ast & \ast \\
\ast & \ast & & & \\
\ast & & \ast & & \\
\ast & & & \ast & \\
\ast & & & & \ast
\end{pmatrix}\xrightarrow{k=1}
M^{(2)}=\begin{pmatrix}
\ast & \ast & \ast & \ast & \ast \\
& \ast & \ast & \ast & \ast \\
& \ast & \ast & \ast & \ast \\
& \ast & \ast & \ast & \ast \\
& \ast & \ast & \ast & \ast
\end{pmatrix}
\]

\[
N^{(1)}=\begin{pmatrix}
\ast & & & & \ast \\
& \ast & & & \ast \\
& & \ast & & \ast \\
& & & \ast & \ast \\
\ast & \ast & \ast & \ast & \ast
\end{pmatrix}\xrightarrow{k=1}
N^{(2)}=\begin{pmatrix}
\ast & & & & \ast \\
& \ast & & & \ast \\
& & \ast & & \ast \\
& & & \ast & \ast \\
& \ast & \ast & \ast & \ast
\end{pmatrix}
\]

可以发现,$M$的消元产生了很多填入,而$N$则没有。由于$N=PMP^T$,其中$P$为[label]排列矩阵[/label],也就是说适当地对$M$进行[label]行列重排[/label]能减少填入量,B包也变得尽可能小。

另一种减少填入量的方法是将矩阵$A$转换为[label]块下三角[/label]或[label]块上三角[/label]矩阵。如找到两个排列矩阵$P$和$Q$,使得:

\[
PAQ=U\equiv\begin{pmatrix}
U_{11} & U_{12} & \cdots & U_{1p} \\
& U_{22} & \cdots & U_{2p} \\
& & \ddots & \vdots \\
& & & U_{pp}
\end{pmatrix}
\]

indent 则方程组$A\boldsymbol{x}=\boldsymbol{b}$,令$Q\boldsymbol{y}=\boldsymbol{x}$,则其等价于:

\[PA(Q\boldsymbol{y})=PA\boldsymbol{x}=P\boldsymbol{b}=\boldsymbol{c}=(PAQ)\boldsymbol{y}=U\boldsymbol{y}\]

indent 换言之,由$P\boldsymbol{b}=\boldsymbol{c}$和$U\boldsymbol{y}=\boldsymbol{c}$,计算得到$\boldsymbol{y}$,再由$Q\boldsymbol{y}=\boldsymbol{x}$计算得到$\boldsymbol{x}$,其中:

\[
U_{ii}\boldsymbol{y}_i=\boldsymbol{c}_i-\sum_{j=i+1}^{p}U_{ij}y_{j},\quad i=p,p-1,\cdots,1
\]

\subsubsection*{主元策略}

上述两种方法发生在消元过程B前,也就是所谓的[label]预处理法[/label],而[label]主元策略[/label]发生在消元B过程中,通过适当地选取主元使得每步消元的填入最小。设完成第$k-1$次消元后得到矩阵$A^{(k)}$,及其右下角$(n-k+1)\times(n-k+1)$阶子矩阵$A_{22}^{(k)}$:

\[
A^{(k)}=\begin{pmatrix}
\ddots & \cdots & \cdots & \cdots & \cdots & \\
\cline{3-6}
\vdots & a^{(k)}_{kk} & \multicolumn{4}{|c|}{\multirow{1}{*}{$\boldsymbol{r}_1$}} \\
\cline{2-6}
\vdots & \multicolumn{1}{|c|}{\multirow{4}{*}{$\boldsymbol{c}_1$}} & & & & \\
\vdots & \multicolumn{1}{|c|}{} & & & & \\
\vdots & \multicolumn{1}{|c|}{} & & & & \\
\vdots & \multicolumn{1}{|c|}{} & & & & \\
\cline{2-2}
\end{pmatrix}
\]

indent 其中,$\boldsymbol{c}_1=\left(a_{k+1,k}^{(k)},\cdots,a_{nk}^{(k)}\right)^T$,$\boldsymbol{r}_1=\left(a_{k,k+1}^{(k)},\cdots,a_{kn}^{(k)}\right)^T$。为保证第$k$次消元填入量最小,规定$nr_i$为子矩阵$A_{22}^{(k)}$中第$i$行非零元的个数,$nc_j$为子矩阵$A_{22}^{(k)}$中第$j$列非零元的个数。B主元策略就是在子矩阵$A_{22}^{(k)}$中寻找一个非零元$a_{ij}^{(k)}$,使得$(nr_i-1)(nc_j-1)$的值最小,并以此$a_{ij}^{(k)}$作为第$k$次消元的主元。

[alert type="error"]B注意:主元策略B往往不能保证数值稳定性,因此需要在局部填入量较少和主元较大之间做平衡。[/alert]


% H2, H3
\subsection*{dsfsdt}
%[latex mode=1]
%[+preamble]
%\usepackage[titlenotnumbered,lined,linesnumbered]{algorithm2e}
\let\oldnl\nl
\newcommand{\nonl}{\renewcommand{\nl}{\let\nl\oldnl}}
%[/preamble]
%\quicklatex{size=14}
\TitleOfAlgo{BLAS-3 Gaussian Elimination with Partial Pivoting}
\begin{algorithm}[H]
\For{$l\gets 1$ \KwTo $n-1$}{
$k\gets (l-1)b+1$\;
factorize $PA^{(l)}=LU$ using BLAS-2\;
store $L$ and $U$ in $A$\;
left multiply $A(1:n,k+b:n)$ by $P$\;
$A(k:k+b-1,k+b:n)\gets T^{-1}A(k:k+b-1,k+b:n)$\;
\nonl\Indp where $T$ is the unit-lower-trian of $A(k:k+b-1,k:k+b-1)$\;
\DontPrintSemicolon\Indm$A(k+b:n,k+b:n)\gets A(k+b:n,k+b:n)$\;
\PrintSemicolon\nonl\Indp$-A(k+b:n,k:k+b-1)A(k:k+b-1,k+b:n)$\;
}
\end{algorithm}
%[/latex]

% H4
% \subsubsection*{}

\end{document}
