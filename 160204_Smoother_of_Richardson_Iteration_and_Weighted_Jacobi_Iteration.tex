\documentclass[12pt, UTF8, nofonts]{ctexart}

\setCJKmainfont[BoldFont=STHeiti,ItalicFont=STKaiti]{STKaiti}
\setCJKsansfont[BoldFont=STHeiti]{STXihei}
\setCJKmonofont{STFangsong}

\usepackage[letterpaper,left=.5in,right=.5in,top=1in,bottom=.5in]{geometry}
\usepackage{afterpage, color, multirow}
\usepackage{amsmath}
\usepackage{amsfonts}
\usepackage{amssymb}
\usepackage[titlenotnumbered,lined,linesnumbered]{algorithm2e}

\definecolor{bgcolor}{RGB}{13, 12, 12}
\definecolor{ttcolor}{RGB}{0, 255, 0}
\pagecolor{bgcolor}\afterpage{\nopagecolor}
\makeatletter
\newcommand{\globalcolor}[1]{\color{#1}\global\let\default@color\current@color}
\makeatother
\AtBeginDocument{\globalcolor{ttcolor}}

\begin{document}

%[aside_content]

\section*{多重网格方法}

基于矩阵理论,上文对一维和二维离散 Poisson 方程进行了详细的特征分析和推导(见多重网格方法相关的矩阵理论一文),这有助于更好地分析在应用了不同迭代格式之后,不同迭代矩阵所具有的特征特性。这些都是理解和分析多重网格方法的必要前提。本文主要就两种基本迭代算法:Richardson 迭代和带权重的 Jacobi 迭代,讨论它们的特征特性。对于 Gauss-Seidel 迭代将另文详述。这三种迭代算法都可作为[label]光滑子[/label](smoother)来使用。

\subsection*{Richardson 迭代}

事实上,此前已经论述了 Richardson 迭代算法及其相关特性,并由此引入了多重网格方法(见由 Richardson 迭代引入多重网格方法),本文在此将做稍加补充和整理。这里将用 Richardson 迭代来分析一维离散 Poisson 方程的情况,并使用一个固定参数$\omega$。[label]Richardson 迭代[/lable]算法的迭代格式及其迭代矩阵$\boldsymbol{G}_{\mathrm{R}}$可以表示为:

\begin{equation}
  \label{eq:richardson}
    \boldsymbol{u}_{j+1} = \boldsymbol{u}_j + \omega(\boldsymbol{b}-\boldsymbol{Au}_j) = \underbrace{(\boldsymbol{I} - \omega\boldsymbol{A})}_{\boldsymbol{G}_{\mathrm{R}}}\boldsymbol{u}_j + \omega\boldsymbol{b} \\
\end{equation}

如由 Richardson 迭代引入多重网格方法一文中给出的分析,Richardson 迭代算法的B收敛条件是$0<\omega<2/\rho(\boldsymbol{A})$。通常,$\rho(\boldsymbol{A})$具有上界$\gamma$,即:$\rho(\boldsymbol{A})\leq\gamma$,因此往往会取$\omega=1/\gamma$,此时它同样满足收敛条件。

\subsection*{带权重的 Jacobi 迭代}

下面将引入带权重的 Jacobi 算法,并将其同时应用于一维和二维离散 Poisson 方程。由标准的 [label]Jacobi 迭代[/label]可知,其迭代格式为:

\begin{equation}
  \label{eq:jacobi}
  \boldsymbol{u}_{j+1} = \boldsymbol{D}^{-1} (\boldsymbol{L}+\boldsymbol{U}) \boldsymbol{u}_j + \boldsymbol{D}^{-1}\boldsymbol{f}
\end{equation}

[label]带权重的 Jacobi 迭代[/label]则是取权重参数$\omega$,并结合式(\ref{eq:jacobi})中的$\boldsymbol{u}_{j+1}$和当前迭代向量$\boldsymbol{u}_j$之后所得到的,因此有:

\begin{equation}
  \label{eq:weightedjacobi}
  \begin{aligned}
    \boldsymbol{u}_{j+1} &= \omega \overbrace{\left[\boldsymbol{D}^{-1} (\boldsymbol{L}+\boldsymbol{U}) \boldsymbol{u}_{j} + \boldsymbol{D}^{-1}\boldsymbol{f}\right]}^{\boldsymbol{u}_{j+1}} + (1-\omega)\boldsymbol{u}_j \\
    &= \underbrace{\left[(1-\omega)\boldsymbol{I} + \omega\boldsymbol{D}^{-1}(\boldsymbol{L}+\boldsymbol{U})\right]}_{\boldsymbol{G}_{\mathrm{J}_\omega}} \boldsymbol{u}_{j} + \omega\boldsymbol{D}^{-1}\boldsymbol{f}
  \end{aligned} \quad\Rightarrow\quad
  \begin{aligned}
    \boldsymbol{G}_{\mathrm{J}_\omega} &= (1-\omega)\boldsymbol{I} + \omega\boldsymbol{D}^{-1}(\boldsymbol{D}-\boldsymbol{A}) \\
    &= \boldsymbol{I} - \omega\boldsymbol{D}^{-1}\boldsymbol{A}
  \end{aligned}
\end{equation}

由式(\ref{eq:weightedjacobi})便可以得到如下带权重的 Jacobi 迭代的收敛条件:

\begin{equation}
  \label{eq:wjitercon}
  -1 < 1 - \omega\rho(\boldsymbol{D}^{-1}\boldsymbol{A}) < 1 \;\Rightarrow\;
  0 < \omega < \dfrac{2}{\rho(\boldsymbol{D}^{-1}\boldsymbol{A})}
\end{equation}

%[/aside_content]

%[aside_content]

\subsubsection*{一维离散 Poisson 方程下的带权重 Jacobi 迭代}

对于一维离散 Poisson 方程,其对角矩阵$\boldsymbol{D}=2\boldsymbol{I}$;并且由上文的讨论可知一维离散 Poisson 方程的特征值$\lambda_k=2(1-\cos\frac{k\pi}{n+1})$(见多重网格方法相关的矩阵理论一文),因此迭代矩阵$\boldsymbol{G}_{\mathrm{J}_\omega}$及其特征值$\mu_k$可以表示为:

\begin{equation}
  \label{eq:1dwjeig}
  \begin{aligned}
    \boldsymbol{G}_{\mathrm{J}_\omega} &= \boldsymbol{I} - \dfrac{\omega}{2}\boldsymbol{A} \\
    \mu_k &= 1 - \dfrac{\omega}{2}(2-2\cos\dfrac{k\pi}{n+1}) = 1 - 2\omega\left(\sin^2\dfrac{k\pi}{2(n+1)}\right) \\
  \end{aligned}
\end{equation}

观察式(\ref{eq:1dwjeig})可知$\mu_k$是一个关于$k$的递减函数,如果令$s=\sin^2\left(\pi/2(n+1)\right)$的话,则有:

\begin{equation}
  \label{eq:1dwjeigrange}
  \begin{aligned}
    & \begin{aligned}
      \max{\mu_k} = \mu_{1} &= 1 - 2\omega\sin^2\dfrac{\pi}{2(n+1)} = 1 - 2\omega s^2 \\
      \min{\mu_k} = \mu_{n} &= 1 - 2\omega\sin^2\dfrac{n\pi}{2(n+1)} = 1 - 2\omega\cos^2\left(\dfrac{\pi}{2}-\dfrac{n\pi}{2(n+1)}\right) \\
      &= 1 - 2\omega\cos^2\left(\dfrac{\pi}{2(n+1)}\right) = (1-2\omega) + 2\omega s^2 \\
    \end{aligned} \\
    \Rightarrow \; & (1-2\omega)+2\omega s^2 \leq \mu_k \leq 1 - 2\omega s^2 \\
    \Rightarrow \; & \rho(\boldsymbol{G}_{\mathrm{J}_\omega}) = \max\Big\{|(1-2\omega)+2\omega s^2|,|1-2\omega s^2|\Big\}
  \end{aligned}
\end{equation}

式(\ref{eq:1dwjeigrange})给出了迭代矩阵$\boldsymbol{G}_{\mathrm{J}_\omega}$的特征值$\mu_k$的范围及其[label]谱半径[/label]。事实上,该谱半径是关于$\omega$的函数:当$\omega<0$或$\omega>1$时,$\rho(\boldsymbol{G}_{\mathrm{J}_\omega})$>1,迭代算法发散;当$0\leq\omega\leq1$时,从下图可以看出,谱半径即为$1-2\omega s^2$。当$n$很大时,$s\approx\frac{1}{2}h\pi$,则谱半径$\rho(\boldsymbol{G}_{\mathrm{J}_\omega}) \approx 1-\frac{1}{2}\omega h^2\pi^2$。

[alert type="error"]B注意:由于谱半径$\rho(\boldsymbol{G}_{\mathrm{J}_\omega})=1-2\omega s^2$是关于参数$\omega$的函数,且其在区间$[0,1]$上是递减函数,换言之,$\omega$的最优值是$1$,此时即为标准 Jacobi 迭代。因此相对标准 Jacobi 迭代,引入参数$\omega$后的带权重的 Jacobi 迭代B不会带来任何加速效果。[/alert]

[alert type="success"]尽管带权重的 Jacobi 迭代本身不会带来任何加速,但如果将其视作B光滑子的话,效果则完全不一样了。[/alert]

前文提到(见由 Richardson 迭代引入多重网格方法和多重网格方法相关的矩阵理论等文),[label]特征函数[/label]在$k>n/2$时处于[label]高频模式[/label](亦即[label]振荡模式[/label])。上文提到$\mu_k$是关于$k$的递减函数,因此针对高频模式,式(\ref{eq:1dwjeigrange})中的$\mu_k$的范围可以进一步缩小为:

\begin{equation}
  \label{eq:1dhighrange}
  \begin{aligned}
    k \geq \dfrac{n+1}{2} \;&\Rightarrow\;
    \left\{\; \begin{aligned}
      \max\mu_k &= \mu_{k=\frac{n+1}{2}} = 1 - 2\omega \sin^2\left(\dfrac{\pi}{4}\right) = 1 - \omega \\
      \min\mu_k &= \mu_{k=n} = (1-2\omega)+2\omega s^2 \\
    \end{aligned}\right. \\
    &\Rightarrow\; 1-2\omega < (1-2\omega) + 2\omega s^2 \leq \mu_{k} \leq 1 - \omega
  \end{aligned}
\end{equation}

[alert type="info"][label]平滑因子[/label](smoothing factor)定义为B高频模式下特征值绝对值的最大值,也可将其视为高频模式下的谱半径。[/alert]

类似于由 Richardson 迭代引入多重网格方法一文中所使用的求解参数$\omega$最优值的情形,式(\ref{eq:1dhighrange})中的$\omega$在$\frac{2}{3}$处取得最优值,即当$\omega=\frac{2}{3}$时谱半径最小,最优平滑因子(与网格大小$h$无关)为$\frac{1}{3}$。

%[/aside_content]

%[aside_content]

\subsubsection*{二维离散 Poisson 方程下的带权重 Jacobi 迭代}

类似地,可以将一维情况拓展至二维情况。此时,对角矩阵$\boldsymbol{D}=4\boldsymbol{I}$,系数矩阵$\boldsymbol{A}$的特征值为$\lambda_{kl}=\sigma_k+\mu_l$(见多重网格方法相关的矩阵理论一文)。因此,二维情况下时的迭代矩阵$\boldsymbol{G}_{\mathrm{J}_\omega}$及其特征值$\mu_{kl}$可以表示为:

\begin{equation}
  \label{eq:2dwjeig}
  \begin{aligned}
    \boldsymbol{G}_{\mathrm{J}_\omega} &= \boldsymbol{I} - \dfrac{\omega}{4}\boldsymbol{A} \\
    \mu_{kl} &= 1 - \dfrac{\omega}{4}\left(4\sin^2\dfrac{k\pi}{2(n+1)} + 4\sin^2\dfrac{l\pi}{2(m+1)}\right) \\
    &= 1 - \omega\left(\sin^2\dfrac{k\pi}{2(n+1)} + \sin^2\dfrac{l\pi}{2(m+1)}\right)
  \end{aligned}
\end{equation}

同样,可以令$s_x$和$s_y$来表示式(\ref{eq:2dwjeig})的特征值范围:

\begin{equation}
  \label{eq:2dwjeigrange}
  \begin{aligned}
    \begin{aligned}
      s_x &= \sin^2\left(\dfrac{\pi}{2(n+1)}\right) \\
      s_y &= \sin^2\left(\dfrac{\pi}{2(m+1)}\right) \\
    \end{aligned} \;&\Rightarrow\;
    \left\{\; \begin{aligned}
      \max\mu_{kl} &= \mu_{11} = 1 - \omega (s_x^2 + s_y^2) \\
      \min\mu_{kl} &= \mu_{nm} = 1 - \omega (1 - s_x^2 + 1 - s_y^2) \\
    \end{aligned} \right. \\
    \;&\Rightarrow\; (1-2\omega) + \omega(s_x^2+s_y^2) \leq \mu_{kl} \leq 1 - \omega(s_x^2+s_y^2) \\
    \;&\Rightarrow\; \rho(\boldsymbol{G}_{\mathrm{J}_\omega}) \approx 1 - \omega \left[ \left(\dfrac{1}{2}h_1\pi\right)^2 + \left(\dfrac{1}{2}h_2\pi\right)^2 \right] \approx 1 - \mathcal{O}(h^2)
  \end{aligned}
\end{equation}

和一维离散 Poisson 方程的结论类似,二维情况下$\omega=1$时最优,即谱半径最小,此时相比标准 Jacobi 迭代没有任何加速效果。但是如果考虑高频模式下($k>n/2$)的特征函数,它们的衰减速度很快,此时它们的特征值范围为($\omega=\frac{2}{3}$时取得最优平滑因子$\frac{1}{3}$):

\begin{equation}
  \label{eq:1dhighrange}
  \begin{aligned}
    \begin{aligned}
      k &\geq (n+1)/2 \\
      l &\geq (m+1)/2 \\
    \end{aligned}
    \;&\Rightarrow\;
    \left\{\; \begin{aligned}
      \max\mu_{kl} &= \mu_{\substack{k=(n+1)/2\\l=(m+1)/2}} = 1 - 2\omega \sin^2\left(\dfrac{\pi}{4}\right) = 1 - \omega \\
      \min\mu_{kl} &= \mu_{\substack{k=n\\l=m}} = (1-2\omega)+\omega (s_x^2+s_y^2) \\
    \end{aligned}\right. \\
    &\Rightarrow\; 1-2\omega < (1-2\omega) + \omega (s_x^2+s_y^2) \leq \mu_{kl} \leq 1 - \omega
  \end{aligned}
\end{equation}

[alert type="info"]事实上,在实际应用中,相比 Jacobi 迭代或者 Richardson 迭代,Gauss-Seidel 迭代或者使用了红黑排序(见详解基于矩阵分裂的经典迭代算法及算法实现一文)的 Gauss-Seidel 迭代B更适合作为光滑子(但分析起来也更为复杂),将另文详述。[/alert]

%[/aside_content]

\end{document}
