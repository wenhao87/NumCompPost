\documentclass[UTF8,nofonts]{ctexart}

\setCJKmainfont[BoldFont=STHeiti,ItalicFont=STKaiti]{STKaiti}
\setCJKsansfont[BoldFont=STHeiti]{STXihei}
\setCJKmonofont{STFangsong}

\usepackage[letterpaper,left=1in,top=1in]{geometry}
\usepackage{multirow}
\usepackage{amsmath}
\usepackage{amsfonts}
\usepackage{amssymb}
\usepackage[titlenotnumbered,lined,linesnumbered]{algorithm2e}

\begin{document}

% Title

\section*{详解Lanczos方法与线性方程组的Lanczos迭代法}

%[aside_content]

本文所要介绍的[label]共轭梯度方法[/label](Conjugate Gradient, [label]CG[/label])主要用于求解稀疏[label]对称正定[/label](sparse Symmetric Positive Definite)线性方程组$A\boldsymbol{x}=\boldsymbol{b}$。用一句话来概括共轭梯度方法(CG)的话,就是该方法实现了一种[label]Krylov子空间[/label]$\mathcal{K}_m(\boldsymbol{r}_0,A)$上的[label]正交投影[/lable]技术,其中$\boldsymbol{r}_0$为初始[label]残差[/label]。显然,在数学上它等价于求解线性方程组的[label]Arnoldi方法[/label],即[label]完全正交化方法[/label]([label]FOM[/label])(见线性方程组的Arnoldi方法之:FOM一文)。另一方面,由于矩阵$A$的一些特殊性质(对称、正定),利用[label]对称Lanczos方法[/label]的B三项递归关系式(见详解Lanczos方法与线性方程组的Lanczos迭代法一文),算法可以得到进一步的简化。

\subsection*{线性方程组的Lanczos迭代法}

如上所述,共轭梯度(CG)方法可以通过以下两种方式推导获得。这两种算法的核心本质都是Arnoldi方法在矩阵$A$是对称矩阵的情况下所得到的简化版本,也就是对称Lanczos方法。不同之处在于近似解$\boldsymbol{x}_m$的求解方式上:第一种基于完全正交化方法(FOM)(见线性方程组的Arnoldi方法之:FOM一文);第二种则基于直接不完全正交化方法(DIOM)(见不完全(IOM)和直接不完全(DIOM)正交化方法一文)。后者便是前文所介绍的直接对称Lanczos迭代法(见详解Lanczos方法与线性方程组的Lanczos迭代法一文)。

\subsubsection*{基于完全正交化方法(FOM)}

共轭梯度(CG)方法可以看作是完全正交化方法(FOM)的一种变形,区别在于共轭梯度(CG)方法得到的[label]上Hessenberg矩阵[/label]$H_m$实际是一个[label]对称三对角矩阵[/label]$T_m$,结合一组向量基底$V_m$,便能进行近似解的计算:$\boldsymbol{y}_m=T^{-1}_m(\beta\boldsymbol{e}_1)$以及$\boldsymbol{x}_m=\boldsymbol{x}_0+V_m\boldsymbol{y}_m$,这种方法也称为线性方程组的[label]Lanczos方法[/label]。下面给出其具体的算法实现:

%[latex mode=1]
%[+preamble]
%\usepackage[titlenotnumbered,lined,linesnumbered]{algorithm2e}
%[/preamble]
\TitleOfAlgo{Lanczos Method for Linear System}
\begin{algorithm}[H]
compute $\boldsymbol{r}_0\gets\boldsymbol{b}-A\boldsymbol{x}_0$, $\beta\gets\|\boldsymbol{r}_0\|_2$, $\boldsymbol{v}_1\gets\boldsymbol{r}_0/\beta$\;
set $\beta_1\gets0$, $\boldsymbol{v}_0\gets0$\;
\For{$j\gets 1$ \KwTo $m$}{
	$\boldsymbol{w}_j\gets A\boldsymbol{v}_j-\beta_j\boldsymbol{v}_{j-1}$\;
	$\alpha_j\gets(\boldsymbol{w}_j,\boldsymbol{v}_j)$\;
	$\boldsymbol{w}_j\gets\boldsymbol{w}_j-\alpha_j\boldsymbol{v}_j$\;
	$\beta_{j+1}\gets\|\boldsymbol{w}_j\|_2$\;
	\If{$\beta_{j+1}=0$}{
		set $m\gets j$; Goto 13\;
	}
	$\boldsymbol{v}_{j+1}\gets\boldsymbol{w}_j/\beta_{j+1}$\;
}
set $T_m\gets\text{tridiag}(\beta_i,\alpha_i,\beta_{i+1})$, and $V_m\gets(\boldsymbol{v}_1,\cdots,\boldsymbol{v}_m)$\;
$\boldsymbol{y}_m\gets H_m^{-1}(\beta\boldsymbol{e}_1)$; $\boldsymbol{x}_m\gets\boldsymbol{x}_0+V_m\boldsymbol{y}_m$\;
\end{algorithm}
%[/latex]

上述算法的第$3\thicksim12$行便是Arnoldi方法简化而来的对称Lanczos方法。算法最后(第$14$行)$\boldsymbol{y}_m$和$\boldsymbol{x}_m$的计算则和完全正交化方法(FOM)的计算形式一致,因此和FOM方法一样,该算法中近似解$\boldsymbol{x}_m$的残差也可以表示为($h_{m+1,m}=\beta_{m+1}$):

\begin{equation}
\label{eq:rm}
\boldsymbol{b}-A\boldsymbol{x}_m=-\beta_{m+1}\left(\boldsymbol{e}^T_m\boldsymbol{y}_m\right)\boldsymbol{v}_{m+1}
\end{equation}

\subsubsection*{基于直接不完全正交化方法(DIOM)}

直接不完全正交化方法(DIOM)在$k=2$时,结合对称Lanczos方法,所得到的便是线性方程组的直接Lanczos迭代法。其具体算法在详解Lanczos方法与线性方程组的Lanczos迭代法一文中已详细给出,这里不再冗述,但需要注意的是,共轭梯度(CG)方法的算法推导主要都将基于直接Lanczos迭代法的以下几个重要性质。

\subsubsection*{直接Lanczos迭代法的重要性质}

- 各个残差之间相互正交

根据式(\ref{eq:rm})近似解$\boldsymbol{x}_m$的残差表达式可知:

\begin{equation}
\label{eq:rsv}
\boldsymbol{r}_m=\sigma_m\boldsymbol{v}_{m+1},\quad\sigma_m=-\beta_{m+1}\left(\boldsymbol{e}^T_m\boldsymbol{y}_m\right)
\end{equation}

indent而由于$\{\boldsymbol{v}_i\}_{i=1}^{m+1}$是一组标准正交基底,因此可以推知,各个残差$\{\boldsymbol{r}_i\}_{i=0}^{m}$之间也都相互正交。并且,由于$\boldsymbol{r}_m$是关于$\boldsymbol{v}_{m+1}$的表达式,因此它和前$m$个$\{\boldsymbol{v}_i\}_{i=1}^m$都正交。

- $\left(A\boldsymbol{p}_i,\boldsymbol{p}_j\right)=0,i \neq j$

利用$P_m$的定义和$T_m$的LU分解,可以推知:

\begin{equation}
\label{eq:pap}
\begin{aligned}
P_m&=V_mU_m^{-1} \\
T_m&=L_mU_m=V_m^TAV_m \\
\end{aligned}\quad\Longrightarrow\quad
\begin{aligned}
P_m^TAP_m &= U_m^{-T}V_m^TAV_mU_m^{-1} \\
&= U_m^{-T}T_mU_m^{-1}=U_m^{-T}L_m
\end{aligned}
\end{equation}

indent显然,式(\ref{eq:pap})的最右项$U_m^{-T}L_m$是一个[label]下三角矩阵[/label],可推至$P_m^TAP_m$也是下三角矩阵。进一步观察$P_m^TAP_m$,由于$A$是对称矩阵,所以$P_m^TAP_m$也是对称矩阵:

\[(P_m^TAP_m)^T=P_m^TA^TP_m=P_m^TAP_m\]

indent于是,$P_m^TAP_m$既是下三角矩阵,又是对称矩阵,所以可以推知,$P_m^TAP_m$必是[label]对角矩阵[/label],即:

\begin{equation}
\label{eq:pipj}
\left(A\boldsymbol{p}_i,\boldsymbol{p}_j\right)=0,\quad i \neq j
\end{equation}

- $\left(\boldsymbol{v}_i,A\boldsymbol{p}_j\right)=0,i<j$

同样,使用类似式(\ref{eq:pap})推导方式,可以推知:

\[V_m^TAP_m=V_m^TAV_mU_m^{-1}=T_mU_m^{-1}=L_m\]

indent所以,由于$L_m$是下三角矩阵,必然可以得知:

\begin{equation}
\label{eq:vap}
\left(\boldsymbol{v}_i,A\boldsymbol{p}_j\right)=\left(A\boldsymbol{p}_j,\boldsymbol{v}_i\right)=0,\quad i<j
\end{equation}

[alert type="success"]根据式(\ref{eq:rsv})和式(\ref{eq:vap}),可以得出关于直接Lanczos迭代法的一个重要结论,即:对于任意的$k$,对应的$\boldsymbol{r}_k$、$A\boldsymbol{p}_{k+1}$都和$\mathcal{K}_k(A,\boldsymbol{r}_0)=\text{span}\{\boldsymbol{v}_1,\cdots,\boldsymbol{v}_k\}$正交。[/alert]

\subsection*{共轭梯度(CG)方法}

\subsubsection*{共轭梯度(CG)方法的推导}

结合直接Lanczos迭代方法以及上面所给出的重要性质,便可以进行共轭梯度(CG)方法的推导。整个推导过程如下所述:

首先,根据直接Lanczos迭代方法可知,近似解$\boldsymbol{x}_m$可以表示为:

\begin{equation}
\label{eq:xk1pk1}
\boldsymbol{x}_{k+1}=\boldsymbol{x}_k+\xi_{k+1}\boldsymbol{p}_{k+1}
\end{equation}

indent于是,引入和式(\ref{eq:xk1pk1})中$\boldsymbol{p}_{k+1}$平行的搜索方向$\boldsymbol{d}_k$,则$\boldsymbol{x}_{k+1}$可以表示为:

\begin{equation}
\label{eq:xk1dk}
\boldsymbol{x}_{k+1}=\boldsymbol{x}_k+\nu_k\boldsymbol{d}_k
\end{equation}

indent同时也可以推得残差$\boldsymbol{r}_{k+1}$的递推表达式:

\begin{equation}
\label{eq:rk1adk}
\begin{aligned}
\boldsymbol{r}_{k+1}-\boldsymbol{r}_k&=(\boldsymbol{b}-A\boldsymbol{x}_{k+1})-(\boldsymbol{b}-A\boldsymbol{x}_{k}) \\
&=-A(\boldsymbol{x}_{k+1}-\boldsymbol{x}_k)=-\nu_kA\boldsymbol{d}_k
\end{aligned}\quad\Longrightarrow\quad
\boldsymbol{r}_{k+1}=\boldsymbol{r}_k-\nu_kA\boldsymbol{d}_k
\end{equation}

indent再利用式(\ref{eq:rsv})各个残差之间相互正交的性质,即:$(\boldsymbol{r}_{k+1},\boldsymbol{r}_k)=0$,结合式(\ref{eq:rk1adk})可以得到:

\begin{equation}
\label{eq:muk}
(\boldsymbol{r}_{k+1},\boldsymbol{r}_k)=(\boldsymbol{r}_k-\nu_kA\boldsymbol{d}_k,\boldsymbol{r}_k)=0\quad\Longrightarrow\quad
\nu_k=\dfrac{(\boldsymbol{r}_k,\boldsymbol{r}_k)}{(A\boldsymbol{d}_k,\boldsymbol{r}_k)}
\end{equation}

接着,考察搜索方向$\boldsymbol{d}_k$。如前所述,它平行于$\boldsymbol{p}_{k+1}$,而根据直接Lanczos迭代方法中$\boldsymbol{p}_k$的表达式,以及式(\ref{eq:rsv})的残差表达式,即:

\[
\boldsymbol{p}_{k+1}=u^{-1}_{k+1,k+1}\left(\boldsymbol{v}_{k+1}-\beta_{k+1}\boldsymbol{p}_k\right)\quad,\quad\boldsymbol{r}_k=\sigma_k\boldsymbol{v}_{k+1}
\]

indent可以推知,$\boldsymbol{d}_k$可以表示成$\boldsymbol{r}_k$和$\boldsymbol{p}_k$的线性组合。换言之,$\boldsymbol{d}_{k+1}$也是$\boldsymbol{r}_{k+1}$和$\boldsymbol{p}_{k+1}$的线性组合,而$\boldsymbol{d}_k$又平行于后者搜索方向$\boldsymbol{p}_{k+1}$,所以可以得到如下$\boldsymbol{d}_{k+1}$的递推关系式:

\begin{equation}
\label{eq:dk1rhodk}
\boldsymbol{d}_{k+1}=\boldsymbol{r}_{k+1}+\omega_k\boldsymbol{d}_k
\end{equation}

indent于是,式(\ref{eq:muk})中的$\boldsymbol{r}_k$可以用式(\ref{eq:dk1rhodk})的关系式来表达;另一方面,由于$\boldsymbol{d}_k$是平行于$\boldsymbol{p}_{k+1}$的搜索方向,所以$\boldsymbol{d}_k$同样满足式(\ref{eq:pipj})的正交性质,于是有:

\begin{equation}
\label{eq:adkdk}
\begin{aligned}
&\boldsymbol{r}_{k}=\boldsymbol{d}_k-\omega_{k-1}\boldsymbol{d}_{k-1} \\
&(A\boldsymbol{d}_i,\boldsymbol{d}_j)=0,i\neq j
\end{aligned}\quad\Longrightarrow\quad
\begin{aligned}
(A\boldsymbol{d}_k,\boldsymbol{r}_k) &= (A\boldsymbol{d}_k,\boldsymbol{d}_k-\omega_{k-1}\boldsymbol{d}_{k-1}) \\
&= (A\boldsymbol{d}_k,\boldsymbol{d}_k)-(A\boldsymbol{d}_k,\omega_{k-1}\boldsymbol{d}_{k-1}) \\
&= (A\boldsymbol{d}_k,\boldsymbol{d}_k)
\end{aligned}
\end{equation}

indent那么,将式(\ref{eq:adkdk})代入式(\ref{eq:muk})后,$\mu_k$可以重新表示为:

\begin{equation}
\label{eq:muknew}
\nu_k=\dfrac{(\boldsymbol{r}_k,\boldsymbol{r}_k)}{(A\boldsymbol{d}_k,\boldsymbol{r}_k)}=\dfrac{(\boldsymbol{r}_k,\boldsymbol{r}_k)}{(A\boldsymbol{d}_k,\boldsymbol{d}_k)}
\end{equation}

最后考察$\omega_k$。利用式(\ref{eq:dk1rhodk})中$\boldsymbol{d}_k$的递推表达式,以及$\boldsymbol{d}_k$之间的正交性,可以推知:

\begin{equation}
\label{eq:omegak}
\left.
\begin{aligned}
(\boldsymbol{d}_k,\boldsymbol{d}_{k+1})=0\quad&\Rightarrow\quad(A\boldsymbol{d}_k,\boldsymbol{d}_{k+1})=0 \\
\boldsymbol{d}_{k+1}=\boldsymbol{r}_{k+1}+\omega_k\boldsymbol{d}_k\quad&\Rightarrow\quad(A\boldsymbol{d}_k,\boldsymbol{r}_{k+1}+\omega_k\boldsymbol{d}_k)=0
\end{aligned}\quad\right\}\Longrightarrow\quad
\omega_k=-\dfrac{(A\boldsymbol{d}_k,\boldsymbol{r}_{k+1})}{(A\boldsymbol{d}_k,\boldsymbol{d}_k)}
\end{equation}

indent再结合式(\ref{eq:muknew})中$\nu_k$的表达式,式(\ref{eq:rk1adk})中的$A\boldsymbol{d}_k$的表达式,以及各个残差之间的正交性,可以将式(\ref{eq:omegak})进一步改写为:

\begin{equation}
\label{eq:omegaknew}
\left.
\begin{aligned}
(A\boldsymbol{d}_k,\boldsymbol{d}_k)=\dfrac{(\boldsymbol{r}_k,\boldsymbol{r}_k)}{\nu_k} \\
A\boldsymbol{d}_k=-\dfrac{1}{\nu_k}(\boldsymbol{r}_{k+1}-\boldsymbol{r}_k) \\
(\boldsymbol{r}_k,\boldsymbol{r}_{k+1})=0
\end{aligned}\quad\right\}\Longrightarrow\quad
\begin{aligned}
\omega_k &= -\dfrac{\left(-\dfrac{1}{\nu_k}(\boldsymbol{r}_{k+1}-\boldsymbol{r}_k),\boldsymbol{r}_{k+1}\right)}{\dfrac{(\boldsymbol{r}_k,\boldsymbol{r}_k)}{\nu_k}} \\
&= \dfrac{(\boldsymbol{r}_{k+1},\boldsymbol{r}_{k+1})}{(\boldsymbol{r}_k,\boldsymbol{r}_k)}
\end{aligned}
\end{equation}

至此,便得到了式(\ref{eq:muknew})关于$\nu_k$的计算方式,以及式(\ref{eq:omegaknew})关于$\omega_k$的计算方式,从而便完成了共轭梯度(CG)方法的推导,而其原理就是:根据当前近似解$\boldsymbol{x}_k$、$\nu_k$、以及当前搜索方向$\boldsymbol{d}_k$,便可以计算得到$\boldsymbol{x}_{k+1}$;而下一步的搜索方向$\boldsymbol{d}_{k+1}$又可以通过$\boldsymbol{x}_{k+1}$的残差$\boldsymbol{r}_{k+1}$、$\omega_{k}$、以及当前搜索方向$\boldsymbol{d}_k$计算得到。

\subsubsection*{共轭梯度(CG)方法的算法实现}

对上述的推导过程稍加总结,便能得到共轭梯度(CG)方法的算法流程,且整个算法流程主要基于以下几个关系式:

\begin{equation}
\renewcommand\arraystretch{1.5}
\begin{array}{lcl}
\nu_k=\dfrac{(\boldsymbol{r}_k,\boldsymbol{r}_k)}{(A\boldsymbol{d}_k,\boldsymbol{d}_k)} & \to & 
\boldsymbol{x}_{k+1}=\boldsymbol{x}_k+\nu_k\boldsymbol{d}_k \\
\omega_k=\dfrac{(\boldsymbol{r}_{k+1},\boldsymbol{r}_{k+1})}{(\boldsymbol{r}_k,\boldsymbol{r}_k)} & \to &
\boldsymbol{d}_{k+1}=\boldsymbol{r}_{k+1}+\omega_k\boldsymbol{d}_k
\end{array}
\end{equation}

从第$k=0$步起,由初始向量$\boldsymbol{x}_0$可以得到初始残差$\boldsymbol{r}_0$,并令其为初始搜索方向,即$\boldsymbol{r}_0\equiv\boldsymbol{d}_0$,于是便能得到$\nu_0$,并计算得到$\boldsymbol{x}_1$;而通过$\boldsymbol{x}_1$的残差$\boldsymbol{r}_1$和$\boldsymbol{x}_0$的残差$\boldsymbol{r}_0$,便能计算得到$\omega_0$,从而得到下一步$\boldsymbol{x}_1$的搜索方向$\boldsymbol{d}_1$。此后便重复该过程,直到达到收敛条件为止。综上所述,给定对称正定矩阵$A\in\mathbb{R}^{n \times n}$,$\boldsymbol{b},\boldsymbol{x}_0\in\mathbb{R}^n$,共轭梯度(CG)方法的算法实现如下:

%[latex mode=1]
%[+preamble]
%\usepackage[titlenotnumbered,lined,linesnumbered]{algorithm2e}
%[/preamble]
\TitleOfAlgo{Conjugate Gradient Method}
\begin{algorithm}[H]
compute $\boldsymbol{r}_0\gets\boldsymbol{b}-A\boldsymbol{x}_0$, $\boldsymbol{d}_0\gets\boldsymbol{r}_0$\;
\For{$k\gets 0,1,\cdots$}{
	$\nu_k\gets(\boldsymbol{r}_k,\boldsymbol{r}_k)/(A\boldsymbol{d}_k,\boldsymbol{d}_k)$\;
	$\boldsymbol{x}_{k+1}\gets\boldsymbol{x}_k+\nu_k\boldsymbol{d}_k$\;
	$\boldsymbol{r}_{k+1}\gets\boldsymbol{r}_k-\nu_kA\boldsymbol{d}_k$\;
	$\omega_k\gets(\boldsymbol{r}_{k+1},\boldsymbol{r}_{k+1})/(\boldsymbol{r}_k,\boldsymbol{r}_k)$\;
	$\boldsymbol{d}_{k+1}\gets\boldsymbol{r}_{k+1}+\omega_k\boldsymbol{d}_k$\;
	\If{$\boldsymbol{x}_m$ has converged}{
	Stop\;
}
}
\end{algorithm}
%[/latex]

[alert type="error"]注意:上述共轭梯度(CG)方法的算法实现第$5$行处,$\boldsymbol{r}_{k+1}$使用的是式(\ref{eq:rk1adk})的计算方式。另外,关于存储量,除了矩阵$A$以外,共轭梯度(CG)方法只需存储四个向量:$\boldsymbol{x}$、$\boldsymbol{d}$、$A\boldsymbol{d}$、$\boldsymbol{r}$。[/alert]

\subsubsection*{共轭梯度(CG)方法的算法分析}

上述的共轭梯度(CG)方法中,初始搜索方向$\boldsymbol{d}_0=\boldsymbol{r}_0$,得到的迭代格式即为没有预处理的共轭梯度方法(预处理的共轭梯度方法将另文详述)。

在不考虑舍入误差的情况下,对于相同初始向量$\boldsymbol{x}_0$,直接Lanczos迭代法和共轭梯度(CG)方法得到的近似解$\boldsymbol{x}_m$相同,而相比前者,共轭梯度(CG)方法更为简洁。由于在共轭梯度(CG)方法中引入的搜索方向$\boldsymbol{d}_k$和直接Lanczos迭代法中的$\boldsymbol{p}_{k+1}$平行,所以对于任意$k$,共轭梯度(CG)方法中的$\boldsymbol{r}_k$和$A\boldsymbol{d}_k$也都和$\mathcal{K}_k(A,\boldsymbol{r}_0)=\text{span}\{\boldsymbol{v}_1,\cdots,\boldsymbol{v}_k\}$正交。

最后需要指出的是,共轭梯度(CG)方法并没有生成直接Lanczos迭代法中的对称三对角矩阵$T_m$,但在矩阵$A$的特征值估计中,$T_m$有着很重要的作用。幸运的是,利用共轭梯度(CG)方法得到的系数,便可以构造出矩阵$T_m$。其原理是:根据式(\ref{eq:rsv})可知,$\boldsymbol{x}_m$的残差$\boldsymbol{r}_m$与$\boldsymbol{v}_{m+1}$平行,而$T_m$又可以表示为$T_m=V_m^TAV_m$,则显然$T_m$可以用$\boldsymbol{r}_m$进行表示,进而可以用共轭梯度(CG)方法中的系数$\nu_k$和$\omega_k$表示。这里具体的推导过程从略,直接给出矩阵$T_m$的系数表示形式:

\[
\renewcommand\arraystretch{1.5}
T_m=\begin{pmatrix}
\frac{1}{\nu_0} & \frac{\sqrt{\omega_0}}{\nu_0} & & & \\
\frac{\sqrt{\omega_0}}{\nu_0} & \frac{1}{\nu_1}+\frac{\omega_0}{\nu_0} & \frac{\sqrt{\omega_1}}{\nu_1} & & \\
& \ddots & \ddots & \ddots & & \\
& & \ddots & \ddots & \frac{\sqrt{\omega_{m-2}}}{\nu_{m-2}} \\
& & & \frac{\sqrt{\omega_{m-2}}}{\nu_{m-2}} & \frac{1}{\nu_{m-1}}+\frac{\omega_{m-2}}{\nu_{m-2}}
\end{pmatrix}\]




%[/aside_content]


% H2, H3
\subsection*{dsfsdt}
%[latex mode=1]
%[+preamble]
%\usepackage[titlenotnumbered,lined,linesnumbered]{algorithm2e}
\let\oldnl\nl
\newcommand{\nonl}{\renewcommand{\nl}{\let\nl\oldnl}}
%[/preamble]
%\quicklatex{size=14}
\TitleOfAlgo{BLAS-3 Gaussian Elimination with Partial Pivoting}
\begin{algorithm}[H]
\For{$l\gets 1$ \KwTo $n-1$}{
$k\gets (l-1)b+1$\;
factorize $PA^{(l)}=LU$ using BLAS-2\;
store $L$ and $U$ in $A$\;
left multiply $A(1:n,k+b:n)$ by $P$\;
$A(k:k+b-1,k+b:n)\gets T^{-1}A(k:k+b-1,k+b:n)$\;
\nonl\Indp where $T$ is the unit-lower-trian of $A(k:k+b-1,k:k+b-1)$\;
\DontPrintSemicolon\Indm$A(k+b:n,k+b:n)\gets A(k+b:n,k+b:n)$\;
\PrintSemicolon\nonl\Indp$-A(k+b:n,k:k+b-1)A(k:k+b-1,k+b:n)$\;
}
\end{algorithm}
%[/latex]

% H4
% \subsubsection*{}

\end{document}
