\documentclass[UTF8,nofonts]{ctexart}

\setCJKmainfont[BoldFont=STHeiti,ItalicFont=STKaiti]{STKaiti}
\setCJKsansfont[BoldFont=STHeiti]{STXihei}
\setCJKmonofont{STFangsong}

\usepackage[letterpaper,left=1in,top=1in]{geometry}
\usepackage{multirow}
\usepackage{amsmath}
\usepackage{amsfonts}
\usepackage{amssymb}
\usepackage[titlenotnumbered,lined,linesnumbered]{algorithm2e}

\begin{document}

% Title
\section*{基本迭代法的收敛性分析}

上文(见求解线性方程组的基本迭代法一文)介绍了几种常用的求解[label]线性方程组[/label]的[label]迭代法[/label],且它们都具有如下的[label]迭代格式[/label]:

\begin{equation}
\label{eq:iter}
\boldsymbol{x}^{(k+1)}=G\boldsymbol{x}^{(k)}+\boldsymbol{f},\quad k=0,1,\cdots
\end{equation}

indent 其中,矩阵$G$称为[label]迭代矩阵[/label]。本文主要通过迭代矩阵$G$,对基本迭代法进行收敛性分析,包括:收敛的一致性问题、宏观收敛条件、收敛速度,以及部分特殊数据的收敛条件等。

\subsection*{收敛的一致性问题}

如果式(\ref{eq:iter})的迭代收敛,则$\boldsymbol{x}$的极限满足$\boldsymbol{x}=G\boldsymbol{x}+\boldsymbol{f}$。另一方面,迭代格式是通过矩阵分裂$A=M-N$得到的,也就是说式(\ref{eq:iter})等价于:

\[\boldsymbol{x}^{(k+1)}=M^{-1}N\boldsymbol{x}^{(k)}+M^{-1}\boldsymbol{f}\]

indent 且其极限满足$M\boldsymbol{x}=N\boldsymbol{x}+\boldsymbol{b}$,也就是线性方程组$A\boldsymbol{x}=\boldsymbol{b}$。所以说,如果迭代收敛的话,便可以说式(\ref{eq:iter})的极限就是原线性方程组的解。

\subsection*{收敛条件分析}

根据迭代法的[label]误差传播方程[/label](见求解线性方程组的基本迭代法一文),有:

\begin{equation}
\label{eq:err}
\boldsymbol{x}^{(k+1)}-\boldsymbol{x}^\ast=G(\boldsymbol{x}^{(k)}-\boldsymbol{x}^\ast)=\cdots=G^{k+1}(\boldsymbol{x}^{(0)}-\boldsymbol{x}^{\ast})
\end{equation}

indent 其中,极限$\boldsymbol{x}^\ast$便是线性方程组的解。另一方面,当且仅当$\rho(G)<1$的时候,$I-G$非奇异,且矩阵序列$G^k,k=0,1,\cdots$收敛到$0$。它等价于$\boldsymbol{x}^{(k)}-\boldsymbol{x}^{\ast}$收敛到$0$,也等价于$\boldsymbol{x}^{(k)}$收敛到原线性方程组的解$\boldsymbol{x}^{\ast}$。所以,便有了如下的[label]迭代法基本定理[/label]:

[alert type="info"]式(\ref{eq:iter})迭代法对于任意初始向量$\boldsymbol{x}^{(0)}$都收敛的一个B充分必要条件是$\rho(G)<1$。[/alert]

另一方面,如果迭代矩阵$G$的某一[label]算子范数[/label]满足$q\equiv\|G\|<1$,则迭代法对于任意初始向量$\boldsymbol{x}^{(0)}$都收敛。这也就是迭代法收敛的B充分条件。

\subsection*{收敛速度分析}

根据式(\ref{eq:err})的误差传播方程,第$k$步迭代的[label]误差[/label]为$\boldsymbol{\epsilon}^{(k)}=G^{k}\boldsymbol{\epsilon}^{(0)}$。定义一个序列的[label]收敛因子[/label]为极限:

\[\rho=\lim_{k\to\infty}\left(\dfrac{\|\boldsymbol{\epsilon^{(k)}}\|}{\|\boldsymbol{\epsilon^{(0)}}\|}\right)^{1/k}\]

indent 它满足$\rho=\rho(G)$,并定义[label]收敛速度[/label]$\tau$为:

\begin{equation}
\label{eq:ratetau}
\tau=-\ln\rho=-\ln\rho(G)
\end{equation}

indent 式(\ref{eq:ratetau})收敛速度的定义B有赖于初始向量$\boldsymbol{x}^{(0)}$,因此也称为[label]特定收敛速度[/label]。相应地,[label]通用收敛速度[/label]可以定义为:

\begin{equation}
\label{eq:ratephi}
\phi=\lim_{k\to\infty}\left(\max_{\boldsymbol{x}^{(0)}\in\mathbb{R}^n}\dfrac{\|\boldsymbol{\epsilon^{(k)}}\|}{\|\boldsymbol{\epsilon^{(0)}}\|}\right)^{1/k}=\lim_{k\to\infty}\left(\max_{\boldsymbol{x}^{(0)}\in\mathbb{R}^n}\dfrac{\|G^k\boldsymbol{\epsilon^{(0)}}\|}{\|\boldsymbol{\epsilon^{(0)}}\|}\right)^{1/k}=\lim_{k\to\infty}\left(\|G^{k}\|\right)^{1/k}=\rho(G)
\end{equation}

[alert type="info"]式(\ref{eq:ratephi})的通用收敛速度比式(\ref{eq:ratetau})的特定收敛速度更为常用。[/alert]

\subsection*{特殊矩阵的收敛条件}

\subsubsection*{对角占优矩阵}

矩阵相关的几个基本概念一文中给出了[label]严格对角占优[/label](strictly diagonally dominant)矩阵、[label]弱对角占优[/label](weakly diagonally dominant)矩阵和[label]不可约[/label](irreducible)矩阵的定义。同时满足弱对角占优和不可约的矩阵,称为[label]不可约的弱对角占优矩阵[/label]。

如果$n \times n$阶矩阵$A$是B严格对角占优矩阵或B不可约的弱对角占优矩阵,则$A$是[label]非奇异[/label]的,并且对于任意的初始向量$\boldsymbol{x}^{(0)}$,相应的[label]Jacobi迭代法[/label]、[label]Gauss-Seidel迭代法[/label],以及关于$0<\omega\leq 1$的[label]SOR迭代法[/label]B均收敛。

\subsubsection*{对称正定矩阵}

如果矩阵$A$是[label]对称正定矩阵[/label](复数域上即为[label]Hermite矩阵[/label]),且$0<\omega<2$的话,则SOR迭代格式对任意初始向量$\boldsymbol{x}^{(0)}$都收敛。另一方面,如果SOR迭代格式对任意初始向量都收敛,则必有$0<\omega<2$。

[alert type="info"]上述两个结论对于[label]SSOR迭代法[/label]也同样成立。[/alert]

对于对称正定矩阵$A$,Gauss-Seidel迭代法也是收敛的,但Jacobi迭代法则不一定收敛,事实上,其收敛的充要条件为$2D-A$也是对称正定矩阵。

对于任意初始向量$\boldsymbol{x}^{(0)}$以及不同的迭代格式,特殊矩阵的收敛条件总结如下:



%\begin{eqnarray*}
%& M=\dfrac{}
%\end{eqnarray*}

% H2, H3
\subsection*{dsfsdt}
%[latex mode=1]
%[+preamble]
%\usepackage[titlenotnumbered,lined,linesnumbered]{algorithm2e}
\let\oldnl\nl
\newcommand{\nonl}{\renewcommand{\nl}{\let\nl\oldnl}}
%[/preamble]
%\quicklatex{size=14}
\TitleOfAlgo{BLAS-3 Gaussian Elimination with Partial Pivoting}
\begin{algorithm}[H]
\For{$l\gets 1$ \KwTo $n-1$}{
$k\gets (l-1)b+1$\;
factorize $PA^{(l)}=LU$ using BLAS-2\;
store $L$ and $U$ in $A$\;
left multiply $A(1:n,k+b:n)$ by $P$\;
$A(k:k+b-1,k+b:n)\gets T^{-1}A(k:k+b-1,k+b:n)$\;
\nonl\Indp where $T$ is the unit-lower-trian of $A(k:k+b-1,k:k+b-1)$\;
\DontPrintSemicolon\Indm$A(k+b:n,k+b:n)\gets A(k+b:n,k+b:n)$\;
\PrintSemicolon\nonl\Indp$-A(k+b:n,k:k+b-1)A(k:k+b-1,k+b:n)$\;
}
\end{algorithm}
%[/latex]

% H4
% \subsubsection*{}

\end{document}
