\documentclass[UTF8,nofonts]{ctexart}

\setCJKmainfont[BoldFont=STHeiti,ItalicFont=STKaiti]{STKaiti}
\setCJKsansfont[BoldFont=STHeiti]{STXihei}
\setCJKmonofont{STFangsong}

\usepackage[letterpaper,left=1in,top=1in]{geometry}
\usepackage{multirow}
\usepackage{amsmath}
\usepackage{amsfonts}
\usepackage{amssymb}
\usepackage[titlenotnumbered,lined,linesnumbered]{algorithm2e}

\begin{document}

% Title
\section*{列主元Gauss消去法的舍入误差分析}

大规模计算过程中,由[label]浮点运算[/labe]引起的[label]舍入误差[/label]可能逐渐积累并影响最终计算结果,因此有必要对算法进行[label]舍入误差分析[/label]。本文主要分析列主元Gauss消去法(见BLAS三个级别的列主元Gauss消去法一文)求解以下[label]线性方程组[/label]时产生的舍入误差:

\[A\boldsymbol{x}=\boldsymbol{b},\quad A\in\mathbb{R}^{n \times n},\quad \boldsymbol{b}\in\mathbb{R}^n\]

indent 其结果表明当计算机的[label]精度[/label]$\epsilon$足够高时,舍入误差便能得到有效控制;而线性方程组解的相对误差往往和$n^k$($n$是方程组的阶数)以及$\epsilon$成比例,特别地,对于列主元消去法有$k=3$。

\subsection*{近似表示}

用$\tilde{x}$或者$\text{fl}(x)$表示$x$在B计算中的表示,或者称为$x$的[label]浮点数[/label]。当计算机因精度有限而不能精确表示$x$时,$\tilde{x}$便是$x$的一个[label]近似[/label]。此外,用$\odot$表示某种算术运算,所以$\text{fl}(x \odot y)$便是两者运算后的结果在计算机中的表示,也就是$(x \odot y)$的近似,并且满足:

\[
\text{fl}(x \odot y)=(x \odot y)(1-\delta)=(x \odot y)/(1-\delta')
\]

indent 其中,$|\delta|,|\delta'|<\epsilon$。机器精度$\epsilon$可以具体写为(其中,$b$为机器使用的浮点数[label]基底[/label],$t$为[label]字长[/label]):

\[
\epsilon=
\begin{cases}
b^{1-t}/2,& \text{round-off error used}\\
b^{1-t},& \text{truncation error used} \\
\end{cases}
\]

\subsection*{LU分解的舍入误差分析}

若$n \times n$阶矩阵$A=A^{(1)}=(a_{ij})$能在计算机中精确表示,则用列主元Gauss消去法得到的[label]上三角矩阵[/label]$\tilde{U}$和B绝对值不超过$1$的[label]单位下三角矩阵[/label]$\tilde{L}$满足:

\[\tilde{L}\tilde{U}=P(A+E)\]

indent 其中,$P$为$n \times n$阶[label]排列矩阵[/label],且$n \times n$阶矩阵$E=(e_{ij})$满足:

\begin{equation}
\label{eq:rho}
|e_{ij}| \leq 2n\epsilon\rho,\quad i,j=1,2,\cdots,n
\end{equation}

indent 其中,$\rho=\max_{1 \leq i,j,k \leq n}|a^{(k)}_{ij}|$,$\tilde{A}^{(k)}=\left(a_{ij}^{(k)}\right)$是第$k-1$步消元后$A$在计算机中的精确表示。

\subsection*{向量点积运算的舍入误差}

设$x_1,x_2,\cdots,x_n$和$y_1,y_2,\cdots,y_n$为计算机的浮点数,若机器精度$\epsilon$B满足$(n+1)\epsilon<1/3$,则[label]向量点积[/label]运算:

\[
\left(\cdots\left(\left(x_1y_1+x_2y_2\right)+x_3y_3\right)+\cdots+x_{n-1}y_{n-1}\right)+x_ny_n
\]

indent 的结果可以表示为$\sum_{i=1}^nx_iy_i(1+\delta_i)$,其中$\delta_i \leq \frac{6}{5}(n+1)\epsilon$。

[alert type="info"]内积的求和次序是可以B任意交换的,结论不变。[/alert]


\subsection*{$L\boldsymbol{y}=\boldsymbol{b}$的舍入误差分析}

LU分解后,线性方程组$A\boldsymbol{x}=\boldsymbol{b}$的求解可以分为两个过程:$L\boldsymbol{y}=P\boldsymbol{b}$和$U\boldsymbol{x}=\boldsymbol{y}$。而有了上述向量点积运算的舍入误差分析后,便可以对这两个过程进行舍入误差分析了。

设$L=(l_{ij})$为$n \times n$阶单位下三角矩阵,$\boldsymbol{b}$为$n$维列向量,机器精度$\epsilon$B满足:$(n+1)\epsilon<1/3$,则线性方程组$L\boldsymbol{y}=\boldsymbol{b}$得到的解$\tilde{\boldsymbol{y}}$满足:

\[(L+\Delta)\tilde{\boldsymbol{y}}=\boldsymbol{b}\]

indent 其中,$n \times n$阶下三角矩阵$\Delta=(\Delta_{ij})$满足:

\[|\Delta| \leq \dfrac{6}{5}(n+1)\epsilon|l_{ij}|,\quad i,j=1,2,\cdots,n\]

\subsection*{$U\boldsymbol{x}=\boldsymbol{y}$的舍入误差分析}

设$U=(u_{ij})$为$n \times n$阶上三角矩阵,$\boldsymbol{y}$为$n$维列向量,机器精度$\epsilon$B满足:$(n+1)\epsilon<1/3$,则线性方程组$U\boldsymbol{x}=\boldsymbol{y}$得到的解$\tilde{\boldsymbol{x}}$满足:

\[(U+\Delta)\tilde{\boldsymbol{x}}=\boldsymbol{y}\]

indent 其中,$n \times n$阶上三角矩阵$\Delta=(\Delta_{ij})$满足:

\[|\Delta| \leq \dfrac{6}{5}(n+1)\epsilon|u_{ij}|,\quad i,j=1,2,\cdots,n\]

\subsection*{线性方程组$A\boldsymbol{x}=\boldsymbol{b}$的舍入误差}

设$A$和$\boldsymbol{b}$为已经B考虑了舍入误差后的计算机数,机器精度$\epsilon$B满足:$(n+1)\epsilon<1/3$,用列主元Gauss消去法求解线性方程组$A\boldsymbol{x}=\boldsymbol{b}$,得到的计算机中表示的解$\tilde{\boldsymbol{x}}$满足[label]扰动方程[/label]:

\[(A+F)\tilde{\boldsymbol{x}}=\boldsymbol{b}\]

indent 其中$n \times n$阶矩阵$F=(f_{ij})$满足:

\[|f_{ij}|\leq 10n^2\epsilon\rho,\quad 1\leq i,j \leq n\]

indent 其中的$\rho$在式(\ref{eq:rho})中定义。如果取适当小的$\|F_{\infty}\|$,确保$\kappa(A)_\infty\|F\|_\infty<\|A\|_\infty$的话,则$\tilde{\boldsymbol{x}}$的相对误差满足:

\begin{eqnarray*}
&\dfrac{\|\tilde{\boldsymbol{x}}-\boldsymbol{x}\|_\infty}{\|\boldsymbol{x}\|_\infty}
\leq 10n^3\epsilon g_n(A)\kappa(A)_\infty\\
&g_n(A)=\dfrac{\max_{1\leq i,j,k \leq n}{|a_{ij}^{(k)}|}}{\max_{1\leq i,j \leq n}{|a_{ij}|}}
\end{eqnarray*}

indent 这里,$g_n(A)$称为[label]增长因子[/label],$A^{(k)}=\left(a_{ij}^{(k)}\right)$是第$k-1$步消元后得到的矩阵在机器中的表示。

[alert type="info"]只要增长因子$g_{n}(A)$不是很大(列主元Gauss消去法中不超过$2^{n-1}$),舍入误差就不会很大。实际应用中$g_n(A)$很少超过$n$的线性增长阶,所以列主元消去法是B稳定的。[/alert]

\end{document}
