\documentclass[UTF8,nofonts]{ctexart}

\setCJKmainfont[BoldFont=STHeiti,ItalicFont=STKaiti]{STKaiti}
\setCJKsansfont[BoldFont=STHeiti]{STXihei}
\setCJKmonofont{STFangsong}

\usepackage[letterpaper,left=1in,top=1in]{geometry}
\usepackage{multirow}
\usepackage{amsmath}
\usepackage{amsfonts}
\usepackage{amssymb}
\usepackage[titlenotnumbered,lined,linesnumbered]{algorithm2e}

\begin{document}

% Title

\section*{详解Lanczos方法与线性方程组的Lanczos迭代法}

%[aside_content]

前面详细讨论了基于[label]Gram-Schmidt正交化[/label]方法的[label]Arnoldi方法[/label],以及改进的和不完全正交化的Arnoldi方法(见基于经典Gram-Schmidt正交化方法的Arnoldi算法一文和改进的和不完全正交化的Arnoldi算法一文)。进而,得到了求解线性方程组的Arnoldi方法,分别对应完全正交化方法([label]FOM[/label])(见线性方程组的Arnoldi方法之:FOM)、不完全正交化方法([label]IOM[/label])和直接不完全正交化方法([label]DIOM[/label])(见不完全(IOM)和直接不完全(DIOM)正交化方法)。此外,还详细讨论了另一种称为广义极小残差方法([label]GMRES[/label])的现代迭代法,以及相关的变形(见基本GMRES方法及其变形一文)。本文将讨论另一种特殊的迭代方法:[label]对称Lanczos方法[/label]。如果矩阵$A$是一个[label]对称矩阵[/label],对称Lanczos方法可以视作是改进的Arnoldi算法的一种简化。

\subsection*{对称Lanczos方法}

\subsubsection*{对称Lanczos方法的算法思想}

根据Arnoldi方法(见基于经典Gram-Schmidt正交化方法的Arnoldi算法一文),对于矩阵$A$和Krylov子空间上正交化后得到的一组基底向量$V_m$,它们满足关系式:

\begin{equation}
\label{eq:arnoldi}
V^T_mAV_m=H_m
\end{equation}

indent其中,$H_m$是一个$m \times m$阶[label]上Hessenberg矩阵[/label]。观察式(\ref{eq:arnoldi}),如果$A$是对称矩阵,则上Hessenberg矩阵$H_m$也是对称矩阵,于是可以推知,$H_{m}=(h_{ij})$是[label]对称三对角矩阵[/label]。如果令$\alpha_j=h_{jj}$,$\beta_j=h_{j-1,j}$的话,则有:

\begin{equation}
\label{eq:tm}
\begin{aligned}
H_m^T&=(V_m^TAV_m)^T=V_m^TA^TV_m \\
&=V_m^TAV_m=H_m
\end{aligned}\quad,\quad
H_m=\begin{pmatrix}
\alpha_1 & \beta_2 & & & \\
\beta_2 & \alpha_2 & \beta_3 & & \\
& \ddots & \ddots & \ddots & \\
& & \beta_{m-1} & \alpha_{m-1} & \beta_m \\
& & & \beta_m & \alpha_m
\end{pmatrix}=T_m
\end{equation}

在改进的Arnoldi方法的算法实现中(见改进的和不完全正交化的Arnoldi算法一文),外循环第$j$步时,内循环$i$从$1\thicksim j$,计算$h_{ij}$和$\boldsymbol{w}_j$;然后得到$h_{j+1,j}$以及正交化向量$\boldsymbol{v}_{j+1}$。观察式(\ref{eq:tm}),显然对于Lanczos方法的外循环第$j$步,其内循环只需执行两步:

- 当$i=j-1$时,$\boldsymbol{w}_j$的更新需要用到$\beta_j=h_{j-1,j}$,根据矩阵$T_m$的对称性,它等于上一步已经得到的$h_{j+1,j}$,因此可直接对$\boldsymbol{w}_j$进行更新;

- 当$i=j$时,需要先计算$\alpha_j=h_{jj}$,而它便等于$(\boldsymbol{w}_j,\boldsymbol{v}_j)$;于是便能进一步更新$\boldsymbol{w}_j$,并得到$h_{j+1,j}=\beta_{j+1}=h_{j,j+1}$,为下一步所用。

\subsubsection*{对称Lanczos方法的算法实现}

基于上述的分析,便很容易得到对称Lanczos方法的算法实现。对于给定对阵矩阵$A\in\mathbb{R}^{n \times n}$,向量$\boldsymbol{v}_1\in\mathbb{R}^n$以及正整数$m$(这里假定$\|\boldsymbol{v}_1\|_2=1$),并且令$\beta_1\equiv0,\boldsymbol{v}_0\equiv0$,即:$\beta_1\boldsymbol{v}_0=0$,则对称Lanczos方法的算法实现如下:

%[latex mode=1]
%[+preamble]
%\usepackage[titlenotnumbered,lined,linesnumbered]{algorithm2e}
%[/preamble]
\TitleOfAlgo{Symmetric Lanczos}
\begin{algorithm}[H]
\For{$j\gets 1$ \KwTo $m$}{
	$\boldsymbol{w}_j\gets A\boldsymbol{v}_j-\beta_j\boldsymbol{v}_{j-1}$\;
	$\alpha_j\gets(\boldsymbol{w}_j,\boldsymbol{v}_j)$\;
	$\boldsymbol{w}_j\gets\boldsymbol{w}_j-\alpha_j\boldsymbol{v}_j$\;
	$\beta_{j+1}\gets\|\boldsymbol{w}_j\|_2$\;
	\If{$\beta_{j+1}=0$}{
		Stop\;
	}
	$\boldsymbol{v}_{j+1}\gets\boldsymbol{w}_j/\beta_{j+1}$\;
}
\end{algorithm}
%[/latex]

上述对称Lanczos方法的算法实现,实际就是改进的Arnoldi方法的一种特殊情况。所以有:

\begin{eqnarray*}
& AV_m=V_mT_m+\boldsymbol{w}_m\boldsymbol{e}_m^T=V_{m+1}\bar{T}_m \\
& V_m^TAV_m=T_m
\end{eqnarray*}

indent其中,$\bar{T}_m$是把$T_m$的最后一行添上一行$(0,\cdots,0,\beta_{m+1})$后得到的$(m+1)\times m$阶矩阵。而如果要得到完整的$\{\boldsymbol{v}_j\}_{j=1}^{m+1}$,则需要有$\{\beta_j\}_{j=2}^{m+1}$都不为$0$。

\subsection*{线性方程组的直接Lanczos迭代方法}

改进的Arnoldi方法可以用于线性方程组$A\boldsymbol{x}=\boldsymbol{b}$的迭代求解,对应的算法称为完全正交化方法(FOM)(见线性方程组的Arnoldi方法之:FOM一文);如果基于不完全正交化的Arnoldi方法(见改进的和不完全正交化的Arnoldi算法一文),对应的迭代算法称为不完全正交化方法(IOM),及其变形直接不完全正交化方法(DIOM)(见不完全(IOM)和直接不完全(DIOM)正交化方法一文)。由于对称Lanczos方法是改进的Arnoldi方法的一种特殊情况,因此线性方程组的Lanczos迭代法的推导几乎和Arnoldi方法推导FOM、IOM、DIOM迭代法的过程完全一致。

[alert type="info"]通过简化重新开始的完全正交化方法(Restarted FOM),即可得到重新开始的Lanczos迭代法,其根本便是将核心的Arnoldi方法替换为上述的Lanczos方法。这里将不再冗述。[/alert]

\subsubsection*{直接Lanczos迭代方法的算法思想}

不完全正交化Arnoldi方法的核心思想是:对于向量$\boldsymbol{v}_j$的正交化,并不要求其与之前所有的$\{\boldsymbol{v}_i\}_{i=1}^{j-1}$都正交,而只需与它前面的$k$个$\boldsymbol{v}_i$正交(见改进的和不完全正交化的Arnoldi算法一文),且最终得到的是一个[label]上带宽[/label]为$k-1$的上Hessenberg矩阵$H_m$。

根据前面所讨论的,若矩阵$A$是[label]对称矩阵[/label],最终得到的$H_m=T_m$是一个[label]对称三对角矩阵[/label],也就是说其上带宽为$1$。因此,可以把对称Lanczos方法看作是$k=2$的不完全正交化Arnoldi方法;而若将后者用于求解线性方程组的话,便可以得到IOM和DIOM方法。因此,如果将对称Lanczos方法用于求解线性方程组,则完全可以看作是基于不完全正交化算法($k=2$)的DIOM算法,并称其为直接Lanczos方法。

根据$k=2$时的直接不完全正交化方法(DIOM)的推导,对$T_m$进行[label]LU分解[/label]的话可以得到:$T_m=L_mU_m$,其矩阵形式可以表示为:

\begin{align}
\label{eq:mattm}
\begin{split}
T_m&=\begin{pmatrix}
\alpha_1 & \beta_2 & & & \\
\beta_2 & \alpha_2 & \beta_3 & & \\
& \ddots & \ddots & \ddots & \\
& & \beta_{m-1} & \alpha_{m-1} & \beta_m \\
& & & \beta_m & \alpha_m
\end{pmatrix} \\
&=\begin{pmatrix}
1 & & & & \\
l_{21} & 1 & & & \\
& \ddots & \ddots & & \\
& & l_{m-1,m-2} & 1 & \\
& & & l_{m,m-1} & 1
\end{pmatrix}
\begin{pmatrix}
u_{11} & \beta_2 & & & \\
& u_{22} & \beta_3 & & \\
& & \ddots & \ddots & \\
& & & u_{m-1,m-1} & \beta_m \\
& & & & u_{mm}
\end{pmatrix}
\end{split}
\end{align}

第$m$步时,需要更新$T_m$的LU分解,也就是计算$l_{m,m-1}$,以及$\{u_{im}\}_{i=\max\{1,m-k+1\}}^m$。当$k=2$时,实际只需要计算$u_{m-1,m}$和$u_{mm}$即可,且其中$u_{m-1,m}=\beta_m$。因此,根据式(\ref{eq:mattm}),$l_{m,m-1}$和$u_{mm}$可以通过如下方式求得:

\begin{equation}
\label{eq:lu}
\left.
\begin{aligned}
l_{m,m-1}u_{m-1,m-1}&=\beta_m \\
l_{m,m-1}\beta_m+u_{mm}&=\alpha_m
\end{aligned}\quad\right\}\Longrightarrow\quad
\begin{aligned}
l_{m,m-1}&=\dfrac{\beta_m}{u_{m-1,m-1}} \\
u_{mm}&=\alpha_m-l_{m,m-1}\beta_m
\end{aligned},\quad
u_{11}=\alpha_1,\quad m\geq 2
\end{equation}

\subsubsection*{直接Lanczos迭代方法的算法实现}

直接不完全正交化方法(DIOM)的算法关键在于更新矩阵$H_m$的LU分解,而上述讨论给出了$k=2$时矩阵$T_m$的LU分解更新流程。在保持DIOM算法中的其他操作都不变的情况下,便得到了直接Lanczos迭代方法的算法实现。对于给定对称矩阵$A\in\mathbb{R}^{n \times n}$,$\boldsymbol{b},\boldsymbol{x}_0\in\mathbb{R}^{n}$,直接Lanczos迭代方法的算法实现如下:

%[latex mode=1]
%[+preamble]
%\usepackage[titlenotnumbered,lined,linesnumbered]{algorithm2e}
%[/preamble]
\TitleOfAlgo{Direct Version of Lanczos}
\begin{algorithm}[H]
compute $\boldsymbol{r}_0\gets\boldsymbol{b}-A\boldsymbol{x}_0$, $\lambda\gets\|\boldsymbol{r}_0\|_2$, $\boldsymbol{v}_1\gets\boldsymbol{r}_0/\lambda$\;
set $\beta_1=0$, $\boldsymbol{v}_0=0$, $\boldsymbol{p}_0=0$\;
\For{$m\gets 1,2,\cdots$}{
	$\boldsymbol{w}_m\gets A\boldsymbol{v}_m-\beta_m\boldsymbol{v}_{m-1}$\;
	$\alpha_m\gets(\boldsymbol{w}_m,\boldsymbol{v}_m)$\;
	$\boldsymbol{w}_m\gets\boldsymbol{w}_m-\alpha_m\boldsymbol{v}_m$\;
	$\beta_{m+1}\gets\|\boldsymbol{w}_m\|_2$\;
	\If{$\beta_{m+1}=0$}{
		Stop\;
	}
	$\boldsymbol{v}_{m+1}\gets\boldsymbol{w}_m/\beta_{m+1}$\;
	\eIf{$m=1$}{
		$u_{mm}\gets\alpha_m$\;
	}{
		$l_{m,m-1}\gets\beta_m/u_{m-1,m-1}$\;
		$u_{m-1,m}\gets\beta_m$\;
		$u_{mm}\gets\alpha_m-l_{m,m-1}\beta_m$\;
	}
	\eIf{$m=1$}{
		$\xi_m\gets\lambda$\;
	}{
		$\xi_m\gets-l_{m,m-1}\xi_{m-1}$\;
	}
	$\boldsymbol{p}_m\gets u_{mm}^{-1}\left(\boldsymbol{v}_m-\beta_m\boldsymbol{p}_{m-1}\right)$\;
	$\boldsymbol{x}_m=\boldsymbol{x}_{m-1}+\xi_m\boldsymbol{p}_m$\;
	\If{$\boldsymbol{x}_m$ has converged}{
		Stop\;
	}
}
\end{algorithm}
%[/latex]

\subsubsection*{直接Lanczos迭代方法的算法分析}

对于上述直接Lanczos迭代方法,有以下几点需要说明:

- 和直接不完全正交化方法(DIOM)类似,直接Lanczos迭代方法也以以$m$作为算法最外层循环(第$3$行)的条件,每一次迭代的结果$\boldsymbol{x}_m$都要用到上一次的$\boldsymbol{x}_{m-1}$;

- 算法的第$4\thicksim11$行便是完整的对称Lanczos方法,并得到对称三对角矩阵$T_M$;

- 算法的第$12\thicksim18$行,便是对$T_m$的LU分解进行更新,即根据式(\ref{eq:lu})计算$l_{m-1,m}$和$u_{mm}$,其中$u_{m-1,m}=\beta_m$。特别地,当$m=1$时,只需要计算$u_{mm}$即可;

- 算法的第$19\thicksim28$行只需看作是直接不完全正交化方法(DIOM)在$k=2$时的特例即可。其主要工作就是计算当前步的$\xi_m$和$\boldsymbol{p}_m$,并结合上一步的$\boldsymbol{x}_{m-1}$,计算得到当前步的$\boldsymbol{x}_m$,并判断其是否满足精度要求。

特别需要注意的是算法的第$24$行,在直接不完全正交化方法(DIOM)中,$\boldsymbol{p}_m$的计算形式为:

\begin{equation}
\label{eq:pmdiom}
\boldsymbol{p}_m=\dfrac{1}{u_{mm}}\left(\boldsymbol{v}_m-\sum_{i=\max\{1,m-k+1\}}^{m-1}u_{im}\boldsymbol{p}_i\right)
\end{equation}

indent而在直接Lanczos迭代法中,由于$k=2$,所以式(\ref{eq:pmdiom})可以进一步简化为:

\begin{align*}
\label{eq:pmdlan}
\begin{split}
\boldsymbol{p}_m&=\dfrac{1}{u_{mm}}\left(\boldsymbol{v}_m-\sum_{i=\max\{1,m-1\}}^{m-1}u_{im}\boldsymbol{p}_i\right) \\
&=\dfrac{1}{u_{mm}}\left(\boldsymbol{v}_m-u_{m-1,m}\boldsymbol{p}_{m-1}\right)=\dfrac{1}{u_{mm}}\left(\boldsymbol{v}_m-\beta_m\boldsymbol{p}_{m-1}\right)
\end{split}
\end{align*}

%[/aside_content]


% H2, H3
\subsection*{dsfsdt}
%[latex mode=1]
%[+preamble]
%\usepackage[titlenotnumbered,lined,linesnumbered]{algorithm2e}
\let\oldnl\nl
\newcommand{\nonl}{\renewcommand{\nl}{\let\nl\oldnl}}
%[/preamble]
%\quicklatex{size=14}
\TitleOfAlgo{BLAS-3 Gaussian Elimination with Partial Pivoting}
\begin{algorithm}[H]
\For{$l\gets 1$ \KwTo $n-1$}{
$k\gets (l-1)b+1$\;
factorize $PA^{(l)}=LU$ using BLAS-2\;
store $L$ and $U$ in $A$\;
left multiply $A(1:n,k+b:n)$ by $P$\;
$A(k:k+b-1,k+b:n)\gets T^{-1}A(k:k+b-1,k+b:n)$\;
\nonl\Indp where $T$ is the unit-lower-trian of $A(k:k+b-1,k:k+b-1)$\;
\DontPrintSemicolon\Indm$A(k+b:n,k+b:n)\gets A(k+b:n,k+b:n)$\;
\PrintSemicolon\nonl\Indp$-A(k+b:n,k:k+b-1)A(k:k+b-1,k+b:n)$\;
}
\end{algorithm}
%[/latex]

% H4
% \subsubsection*{}

\end{document}
