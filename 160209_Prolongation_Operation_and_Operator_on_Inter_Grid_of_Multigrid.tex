\documentclass[12pt, UTF8, nofonts]{ctexart}

\setCJKmainfont[BoldFont=STHeiti,ItalicFont=STKaiti]{STKaiti}
\setCJKsansfont[BoldFont=STHeiti]{STXihei}
\setCJKmonofont{STFangsong}

\usepackage[letterpaper,left=.5in,right=.5in,top=1in,bottom=.5in]{geometry}
\usepackage{afterpage, color, multirow}
\usepackage{amsmath}
\usepackage{amsfonts}
\usepackage{amssymb}
\usepackage[titlenotnumbered,lined,linesnumbered]{algorithm2e}

\definecolor{bgcolor}{RGB}{13, 12, 12}
\definecolor{ttcolor}{RGB}{0, 255, 0}
\pagecolor{bgcolor}\afterpage{\nopagecolor}
\makeatletter
\newcommand{\globalcolor}[1]{\color{#1}\global\let\default@color\current@color}
\makeatother
\AtBeginDocument{\globalcolor{ttcolor}}

\begin{document}

%[aside_content]

\section*{详解多重网格内部网格上的延拓操作}

上文概述了基于 Gauss-Seidel 迭代[label]光滑子[/label](smoother)的多重网格(Multigrid)方法,及其通用框架下的算法描述,并由此引申出了两种类型的多重网格:V-型和 W-型多重网格(见几何多重网格方法概述一文)。事实上,针对同一个原始离散线性方程组,多重网格方法需要在不同的网格问题间进行来回转换。简化来看,就是在两种网格之间:$\Omega_h$代表的[label]细网格[/label]和$\Omega_{H}$代表的[label]粗网格[/label]之间,进行B网格转换。进一步简化问题模型,可以仅考虑$H=2h$时的情形。

[alert type="info"]通常来说,粗网格问题对应的网格大小是细网格问题对应的网格大小的$2^d$倍,其中$d$代表的是空间维数。[/alert]

上文残留的一个问题是多重网格方法内部网格上的的两个核心操作:[label]限制操作[/label](restriction)和[label]延拓操作[/label]](prolongation)。本文将详细讨论延拓操作,及相应的延拓算子。限制操作将另文阐述。

\subsection*{延拓操作和延拓算子}

延拓操作是将粗网格$\Omega_H$上的一个向量(在多重网格方法中就是求解粗网格上的残差方程后所得到的误差的精确解)延拓到细网格上相应的一个向量,它可以用[label]延拓算子[/label](prolongation operator)$\boldsymbol{P}_H^h$来表示,记为如下形式:

\begin{equation}
  \label{eq:prolong}
  \boldsymbol{P}_H^h\;:\;\Omega_H \to \Omega_h
\end{equation}

延拓算子的一个最简单的实现方式是通过[label]线性插值[/label](linear interpolation)。下面将分别讨论一维和二维离散椭圆方程上的延拓算子。

\subsubsection*{一维离散椭圆方程的延拓算子}

考虑一维离散椭圆方程问题模型,通常将其离散化为$x_0,\ldots,x_{n+1}$这样的$n+2$个点,其中$x_0$和$x_{n+1}$是边界点。换言之,B内部网格上共有$n$个节点($x_1,\ldots,x_n$)。通常假定$n$的值为B奇数,方便计算中间点的位置。根据此前给出延拓操作的定义,对于给定的两个向量,即粗网格上的向量$\boldsymbol{v}^{2h}$和细网格上的向量$\boldsymbol{v}^h$,它们分量之间的对应关系(从粗网格延拓到细网格上)可以表示为:

\begin{equation}
  \label{eq:1dv2hvh}
  \left\{\; \begin{array}{lcl}
    v_{2j}^{h} &=& v_{j}^{2h} \\
    v_{2j+1}^{h} &=& \left( v_{j}^{2h} + v_{j+1}^{2h} \right) \big/ 2 \\
  \end{array} \right. \quad j = 0,\ldots,\dfrac{n+1}{2}
\end{equation}

这里需要注意的是下标$j$的范围。由于细网格上共有$n+2$个点,且细网格上最后一个点($n+1$)对应于粗网格上的最后一个点,因此粗网格上$j$的下标为$0,\ldots,(n+1)/2$。换言之,粗网格上$\left[v_{j}^{2h}\right]_{j=0}^{(n+1)/2}$共$(n+3)/2$个点,可以延拓为细网格上的$n+2$个点。

式(\ref{eq:1dv2hvh})可以表示为[label]线性变换[/label]的形式,即:$\boldsymbol{v}^h=\boldsymbol{P}_{2h}^{h}\boldsymbol{v}^{2h}$,其中系数矩阵$\boldsymbol{P}_{2h}^{h}$即为延拓算子。进一步,可以将其表示为如下的矩阵形式(只考虑B内部节点):

\begin{equation}
  \label{eq:1dmatform}
  \boldsymbol{v}^h = \dfrac{1}{2}
  \begin{pmatrix}
    1 & & & \\
    2 & & & \\
    1 & 1 & & \\
    & 2 & & \\
    & 1 & 1 & \\
    & & \vdots & \\
    & & 1 & 1 \\
    & & & 2 \\
    & & & 1 \\
  \end{pmatrix} \boldsymbol{v}^{2h}
\end{equation}

以一维离散椭圆方程上细网格内部节点个数为$n=7$的情况为例。对应的粗网格节点为$v^{2h}_{0},\ldots,v^{2h}_{4}$,则其内部节点为$\{v^{2h}_{1},v^{2h}_{2},v^{2h}_{3}\}$,则延拓操作可以表示为如下的矩阵形式:

\[
  \dfrac{1}{2}
  \begin{pmatrix}
    1 & & \\ 2 & & \\ 1 & 1 & \\ & 2 & \\ & 1 & 1 \\ & & 2 \\ & & 1 \\
  \end{pmatrix}
  \begin{pmatrix}
    v^{2h}_{1} \\ v^{2h}_{2} \\ v^{2h}_{3}
  \end{pmatrix} =
  \begin{pmatrix}
    v^{h}_{1} \\ v^{h}_{2} \\ v^{h}_{3} \\ v^{h}_{4} \\ v^{h}_{5} \\
    v^{h}_{6} \\ v^{h}_{7} \\
  \end{pmatrix}
\]

【Figure】

[alert type="error"]B注意:由于网格边界上的点为$0$,因此有$v^{h}_{1}=(v^{2h}_{0}+v^{2h}_{1})/2=v^{2h}_{1}/2$。上图给出了该例对应的节点示意图。[/alert]

[alert type="info"]如果经过延拓操作后得到的细网格内部节点个数为$n$,则延拓算子$\boldsymbol{P}_{2h}^{h}$可以视作是$\mathbb{R}^{(n-1)/2}\to\mathbb{R}^{n}$的线性变换,其矩阵形式的阶数为$n \times (n-1)/2$。[/alert]

\subsubsection*{二维离散椭圆方程的延拓算子}

二维离散椭圆方程上的延拓算子也可以通过类似的方式得到。如果用$v_{ij}$来表示坐标点$(x_i,y_j)$关于$\boldsymbol{v}$的函数,那么二维空间上的延拓算子可以使用[label]双线性插值[/label]来实现。

具体来说,用$\boldsymbol{P}_{x,2h}^{h}$来表示$x$方向上的插值;而用$\boldsymbol{P}_{y,2h}^{h}$来表示$y$方向上的插值。依据此前的惯例,二维空间上的细网格在$x$方向上共有$n+2$个点,$y$方向上共有$m+2$个点,那么对应的粗网格在$x$方向上共有$(n+3)/2$个节点,在$y$方向上共有$(m+3)/2$个节点。于是,二维空间上粗网格的点延拓到细网格上的点的对应关系可以表示为:

\begin{equation}
  \label{eq:2dv2hvh}
  \left\{\; \begin{array}{lcl}
    v_{2i,2j}^{h} &=& v_{ij}^{2h} \\
    v_{2i+1,2j}^{h} &=& \left(v_{ij}^{2h} + v_{i+1,j}^{2h}\right) \big/ 2 \\
    v_{2i,2j+1}^{h} &=& \left(v_{ij}^{2h} + v_{i,j+1}^{2h}\right) \big/ 2 \\
    v_{2i+1,2j+1}^{h} &=& \left(v_{ij}^{2h} + v_{i+1,j}^{2h} + v_{i,j+1}^{2h} + v_{i+1,j+1}^{2h} \right) \big/ 4 \\
  \end{array} \right. \quad
  \left\{ \begin{array}{lcl}
    i &=& 0,\ldots,\frac{n+1}{2} \\
    j &=& 0,\ldots,\frac{m+1}{2} \\
  \end{array} \right.
\end{equation}

事实上,二维空间上的延拓可以先沿着$x$方向使用延拓算子$\boldsymbol{P}_{x,2h}^{h}$进行插值,然后再沿着$y$方向使用延拓算子$\boldsymbol{P}_{y,2h}^{h}$插值。更为重要的是,二维空间上的双线性插值可以表示为两个一维空间延拓算子的 [label]Kronecker 积[/label]的形式,即:

\begin{equation}
  \label{eq:2dkronecker}
  \boldsymbol{P}_{2h}^{h} = \boldsymbol{P}_{y,2h}^{h} \otimes \boldsymbol{P}_{x,2h}^{h}
\end{equation}

以$n=7,m=5$为例进行说明。细网格总节点数(边界网格节点$+$内部网格节点)为$(n+2)\times(m+2)=9\times7$,内部网格节点数为$n \times m=7\times5$;粗网格总结点数为$(n+3)/2 \times (m+3)/2=5\times4$,其内部网格节点数为$(n-1)/2\times(m-1)/2=3\times2$。下图给出了该情况下的节点对应示意图,它与式(\ref{eq:2dv2hvh})给出的关系式是相一致的。

进一步,可以验证式(\ref{eq:2dkronecker})。前文提到,一维延拓算子$\boldsymbol{P}_{2h}^{h}$可以视作是$\mathbb{R}^{(n-1)/2}\to\mathbb{R}^{n}$的线性变换。那么在这里,由$n=7$可知$\boldsymbol{P}_{x,2h}^{h}$是$\mathbb{R}^{3}\to\mathbb{R}^{7}$的线性变换;由$m=5$可知$\boldsymbol{P}_{y,2h}^{h}$是$\mathbb{R}^{2}\to\mathbb{R}^{5}$的线性变换。那么它们的 Kronecker 积,也就是二维延拓算子$\boldsymbol{P}_{2h}^{h}$的矩阵形式为:

\begin{equation}
  \label{eq:2dmatform}
  \begin{array}{lcl}
    \boldsymbol{P}_{x,2h}^{h} &=& \dfrac{1}{2}
    \begin{pmatrix}
      1 & & \\ 2 & & \\ 1 & 1 & \\ & 2 & \\ & 1 & 1 \\ & & 2 \\ & & 1 \\
    \end{pmatrix} = \dfrac{1}{2}\boldsymbol{I}_{x,2h}^{h}\\
    \boldsymbol{P}_{y,2h}^{h} &=& \dfrac{1}{2}
    \begin{pmatrix}
      1 & \\ 2 & \\ 1 & 1 \\ & 2 \\ & 1 \\
    \end{pmatrix} \\
  \end{array} \quad\Rightarrow\quad
  \begin{aligned}
    \boldsymbol{v}^{h} &= \boldsymbol{P}_{2h}^{h}\boldsymbol{v}^{2h} = \boldsymbol{P}_{y,2h}^{h} \otimes \boldsymbol{P}_{x,2h}^{h} \boldsymbol{v}^{2h} \\
    &= \dfrac{1}{4}
    \begin{pmatrix}
      \boldsymbol{I}_{x,2h}^{h} & \\ 2\boldsymbol{I}_{x,2h}^{h} & \\ \boldsymbol{I}_{x,2h}^{h} & \boldsymbol{I}_{x,2h}^{h} \\ & \boldsymbol{I}_{x,2h}^{h} \\ & \boldsymbol{I}_{x,2h}^{h} \\
    \end{pmatrix}
    \begin{pmatrix}
      v_{11}^{2h} \\ v_{21}^{2h} \\ v_{31}^{2h} \\ \cline{1-1} v_{12}^{2h} \\ v_{22}^{2h} \\ v_{32}^{2h}
    \end{pmatrix} =
    \begin{pmatrix}
      v_{11}^{h} \\ \vdots \\ v_{71}^{h} \\ \cline{1-1} \vdots \\ \cline{1-1}
      v_{13}^{h} \\ \vdots \\ v_{73}^{h} \\ \cline{1-1} \vdots \\ \cline{1-1}
      v_{15}^{h} \\ \vdots \\ v_{75}^{h}
    \end{pmatrix} \\
  \end{aligned}
\end{equation}

已知$m \times n$阶矩阵$\boldsymbol{A}$和$p \times q$阶矩阵$\boldsymbol{B}$的 Kronecker 积的阶数为$mp \times nq$。这里$\boldsymbol{P}_{x,2h}^{h}$的阶数为$n\times(n-1)/2=7\times3$,$\boldsymbol{P}_{y,2h}^h$的阶数为$m\times(m-1)/2=5\times2$,那么二维延拓算子$\boldsymbol{P}_{2h}^{h}$的阶数为:$nm \times (n-1)(m-1)/4 = 35 \times 6$。可以验证如下两个节点满足式(\ref{eq:2dv2hvh})的关系式:

\[
  \begin{aligned}
    v_{33}^{h} &= \dfrac{1}{4}\left(v_{11}^{2h} + v_{21}^{2h} + v_{12}^{2h} + v_{22}^{2h}\right) \\
    v_{53}^{h} &= \dfrac{1}{4}\left(v_{21}^{2h} + v_{31}^{2h} + v_{22}^{2h} + v_{32}^{2h}\right) \\
  \end{aligned}
\]

\subsection*{延拓算子的 Stencil 记号}

通常可以使用 [label]stencil[/label] 记号来表示延拓算子。对于式(\ref{eq:1dmatform})中的延拓算子,可以使用如下记号表示:

\begin{equation}
  \label{eq:1dstencil}
  p = \left]\begin{matrix}
    \dfrac{1}{2} & 1 & \dfrac{1}{2}
  \end{matrix}\right[
\end{equation}

[alert type="error"]B注意:式(\ref{eq:1dstencil})中的B开括号代表[label]扇出[/label](fan-out)操作,其含义是它表示的是B列操作。换言之,每一个 stencil 都对应于粗网格中的一个点。具体来说,粗网格上的点$x_{i}^{2h}$对于细网格上的三个点$x_{2i-1}^{h},x_{2i}^h,x_{2i+1}^h$的贡献分别是$0.5,1,0.5$。[/alert]

那么,由一维情况下的延拓算子的 stencil 记号很容易推得二维情况下的 stencil 记号,粗网格上的每一个点对细网格上周围$9$个点都有贡献,即:

\begin{equation}
  \label{eq:2dstencil}
  p = \dfrac{1}{4}
  \left]\begin{matrix}
    1 & 2 & 1 \\
    2 & 4 & 2 \\
    1 & 2 & 1 \\
  \end{matrix}\right[
\end{equation}

[alert type="info"]值得一提的是,二维 stencil 可以看作是一维 stencil $p$及其转置$p^T$的 Kronecker 积。证明如下。[/alert]

\[
  p \otimes p^T =
  \left]\begin{matrix}
    \frac{1}{2} & 1 & \frac{1}{2}
  \end{matrix}\right[ \otimes
  \left]\begin{matrix}
    \frac{1}{2} \\ 1 \\ \frac{1}{2}
  \end{matrix}\right[ =
  \left]\begin{matrix}
    \frac{1}{4} & \frac{1}{2} & \frac{1}{4} \\
    \frac{1}{2} & 1 & \frac{1}{2} \\
    \frac{1}{4} & \frac{1}{2} & \frac{1}{4} \\
  \end{matrix}\right[
\]

%[/aside_content]

\end{document}
