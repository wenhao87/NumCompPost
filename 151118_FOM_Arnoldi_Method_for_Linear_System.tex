\documentclass[UTF8,nofonts]{ctexart}

\setCJKmainfont[BoldFont=STHeiti,ItalicFont=STKaiti]{STKaiti}
\setCJKsansfont[BoldFont=STHeiti]{STXihei}
\setCJKmonofont{STFangsong}

\usepackage[letterpaper,left=1in,top=1in]{geometry}
\usepackage{multirow}
\usepackage{amsmath}
\usepackage{amsfonts}
\usepackage{amssymb}
\usepackage[titlenotnumbered,lined,linesnumbered]{algorithm2e}

\begin{document}

% Title

\section*{FOM}

%[aside_content]

求解[label]线性方程组[/label]$A\boldsymbol{x}=\boldsymbol{b}$的[label]现代迭代法[/label]都是在[label]Krylov子空间[/label]框架下进行推导的(见由相伴矩阵引入Krylov子空间法一文)。在$\mathbb{R}^n$中,关于$A$和$\boldsymbol{r}_0$的Krylov子空间定义为:

\[
\mathcal{K}_m\equiv\mathcal{K}_{m}(A,\boldsymbol{r}_0)=
\text{span}\{\boldsymbol{r}_0,A\boldsymbol{r}_0,\cdots,A^{m-1}\boldsymbol{r}_0\}
\]

indent其中,$m$为给定正整数,$\boldsymbol{r}_0=\boldsymbol{b}-A\boldsymbol{x}_0$为任意选取的$n$维初始向量$\boldsymbol{x}_0$的初始[label]残差[/label],并以$\boldsymbol{r}_0$作为Krylov子空间的[label]种子向量[/label],也称为[label]循环向量[/label]。

为了结合[label]投影方法[/label](见迭代法中投影过程的基本框架一文),需要选取Krylov子空间$\mathcal{K}_m$上的一组[label]正交基[/label]。主要有两种求解正交基的正交化方法:基于[label]Gram-Schmidt正交化[/label]方法的Arnoldi方法,简称[label]Arnoldi方法[/label],它又可以分为经典的和改进的两种Gram-Schmidt正交化(见基于经典Gram-Schmidt正交化方法的Arnoldi算法和改进的和不完全正交化的Arnoldi算法两篇文章);另一种则是基于[label]Householder正交化[/label]的Arnoldi方法,简称为Householder正交化。

相比基于Gram-Schmidt正交化的Arnoldi方法,Householder正交化具有更好的数值稳定性,但计算量近乎于Arnoldi方法的两倍(见基于Householder正交化方法的Arnoldi算法一文),但这两者在数学上是B完全等价的。设$V_m=\{\boldsymbol{v}_1,\boldsymbol{v}_2,\cdots,\boldsymbol{v}_m\}$中的列向量为Krylov子空间$\mathcal{K}_m$的[label]标准正交基[/label],有:

\begin{equation}
\label{eq:avvh}
AV_m=V_{m+1}\bar{H}_m=V_mH_m+h_{m+1,m}\boldsymbol{v}_{m+1}\boldsymbol{e}_m^T
\quad\Longleftrightarrow\quad
V_m^TAV_m=H_m
\end{equation}

indent其中,$\boldsymbol{v}_1=\boldsymbol{r}_0/\|\boldsymbol{r}_0\|_2$,而$H_m$为$m \times m$阶上[label]Hessenberg矩阵[/label]。

\subsection*{Arnoldi方法的Krylov子空间框架}

在迭代法中投影过程的基本框架一文中提到,投影过程就是在如下的[label]仿射空间[/label](affine space)中寻找近似解$\boldsymbol{x}_m$:

\begin{equation}
\label{eq:xkm}
\boldsymbol{x}_0+\mathcal{K}_m=\{\boldsymbol{x}_0+\boldsymbol{x}:\boldsymbol{x}\in\mathcal{K}_m\}
\end{equation}

indent并使其残差满足[label]Petrov-Galerkin条件[/label]:$\boldsymbol{b}-A\boldsymbol{x}_m\perp\mathcal{L}_m$,其中$\mathcal{L}_m$是另一个Krylov子空间。本文所要讨论的用于求解线性方程组的Arnoldi方法,都要求$\mathcal{L}_m=\mathcal{K}_m$。换言之,残差满足条件:

\begin{equation}
\label{eq:galer}
\boldsymbol{b}-A\boldsymbol{x}_m\perp\mathcal{K}_m
\end{equation}

indent式(\ref{eq:galer})也称为[label]Galerkin条件[/label]。有了Galerkin条件之后,便可以进行线性方程组$A\boldsymbol{x}=\boldsymbol{b}$的求解了,其方法统称为线性方程组的Arnoldi方法,主要包括完全正交化方法([label]FOM[/label])和相关的变形版本,如:重新开始的FOM方法([label]Restarted FOM[/label])、不完全正交化方法([label]IOM[/label])、直接不完全正交分解([label]DIOM[/label])等。本文主要讨论前两种Arnoldi方法,IOM和DIOM方法将另文详述。

[alert type="error"]注意:如前所述,这里所讨论的Arnoldi方法及其相关算法都指的是基于Gram-Schmidt正交化方法的Arnoldi方法。[/alert]

\subsection*{线性方程组的Arnoldi方法的基本原理}

由于式(\ref{eq:avvh})中$V_m$的列向量是$\mathcal{K}_m$的标准正交基,所以式(\ref{eq:galer})等价于:

\begin{equation}
\label{eq:vtkm}
V_m^T(\boldsymbol{b}-A\boldsymbol{x}_m)=0
\end{equation}

indent而另一方面,由于$\boldsymbol{x}_m$在仿射空间$\boldsymbol{x}_0+\mathcal{K}_m$中,而$\mathcal{K}_m$中的任意向量都可以表示为其标准向量基的[label]线性组合[/label],因此根据式(\ref{eq:xkm}),$\boldsymbol{x}_m$可以表示为如下形式:

\begin{equation}
\label{eq:xvy}
\boldsymbol{x}_m=\boldsymbol{x}_0+V_m\boldsymbol{y}_m,\quad\boldsymbol{y}_m\in\mathbb{R}^m
\end{equation}

indent结合式(\ref{eq:vtkm})和式(\ref{eq:xvy}),可以得到:

\begin{align}
\label{eq:vrhy}
\begin{split}
0=V_m^T(\boldsymbol{b}-A\boldsymbol{x}_m)&=V_m^T(\boldsymbol{b}-A(\boldsymbol{x}_0+V_m\boldsymbol{y}_m)) \\
&=V_m^T(\boldsymbol{r}_0-AV_m\boldsymbol{y}_m)=V_m^T\boldsymbol{r}_0-V_m^TAV_m\boldsymbol{y}_m
\end{split}
\end{align}

indent观察式(\ref{eq:vrhy})的最右端,结合此前的假定$\boldsymbol{v}_1=\boldsymbol{r}_0/\|\boldsymbol{r}_0\|_2$其中$V_m^T\boldsymbol{r}_0$可以表示为:

\[
\begin{pmatrix}\boldsymbol{v}_1^T\\\boldsymbol{v}_2^T\\\vdots\\\boldsymbol{v}^T_{m-1}\\\boldsymbol{v}_m^T\end{pmatrix}
\begin{pmatrix}~\\~\\\boldsymbol{v}_1\|\boldsymbol{r}_0\|_2\\~\\~\end{pmatrix}=
\begin{pmatrix}\|\boldsymbol{r}_0\|_2\\0\\\vdots\\0\\0\end{pmatrix}=\beta\boldsymbol{e}_1,\quad\beta=\|\boldsymbol{r}_0\|_2
\]

indent再结合式(\ref{eq:avvh})的$V_m^TAV_m=H_m$,于是最终式(\ref{eq:vrhy})可以表示为如下形式:

\begin{equation}
\label{eq:ym}
\begin{array}{lc}
& \beta\boldsymbol{e}_1-H_m\boldsymbol{y}_m=0 \\
\Longrightarrow & \boldsymbol{y}_m=(H_m)^{-1}(\beta\boldsymbol{e}_1)
\end{array}\quad
,\quad\beta=\|\boldsymbol{r}_0\|_2
\end{equation}

indent按式(\ref{eq:ym})求得$\boldsymbol{y}_m$后,便可根据式(\ref{eq:xvy})求得$\boldsymbol{x}_m$,并由$\boldsymbol{b}-A\boldsymbol{x}_m$确定其残差。然而观察式(\ref{eq:vrhy})的话可以发现,$\boldsymbol{x}_m$的残差$\boldsymbol{r}_m$的计算并不需要显式用到$\boldsymbol{x}_m$,因为:

\begin{align*}
\boldsymbol{b}-A\boldsymbol{x}_m&=\boldsymbol{b}-A(\boldsymbol{x}_0+V_m\boldsymbol{y}_m)=\boldsymbol{r}_0-AV_m\boldsymbol{y}_m \\
&=\boldsymbol{r}_0-(V_mH_m+h_{m+1,m}\boldsymbol{v}_{m+1}\boldsymbol{e}_m^T)\boldsymbol{y}_m \\
&=\boldsymbol{r}_0-V_mH_m(H_m)^{-1}(\beta\boldsymbol{e}_1)-(h_{m+1,m}\boldsymbol{v}_{m+1}\boldsymbol{e}_m^T)\boldsymbol{y}_m \\
&=\boldsymbol{r}_0-\beta\boldsymbol{v}_1-(h_{m+1,m}\boldsymbol{v}_{m+1}\boldsymbol{e}_m^T)\boldsymbol{y}_m \\
&=\boldsymbol{r}_0-\|\boldsymbol{r}_0\|_2(\boldsymbol{r}_0/\|\boldsymbol{r}_0\|_2)-(h_{m+1,m}\boldsymbol{v}_{m+1}\boldsymbol{e}_m^T)\boldsymbol{y}_m \\
&=-(h_{m+1,m}\boldsymbol{v}_{m+1}\boldsymbol{e}_m^T)\boldsymbol{y}_m
=-h_{m+1,m}(\boldsymbol{e}_m^T\boldsymbol{y}_m)\boldsymbol{v}_{m+1}
\end{align*}

indent另外,根据Arnoldi算法有:$h_{m+1,m}\geq0$,$\boldsymbol{v}_{m+1}$是[label]单位向量[/label],所以有:

\begin{equation}
\label{eq:rmnorm}\|\boldsymbol{b}-A\boldsymbol{x}_m\|_2=h_{m+1,m}|\boldsymbol{e}_m^T\boldsymbol{y}_m|
\end{equation}

indent所以,当$h_{m+1,m}$很小时,近似解$\boldsymbol{x}_m$的残差就很小。因此,可以用$h_{m+1,m}$是否很小来作为迭代是否终止的判断依据。特别地,当$h_{m+1,m}=0$时,此时的$\boldsymbol{x}_m$便是$A\boldsymbol{x}=\boldsymbol{b}$的解。

\subsection*{完全正交化方法(FOM)}

上述过程便是求解线性方程组的Arnoldi方法的基本原理和过程,结合其中的Krylov子空间$\mathcal{K}_m$的正交化过程后,便能得到[label]完全正交化方法[/label]([label]FOM[/label])。这里主要采用改进的Gram-Schmidt正交化方法来对Krylov子空间$\mathcal{K}_m$进行正交化,其算法和原理可以参见改进的和不完全正交化的Arnoldi算法一文。

\subsubsection*{完全正交化方法(FOM)算法流程}

完全正交化方法(FOM)的算法流程可以描述为:

- 给定一个正整数$m$和一个初始向量$\boldsymbol{x}_0$,计算其残差$\boldsymbol{r}_0$,将其归一化后作为Krylov子空间$\mathcal{K}_m$的种子向量$\boldsymbol{v}_1$;

- 根据改进的Gram-Schmidt正交化的Arnoldi方法,对Krylov子空间$\mathcal{K}_m$进行正交化,得到该子空间的一组[label]基底向量[/label]$V_m=\{\boldsymbol{v}_1,\boldsymbol{v}_2,\cdots,\boldsymbol{v}_m\}$,以及[label]上Hessenberg矩阵[/label]$H_m=\{h_{ij}\}$;

- 根据式(\ref{eq:ym}),利用求得的$H_m$可以计算得到$\boldsymbol{y}_m$;

- 根据式(\ref{eq:xvy}),计算得到$\boldsymbol{x}_m$,它便是线性方程组的近似解。

\subsubsection*{完全正交化方法(FOM)算法实现}

对于给定的$A\in\mathbb{R}^{n \times n}$,$\boldsymbol{b},\boldsymbol{x}_0\in\mathbb{R}^{n}$,以及正整数$m$,完全正交化方法(FOM)算法实现如下:

%[latex mode=1]
%[+preamble]
%\usepackage[titlenotnumbered,lined,linesnumbered]{algorithm2e}
%[/preamble]
\TitleOfAlgo{Full Orthogonalization Method(FOM)}
\begin{algorithm}[H]
compute $\boldsymbol{r}_0\gets\boldsymbol{b}-A\boldsymbol{x}_0$, $\beta\gets\|\boldsymbol{r}_0\|_2$, $\boldsymbol{v}_1\gets\boldsymbol{r}_0/\beta$\;
define $H_m=\{h_{ij}\}_{i,j=1,\cdots,m}$; set $H_m=0$\;
\For{$j\gets 1$ \KwTo $m$}{
	$\boldsymbol{w}_j\gets A\boldsymbol{v}_j$\;
	\For{$i\gets 1$ \KwTo $j$}{
		$h_{ij}\gets(\boldsymbol{w}_j,\boldsymbol{v}_i)$\;
		$\boldsymbol{w}_j\gets\boldsymbol{w}_j-h_{ij}\boldsymbol{v}_i$\;
	}
	$h_{j+1,j}\gets\|\boldsymbol{w}_j\|_2$\;
	\If{$h_{j+1,j}=0$}{
		set $m\gets j$; Goto 15\;
	}
	$\boldsymbol{v}_{j+1}\gets\boldsymbol{w}_j/h_{j+1,j}$\;
}
$\boldsymbol{y}_m\gets H_m^{-1}(\beta\boldsymbol{e}_1)$; $\boldsymbol{x}_m\gets\boldsymbol{x}_0+V_m\boldsymbol{y}_m$\;
\end{algorithm}
%[/latex]

\subsubsection*{完全正交化方法(FOM)算法说明}

对于上述算法,有以下几点需要特别说明:

- 算法第$3\thicksim14$行的循环便是基于改进的Gram-Schmidt正交化方法的Arnoldi方法,具体体现在内循环(第$5\thicksim8$行)中$\boldsymbol{w}_j$和$h_{ij}$的耦合计算;

- 第$j$步正交化过程中,如果$h_{j+1,j}=0$,则说明找到了精确解$\boldsymbol{x}_m$,将$j$作为Krylov子空间$\mathcal{K}_m$新的维数后,便可退出正交化过程(第$11\thicksim13$行);

- 第$15$行得到的$\boldsymbol{x}_m$,便是线性方程组在以$\boldsymbol{r}_0$作为种子向量的Krylov子空间$\mathcal{K}_m$上寻找到的近似解;

[alert type="error"]注意:如果$\boldsymbol{x}_m$的残差很大(如前所述,可以以$h_{m+1,m}$的值作为判断依据),那么可以将得到的$\boldsymbol{x}_m$作为新的$\boldsymbol{x}_0$,并重新开始新一轮的Arnoldi正交化计算。这就是所谓的[label]重新开始的FOM方法[/label]。[/alert]

\subsection*{重新开始的FOM方法}

\subsubsection*{重新开始的FOM方法算法实现}

显然,只需要在上述完全正交化方法(FOM)的最后加入精度判断条件,便能得到[label]重新开始的FOM方法[/label](这里以$h_{m+1,m}$的值作为判断依据),其算法实现如下:

%[latex mode=1]
%[+preamble]
%\usepackage[titlenotnumbered,lined,linesnumbered]{algorithm2e}
%[/preamble]
\TitleOfAlgo{Full Orthogonalization Method(FOM)}
\begin{algorithm}[H]
compute $\boldsymbol{r}_0\gets\boldsymbol{b}-A\boldsymbol{x}_0$, $\beta\gets\|\boldsymbol{r}_0\|_2$, $\boldsymbol{v}_1\gets\boldsymbol{r}_0/\beta$\;
define $H_m=\{h_{ij}\}_{i,j=1,\cdots,m}$; set $H_m=0$\;
\For{$j\gets 1$ \KwTo $m$}{
	\tcc{exactly the same procedure as FOM}
}
$\boldsymbol{y}_m\gets H_m^{-1}(\beta\boldsymbol{e}_1)$; $\boldsymbol{x}_m\gets\boldsymbol{x}_0+V_m\boldsymbol{y}_m$\;
\eIf{$h_{m+1,m}$ is small enough}{
	Stop\;
	}{
	$\boldsymbol{x}_0\gets\boldsymbol{x}_m$ and Goto 1\;
}
\end{algorithm}
%[/latex]

\subsubsection*{重新开始的FOM方法算法说明}

正如在基于Householder正交化方法的Arnoldi算法一文最后所提到的,随着$m$的增大,基于Gram-Schmidt正交化的Arnoldi方法,其计算量按$\mathcal{O}(m^2n)$级别增加;而存储量则以$\mathcal{O}(mn)$级别增加。因此,对于$n$很大的大型线性方程组来说,它同时也限制了最大可使用的$m$的值。此时,有两种补救办法:一种便是使用上述的重新开始的FOM方法,在$m$固定的情况下(计算量和存储量也固定),通过连续使用FOM方法,使最终结果达到满意的精度;另一种方法则是“缩减”正交化的工作量,也就是所谓的[label]不完全正交化方法(IOM)[/label],后续将另文详述。

其实,重新开始的FOM方法也有一种变形。由于合适的$m$值事先很难确定(可大可小),因此可以尝试从$m=1$开始,并在一次FOM方法完成后,将$m$的值增加$1$(改写算法第$9$行处),然后进行下一次的FOM方法,直到$m$达到预设的最大值$m_{max}$。

%[/aside_content]


% H2, H3
\subsection*{dsfsdt}
%[latex mode=1]
%[+preamble]
%\usepackage[titlenotnumbered,lined,linesnumbered]{algorithm2e}
\let\oldnl\nl
\newcommand{\nonl}{\renewcommand{\nl}{\let\nl\oldnl}}
%[/preamble]
%\quicklatex{size=14}
\TitleOfAlgo{BLAS-3 Gaussian Elimination with Partial Pivoting}
\begin{algorithm}[H]
\For{$l\gets 1$ \KwTo $n-1$}{
$k\gets (l-1)b+1$\;
factorize $PA^{(l)}=LU$ using BLAS-2\;
store $L$ and $U$ in $A$\;
left multiply $A(1:n,k+b:n)$ by $P$\;
$A(k:k+b-1,k+b:n)\gets T^{-1}A(k:k+b-1,k+b:n)$\;
\nonl\Indp where $T$ is the unit-lower-trian of $A(k:k+b-1,k:k+b-1)$\;
\DontPrintSemicolon\Indm$A(k+b:n,k+b:n)\gets A(k+b:n,k+b:n)$\;
\PrintSemicolon\nonl\Indp$-A(k+b:n,k:k+b-1)A(k:k+b-1,k+b:n)$\;
}
\end{algorithm}
%[/latex]

% H4
% \subsubsection*{}

\end{document}
