\documentclass[UTF8,nofonts]{ctexart}

\setCJKmainfont[BoldFont=STHeiti,ItalicFont=STKaiti]{STKaiti}
\setCJKsansfont[BoldFont=STHeiti]{STXihei}
\setCJKmonofont{STFangsong}

\usepackage[letterpaper,left=1in,top=1in]{geometry}
\usepackage{multirow}
\usepackage{amsmath}
\usepackage{amsfonts}
\usepackage{amssymb}
\usepackage[titlenotnumbered,lined,linesnumbered]{algorithm2e}

\begin{document}

\section*{迭代法整理(一):基本迭代法、Krylov子空间和Arnoldi方法}

%[aside_content]

迭代法通常用于求解两类B大规模数值问题:1)线性方程组求解;2)特征值问题求解,即:

\[
A\boldsymbol{x}=\boldsymbol{b},\quad
A\boldsymbol{x}=\lambda\boldsymbol{x}\quad
A\in\mathbb{R}^{n \times n}
\]

indent其中,$n$很大(如:$n=10^9$),且通常$A$是[label]稀疏矩阵[/label]。对于线性方程组而言,希望通过$A$和$\boldsymbol{b}$求解得到$\boldsymbol{x}$,而对于特征值问题,则希望找到$\lambda$。相比[label]直接法[/label](如:Gauss消去法(见不选主元的Gauss消去法、完全主元的Gauss消去法、BLAS三个级别的列主元Gauss消去法等文)、QR分解法(见基于Householder和Givens正交化的QR分解一文)等),迭代法的求解时间更快。对于稀疏矩阵,迭代法甚至能取得线性求解时间,即$\mathcal{O}(n)$,在一般情况下则为$\mathcal{O}(n^2)$;而直接法在矩阵是稀疏和稠密的两种情况下,求解时间分别为$\mathcal{O}(n^2)$和$\mathcal{O}(n^3)$。此外,在存储量方面,直接法B总是需要$\mathcal{O}(n^2)$的存储,而迭代法在矩阵为稀疏的情况下通常只需要$\mathcal{O}(n)$的存储(非稀疏时同样也需要$\mathcal{O}(n^2)$)。

上面提到的[label]稀疏性[/label](sparsity)通常指的是矩阵$A$的每一行最多只有$c$个非零元,且满足$c \ll n$。一般来说,如果矩阵大部分都是零元,则可以实现[label]线性时间[/label]]的矩阵向量乘。因此,迭代法的理念是在算法中以这种计算量较少的矩阵向量乘为主导,并最终得到$\boldsymbol{x}=A^{-1}\boldsymbol{b}$的一个极佳近似解。

\subsection*{基本迭代法}

[label]Jacobi迭代[/label]和[label]Gauss-Seidel[/label]迭代是两种最基本的迭代方法(见求解线性方程组的基本迭代法一文),但通常它们都无法得到最优的性能,一是因为它们需要较多的迭代步数,二是由于随着迭代步数的增加,迭代的结果往往会停滞不前,无法进一步接近真实解。但这两种技术在“[label]平滑[/label]”(smooth)矩阵的误差上十分有效,并主要被应用于[label]多重网格[/label](multigrid)方法之上。

基本迭代法的迭代格式可以从[label]矩阵分裂[/label]的角度推导,即:把求解线性方程组问题转换为求线性方程组的不动点问题,以此来构造迭代格式。通常,将矩阵$A$分裂为$A=M-N$,则原线性方程组等价于$M\boldsymbol{x}=N\boldsymbol{x}+b$,并构造如下迭代格式:

\begin{equation*}
\begin{aligned}
M\boldsymbol{x}_{k+1}&=N\boldsymbol{x}_{k}+\boldsymbol{b}\\
\Longrightarrow\quad\boldsymbol{x}_{k+1}&=B\boldsymbol{x}_{k}+\boldsymbol{f}
\end{aligned}\quad,\quad\text{where}
\left\{\begin{aligned}k&=0,1,\cdots\\B&=M^{-1}N\\\boldsymbol{f}&=M^{-1}\boldsymbol{b}\end{aligned}\right.
\end{equation*}

[alert type="info"]一般会将矩阵$A$分解为三个部分:$A=D+L+U$,其中$D$是矩阵$A$的对角部分,$L$和$U$则分别是$A$的严格下三角部分和严格上三角部分。[/alert]

\subsubsection*{Jacobi迭代}

在Jacobi迭代法中,$M=D$,$N=-(L+U)$,于是有:

\begin{equation*}
\begin{aligned}
\boldsymbol{x}_{k+1}&=-D^{-1}(L+U)\boldsymbol{x}_k+D^{-1}\boldsymbol{b} \\
&=D^{-1}\big(\boldsymbol{b}-(L+U)\boldsymbol{x}_k\big)=D^{-1}\big(\boldsymbol{b}-A\boldsymbol{x}_k+D\boldsymbol{x}_k\big) \\
&=D^{-1}(\boldsymbol{r}_k+D\boldsymbol{x}_k)=D^{-1}\boldsymbol{r}_k+\boldsymbol{x}_k
\end{aligned}
\end{equation*}

于是,Jacobi迭代法的算法实现如下:

%[latex mode=1]
%[+preamble]
%\usepackage[titlenotnumbered,lined,linesnumbered]{algorithm2e}
%[/preamble]
\TitleOfAlgo{Jacobi Method}
\begin{algorithm}[H]
\For{$k\gets1,\cdots$}{
	$\boldsymbol{r}\gets\boldsymbol{b}-A\boldsymbol{x}$\;
	\For{$i\gets 1$ \KwTo $n$}{
		$\delta\gets r_i/a_{ii}$\;
		$x_i\gets x_i+\delta$\;
	}
}
\end{algorithm}
%[/latex]

[alert type="success"]Jacobi迭代法中$\boldsymbol{x}_k$的所有分量$\{x_{i}\}_{i=1}^n$的更新可以并行执行。[/alert]

\subsubsection*{Gauss-Seidel迭代}

在Gauss-Seidel迭代法中,$M=D+L$,$N=-U$,于是有:

\[
(D+L)\boldsymbol{x}_{k+1}=-U\boldsymbol{x}+\boldsymbol{b}
\quad\Longrightarrow\quad
D\boldsymbol{x}_{k+1}=\boldsymbol{b}-L\boldsymbol{x}_{k+1}-U\boldsymbol{x}
\]

indent所以,Gauss-Seidel的迭代格式中当前分量的计算需要使用最新更新的分量,所以其算法实现如下:

%[latex mode=1]
%[+preamble]
%\usepackage[titlenotnumbered,lined,linesnumbered]{algorithm2e}
%[/preamble]
\TitleOfAlgo{Gauss-Seidel Method}
\begin{algorithm}[H]
\For{$k\gets1,\cdots$}{
\For{$i\gets 1$ \KwTo $n$}{
	$\delta\gets r_i/a_{ii}$\;
	$x_i\gets x_i+\delta$\;
	$\boldsymbol{r}\gets\boldsymbol{b}-A\boldsymbol{x}$\;
	}
}
\end{algorithm}
%[/latex]

[alert type="error"]注意:与Jacobi迭代法相比,Gauss-Seidel迭代法中每一个分量的计算都要使用最新的残差,因此各个分量的计算不是独立的,其并行性没有Jacobi迭代法好。[/alert]

\subsection*{Krylov子空间和Arnoldi方法}

[label]Krylov子空间法[/label]具有与问题规模$n$B无关的较少迭代步数,并且根据矩阵的特性,它还能被进一步优化。例如,当矩阵是[label]对称正定矩阵[/label](SPD)时,对应的Krylov方法便是[label]共轭梯度[/label](CG)方法;如果只是对称矩阵,对应的方法是[label]极小残差[/label](MINRES)方法;更一般的情况下时,则是[label]广义极小残差法[/label](GMRES)。无论从计算量还是存储量来说,共轭梯度(CG)方法最高效,其次是极小残差(MINRES)方法,最后则是广义极小残差法(GMRES)。这些方法都基于[label]Krylov子空间[/label]$\mathcal{K}_k$,其形式为:

\[\mathcal{K}_k=\text{span}\{\boldsymbol{b},A\boldsymbol{b},A^2\boldsymbol{b},\cdots,A^{k-1}\boldsymbol{b}\}\]

indent且它具有以下两个特性:1)对于任意的$\hat{k}>k$,有$\mathcal{K}_k\subset\mathcal{K}_{\hat{k}}$;2)存在$k$,使得$A^{-1}\boldsymbol{b}\in\mathcal{K}_k$。这里的第一条特性显而易见,而第二条特性的理由如下(考虑$A$为[label]满秩[/label]的情况):构造一个以$A$为根的多项式,并且这里以[lable]可对角化矩阵[/label]$A$为例,则$A$可以表示为$A=Q\Lambda Q^{-1}$的形式,其中$\Lambda$是对角元为$\lambda_1,\cdots,\lambda_n$的[label]对角矩阵[/label]。如果令$\lambda_1,\cdots,\lambda_k$为所有B相异特征值,则$A$可以表示为以下$k$阶多项式的根:

\begin{equation}
\label{eq:mdgpoly}
p(\lambda)=(\lambda-\lambda_1)(\lambda-\lambda_2)\cdots(\lambda-\lambda_k)=\sum_{i=0}^{k-1}b_i\lambda^i+\lambda^k
\end{equation}

indent其中,$k$便是相异特征值的个数。因为$A$是式(\ref{eq:mdgpoly})的根,所以有:

\[
\begin{aligned}
\sum_{i=0}^{k-1}b_iA^i+A^k &= b_0+b_1A^1+\cdots+b_{k-1}A^{k-1}+A^k=0 \\
\Longrightarrow\quad & A^{-1}(b_0+b_1A^{1}+\cdots+b_{k-1}A^{k-1}+A^k)=0 \\
\Longrightarrow\quad & A^{-1}=-\dfrac{1}{b_0}\left(\sum_{i=1}^{k-1}b_iA^{i-1}+A^{k-1}\right)
\end{aligned}
\]

indent其中$b_0$是所有相异特征值的乘积,并且在满秩情况下其不为零。因此,$\boldsymbol{x}=A^{-1}\boldsymbol{b}$可以表示为:

\[
\boldsymbol{x}=A^{-1}\boldsymbol{b}=-\dfrac{1}{b_0}\left(\sum_{i=1}^{k-1}b_iA^{i-1}\boldsymbol{b}+A^{k-1}\boldsymbol{b}\right)
\]

indent换言之,$A^{-1}\boldsymbol{b}$可以表示为$\{\boldsymbol{b},A\boldsymbol{b},\cdots,A^{k-1}\boldsymbol{b}\}$的线性组合,所以说存在这样的$k$,使得$\boldsymbol{x}=A^{-1}\boldsymbol{b}\in\mathcal{K}_k$。而对于一般情况(秩亏时),利用Jordan分块也可以得到同样的结论。

\subsubsection*{Krylov子空间迭代法}

在求解线性方程组$A\boldsymbol{x}=\boldsymbol{b}$时,Krylov方法会在第$k$步迭代时计算得到最优(从最小二乘法角度出发)的$\boldsymbol{x}^k$。令$Q_k=(\boldsymbol{q}_1,\boldsymbol{q}_2,\cdots,\boldsymbol{q}_k)\in\mathbb{R}^{n\times k}$为$\mathcal{K}_k$的一组标准正交基,则Krylov方法的第$k$步迭代是所求解的问题可以写为如下形式:

\begin{equation}
\label{eq:argminwm}
\boldsymbol{w}_k\in\mathbb{R}^k=\text{argmin}\|\boldsymbol{b}-AQ_k\boldsymbol{w}_k\|_2,\quad\boldsymbol{x}_k=Q_k\boldsymbol{w}_k
\end{equation}

如上所述,因为线性方程组的解一定在某个Krylov子空间内,因此式(\ref{eq:argminwm})的迭代过程一定会终止。事实上,并不需要精确地求解该最小二乘问题,而是希望下一步的Krylov子空间能使误差尽可能减小,并且迭代过程能在一定迭代步数(前面提到的多项式的阶数)后终止。

实际过程中并不是直接使用$\{\boldsymbol{b},A\boldsymbol{b},\cdots,A^{k-1}\boldsymbol{b}\}$作为基底向量,因为当$k$增大时,$A^k\boldsymbol{b}$会变得十分病态,从而造成基底的线性相关。取而代之的方法是生成一组[label]正交向量[/label]作为Krylov子空间$\mathcal{K}_k$的基底向量,这个过程便称为[label]Arnoldi方法[/label],并且它一般基于[label]改进的Gram-Schmidt正交化[/label]方法。

\subsubsection*{Arnoldi方法}

Arnoldi迭代方法在计算Krylov子空间$\mathcal{K}_k$的正交基底向量$Q_k$过程中,利用了前面提到的Krylov子空间的一个特性,即:$\mathcal{K}_k\subset\mathcal{K}_{k+1}\subset\mathcal{K}_{k+2}$。因而在构造下一组标准正交基底向量时,其实只需要计算一个新的$\boldsymbol{q}_{k+1}$向量即可,从而有:$Q_{k+1}=(Q_k,\boldsymbol{q}_{k+1})$。显然,最直接的方法是加入不在$\mathcal{K}_k$中的向量$A^k\boldsymbol{b}$,并利用Gram-Schmidt正交化方法将其正交化后,便能得到$\mathcal{K}_{k+1}$。但正如前面所说的,实际操作中由于$A^{k}\boldsymbol{b}$的计算量较大,且数值上具有不稳定性,所以通常不会显式计算$A^k\boldsymbol{b}$。取而代之的是,Arnoldi方法会利用向量$A\boldsymbol{q}_k$实现正交化,即:

\begin{equation}
\label{eq:qm1}
\begin{aligned}
Q_0 &=
\begin{pmatrix}
	\dfrac{\boldsymbol{b}}{\|\boldsymbol{b}\|_2}
\end{pmatrix},\quad
Q_{k+1}=
\begin{pmatrix}
	Q_{k}&\boldsymbol{q}_{k+1}
\end{pmatrix},\quad k=0,1,\cdots \\
\boldsymbol{q}_{k+1} &= \dfrac{\boldsymbol{v}_{k+1}}{\|\boldsymbol{v}_{k+1}\|_2},\quad
\boldsymbol{v}_{k+1}=A\boldsymbol{q}_k-h_{1k}\boldsymbol{q}_1-h_{2k}\boldsymbol{q}_2-\cdots-h_{kk}\boldsymbol{q}_k
\end{aligned}
\end{equation}

indent也就是说,对于新加入的向量$A\boldsymbol{q}_k$,希望将其同已正交化的$k$个基底向量$\{\boldsymbol{q}_i\}_{i=1}^k$正交化后得到$\boldsymbol{v}_{k+1}$,以及对应的单位向量$\boldsymbol{q}_{k+1}$作为新加入的基底向量,以此来扩充Krylov子空间。因此需要确定系数$h_{ik}$来确保$\boldsymbol{q}_{k+1}$和$\{\boldsymbol{q}_i\}_{i=1}^{k}$的正交(此即线性无关向量组的Gram-Schmidt正交化方法)。根据Gram-Schmidt正交化方法,向量$\boldsymbol{q}_i$的系数$h_{ik}$为其和所要正交化的向量(这里便是$A\boldsymbol{q}_k$)的内积,即:

\[h_{ik}=(A\boldsymbol{q}_k,\boldsymbol{q}_i)\]

[alert type="info"]Arnoldi方法的特点在于只需要向$\mathcal{K}_k$中加入一个新的搜索方向$\boldsymbol{q}_{k+1}$便能得到$\mathcal{K}_{k+1}$的标准正交基底向量,并且避免了病态向量$A^k\boldsymbol{b}$的计算。此外,利用上一步的迭代结果并结合[label]Givens平面旋转[/label],还能简化最小二乘问题的求解。[/alert]

最后,对式(\ref{eq:qm1})进行改写,便可以得到如下递推表达式:

\begin{equation}
\label{eq:aqk}
\begin{array}{rrcl}
	& \boldsymbol{v}_{k+1} & = & A\boldsymbol{q}_k-h_{1k}\boldsymbol{q}_1-\cdots-h_{kk}\boldsymbol{q}_k\quad \\
	\Longleftrightarrow & A\boldsymbol{q}_k & = & h_{1k}\boldsymbol{q}_1+\cdots+h_{kk}\boldsymbol{q}_k+\|\boldsymbol{v}_{k+1}\|_2\boldsymbol{q}_{k+1} \\
	\Longleftrightarrow & A\boldsymbol{q}_k & = & \displaystyle \sum_{i=1}^{k+1}h_{ik}\boldsymbol{q}_i
\end{array} \quad,\quad
\begin{aligned}
	& k = 1,\cdots,n-1 \\
	& h_{k+1,k} = \|\boldsymbol{v}_{k+1}\|_2 \\
\end{aligned}
\end{equation}

indent式(\ref{eq:aqk})给出了搜索方向$\boldsymbol{q}_i$之间的关系表达式,而它也就是Arnoldi迭代方法:先通过递推关系式得到$\boldsymbol{v}_{k+1}$,并计算其$2$-范数,也就是新的基底向量$\boldsymbol{q}_{k+1}$对应的系数$h_{k+1,k}$,最后对其规格化便能得到新的标准正交基底向量$\boldsymbol{q}_{k+1}=\boldsymbol{v}_{k+1}/h_{k+1,k}$。进一步观察式(\ref{eq:aqk}),它也可以表示成矩阵乘积的形式,即:$AQ_k=Q_{k+1}H_k$,其中$H_k$为$(k+1) \times k$阶的[label]上Hessenberg矩阵[/label],其矩阵表示形式为:

\[
H_k=
\begin{pmatrix}
h_{11} & h_{21} & \cdots & \cdots & h_{1k} \\
h_{21} & h_{22} & \cdots & \cdots & \vdots \\
& h_{32} & \ddots & & \vdots \\
& & \ddots & \ddots & \vdots \\
& & & h_{k,k-1} & h_{kk} \\
& & & & h_{k+1,k}
\end{pmatrix}
\]

%[/aside_content]

% H2, H3
% \subsection*{dsfsdt}
%[latex mode=1]
%[+preamble]
%\usepackage[titlenotnumbered,lined,linesnumbered]{algorithm2e}
\let\oldnl\nl
\newcommand{\nonl}{\renewcommand{\nl}{\let\nl\oldnl}}
%[/preamble]
%\quicklatex{size=14}
\TitleOfAlgo{Jacobi Method}
\begin{algorithm}[H]
\For{$k\gets1,\cdots$}{
	$\boldsymbol{r}\gets\boldsymbol{b}-A\boldsymbol{x}$\;
	\For{$i\gets 1$ \KwTo $n$}{
		$\delta\gets r_i/A_{ii}$\;
		$x_i\gets x_i+\delta$\;
	}
}
\end{algorithm}
%[/latex]

\end{document}
