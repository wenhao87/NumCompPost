\documentclass[12pt, UTF8, nofonts]{ctexart}

\setCJKmainfont[BoldFont=STHeiti,ItalicFont=STKaiti]{STKaiti}
\setCJKsansfont[BoldFont=STHeiti]{STXihei}
\setCJKmonofont{STFangsong}

\usepackage[letterpaper,left=.5in,right=.5in,top=1in,bottom=.5in]{geometry}
\usepackage{afterpage, color, multirow}
\usepackage{amsmath}
\usepackage{amsfonts}
\usepackage{amssymb}
\usepackage[titlenotnumbered,lined,linesnumbered]{algorithm2e}

\definecolor{bgcolor}{RGB}{13, 12, 12}
\definecolor{ttcolor}{RGB}{0, 255, 0}
\pagecolor{bgcolor}\afterpage{\nopagecolor}
\makeatletter
\newcommand{\globalcolor}[1]{\color{#1}\global\let\default@color\current@color}
\makeatother
\AtBeginDocument{\globalcolor{ttcolor}}

\begin{document}

%[aside_content]

\section*{预处理的共轭梯度法}

从直接Lanczos迭代法推导共轭梯度(CG)方法一文中,详细介绍了[label]共轭梯度[/label](CG)方法,它主要用于求解稀疏[label]对称正定[/label]矩阵方程组。此外,在详解基于矩阵分裂的经典迭代算法及算法实现一文中,详细介绍了几种经典迭代算法。本文将着重讨论在使用共轭梯度(CG)方法之前,先使用诸如(SSOR 迭代算法)的经典迭代格式做预处理,通常可以加快收敛速度。

\subsection*{共轭梯度(CG)方法}

考虑线性方程组$\boldsymbol{Ax}=\boldsymbol{b}$,其中$\boldsymbol{A}\in\mathbb{R}^{n \times n}$为对称正定矩阵,$\boldsymbol{b}\in\mathbb{R}^{n}$。该问题的解是无约束[label]极小问题[/label]$\min_{\boldsymbol{x}\in\mathbb{R}^n}f(\boldsymbol{x})$的解,其中$f(\boldsymbol{x})=\frac{1}{2}\boldsymbol{x}^T\boldsymbol{Ax}-\boldsymbol{b}^T\boldsymbol{x}$,因此为题转换为了对该极小问题的[label]极小解[/label]$\boldsymbol{x}^{\ast}$的求解。

极小问题最优解的迭代求解过程为:设$\boldsymbol{x}_{k}\in\mathbb{R}^{n}$是$\boldsymbol{x}^{\ast}$的一个近似,从$\boldsymbol{x}_k$出发沿着[label]搜索方向[/label]$\boldsymbol{d}_k$寻找目标函数值更小的点$\boldsymbol{x}_{k+1}$,从而$\boldsymbol{x}_{k+1}$更接近$f$的极小点$\boldsymbol{x}^{\ast}$。共轭梯度(CG)方法主要用到以下几个关系式(详细推导过程以及算法实现可以参见从直接Lanczos迭代法推导共轭梯度(CG)方法一文):

\begin{equation}
    \label{eq:cgrelations}
    \alpha_k = \dfrac{\langle \boldsymbol{r}_k, \boldsymbol{r}_k \rangle}{\langle \boldsymbol{Ad}_k, \boldsymbol{d}_k \rangle}
    \; \Rightarrow \; \begin{aligned}
        \boldsymbol{x}_{k+1} &= \boldsymbol{x}_{k}+\alpha_k\boldsymbol{d}_k \\
        \boldsymbol{r}_{k+1} &= \boldsymbol{r}_{k}-\alpha_k\boldsymbol{Ad}_k
    \end{aligned} \; \Rightarrow \;
    \beta_k = \dfrac{\langle \boldsymbol{r}_{k+1},\boldsymbol{r}_{k+1} \rangle}{\langle \boldsymbol{r}_k,\boldsymbol{r}_k \rangle}
    \; \Rightarrow \;
    \boldsymbol{d}_{k+1} = \boldsymbol{r}_{k+1} + \beta_k\boldsymbol{d}_k
\end{equation}

[alert type="success"]B共轭梯度迭代方法:给定$\boldsymbol{x}_0$,计算残差$\boldsymbol{r}_0$,也即搜索方向$\boldsymbol{d}_0$。此后,根据式(\ref{eq:cgrelations})计算$\boldsymbol{x}_1$、残差$\boldsymbol{r}_1$、和搜索方向$\boldsymbol{d}_1$,如此不断循环,直到达到精度为止。[/alert]

设$\boldsymbol{r}_i\neq0$,$0 \leq i \leq n-1$,那么由于$n$维向量$\boldsymbol{r}_n$和前面$n$个残差向量都正交,则$\boldsymbol{r}_n$B必为零向量。所以说,共轭梯度(CG)方法最多经过$n$次迭代后,必定能得到精确解$\boldsymbol{x}^{\ast}$。当阶数较高时,迭代次数往往比阶数少很多。

另外,关于共轭梯度(CG)方法的收敛性估计,具体可以表示为:

\[
    \|\boldsymbol{x}_k-\boldsymbol{x}^{\ast}\|_{\boldsymbol{A}} \leq
    2\left( \dfrac{\sqrt{K}-1}{\sqrt{K}+1} \right)^k \|\boldsymbol{x}_0-\boldsymbol{x}^{\ast}\|_{\boldsymbol{A}} \quad
    \left\{ \; \begin{aligned}
        \|\boldsymbol{x}\|_{\boldsymbol{A}} &= \sqrt{\boldsymbol{x}^T\boldsymbol{Ax}} \\
        K &= \kappa(\boldsymbol{A})_2
    \end{aligned} \right.
\]

\subsection*{预处理的共轭梯度(CG)方法}

\subsubsection*{预处理的共轭梯度(CG)方法的推导}

引入[label]]预处理矩阵[/label]$\boldsymbol{M}\in\mathbb{R}^{n \times n}$来对方程组$\boldsymbol{Ax}=\boldsymbol{b}$做预处理,其中$\boldsymbol{M}$也是对称正定矩阵,且预处理后方程的系数矩阵还是对称正定矩阵。若$\boldsymbol{M}$有分解:$\boldsymbol{M}=\boldsymbol{SS}^T$,则线性方程组$\boldsymbol{Ax}=\boldsymbol{b}$经过预处理后变为:

\begin{equation}
    \label{eq:prels}
    \underbrace{\boldsymbol{S}^{-1}\boldsymbol{A}(\boldsymbol{S}^{-1})^T}_{\tilde{\boldsymbol{A}}}\boldsymbol{u} = \underbrace{\boldsymbol{S}^{-1}\boldsymbol{b}}_{\tilde{\boldsymbol{b}}} \;, \quad
    \boldsymbol{x} = (\boldsymbol{S}^{-1})^T\boldsymbol{u}
\end{equation}

显然,式(\ref{eq:prels})中的系数矩阵$\boldsymbol{S}^{-1}\boldsymbol{A}(\boldsymbol{S}^{-1})^T$仍为对称正定矩阵。预处理的目标是通过$\boldsymbol{M}$和$\boldsymbol{S}$的选取,使得$\boldsymbol{S}^{-1}\boldsymbol{A}(\boldsymbol{S}^{-1})^T$比$\boldsymbol{A}$具有更好的[label]条件数[/label]。对式(\ref{eq:prels})使用式(\ref{eq:cgrelations})中给出的共轭梯度(CG)方法的迭代计算过程,则其迭代过程可以描述为:

\begin{equation}
    \label{eq:cgprerelations}
    \begin{aligned}
        \boldsymbol{u}_0 &\;\Rightarrow \;
        \left\{ \; \begin{aligned}
            \tilde{\boldsymbol{r}}_0 &= \tilde{\boldsymbol{b}}-\tilde{\boldsymbol{A}}\boldsymbol{u}_0 =
            \boldsymbol{S}^{-1}\boldsymbol{b} - \boldsymbol{S}^{-1}\boldsymbol{A}(\boldsymbol{S}^{-1})^T\boldsymbol{u}_0 \\
            \tilde{\boldsymbol{d}}_0 &= \tilde{\boldsymbol{r}}_0
        \end{aligned}\right. \;\Rightarrow\;
        \tilde{\alpha}_k = \dfrac{\Big\langle \tilde{\boldsymbol{r}}_k, \tilde{\boldsymbol{r}}_k \Big\rangle}{\Big\langle \underbrace{\left(\boldsymbol{S}^{-1}\boldsymbol{A}(\boldsymbol{S}^{-1})^T\right)}_{\tilde{\boldsymbol{A}}}\tilde{\boldsymbol{d}}_k , \tilde{\boldsymbol{d}}_k \Big\rangle} \\
        &\;\Rightarrow \;
        \left\{ \; \begin{aligned}
            \boldsymbol{u}_{k+1} &= \boldsymbol{u}_k + \tilde{\alpha}_k\tilde{\boldsymbol{d}}_k \\
            \tilde{\boldsymbol{r}}_{k+1} &= \tilde{\boldsymbol{r}}_k - \tilde{\alpha}_k \overbrace{\left( \boldsymbol{S}^{-1}\boldsymbol{A}(\boldsymbol{S}^{-1})^T \right)}^{\tilde{\boldsymbol{A}}} \tilde{\boldsymbol{d}}_k
        \end{aligned}\right. \;\Rightarrow\;
        \tilde{\beta}_k = \dfrac{\Big\langle \tilde{\boldsymbol{r}}_{k+1}, \tilde{\boldsymbol{r}}_{k+1} \Big\rangle}{\Big\langle \tilde{\boldsymbol{r}}_{k}, \tilde{\boldsymbol{r}}_{k} \Big\rangle} \;\Rightarrow\; \tilde{\boldsymbol{d}}_{k+1} = \tilde{\boldsymbol{r}}_{k+1} + \tilde{\beta}_k\tilde{\boldsymbol{d}}_k \\
    \end{aligned}
\end{equation}

如果令$\boldsymbol{x}_k=(\boldsymbol{S}^{-1})^T\boldsymbol{u}_k$,$\boldsymbol{d}_k=(\boldsymbol{S}^{-1})^T\tilde{\boldsymbol{d}}_k$的话,则有:$\boldsymbol{u}_k=\boldsymbol{S}^T\boldsymbol{x}_k$,$\tilde{\boldsymbol{d}}_k=\boldsymbol{S}^T\boldsymbol{d}_k$,将它们代入式(\ref{eq:cgprerelations}),便可以得到:

\begin{equation}
    \label{eq:cgpredev}
    \begin{aligned}
        \tilde{\boldsymbol{r}}_k &= \tilde{\boldsymbol{b}}_k - \tilde{\boldsymbol{A}}\boldsymbol{u}_k = \boldsymbol{S}^{-1}\boldsymbol{b}-\Big(\boldsymbol{S}^{-1}\boldsymbol{A}\overbrace{(\boldsymbol{S}^{-1})^T\Big)\boldsymbol{u}_k}^{\boldsymbol{x}_k} = \boldsymbol{S}^{-1}(\boldsymbol{b}-\boldsymbol{Ax}_k) = \boldsymbol{S}^{-1}\boldsymbol{r}_k \\
        \alpha_k &= \dfrac{\Big\langle \boldsymbol{S}^{-1}\boldsymbol{r}_k, \boldsymbol{S}^{-1}\boldsymbol{r}_k \Big\rangle}{\Big\langle \boldsymbol{S}^{-1}\boldsymbol{A}\boldsymbol{d}_k, \boldsymbol{S}^T\boldsymbol{d}_k \Big\rangle} = \dfrac{\Big\langle \boldsymbol{r}_k, (\boldsymbol{SS}^{T})^{-1}\boldsymbol{r}_k \Big\rangle}{\Big\langle \boldsymbol{Ad}_k, (\boldsymbol{S}^{-1}\boldsymbol{S})^T\boldsymbol{d}_k \Big\rangle} = \dfrac{\Big\langle \boldsymbol{r}_k, \boldsymbol{M}^{-1}\boldsymbol{r}_k \Big\rangle}{\Big\langle \boldsymbol{Ad}_k, \boldsymbol{d}_k \Big\rangle} \\
        \boldsymbol{u}_{k+1} &= \boldsymbol{u}_k + \tilde{\alpha}_k\tilde{\boldsymbol{d}}_k \;\Rightarrow\;
        \boldsymbol{S}^T\boldsymbol{x}_{k+1} = \boldsymbol{S}^T\boldsymbol{x}_k + \alpha_k\boldsymbol{S}^T\boldsymbol{d}_k \;\Rightarrow\; \boldsymbol{x}_{k+1} = \boldsymbol{x}_k + \alpha_k\boldsymbol{d}_k \\
        \tilde{\boldsymbol{r}}_{k+1} &= \tilde{\boldsymbol{r}}_k-\tilde{\alpha}_k\left( \boldsymbol{S}^{-1}\boldsymbol{A}(\boldsymbol{S}^{-1})^T\tilde{\boldsymbol{d}}_k \right)
        \begin{aligned}
            & \;\Rightarrow\; \boldsymbol{S}^{1-}\boldsymbol{r}_{k+1} = \boldsymbol{S}^{-1}\boldsymbol{r}_k - \alpha_k\boldsymbol{S}^{-1}\boldsymbol{A}\boldsymbol{d}_k \\
            & \;\Rightarrow\; \boldsymbol{r}_{k+1} = \boldsymbol{r}_k - \alpha_k\boldsymbol{Ad}_k
        \end{aligned} \\
        \beta_{k} &= \dfrac{\Big\langle \boldsymbol{S}^{-1}\boldsymbol{r}_{k+1}, \boldsymbol{S}^{-1}\boldsymbol{r}_{k+1} \Big\rangle}{\Big\langle \boldsymbol{S}^{-1}\boldsymbol{r}_k, \boldsymbol{S}^{-1}\boldsymbol{r}_k \Big\rangle} = \dfrac{\Big\langle \boldsymbol{r}_{k+1}, \boldsymbol{M}^{-1}\boldsymbol{r}_{k+1} \Big\rangle}{\Big\langle \boldsymbol{r}_k,\boldsymbol{M}^{-1}\boldsymbol{r}_k \Big\rangle} \\
        \tilde{\boldsymbol{d}}_{k+1} &= \tilde{\boldsymbol{r}}_{k+1} + \tilde{\beta}_k\tilde{\boldsymbol{d}}_k \;\Rightarrow\; \boldsymbol{S}^T\boldsymbol{d}_{k+1} = \boldsymbol{S}^{-1}\boldsymbol{r}_{k+1} + \beta_{k}\boldsymbol{S}^T\boldsymbol{d}_k \;\Rightarrow\; \boldsymbol{d}_{k+1} = \boldsymbol{M}^{-1}\boldsymbol{r}_{k+1} + \beta_k\boldsymbol{d}_k
    \end{aligned}
\end{equation}

\subsubsection*{预处理的共轭梯度(CG)方法的算法实现}

如果令$\boldsymbol{z}_{k}=\boldsymbol{M}^{-1}\boldsymbol{r}_k$的话,则式(\ref{eq:cgpredev})中的$\alpha_k$、$\beta_k$、和$\boldsymbol{d}_{k+1}$可以得到进一步简化;而在初始化方面,对于给定的$\boldsymbol{x}_0\in\mathbb{R}^n$,有:$\boldsymbol{r}_0=\boldsymbol{b}-\boldsymbol{Ax}_0$,$\boldsymbol{z}_0=\boldsymbol{M}^{-1}\boldsymbol{r}_0$,而B初始搜索方向为$\boldsymbol{d}_0=\boldsymbol{z}_0$。由此便得到了预处理的共轭梯度(CG)方法的算法。对于给定线性方程组$\boldsymbol{Ax}=\boldsymbol{b}$,其算法具体实现如下:

%[latex mode=1]
%[+preamble]
%\usepackage[titlenotnumbered,lined,linesnumbered]{algorithm2e}
%[/preamble]
\TitleOfAlgo{Preprocessed Conjugate Gradient Method}
\begin{algorithm}[H]
    choose an initial guess $\boldsymbol{x}^{(0)}\in\mathbb{R}^{n}$ \;
    $\boldsymbol{r}_0 \gets \boldsymbol{b}-A\boldsymbol{x}_0 \;,\; \boldsymbol{z}_0 \gets \boldsymbol{M}^{-1}\boldsymbol{r}_0 \;,\; \boldsymbol{d}_0 \gets \boldsymbol{z}_0$ \;
    \For{$k \gets 0$ \KwTo $\cdots$}{
        $\alpha_k \gets \big\langle \boldsymbol{r}_k, \boldsymbol{z}_k \big\rangle \Big/ \big\langle \boldsymbol{Ad}_k, \boldsymbol{d}_k \big\rangle$ \;
        $\boldsymbol{x}_{k+1} \gets \boldsymbol{x}_k + \alpha_k\boldsymbol{d}_k$ \;
        $\boldsymbol{r}_{k+1} \gets \boldsymbol{r}_k - \alpha_k\boldsymbol{Ad}_k$ \;
        $\boldsymbol{z}_{k+1} \gets \boldsymbol{M}^{-1}\boldsymbol{r}_{k+1}$ \;
        $\beta_k \gets \big\langle \boldsymbol{r}_{k+1}, \boldsymbol{z}_{k+1} \big\rangle \Big/ \big\langle \boldsymbol{r}_k, \boldsymbol{z}_k \big\rangle$ \;
        $\boldsymbol{d}_{k+1} \gets \boldsymbol{z}_{k+1} + \beta_k\boldsymbol{d}_k$ \;
    }
\end{algorithm}
%[/latex]

\subsubsection*{预处理的共轭梯度(CG)方法的算法分析}

观察上述预处理的共轭梯度(CG)方法的算法实现,虽然预处理矩阵$\boldsymbol{M}$具有$\boldsymbol{M}=\boldsymbol{SS}^T$的分解形式,但算法本身仅和$\boldsymbol{M}$有关,而和$\boldsymbol{S}$无关。如果$\boldsymbol{M}$为[label]单位矩阵[/label],上述算法就是无预处理的共轭梯度(CG)方法(见从直接Lanczos迭代法推导共轭梯度(CG)方法一文)。

此外,观察算法的第$7$行,它和无预处理的共轭梯度(CG)方法相比,仅仅多了每一步求解$\boldsymbol{z}_{k+1}$的计算量,且算法每一步的B主要工作量集中在$\boldsymbol{Ad}_k$和$\boldsymbol{M}^{-1}\boldsymbol{r}_{k+1}$这两次的[label]矩阵向量乘[/label]。

最后需要考虑的是预处理矩阵$\boldsymbol{M}=\boldsymbol{SS}^T$的选取,其中一种选取方法是对$\boldsymbol{A}$进行不完全$\boldsymbol{LU}$分解(见稀疏模式下的不完全三角分解法一文),并且由于$\boldsymbol{A}$的对称性,则$\boldsymbol{A}$的不完全$\boldsymbol{LU}$分解可以表示为:$\boldsymbol{A}=\boldsymbol{LL}^T+\boldsymbol{R}$,$\boldsymbol{M}=\boldsymbol{LL}^T$。

另一种方法则是对$\boldsymbol{A}$进行[label]矩阵分裂[/label](见详解基于矩阵分裂的经典迭代算法及算法实现一文)。对于基于矩阵分裂的不同经典迭代格式$\boldsymbol{A}=\boldsymbol{M}-\boldsymbol{N}$,需要考察$\boldsymbol{M}$是否满足对称正定性。

- [label]Jacobi 迭代[label]:$\boldsymbol{M}=\boldsymbol{D}$,满足对称正定性,对应的预处理方法称为 [label]Jacobi 预处理[/label];

[alert type="error"]B注意:如果对角矩阵$\boldsymbol{M}$的对角元都相同(如二维离散Poisson方程,见详解一维和二维离散Poisson方程一文),即:$\boldsymbol{M}=n\boldsymbol{I}$,此时做 Jacobi 预处理是没有必要的。[/alert]

- [label]Gauss-Seidel 迭代[/label]:$\boldsymbol{M}=\boldsymbol{D}-\boldsymbol{L}$,不满足对称正定性;

- [label]SOR 迭代[/label]:$\boldsymbol{M}=\frac{1}{\omega}\boldsymbol{D}-\boldsymbol{L}$,不满足对称正定性;

- [label]SSOR 迭代[/label]:$\boldsymbol{M} = \frac{1}{\omega(2-\omega)} (\boldsymbol{D}-\omega\boldsymbol{L})\boldsymbol{D}^{-1}(\boldsymbol{D}-\omega\boldsymbol{U})$;满足对称正定性,对应的与处理方法称为 [label]SSOR 预处理[/label];

[alert type="success"]综上所述,对于对称正定线性方程组,可以使用诸如不完全$\boldsymbol{LU}$分解或 SSOR 迭代算法对共轭梯度(CG)方法进行预处理,来加快迭代的收敛速度。[/alert]

%[/aside_content]

\end{document}
