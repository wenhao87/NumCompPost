\documentclass[UTF8,nofonts]{ctexart}

\setCJKmainfont[BoldFont=STHeiti,ItalicFont=STKaiti]{STKaiti}
\setCJKsansfont[BoldFont=STHeiti]{STXihei}
\setCJKmonofont{STFangsong}

\usepackage[letterpaper,left=1in,top=1in]{geometry}
\usepackage{multirow}
\usepackage{amsmath}
\usepackage{amsfonts}
\usepackage{amssymb}
\usepackage[titlenotnumbered,lined,linesnumbered]{algorithm2e}

\begin{document}

% Title
\section*{迭代法中的投影方法}

大部分用于求解大型线性方程组的迭代技术都使用了[label]投影方法[/label],它利用[label]子空间[/label]来得到线性方程组的近似解。本文主要给出投影方法及其过程的一个通用框架以及相关理论。

\subsection*{两个字空间}

对于一个线性方程组$A\boldsymbol{x}=\boldsymbol{b}$,这里假定$A$是一个$n \times n$阶的实矩阵,[label]投影方法[/label](projection method)的B核心思想是从$\mathbb{R}^n$的一个[label]子空间[/label]内提取得到该线性方程组的一个[label]近似解[/label]。投影方法的基本框架使用了两个子空间:

- 子空间$\mathcal{K}$:候选近似解的子空间,也称为[label]搜索子空间[/label]。它是$\mathbb{R}^n$中的一个$m$维子空间,表示为了得到近似解需要用到$m$个限制条件,通常用[label]正交化[/label]条件作为此限制;

- 子空间$\mathcal{L}$:如果要求残差向量$\boldsymbol{b}-A\boldsymbol{x}$正交于$m$个线性无关向量,则这$m$个线性无关向量构成了$\mathbb{R}^n$中的另一个$m$维的子空间$\mathcal{L}$,也称为[label]左子空间[/label];

投影方法一般可以分为[label]正交投影[/label]和[label]斜投影[/lable]两类:

- 在正交投影中,两个子空间$\mathcal{K}$和$\mathcal{L}$相同;

- 在斜投影中,两个子空间$\mathcal{K}$和$\mathcal{L}$不同,甚至可以完全没有任何关系;

\subsection*{通用投影方法}

假定$A$是$\mathbb{R}^n$上的一个$n \times n$阶矩阵,$\mathcal{K}$和$\mathcal{L}$是$\mathbb{R}^n$上的两个$m$维子空间。投影过程就是在寻找线性方程组近似解$\boldsymbol{x}^\ast$的过程中,要求其满足$\boldsymbol{x}^\ast$属于子空间$\mathcal{K}$,同时其残差向量正交于$\mathcal{L}$;另一方面,由于初始向量$\boldsymbol{x}^{(0)}$的存在,所以准确来说,近似解$\boldsymbol{x}^\ast$的搜索应该在[label]仿射空间[/label](affine space)$\boldsymbol{x}^{(0)}+\mathcal{K}$中进行,即:

\begin{equation}
\label{eq:mkl}
\boldsymbol{x}^\ast\in\boldsymbol{x}^{(0)}+\mathcal{K},\quad\boldsymbol{b}-A\boldsymbol{x}^\ast\perp\mathcal{L}
\end{equation}

如果$\boldsymbol{x}^\ast$具有形式:$\boldsymbol{x}^\ast=\boldsymbol{x}^{(0)}+\boldsymbol{\delta}$,初始残差向量$\boldsymbol{r}^{(0)}$具有形式:$\boldsymbol{r}^{(0)}=\boldsymbol{b}-A\boldsymbol{x}^{(0)}$,则由式(\ref{eq:mkl})可以推得:

\[
\boldsymbol{b}-A\boldsymbol{x}^\ast=\boldsymbol{b}-A(\boldsymbol{x}^{(0)}+\boldsymbol{\delta})=(\boldsymbol{r}^{(0)}-A\boldsymbol{\delta})\perp\mathcal{L}
\]

indent 换言之,近似解的求解过程可以表示为:

\begin{align}
\label{eq:cond}
\begin{split}
\boldsymbol{x}^\ast=\boldsymbol{x}^{(0)}+\boldsymbol{\delta},&\quad\boldsymbol{\delta}\in\mathcal{K} \\
\boldsymbol{w}^T(\boldsymbol{r}^{(0)}-A\boldsymbol{\delta})=0,&\quad\forall\boldsymbol{w}\in\mathcal{L}
\end{split}
\end{align}

indent 式(\ref{eq:cond})便是一个基本投影步骤,而投影过程就是执行一系列的该基本投影步骤,且通常每一步都会使用一组新的子空间$\mathcal{K}$和$\mathcal{L}$,并将上一步投影步骤得到的近似解$\boldsymbol{x}^\ast$作为新的初始向量$\boldsymbol{x}^{(0)}$。这就是投影过程的一个基本框架,事实上,几乎所有的基本迭代法都是基于此框架改进而来的。

[alert type="info"]式(\ref{eq:cond})也称为Petrov-Galerkin条件。特别地,当$\mathcal{K}=\mathcal{L}$时,该条件通常就称为Galerkin条件。[/alert]

\subsection*{投影方法的矩阵表示}

矩阵表示的原理就是用一组[label]基底向量[/label]来表示子空间$\mathcal{K}$和$\mathcal{L}$。通常,令$n \times m$阶矩阵$V=(\boldsymbol{v}_1,\cdots,\boldsymbol{v}_m)$和$W=(\boldsymbol{w}_1,\cdots,\boldsymbol{w}_m)$,它们的列向量分别是构成子空间$\mathcal{K}$和$\mathcal{L}$的一组基底。因此,式(\ref{eq:cond})的近似解求解过程可以改写为:

\begin{align}
\label{eq:condmat}
\begin{split}
\boldsymbol{x}^\ast&=\boldsymbol{x}^{(0)}+V\boldsymbol{y} \\
W^T\boldsymbol{r}^{(0)}&=W^TAV\boldsymbol{y}
\end{split}
\end{align}

indent 如果$W^TAV$[label]非奇异[/label],则式(\ref{eq:condmat})可以合并为:

\begin{equation}
\label{eq:algo}
\boldsymbol{x}^\ast=\boldsymbol{x}^{(0)}+V(W^TAV)^{-1}W^T\boldsymbol{r}^{(0)}
\end{equation}

\subsubsection*{投影方法算法原型}

根据式(\ref{eq:algo}),投影方法的B算法原型可以描述为如下:

%[latex mode=1]
%[+preamble]
%\usepackage[titlenotnumbered,lined,linesnumbered]{algorithm2e}
%[/preamble]
\TitleOfAlgo{Prototype Projection Method}
\begin{algorithm}[H]
\Repeat{convergence}{
	select a pair of subspaces $\mathcal{K}$ and $\mathcal{L}$\;
	choose bases $V=(\boldsymbol{v}_1,\cdots,\boldsymbol{v}_m)$ for $\mathcal{K}$\;
	choose bases $W=(\boldsymbol{w}_1,\cdots,\boldsymbol{w}_m)$ for $\mathcal{L}$\;
	$\boldsymbol{r}\gets\boldsymbol{b}-A\boldsymbol{x}$\;
	$\boldsymbol{y}\gets(W^TAV)^{-1}W^T\boldsymbol{r}$\;
	$\boldsymbol{x}\gets\boldsymbol{x}+V\boldsymbol{y}$\;
  }
\end{algorithm}
%[/latex]

[alert type="error"]注意:该投影过程只有在$W^TAV$B非奇异时才有定义,并且即使$A$非奇异,也B不能保证$W^TAV$非奇异。但是,如果子空间$A\mathcal{K}$中的任意向量都B不正交于子空间$\mathcal{L}$,显然此时$W^TAV$非奇异。[/alert]

\subsubsection*{$W^TAV$的非奇异性}

两类特殊情况下时,可以保证$W^TAV$的非奇异性:

- 矩阵$A$正定,且$\mathcal{K}=\mathcal{L}$;

- 矩阵$A$非奇异,且$\mathcal{L}=A\mathcal{K}$;

indent 在这两种情况下,子空间$\mathcal{K}$和$\mathcal{L}$内选取的任意基底$V$和$W$,都能保证$B=W^TAV$非奇异。特别地,如果$A$是[label]对称正定矩阵[/label],且使用B正交投影方法,即子空间$\mathcal{K}$和$\mathcal{L}$的B基底相同,则有$B=V^TAT$,此时$B$也是对称正定矩阵。

\subsection*{最优化观点下的投影方法}

\subsubsection*{正交投影方法}

[alert type="info"]若$A$为对称正定矩阵,定义$(\boldsymbol{x},\boldsymbol{y})_A\equiv(A\boldsymbol{x},\boldsymbol{y})$,相应的范数称为[label]$A$-范数[/label]。[/alert]

设$A$是一个对称正定矩阵,且$\mathcal{K}=\mathcal{L}$,若初始向量为$\boldsymbol{x}^{(0)}$,$\tilde{\boldsymbol{x}}$为精确解,则$\mathcal{K}$上的B正交投影结果$\boldsymbol{x}^\ast$,便是使$\boldsymbol{x}^{(0)}+\mathcal{K}$上的[label]误差$A$-范数[/label]$E(\boldsymbol{x})$达到B最小的一个向量,即:

\[
E(\boldsymbol{x}^\ast)=\min_{\boldsymbol{x}\in\boldsymbol{x}^{(0)}+\mathcal{K}}E(\boldsymbol{x})=\min_{\boldsymbol{x}\in\boldsymbol{x}^{(0)}+\mathcal{K}}\big(A(\tilde{\boldsymbol{x}}-\boldsymbol{x}),\tilde{\boldsymbol{x}}-\boldsymbol{x}\big)^{1/2}
\]

indent 理由是:因为$\tilde{\boldsymbol{x}}-\boldsymbol{x}^\ast$[label]A-正交于[/label]$\mathcal{K}$上的所有向量,所以有:

\[
\big(A(\tilde{\boldsymbol{x}}-\boldsymbol{x}^\ast),\boldsymbol{v}\big)=\big(\boldsymbol{b}-A\boldsymbol{x}^\ast,\boldsymbol{v}\big)=0,\quad\forall\boldsymbol{v}\in\mathcal{K}
\]

indent 而上式也正是正交投影过程中的Galerkin条件。所以,正交投影方法等价于最小化B误差的$A$-范数。

\subsubsection*{斜投影方法}

如果考虑$\mathcal{L}=A\mathcal{K}$的情况,同样,对于初始向量$\boldsymbol{x}^{(0)}$和精确解$\tilde{\boldsymbol{x}}$来说,正交于$\mathcal{L}$且在$\mathcal{K}$上的B斜投影结果$\boldsymbol{x}^\ast$,是使得$\boldsymbol{x}^{(0)}+\mathcal{K}$上的[label]残差向量[/label]$\boldsymbol{b}-A\boldsymbol{x}$的[label]$2$-范数[/label]$E(\boldsymbol{x})$达到B最小的一个向量,即:

\[
R(\boldsymbol{x}^\ast)=\min_{\boldsymbol{x}\in\boldsymbol{x}^{(0)}+\mathcal{K}}R(\boldsymbol{x})=\min_{\boldsymbol{x}\in\boldsymbol{x}^{(0)}+\mathcal{K}}\|\boldsymbol{b}-A\boldsymbol{x}\|_2
\]

indent 理由是:如果$\boldsymbol{x}^\ast$最小化了$R(x)=\|\boldsymbol{b}-A\boldsymbol{x}\|_2$,则表明$\boldsymbol{b}-A\boldsymbol{x}^\ast$正交于$\mathcal{L}=A\mathcal{K}$上的所有向量$\boldsymbol{v}=A\boldsymbol{y}$,其中$\boldsymbol{y}\in\mathcal{K}$,即:

\[
\big(\boldsymbol{b}-A\boldsymbol{x}^\ast,\boldsymbol{v}\big)=0,\quad\forall\boldsymbol{v}\in A\mathcal{K}
\]

indent 同样,上式便是斜投影过程中的Petrov-Galerkin条件。所以,斜投影方法等价于最小化B残差的$2-$范数。

\subsection*{投影算子观点下的投影方法}

\subsubsection*{斜投影方法}

先考虑$\mathcal{L}=A\mathcal{K}$的情况,令初始残差$\boldsymbol{r}^{(0)}=\boldsymbol{b}-A\boldsymbol{x}^{(0)}$,利用投影过程得到近似解$\boldsymbol{x}^\ast$后,其残差表示为:

\begin{equation}
\label{eq:rast}
\boldsymbol{r}^\ast=\boldsymbol{b}-A\boldsymbol{x}^\ast=\boldsymbol{b}-A(\boldsymbol{x}^{(0)}+\boldsymbol{\delta})=\boldsymbol{r}^{(0)}-A\boldsymbol{\delta}
\end{equation}

indent 根据投影算法,残差$\boldsymbol{r}^\ast$需正交于$\mathcal{L}=A\mathcal{K}$,所以,$\boldsymbol{r}^{(0)}-A\boldsymbol{\delta}$正交于$A\mathcal{K}$,而其含义便是$A\boldsymbol{\delta}$为$\boldsymbol{r}^{(0)}$在$A\mathcal{K}$上的正交投影,所以$A\boldsymbol{\delta}$可以表示为$P\boldsymbol{r}^{(0)}$,其中$P$为子空间$\mathcal{L}=A\mathcal{K}$上的[label]正交投影算子[/label]。最后,式(\ref{eq:rast})可以简化为:

\[
\boldsymbol{r}^\ast=\boldsymbol{r}^{(0)}-P\boldsymbol{r}^{(0)}=(I-P)\boldsymbol{r}^{(0)}
\]

\subsubsection*{正交投影方法}

考虑$\mathcal{L}=\mathcal{K}$且$A$为对称正定矩阵的情况,令初始误差$\boldsymbol{\epsilon}^{(0)}=\tilde{\boldsymbol{x}}-\boldsymbol{x}^{(0)}$,其中$\tilde{\boldsymbol{x}}$为精确解。利用投影过程得到近似解$\boldsymbol{x}^\ast$后,其误差可以表示为:$\boldsymbol{\epsilon}^\ast=\tilde{\boldsymbol{x}}-\boldsymbol{x}^\ast$,其中$\boldsymbol{x}^\ast=\boldsymbol{x}^{(0)}+\boldsymbol{\delta}$。由此,可以推得以下关系式:

\begin{align*}
A\boldsymbol{\epsilon}^\ast&=A(\tilde{\boldsymbol{x}}-\boldsymbol{x}^\ast)=\boldsymbol{b}-A\boldsymbol{x}^\ast=\boldsymbol{r}^\ast \\
&=A\Big((\tilde{\boldsymbol{x}}-\boldsymbol{x}^{(0)})-(\boldsymbol{x}^\ast-\boldsymbol{x}^{(0)})\Big)=A(\boldsymbol{\epsilon}^{(0)}-\boldsymbol{\delta})
\end{align*}

indent 由于残差$\boldsymbol{r}^\ast$正交于$\mathcal{L}=\mathcal{K}$,则$A(\boldsymbol{\epsilon}^{(0)}-\boldsymbol{\delta})$也正交于$\mathcal{L}=\mathcal{K}$,于是有:

\[
\Big(A(\boldsymbol{\epsilon}^{(0)}-\boldsymbol{\delta}),\boldsymbol{w}\Big)=0,\quad\forall\boldsymbol{w}\in\mathcal{K}
\]

indent 由于$A$是对称正定矩阵,上式也可以写为$\big(\boldsymbol{\epsilon}^{(0)}-\boldsymbol{\delta},\boldsymbol{w}\big)_A=0$,其含义为$\boldsymbol{\delta}$是$\boldsymbol{\epsilon}^{(0)}$在子空间$\mathcal{K}$上的$A$-正交投影,于是$\boldsymbol{\delta}$可以表示为$P_A\boldsymbol{\epsilon}^{(0)}$,其中$P_A$为子空间$\mathcal{K}=\mathcal{L}$上的[label]$A$-正交投影算子[/label]。最后,可以得到:

\[
\boldsymbol{\epsilon}^\ast=\boldsymbol{\epsilon}^{(0)}-\boldsymbol{\delta}=\boldsymbol{\epsilon}^{(0)}-P_A\boldsymbol{\epsilon}^{(0)}=(I-P)\boldsymbol{\epsilon}^{(0)}
\]

% H2, H3
\subsection*{dsfsdt}
%[latex mode=1]
%[+preamble]
%\usepackage[titlenotnumbered,lined,linesnumbered]{algorithm2e}
\let\oldnl\nl
\newcommand{\nonl}{\renewcommand{\nl}{\let\nl\oldnl}}
%[/preamble]
%\quicklatex{size=14}
\TitleOfAlgo{BLAS-3 Gaussian Elimination with Partial Pivoting}
\begin{algorithm}[H]
\For{$l\gets 1$ \KwTo $n-1$}{
$k\gets (l-1)b+1$\;
factorize $PA^{(l)}=LU$ using BLAS-2\;
store $L$ and $U$ in $A$\;
left multiply $A(1:n,k+b:n)$ by $P$\;
$A(k:k+b-1,k+b:n)\gets T^{-1}A(k:k+b-1,k+b:n)$\;
\nonl\Indp where $T$ is the unit-lower-trian of $A(k:k+b-1,k:k+b-1)$\;
\DontPrintSemicolon\Indm$A(k+b:n,k+b:n)\gets A(k+b:n,k+b:n)$\;
\PrintSemicolon\nonl\Indp$-A(k+b:n,k:k+b-1)A(k:k+b-1,k+b:n)$\;
}
\end{algorithm}
%[/latex]

% H4
% \subsubsection*{}

\end{document}
