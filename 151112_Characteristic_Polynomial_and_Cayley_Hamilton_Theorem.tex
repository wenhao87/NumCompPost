\documentclass[UTF8,nofonts]{ctexart}

\setCJKmainfont[BoldFont=STHeiti,ItalicFont=STKaiti]{STKaiti}
\setCJKsansfont[BoldFont=STHeiti]{STXihei}
\setCJKmonofont{STFangsong}

\usepackage[letterpaper,left=1in,top=1in]{geometry}
\usepackage{multirow}
\usepackage{amsmath}
\usepackage{amsfonts}
\usepackage{amssymb}
\usepackage[titlenotnumbered,lined,linesnumbered]{algorithm2e}

\begin{document}

% Title

\subsection*{特征多项式}

设$n \times n$阶矩阵$A$的[label]特征值[/label]为$\lambda$,其对应的[label]特征向量[/label]为$\boldsymbol{x}$,则它们满足[label]特征方程[/label]$A\boldsymbol{x}=\lambda\boldsymbol{x}$,改写后可得:$(A-\lambda I)\boldsymbol{x}=0$。因为特征向量$\boldsymbol{x}$不为零,所以要求$A-\lambda I$[label]不可逆[/label],其等价于:$\det(A-\lambda I)=0$,特征值$\lambda$即为该特征方程的解。因此,可以定义矩阵$A$的[label]特征多项式[/label]为:

\begin{equation}
\label{eq:chpoly}
p(t)=\det(A-tI)
\end{equation}

indent这里用$t$强调它是多项式的变量,显然特征值$\lambda$就是$p(t)$的根。如果将式(\ref{eq:chpoly})展开的话,可以得到关于$(-t)$的$n$次多项式:

\[
p(t)=(-t)^n+a_{n-1}(-t)^{n-1}+\cdots+a_1(-t)+a_0
\]

indent下面将重点讨论特征多项式$p(t)$的系数$a_i,i=0,1,\cdots,n-1$和矩阵$A$的[label]行列式[/label]以及特征值之间的关系。这里,以$3 \times 3$阶矩阵$A=(a_{ij})$为例:

\begin{equation}
\label{eq:mata}
A=\begin{bmatrix}
a_{11}&a_{12}&a_{13}\\a_{21}&a_{22}&a_{23}\\a_{31}&a_{32}&a_{33}
\end{bmatrix}
\end{equation}

\subsubsection*{特征多项式的系数与矩阵行列式的关系}

对于式(\ref{eq:mata})中的$A$,其特征多项式的形式为:

\begin{align*}
p(t)=&\det(A-tI)=
\begin{vmatrix}
a_{11}-t & a_{12} & a_{13} \\
a_{21} & a_{22}-t & a_{23} \\
a_{31} & a_{32} & a_{33}-t
\end{vmatrix} \\
=&
\begin{vmatrix}-t&0&0\\0&-t&0\\0&0&-t\end{vmatrix}+
\begin{vmatrix}-t&0&0\\0&-t&0\\a_{31}&a_{32}&a_{33}\end{vmatrix}+
\begin{vmatrix}-t&0&0\\a_{21}&a_{22}&a_{23}\\0&0&-t\end{vmatrix}+
\begin{vmatrix}-t&0&0\\a_{21}&a_{22}&a_{23}\\a_{31}&a_{32}&a_{33}\end{vmatrix} \\
&+
\begin{vmatrix}a_{11}&a_{12}&a_{13}\\0&-t&0\\0&0&-t\end{vmatrix}+
\begin{vmatrix}a_{11}&a_{12}&a_{13}\\0&-t&0\\a_{31}&a_{32}&a_{33}\end{vmatrix}+
\begin{vmatrix}a_{11}&a_{12}&a_{13}\\a_{21}&a_{22}&a_{23}\\0&0&-t\end{vmatrix}+
\begin{vmatrix}a_{11}&a_{12}&a_{13}\\a_{21}&a_{22}&a_{23}\\a_{31}&a_{32}&a_{33}\end{vmatrix} \\
=&
(-t)^3+(-t)^2(a_{11}+a_{22}+a_{33})+
(-t)\left(\begin{vmatrix}a_{11}&a_{12}\\a_{21}&a_{22}\end{vmatrix}+\begin{vmatrix}a_{11}&a_{13}\\a_{31}&a_{33}\end{vmatrix}+\begin{vmatrix}a_{22}&a_{23}\\a_{32}&a_{33}\end{vmatrix}\right) \\
&+
\begin{vmatrix}a_{11}&a_{12}&a_{13}\\a_{21}&a_{22}&a_{23}\\a_{31}&a_{32}&a_{33}\end{vmatrix}
\end{align*}

indent可以看到,特征多项式的系数和$k \times k$阶[label]主子矩阵[/label]的行列式(称为$k \times k$阶[label]主余子式[/label]),且对于$n \times n$阶矩阵$A$来说,$k \times k$阶的主余子式有$C_n^k$个,并用$E_k(A)$表示这$C_n^k$个$k \times k$阶主余子式的和。因此,特征多项式可以表示为如下形式:

\begin{equation}
\label{eq:detchpoly}
p(t)=(-t)^n+E_1(A)(-t)^{n-1}+E_2(A)(-t)^{n-2}+\cdots+E_{n-1}(-t)+E_n(A)
\end{equation}

indent这里有两个特别的系数:$E_1(A)$和$E_n(A)$。$E_1(A)$为$n$个$1$阶主余子式之和,也就是矩阵的[label]迹[/label],而$E_n(A)$即为原$n \times n$阶矩阵$A$的行列式,所以有:

\[E_1(A)=a_{11}+a_{22}+\cdots+a_{nn}=\text{trace}A,\quad E_n(A)=\det A\]

\subsubsection*{特征多项式的系数与特征值的关系}

另一方面,如果$A$有$n$个特征值$\lambda_1,\cdots,\lambda_n$,换言之,它们是特征方程的$n$个解,那么根据[label]代数基本定理[/label],特征多项式可以分解为:

\[p(t)=(\lambda_1-t)(\lambda_2-t)\cdots(\lambda_n-t)\]

indent如果以式(\ref{eq:mata})的$3 \times 3$阶矩阵$A$为例,其特征多项式可以展开为:

\begin{align*}
p(t)
&=(\lambda_1-t)(\lambda_2-t)(\lambda_3-t) \\
&=(-t)^3+(\lambda_1+\lambda_2+\lambda_3)(-t)^2+(\lambda_1\lambda_2+\lambda_1\lambda_3+\lambda_2\lambda_3)(-t)+\lambda_1\lambda_2\lambda_3
\end{align*}

indent为了表征系数,对于特征值构成的集合$L=\{\lambda_1,\lambda_2,\cdots,\lambda_n\}$,定义如下形式的[label]基本对称函数[/label]:

\begin{equation}
\label{eq:esf}
\displaystyle{S_k(L)=\sum_{1 \leq i_1<i_2<\cdots<i_k\leq n}\prod_{j=1}^k\lambda_{i_j}}
\end{equation}

indent于是有:$S_1(L)=\lambda_1+\cdots+\lambda_n$,$S_n(L)=\lambda_1\cdots\lambda_n$。因此,特征多项式也可以表示为如下形式:

\begin{equation}
\label{eq:eigchpoly}
p(t)=(-t)^n+S_1(L)(-t)^{n-1}+S_2(L)(-t)^{n-2}+\cdots+S_{n-1}(L)(-t)+S_n(L)
\end{equation}

[alert type="success"]对比式(\ref{eq:detchpoly})和式(\ref{eq:eigchpoly}),可以发现两种特征多项式(主余子式和基本对称函数)具有相同的形式,且$E_i(A)=S_i(L),i=1,2,\cdots,n$。[/alert]

\subsection*{Cayley-Hamilton定理}

\subsubsection*{Cayley-Hamilton定理的描述}

关于矩阵$A$的特征多项式,有一个极为重要的定理,即:[label]Cayley-Hamilton定理[/label]。该定理表述为:任何矩阵都满足与它次数相同的一个[label]代数方程[/label],其中最高次幂的系数为1,且最后一项的系数正是其行列式。这里所说的代数方程,恰好和前面介绍的矩阵$A$的特征多项式相一致。因此,Cayley-Hamilton定理也可以表述为:

[alert type="success"]Cayley-Hamilton定理:令$p(t)$为$n \times n$阶矩阵$A$的特征多项式,以矩阵$A$替换$t$所得到的[label]矩阵多项式[/label],满足$p(A)=0$。[/alert]

\subsubsection*{Cayley-Hamilton定理的证明}

本文主要介绍基于[label]酉相似变换[/label]的Cayley-Hamilton定理证明。根据之前的讨论,利用代数基本定理,矩阵$A$的特征多项式$p(t)$可以表示为:

\[p(t)=(\lambda_1-t)(\lambda_2-t)\cdots(\lambda_n-t)\]

indent其中,$\lambda_1,\cdots,\lambda_n$为矩阵$A$的特征值;而对应的[label]矩阵多项式[/label]$p(A)$可以表示为:

\begin{equation}
\label{eq:matpoly}
p(A)=(\lambda_1I-A)(\lambda_2I-A)\cdots(\lambda_nI-A)
\end{equation}

indent而Cayley-Hamilton定理的意思就是式(\ref{eq:matpoly})的值为零,即:$p(A)=0$。证明如下:首先可以通过[label]酉相似变换[/label]将任意矩阵$A$转换为一个[label]上三角矩阵[/label]$T$,即:$A=UTU^{-1}$,其中,$T$的对角元即为$A$的特征值$\lambda_1,\cdots,\lambda_n$,而$U$为[label]酉矩阵[/label],满足$U^{-1}=U^T$(可参见矩阵的三种标准分解一文)。于是,式(\ref{eq:matpoly})的矩阵多项式$p(A)$可以改写为:

\begin{align*}
p(A)&=p(UTU^{-1})=\left(\lambda_1I-UTU^{-1}\right)\cdots\left(\lambda_nI-UTU^{-1}\right) \\
&=\left[U(\lambda_1I-T)U^{-1}\right]\cdots\left[U(\lambda_nI-T)U^{-1}\right]=U\left[(\lambda_1I-T)\cdots(\lambda_nI-T)\right]U^{-1} \\
&=Up(T)U^{-1}
\end{align*}

indent所以,如果能证明$p(T)=0$,也就证得了$p(A)=0$。这里考虑$T$为$3 \times 3$阶上三角的情形,即:

\[
T=\begin{bmatrix}
\lambda_1 & \ast & \ast \\
0 & \lambda_2 & \ast \\
0 & 0 & \lambda_3
\end{bmatrix}
\]

indent将$T$代入式(\ref{eq:matpoly})的矩阵多项式,可以得到:

\begin{align*}
p(T) &= (\lambda_1I-T)(\lambda_2I-T)(\lambda_3I-T) \\
&=
\begin{bmatrix}0 &\ast & \ast \\ 0 & \ast & \ast \\ 0 & 0 & \ast\end{bmatrix}
\begin{bmatrix}\ast & \ast & \ast \\ 0 & 0 & \ast \\ 0 & 0 & \ast\end{bmatrix}
\begin{bmatrix}\ast & \ast & \ast \\ 0 & \ast & \ast \\ 0 & 0 & 0\end{bmatrix} \\
&=
\begin{bmatrix}0 & 0 & \ast \\ 0 & 0 & \ast \\ 0 & 0 & \ast \end{bmatrix}
\begin{bmatrix}\ast & \ast & \ast \\ 0 & \ast & \ast \\ 0 & 0 & 0\end{bmatrix}
=\begin{bmatrix}0 & 0 & 0 \\ 0 & 0 & 0 \\ 0 & 0 & 0 \end{bmatrix}
\end{align*}

除此之外,Cayley-Hamilton定理有多种证明方法,如:代数证明法、[label]连续论[/label]证明法、基于[label]循环子空间[/label]证明法等。后文在介绍了循环子空间之后,将另行给出基于循环子空间的Cayley-Hamilton定理证明。


% H2, H3
\subsection*{dsfsdt}
%[latex mode=1]
%[+preamble]
%\usepackage[titlenotnumbered,lined,linesnumbered]{algorithm2e}
\let\oldnl\nl
\newcommand{\nonl}{\renewcommand{\nl}{\let\nl\oldnl}}
%[/preamble]
%\quicklatex{size=14}
\TitleOfAlgo{BLAS-3 Gaussian Elimination with Partial Pivoting}
\begin{algorithm}[H]
\For{$l\gets 1$ \KwTo $n-1$}{
$k\gets (l-1)b+1$\;
factorize $PA^{(l)}=LU$ using BLAS-2\;
store $L$ and $U$ in $A$\;
left multiply $A(1:n,k+b:n)$ by $P$\;
$A(k:k+b-1,k+b:n)\gets T^{-1}A(k:k+b-1,k+b:n)$\;
\nonl\Indp where $T$ is the unit-lower-trian of $A(k:k+b-1,k:k+b-1)$\;
\DontPrintSemicolon\Indm$A(k+b:n,k+b:n)\gets A(k+b:n,k+b:n)$\;
\PrintSemicolon\nonl\Indp$-A(k+b:n,k:k+b-1)A(k:k+b-1,k+b:n)$\;
}
\end{algorithm}
%[/latex]

% H4
% \subsubsection*{}

\end{document}
