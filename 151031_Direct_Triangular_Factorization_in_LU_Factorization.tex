\documentclass[UTF8,nofonts]{ctexart}

\setCJKmainfont[BoldFont=STHeiti,ItalicFont=STKaiti]{STKaiti}
\setCJKsansfont[BoldFont=STHeiti]{STXihei}
\setCJKmonofont{STFangsong}

\usepackage[letterpaper,left=1in,top=1in]{geometry}
\usepackage{multirow}
\usepackage{amsmath}
\usepackage{amsfonts}
\usepackage{amssymb}
\usepackage{algorithm2e}

\begin{document}

\section*{LU分解的直接三角法}

此前三篇文章(见BLAS三个级别的列主元Gauss消去法、完全主元的Gauss消去法、不选主元的Gauss消去法)都是利用[label]Gauss消去法[/label]通过[label]初等行变换[/label]将矩阵变为了[label]上三角矩阵[/label],从而实现了矩阵的LU分解:$A=LU$。但其实这种分解也可以从B方程的角度进行直接求解,也就是所谓的[label]直接三角分解[/label]。同样,考虑以下三种情形:[label]不选主元[/label]、[label]完全主元[/label]和[label]列主元[/label]直接三角分解。

\subsection*{不选主元的三角分解}

设$n \times n$阶矩阵$A$[label]非奇异[/label],并有分解$A=LU$,其中$L$为[label]单位下三角矩阵[/label],$U$为上三角矩阵,即:

\begin{align*}
A &= 
\begin{pmatrix}a_{11}&a_{12}&a_{13}&\cdots&a_{1n}\\a_{21}&a_{22}&a_{23}&\cdots&a_{2n}\\a_{31}&a_{32}&a_{33}&\cdots&a_{3n}\\\vdots&\vdots&\vdots&\cdots&\vdots\\a_{n1}&a_{n2}&a_{n3}&\cdots&a_{nn}\end{pmatrix} \\
 &=
\begin{pmatrix}1&&&\\l_{21}&1&&&\\l_{31}&l_{32}&1&&\\\vdots&\vdots&\vdots&\ddots&\\l_{n1}&l_{n2}&\cdots&\cdots&1\end{pmatrix}
\begin{pmatrix}u_{11}&u_{12}&u_{13}&\cdots&u_{1n}\\&u_{22}&u_{23}&\cdots&u_{2n}\\&&u_{33}&\cdots&u_{3n}\\&&&\ddots&\vdots\\&&&&u_{nn}\end{pmatrix}=LU
\end{align*}

由此,可以得到$L$和$U$的计算顺序如下:

\begin{align*}
(u_{11},u_{12},\cdots,u_{1n}) &= (a_{11},a_{12}\cdots,a_{1n}) \\
(l_{21},l_{31},\cdots,l_{n1}) &= (a_{21},a_{31}\cdots,a_{n1})/u_{11} \\
(u_{22},u_{23},\cdots,u_{2n}) &= (a_{22}-l_{21}u_{12},a_{23}-l_{21}u_{13}\cdots,a_{2n}-l_{21}u_{1n}) \\
(l_{32},l_{42},\cdots,l_{n2}) &= (a_{32}-l_{31}u_{12},a_{42}-l_{41}u_{12},\cdots,a_{n2}-l_{n1}u_{12})/u_{22} \\
\cdots &= \cdots
\end{align*}

一行$U$和一列$L$相继计算,直到求得整个$L$和$U$。该算法也称为[label]Doolittle算法[/label],相应的分解也称为Doolittle分解。计算完成后,$A$的下三角矩阵和上三角矩阵(包括对角线)分别存放$L$和$U$。

需要注意的是,由于该解法B不选主元,所以仅当$u_{ii} \neq 0$时才可以进行($i=1,2,\cdots,n$)。若$A$的各阶顺序主子式均不为$0$,上述算法B理论上可行,因为如果$u_{ii}$绝对值很小时,会产生很大的[label]舍入误差[/label]。

\subsection*{列主元的三角分解}

如前所述,不选主元的直接三角分解法中,当$u_{ii}=0$时计算中断,或者绝对值很小的话,舍入误差很大。下面给出[label]列主元的三角分解[/label]:$PA=LU$,其中$P$为行交换矩阵。设完成第$k-1$步后,矩阵形式如下:

\[
\begin{pmatrix}
u_{11} & u_{12} & \cdots & u_{1,k-1} & u_{1k} & \cdots & u_{1n} \\
l_{21} & u_{22} & \cdots & u_{2,k-1} & u_{2k} & \cdots & u_{2n} \\
\vdots & \vdots & & \vdots & \vdots & & \vdots \\
l_{k-1,1} & l_{k-1,2} & \cdots & u_{k-1,k-1} & u_{k-1,k} & \cdots & u_{k-1,n} \\
\cline{5-7}
l_{k1} & l_{k2} & \cdots & l_{k,k-1} & \multicolumn{1}{|c|}{\multirow{1}{*}{$a_{kk}$}} & \cdots & a_{kn} \\
\vdots & \vdots & & \vdots & \multicolumn{1}{|c|}{\multirow{1}{*}{\vdots}} & & \vdots \\
l_{i1} & l_{i2} & \cdots & l_{i,k-1} & \multicolumn{1}{|c|}{\multirow{1}{*}{$\boldsymbol{a_{ik}}$}} & \cdots & a_{in} \\
\vdots & \vdots & & \vdots & \multicolumn{1}{|c|}{\multirow{1}{*}{\vdots}} & & \vdots \\
l_{n1} & l_{n2} & \cdots & l_{n,k-1} & \multicolumn{1}{|c|}{\multirow{1}{*}{$a_{nk}$}} & \cdots & a_{nn} \\
\end{pmatrix}
\]

为了进行列主元交换,按下式取绝对值最大者:

\begin{equation}
\label{eq:si}
s_i=a_{ik}-\sum_{t=1}^{k-1}l_{it}u_{tk},\quad i=k,k+1,\cdots,n
\end{equation}

\begin{equation}
\label{eq:sk}
|s_{i_k}|=\max_{k \leq i \leq n}|s_i|,\quad k \leq i_k \leq n
\end{equation}

indent然后将B整个矩阵的第$k$行和第$i_k$行交换,从而保证$|l_{ik}|=|s_i/s_k|\leq 1$,其中$i=k+1,k+2,\cdots,n)$。这样,当进行第$k$步分解时,有:

\begin{equation}
\label{eq:uandl}
\begin{cases}
u_{kk} &= s_{k},\quad u_{ki} = a_{ki}-\sum_{t=1}^{k-1}l_{kt}u_{ti} \\
l_{ik} &= s_i/s_k
\end{cases}
\quad i=k+1,k+2,\cdots,n
\end{equation}

简单来说,列主元直接三角分解先有式(\ref{eq:si}、\ref{eq:sk})确定所需交换的行,然后根据式(\ref{eq:uandl})产生$U$的一行和$L$的一列。

[alert type="info"]需要注意的是,[label]列主元Gauss消去法[/label]每一步不仅要计算相应$U$的某一行和$L$的某一列,同时还需要更新右下角的子矩阵;而[label]列主元直接三角分解[/label]则不需要进行这样的更新。所以,列主元直接三角分解也是一种“[label]需求驱动[/label]”的算法。但两种方法的计算量是B完全相同的。[/alert]

\subsection*{完全主元的三角分解}

[label]完全主元的三角分解[/label]和上述的[label]列主元的三角分解[/label]类似,只需要改变取绝对最大值时的计算方式即可,即将式(\ref{eq:si}、\ref{eq:sk})改为:

\begin{equation}
\label{eq:sij}
s_{ij}=a_{ij}-\sum_{t=1}^{k-1}l_{it}u_{tj},\quad i,j=k,k+1,\cdots,n
\end{equation}

\[
|s_{pq}|=\max_{k\leq i,j \leq n}|s_{ij}|,\quad k\leq p,q \leq n
\]

之后交换整个矩阵的第$k$行和第$p$行,第$k$列和第$q$列即可。而更新$L$和$U$的过程也和式(\ref{eq:uandl})类似,但需稍加改变如下:

\[
u_{kk}=s_{kk},\quad u_{ki}=s_{ki},\quad l_{ik}=s_i/s_k,\quad i=k+1,k+2,\cdots,n
\]

需要注意的是,分析式(\ref{eq:sij}),即第$k$步完全选主元的过程,每一个$s_{ij}$的计算量为$k-1$次乘法,而子矩阵中共有$(n-k+1)^2$个元素需要计算,所以,整个完全主元的三角分解计算量为:

\[
\sum_{k=1}^{n-1}(n-k+1)^2(k-1)
\]

[alert type="error"]注意:完全主元的直接三角分解法比完全主元的Gauss消去法的乘法次数要高一个数量级。原因在于计算的$(n-k+1)^2$个$s_{ij}$没有在下一步得到充分利用。[/alert]

\end{document}
