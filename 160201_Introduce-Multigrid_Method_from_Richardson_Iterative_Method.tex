\documentclass[12pt, UTF8, nofonts]{ctexart}

\setCJKmainfont[BoldFont=STHeiti,ItalicFont=STKaiti]{STKaiti}
\setCJKsansfont[BoldFont=STHeiti]{STXihei}
\setCJKmonofont{STFangsong}

\usepackage[letterpaper,left=.5in,right=.5in,top=1in,bottom=.5in]{geometry}
\usepackage{afterpage, color, multirow}
\usepackage{amsmath}
\usepackage{amsfonts}
\usepackage{amssymb}
\usepackage[titlenotnumbered,lined,linesnumbered]{algorithm2e}

\definecolor{bgcolor}{RGB}{13, 12, 12}
\definecolor{ttcolor}{RGB}{0, 255, 0}
\pagecolor{bgcolor}\afterpage{\nopagecolor}
\makeatletter
\newcommand{\globalcolor}[1]{\color{#1}\global\let\default@color\current@color}
\makeatother
\AtBeginDocument{\globalcolor{ttcolor}}

\begin{document}

%[aside_content]

\section*{由Richardson 迭代引入多重网格方法}

在详解基于矩阵分裂的经典迭代算法及算法实现一文中,详细介绍了四种主要的基于矩阵分裂的经典迭代法:[label]Jacobi 迭代[/label]、[label]Gauss-Seidel 迭代[/label]、[label]SOR 迭代[/label]、以及 [label]SSOR 迭代[/label]。本文将详细介绍分析另一种经典迭代算法:[label]Richardson 迭代[/label]。事实上,通过深入剖析形式较为简单的迭代模式,如:Richardson 迭代和 Jacobi 迭代算法,更有助于理解[label]多重网格[/label]方法。

\subsection*{Richardon 迭代算法}

\subsubsection*{Richardson 算法迭代格式}

[label]Richardson 迭代[/label]算法是一类形式较为简单的迭代算法,其迭代格式为:

\begin{eqnarray}
  \label{eq:richarditer}
  & \boldsymbol{x}_{k+1} = \boldsymbol{x}_k + \omega(\boldsymbol{b} - \boldsymbol{Ax}_k) = (\boldsymbol{I}-\omega\boldsymbol{A})\boldsymbol{x}_k + \omega\boldsymbol{b} \;,\quad k=0,1,2,\ldots \\
  & \boldsymbol{G}_{\mathrm{R}} = \boldsymbol{I} - \omega\boldsymbol{A}
\end{eqnarray}

noindent其中,$\boldsymbol{G}_{\mathrm{R}}$为 Richardson 迭代算法对应的迭代矩阵。从矩阵分裂$\boldsymbol{A}=\boldsymbol{M}-\boldsymbol{N}$角度来看,Richardson 迭代算法对应的矩阵分裂为:

\begin{equation*}
  \boldsymbol{M} = \dfrac{1}{\omega}\boldsymbol{I} \quad,\quad
  \boldsymbol{N} = \dfrac{1}{\omega}\boldsymbol{I} - \boldsymbol{A}
\end{equation*}

\subsubsection*{Richardson 迭代收敛性分析}

考察迭代矩阵$\boldsymbol{G}_{\mathrm{R}}=\boldsymbol{I}-\omega\boldsymbol{A}$,显然其[label]收敛因子[/label]为$\rho(\boldsymbol{I}-\omega\boldsymbol{A})$。假定矩阵$\boldsymbol{A}$的特征值为$\lambda_i,i=1,\ldots,n$,则有$\lambda_{\min}\leq\lambda_i\leq\lambda_{\max}$,那么迭代矩阵$\boldsymbol{G}_{\mathrm{R}}$的特征值$\mu_i$满足:

\begin{equation}
  \label{eq:greigval}
  1-\omega\lambda_{\max} \leq \mu_i \leq 1-\omega\lambda_{\min}
\end{equation}

观察式(\ref{eq:greigval}),如果$\lambda_{\min}<0$或者$\lambda_{\max}>0$,则至少存在一个$\mu_i>1$,此时对于任意的$\omega$,都有$\rho(\boldsymbol{G}_{\mathrm{R}})>1$,则对于给定的初始向量$\boldsymbol{x}_0$,该方法始终发散。如果假定所有特征值都为正,即:$\lambda_{\min}>0$,那么为了保证迭代算法的收敛性,必须同时满足以下两个条件:

\begin{equation}
  \label{eq:omegarange}
  \left.\begin{array}{ccc}
    1 - \omega\lambda_{\min} & < & 1 \\
    1 - \omega\lambda_{\max} & > & -1 \\
  \end{array}\right\} \;\Rightarrow\; 0 < \omega < \dfrac{2}{\lambda_{\max}}
\end{equation}

\subsubsection*{Richardson 迭代最优参数}

式(\ref{eq:omegarange})给出了 Richardson 迭代方法收敛是参数$\omega$的范围。进一步,希望可以确定$\omega$的最优解$\omega_{opt}$,即:$\omega_{opt}$是使$\rho(\boldsymbol{G}_{\mathrm{R}})$的值达到最小时的值。根据式(\ref{eq:greigval})可知,迭代矩阵$\boldsymbol{G}_{\mathrm{R}}$的[label]谱半径[/label]为:

\begin{equation}
  \label{eq:omegafun}
  \rho(\boldsymbol{G}_{\mathrm{R}}) = \max\Big\{
    |1-\omega\lambda_{\min}|,\;|1-\omega\lambda_{\max}|
  \Big\}
\end{equation}

式(\ref{eq:omegafun})可以看作是关于$\omega$的函数,其函数图像如下所示。从图上可以看出,当处于正斜率方向上的曲线$|1-\omega\lambda_{\max}|$和处于﹣斜率方向的曲线$|1-\omega\lambda_{\min}|$相交时,它们的交点便是$\omega$达到最优时的值,即:

\begin{equation}
  \label{eq:omegaopt}
  -1 + \omega\lambda_{\max} = 1 - \omega\lambda_{\min} \;\Rightarrow\;
  \omega_{opt} = \dfrac{2}{\lambda_{\min}+\lambda_{\max}}
\end{equation}

最终,将式(\ref{eq:omegaopt})代入式(\ref{eq:omegafun})后,便可以得到迭代矩阵$\boldsymbol{G}_{\mathrm{R}}$对应的最优谱半径$\rho_{opt}$:

\begin{equation}
  \label{eq:rhoopt}
  \rho_{opt} = \dfrac{\lambda_{\max}-\lambda_{\min}}{\lambda_{\ max}+\lambda_{\min}}
\end{equation}

[alert type="info"]事实上,在实际应用中,式(\ref{eq:omegaopt})给出的最优参数$\omega_{opt}$是很难求得的。取而代之的是,利用上界$\rho(\boldsymbol{A})\leq\gamma$,其中$\gamma$可以通过 Gershgorin 定理求解,并取$\omega=1/\gamma$。由此可以确保迭代是收敛的,因为其满足式(\ref{eq:omegarange})的收敛条件,即:\[0<\omega=\dfrac{1}{\gamma}\leq\dfrac{1}{\rho(\boldsymbol{A})}<\dfrac{2}{\rho(\boldsymbol{A})}\][alert]

[alert type="info"]如果每次迭代时所取的参数$\omega$不同,则成为[label]非固定 Richardson 迭代算法[/label],即:$\boldsymbol{x}_{k+1}=\boldsymbol{x}_k+\omega_k(\boldsymbol{b}-\boldsymbol{Ax}_k)\;,\;k=0,1,2,\ldots$。[/alert]

\subsection*{多重网格方法简介}

以一维离散Poisson方程为模型(见详解一维和二维离散Poisson方程一文),其特征值$\lambda_k$和对应的特征向量$\boldsymbol{w}_k$为($k=1,2,\ldots,n$):

\begin{equation}
  \label{eq:eigvalvec}
  \begin{aligned}
    \lambda_k & = 2 - 2\cos\dfrac{k\pi}{n+1} = 4\sin^2\dfrac{k\pi}{2(n+1)} \\
    \boldsymbol{w}_k & =
      \begin{pmatrix}
        \sin\dfrac{k\pi}{n+1}, \sin\dfrac{2k\pi}{n+1}, \cdots, \sin\dfrac{nk\pi}{n+1}
      \end{pmatrix}^T \\
  \end{aligned}
  \;\underset{\theta_k=\frac{k\pi}{n+1}}{\Longrightarrow}
  \begin{aligned}
    \lambda_k & = 4\sin^2\dfrac{\theta_k}{2} \\
    \boldsymbol{w}_k & =
      \begin{pmatrix}
        \sin\theta_k, \sin(2\theta_k), \cdots, \sin(n\theta_k)
      \end{pmatrix}^T \\
  \end{aligned}
\end{equation}

根据式(\ref{eq:richarditer})的 Richardson 迭代格式,并结合式(\ref{eq:eigvalvec})可知,其迭代矩阵$\boldsymbol{G}_{\mathrm{R}}=\boldsymbol{I}-\omega\boldsymbol{A}$的特征值为:$\mu_k=1-\omega\lambda_k$,且其具有同$\boldsymbol{A}$一样的特征向量。如果记$\boldsymbol{x}^{\ast}$为一维离散Poisson方程的精确解,那么第$j$步迭代的[label]误差向量[/label]可以表示为:

\begin{equation}
  \label{eq:errorej}
  \left.\begin{aligned}
    \boldsymbol{x}_{k+1} &= \boldsymbol{G}_{\mathrm{R}}\boldsymbol{x}_k + \boldsymbol{f} \\
    \boldsymbol{x}^{\ast} &= \boldsymbol{G}_{\mathrm{R}}\boldsymbol{x}^{\ast} + \boldsymbol{f} \\
  \end{aligned}\;\right\} \;\Rightarrow\;
  \begin{aligned}
    \boldsymbol{x}_{k+1} - \boldsymbol{x}^{\ast} &= \boldsymbol{G}_{\mathrm{R}} (\boldsymbol{x}_k-\boldsymbol{x}^{\ast}) \\
    &= \boldsymbol{G}_{\mathrm{R}}^{k+1} (\boldsymbol{x}_0-\boldsymbol{x}^{\ast}) \\
    &= \boldsymbol{G}_{\mathrm{R}}^{k+1}\boldsymbol{e}_0
  \end{aligned} \;\Rightarrow\;
  \begin{aligned}
    \boldsymbol{e}_j &= \boldsymbol{x}_j-\boldsymbol{x}^{\ast} = \boldsymbol{G}_{\mathrm{R}}^{j}(\boldsymbol{x}_0-\boldsymbol{x}^{\ast}) \\
    &= \boldsymbol{G}_{\mathrm{R}}^j\boldsymbol{e}_0 \\
  \end{aligned}
\end{equation}

另一方面,由于初始误差向量$\boldsymbol{e}_0$可以表示为迭代矩阵$\boldsymbol{G}_{\mathrm{R}}$的[label]基底向量[/label]的B线性组合,即:

\begin{equation}
  \label{eq:ejexpand}
  \begin{aligned}
    \boldsymbol{e}_0 = \sum_{k=1}^n \xi_k\boldsymbol{w}_k \;\Longrightarrow&\;
    \boldsymbol{e}_1 = \boldsymbol{G}_{\mathrm{R}}\boldsymbol{e}_0 = \sum_{k=1}^n \xi_k\boldsymbol{G}_{\mathrm{R}}\boldsymbol{w}_k = \sum_{k=1}^n \xi_k\mu_k\boldsymbol{w}_k \\
    \;\Longrightarrow&\; \boldsymbol{e}_2 = \boldsymbol{G}_{\mathrm{R}} \boldsymbol{e}_1 = \sum_{k=1}^n \xi_k \mu_k^2 \boldsymbol{w}_k \;\Longrightarrow\; \cdots \\
    \;\Longrightarrow&\; \boldsymbol{e}_j = \sum_{k=1}^n \xi_k \mu_k^j \boldsymbol{w}_j = \sum_{k=1}^n \xi_k \left(1-\omega\lambda_k\right)^j \boldsymbol{w}_k \\
    \;\underset{\omega=1/\gamma}{\Longrightarrow}&\; \boldsymbol{e}_j = \sum_{k=1}^n \xi_k \left(1-\dfrac{\lambda_k}{\gamma}\right)^j \boldsymbol{w}_k
  \end{aligned}
\end{equation}

noindent其中,$\boldsymbol{w}_i$是迭代矩阵$\boldsymbol{G}_{\mathrm{R}}$的特征值$\mu_i=1-\omega\lambda_i$对应的特征向量。从式(\ref{eq:ejepand})最后$\boldsymbol{e}_j$的表达式可以看出:其每一个分量都减少了$(1-\lambda_k/\gamma)^j$倍,而根据式(\ref{eq:eigvalvec})中的$\lambda_k$的表达式$\lambda_{k}=4\sin^2(k\pi/2(n+1))$可知,$\lambda_1$具有最小值,且$|\lambda_1/\gamma|\ll1$,此时分量的收敛速度最慢。

[alert type="info"][label]Gershgorin 定理[/label]:矩阵$\boldsymbol{A}$的任一特征值$\lambda$必定位于复平面中$a_{ii}$为圆心的闭环内,且其半径为:$\rho_i=\sum_{j=1,j\neq i}^{j=n}|a_{ij}|$。[/alert]

在一维离散Poisson方程模型中,系数矩阵$\boldsymbol{A}$的最大特征值$\lambda_{\max}=\lambda_{k=n}\approx 4\sin^2(\pi/2)=4$,并且根据此前的假定,取$\rho(\boldsymbol{A})=\gamma=4$。那么,Richardson 迭代算法的[label]收敛因子[/label]为:

\begin{equation}
  \label{eq:convrate}
  \begin{aligned}
    \rho = \rho(\boldsymbol{G}) = \rho(\boldsymbol{I}-\omega\boldsymbol{A}) &= 1 - \dfrac{1}{\gamma}\lambda_{\min} = 1 - \dfrac{\lambda_{k=1}}{4} = 1 - \sin^2\dfrac{\theta_1}{2} \\
    &= 1 - \sin^2\dfrac{\pi}{2(n+1)} \overset{h=1/(n+1)}{\underset{\left.\sin\theta\right|_{\theta \to 0} \approx \theta}{\approx}} 1 - \left(\dfrac{h\pi}{2}\right)^2 = 1 - \mathcal{O}(h^2)
  \end{aligned}
\end{equation}

[alert type="error"]B注意:式(\ref{eq:convrate})给出的重要信息是当网格粒度很细时(即$h$很小),此时的收敛速度很慢。[/alert]

然而,多重网格方法所基于的思想是:不同的分量$\boldsymbol{w}_{k},k=1,2,\ldots,n$具有不同的收敛速度,其中有一半的误差分量具有很快的衰减速度($k>n/2$),并将其称为[label]高频部分[/label](high frequency components),或者也称之为[label]振荡部分[/label](oscillatory part)。对于$k>n/2$的分量来说,它们的收敛因子可以表示为:

\[
  \begin{aligned}
    \eta_k &= 1-\dfrac{\lambda_k}{\gamma} = 1 - \dfrac{4\sin^2\dfrac{\theta_k}{2}}{4} = 1 - \sin^2\dfrac{k\pi}{2(n+1)} \\
    &= \left.\cos^2\dfrac{k\pi}{2(n+1)}\right|_{k>\frac{n}{2}} \leq \dfrac{1}{2}
  \end{aligned}
\]

【Figure1】

\subsubsection*{收敛性分析}

上图显示了 Richardson 迭代算法应用于一维离散 Poisson 问题上时的收敛因子。在深入分析该图之前,需要先引入粗网格和细网格的概念。

令[label]粗网格[/label]$\Omega_{2h}$的网格大小为$2h$,[label]细网格[/label]$\Omega_h$的网格大小则为$h$。针对一维离散 Poisson 方程,粗网格上的第$i$个网格点可以表示为$x_i^{2h}=i*(2h)$。后续的讨论将基于以下一个重要事实,即:$x_i^{2h}=x_{2i}^h$。换言之,粗网格上的第$i$个点对应于细网格上的第$2i$个点。那么,粗网格第$k$个特征向量的第$i$个分量$\left.\boldsymbol{w}_k^{2h}\right|_{x_{i}^{2h}}$与细网格上第$k$个特征向量的第$2i$个分量$\left.\boldsymbol{w}_k^h\right|_{x_{2i}^h}$,具有如下重要关系式:

\begin{equation}
  \label{eq:wiw2i}
  \left.\begin{aligned}
    \left.\boldsymbol{w}_{k}\right|_{i} &= \sin(i\theta_k) = \sin(i\dfrac{k\pi}{n+1}) \\
    x_{i} &= i/h = i/(n+1) \\
  \end{aligned} \;\right\} \;\Rightarrow\;
  \left.\boldsymbol{w}_k\right|_{i} = \sin(k\pi x_i) \;\Rightarrow\;
  \left\{\; \begin{array}{ccc}
    \left.\boldsymbol{w}_{k}^{h}\right|_{2i} & = & \sin(k\pi x_{2i}^h) \\
    \| & & \\
    \left.\boldsymbol{w}_{k}^{2h}\right|_{i} & = & \sin(k\pi x_{i}^{2h}) \\
  \end{array}\right.
\end{equation}

【Figure 2】

如果取细网格上的B平滑模式(即$k\leq n/2$的情况,见上图),便能得到与之相对应的粗网格上的第$k$个模式。举例而言,对于$k=2$的$\boldsymbol{w}_2$模式,$9$个点的细网格($n=7$)和$5$个点的粗网格($n=3$)示意图如下:

【Figure3】

综上所述,可以得出以下两个重要结论:

1. 高频(振荡)部分(即:$\theta_{n/2+1},\ldots,\theta_n$部分分量)收敛速度很快(每一个迭代步$j=1,2,\ldots$,该部分分量的收敛因子都小于$1/2$),并且该收敛因子B与步长$h$无关;

2. 减少低频部分的方法是从粗网格转换到细网格;

[alert type="success"][label]多重网格[/label]的B基本思想是:在细网格上经过若干次迭代后,得到近似解$\tilde{\boldsymbol{x}}_h$和迭代残差$\boldsymbol{d}_h$,然后把细网格上的残差方程变到粗网格上,再求解粗网格上的残差方程和对应残差$\boldsymbol{d}_{2h}$,最后把该残差$\boldsymbol{d}_{2h}$通过B插值变为细网格上的残差$\tilde{\boldsymbol{d}}_h$,它是$\boldsymbol{d}_h$的一个近似,且相比$\tilde{\boldsymbol{x}}_h$,$\tilde{\boldsymbol{u}}_h+\boldsymbol{d}_h$是一个更好的近似。[/alert]

%[/aside_content]

\end{document}
