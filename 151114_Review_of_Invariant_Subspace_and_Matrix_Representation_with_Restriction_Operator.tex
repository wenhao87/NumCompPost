\documentclass[UTF8,nofonts]{ctexart}

\setCJKmainfont[BoldFont=STHeiti,ItalicFont=STKaiti]{STKaiti}
\setCJKsansfont[BoldFont=STHeiti]{STXihei}
\setCJKmonofont{STFangsong}

\usepackage[letterpaper,left=1in,top=1in]{geometry}
\usepackage{multirow}
\usepackage{amsmath}
\usepackage{amsfonts}
\usepackage{amssymb}
\usepackage[titlenotnumbered,lined,linesnumbered]{algorithm2e}

\begin{document}

% Title

\section*{常见不变子空间及其主要应用}

\subsection*{线性变换$T$下的不变子空间}

考虑一个[label]线性变换[/label]$T$,它能将[label]向量空间[/label]$\mathcal{V}$映射至向量空间$\mathcal{W}$,如果$\mathcal{V}=\mathcal{W}$,则称$T$为定义于$\mathcal{V}$上的[label]]线性算子[/label](linear operator)。如果将向量空间$\mathcal{V}$分解为$m$个B不交集子空间的[label]直和[/label](见子空间与投影算子一文),即:

\[\mathcal{V}=\mathcal{X}_1\oplus\cdots\mathcal{X}_m\]

indent那么对于任意$\boldsymbol{v}\in\mathcal{V}$,仅有唯一的$\boldsymbol{x}_i\in\mathcal{X}_i$($i=1,\cdots,m$),满足$\boldsymbol{v}=\boldsymbol{x}_1+\cdots+\boldsymbol{x}_m$。如果用$T\boldsymbol{v}$代表向量$\boldsymbol{v}$经过$T$映射后得到的[label]像[/label]$T(\boldsymbol{v})$,那么根据线性变换的基本性质,有:

\[
T\boldsymbol{v}=T(\boldsymbol{x}_1+\cdots+\boldsymbol{x}_m)=T\boldsymbol{x}_1+\cdots+T\boldsymbol{x}_m
\]

indent于是,线性算子的讨论只需将线性算子$T$限定在子空间$\mathcal{X}_i$上即可,称之为[label]限定算子[/label](restriction),记为$T_{/\mathcal{X}_i}:\mathcal{X}_i\to\mathcal{X}_i$。限定算子的含义是对于任意$\boldsymbol{x}\in\mathcal{X}_i$,都有$T\boldsymbol{x}\in\mathcal{X}_i$,其等价于:$T(\mathcal{X}_i)\subseteq\mathcal{X}_i$,而满足该性质的子空间$\mathcal{X}_i$称为[label]$T$的不变子空间[/label](invariant subspace)。

\subsection*{几种常见的不变子空间}

要证明一个子空间$\mathcal{S}$是线性算子$T$的不变子空间,其思路为:设$\boldsymbol{s}\in\mathcal{S}$,如果能证明$T\boldsymbol{s}\in\mathcal{S}$,则$\mathcal{S}$是$T$的不变子空间。

\subsubsection*{$T$的列空间和零空间}

设$T$为定义于向量空间$\mathcal{V}$的线性算子。线性算子$T$的[label]列空间[/label](值域)为$R(T)=\{T\boldsymbol{v}|\boldsymbol{v}\in\mathcal{V}\}$,而如果$\boldsymbol{v}\in R(T)$,则根据值域的定义,有$T\boldsymbol{v}\in R(T)$,所以说$T$的列空间是$T$的一个不变子空间。

而对于[label]零空间[/label]$N(T)=\{\boldsymbol{v}|T\boldsymbol{v}=0\}$,同样如果$\boldsymbol{v}\in N(T)$,则有$T\boldsymbol{v}=0\in N(T)$(任何子空间都包含零向量)。所以,$T$的零空间也是$T$的一个不变子空间。

\subsubsection*{$TL=LT$时}

设$T$和$V$为定义于向量空间$\mathcal{V}$的线性算子,且满足$TL=LT$,则线性算子$L$的值域$R(L)$和零空间$N(L)$也是$T$的不变子空间。理由如下:

如果$\boldsymbol{v}\in R(L)$,则$\boldsymbol{v}$可以表示为$L\boldsymbol{w}=\boldsymbol{v}$,于是有:$T\boldsymbol{v}=TL\boldsymbol{w}=LT\boldsymbol{w}\in R(L)$,所以$L$的值域$R(L)$是$T$的不变子空间。

另一方面,如果$\boldsymbol{v}\in N(L)$,即$L\boldsymbol{v}=0$,则有$LT\boldsymbol{v}=TL\boldsymbol{v}=0$,所以$T\boldsymbol{v}\in N(L)$,也就说是$L$的零空间$N(L)$也是$T$的不变子空间。

\subsubsection*{$R(T-\lambda I)$和$N(T-\lambda I)$}

设$\lambda$为线性算子$T:\mathcal{V}\to\mathcal{V}$的任意[label]特征值[/label],则$R(T-\lambda I)$和$N(T-\lambda I)$都是$T$的不变子空间。理由如下:

$T-\lambda I$的值域$R(T-\lambda I)$可以表示为$T(T-\lambda I)$,而它也等于$(T-\lambda I)T$,令$L=T-\lambda I$的话,利用之前的结论,便可直接得到$R(T-\lambda I)$以及$N(T-\lambda I)$都是$T$的不变子空间。事实上,该结论可推广至一般情况下,即:$R(T-\lambda I)^k$和$N(T-\lambda I)^k$也都是$T$的不变子空间($k=1,\cdots,n$)。

\subsubsection*{循环子空间}

设$\boldsymbol{v}\in\mathcal{V},\boldsymbol{v} \neq 0$,向量$\boldsymbol{v}$在线性算子$T$下生成一个[label]循环子空间[/label](cyclic subspace)$\mathcal{X}$也是$T$的不变子空间。循环子空间具有如下形式:

\begin{equation}
\label{eq:cysp}\mathcal{X}=\text{span}\{\boldsymbol{v},T\boldsymbol{v},T^2\boldsymbol{v},\cdots\}
\end{equation}

indent如果令$k$是使得$\boldsymbol{v},T\boldsymbol{v},\cdots,T^{k-1}\boldsymbol{v}$为[label]线性无关向量集[/label]的最小整数,则有$c_0,c_1,\cdots,c_{k-1}$使得:

\begin{equation}
\label{eq:tkv}
T^k\boldsymbol{v}=c_0\boldsymbol{v}+c_1T\boldsymbol{v}+\cdots+c_{k-1}T^{k-1}\boldsymbol{v}
\end{equation}

indent如果将上式左乘一个$T$,便可以得到$T^{k+1}\boldsymbol{v}=c_0T\boldsymbol{v}+c_1T^2\boldsymbol{v}+\cdots+c_{k-1}T^k\boldsymbol{v}$,其中最后一项$T^k\boldsymbol{v}$又可以由式(\ref{eq:tkv})所替换。因此,利用归纳法可以推得,对于任意的$m>k$,只需将式(\ref{eq:tkv})连续左乘$m-k$次$T$,等式左边便是$T^m\boldsymbol{v}$,而等式右边则始终可以表示为$\boldsymbol{v},T\boldsymbol{v},\cdots,T^{k-1}\boldsymbol{v}$的线性组合,从而证得$T(\mathcal{X})\subseteq\mathcal{X}$。

\subsection*{不变子空间的主要应用}

\subsubsection*{表示矩阵}

设向量空间$\mathcal{X}$的基底向量为$\boldsymbol{\beta}=\{\boldsymbol{x}_1,\cdots,\boldsymbol{x}_n\}$,则$\mathcal{X}$中的任意向量$\boldsymbol{x}$都可以表示为基底向量的线性组合,即:$\boldsymbol{x}=c_1\boldsymbol{x}_1+\cdots+c_n\boldsymbol{x}_n$,而$(c_1,\cdots,c_n)^T$则称为$\boldsymbol{x}$关于基底$\boldsymbol{\beta}$的[label]坐标向量[/label]。

进一步,给定$T:\mathcal{V}\to\mathcal{W}$,向量空间$\mathcal{V}$和$\mathcal{W}$的基底分别为:$\boldsymbol{\beta}=\{\boldsymbol{v}_1,\cdots,\boldsymbol{v}_n\}$,$\boldsymbol{\gamma}=\{\boldsymbol{w}_1,\cdots,\boldsymbol{w}_m\}$,因为$T(\boldsymbol{v}_i)\in\mathcal{W}$,所以,存在唯一数组$a_{1i},\cdots,a_{mi}$,使得:

\[
T(\boldsymbol{v}_i)=a_{1i}\boldsymbol{w}_1+a_{2i}\boldsymbol{w}_2+\cdots+a_{mi}\boldsymbol{w}_m,\quad j=1,\cdots,n
\]

indent这样便可以得到$T(\boldsymbol{v}_i)$关于基底$\boldsymbol{\gamma}$的坐标向量。如果将所有这些坐标向量组合成$m \times n$阶矩阵的话,便能得到$T$关于基底$\boldsymbol{\beta}$和$\boldsymbol{\gamma}$的[label]表示矩阵[/label](matrix representation):

\[\Big(T(\boldsymbol{v}_i)\Big)_\gamma=\begin{pmatrix}a_{1i}\\a_{2i}\\\vdots\\a_{mi}\end{pmatrix},\quad
\Big(T(\boldsymbol{v}_1),\cdots,T(\boldsymbol{v}_n)\Big)=\Big(T\Big)_{\boldsymbol{\beta}}^{\boldsymbol{\gamma}}=A=
\begin{bmatrix}a_{11}&a_{12}&\cdots&a_{1n}\\a_{21}&a_{22}&\cdots&a_{2n}\\\vdots&\vdots&\ddots&\vdots\\a_{m1}&a_{m2}&\cdots&a_{mn}\end{bmatrix}
\]

\subsubsection*{化简表示矩阵}

不变子空间的一个最主要应用是B简化线性算子的表示矩阵。由于是不变子空间,所以有:$\mathcal{V}=\mathcal{W}$,以及$\mathfrak{B}=\boldsymbol{\beta}=\boldsymbol{\gamma}=\{\boldsymbol{v}_1,\cdots,\boldsymbol{v}_n\}$,且:

\[
\Big(T\Big)_{\mathfrak{B}}=
\begin{pmatrix}a_{11}&\cdots&a_{1n}\\\vdots&\ddots&\vdots\\a_{n1}&\cdots&a_{nn}\end{pmatrix}
\]

indent此时,如果$\mathfrak{B}$中的前$m<n$个向量$\boldsymbol{v_1},\cdots,\boldsymbol{v}_m$能扩张成一个不变子空间$\mathcal{X}$,则对于$i=1,\cdots,m$,始终有:$T\boldsymbol{v}_i\subseteq\mathcal{X}$,即:

\[
T(\boldsymbol{v}_i)=a_{1i}\boldsymbol{v}_1+\cdots+a_{mi}\boldsymbol{v}_m,\quad j=1,\cdots,n
\]

indent换言之,对于前$m$个$T\boldsymbol{v}_i,(i=1,\cdots,m)$坐标向量,它们的后$n-m$个元素都为零,所以,$T_{\mathfrak{B}}$具有如下分块形式:


\[
\Big(T\Big)_{\mathfrak{B}}=
\begin{pmatrix}
a_{11} & \cdots & a_{1m} & a_{1,m+1} & \cdots & a_{1n} \\
\vdots & \ddots & \vdots & \vdots & \ddots & \vdots \\
a_{m1} & \cdots & a_{mm} & a_{m,m+1} & \cdots & a_{mn} \\
0 & \cdots & 0 & a_{m+1,m+1} & \cdots & a_{m+1,n} \\
\vdots & \ddots & \vdots & \vdots & \ddots & \vdots \\
0 & \cdots & 0 & a_{n,m+1} & \cdots & a_{nn}
\end{pmatrix}=
\begin{pmatrix}B & C \\ 0 & D\end{pmatrix}
\]

其中,分块$B$可以看作是限定算子$T_{/\mathcal{X}}$关于基底$\mathfrak{B}=\{\boldsymbol{v}_1,\cdots,\boldsymbol{v}_m\}$的表示矩阵$\Big(T_{/\mathcal{X}}\Big)_{\mathfrak{B}_\mathcal{X}}$。更进一步,如果后$n-m$个基地向量$\boldsymbol{v}_{m+1},\boldsymbol{v}_{m+2},\cdots,\boldsymbol{v}_{n}$也扩张成一个不变子空间$\mathcal{Y}$,则令$\mathfrak{B}_\mathcal{Y}=\{\boldsymbol{v}_{m+1},\boldsymbol{v}_{m+2},\cdots,\boldsymbol{v}_{n}\}$,那么对于$j=m+1,m+2,\cdots,n$,有:

\[
T\boldsymbol{v}_{j}=a_{m+1,j}\boldsymbol{v}_{m+1}+a_{m+2,j}\boldsymbol{v}_{m+2}+\cdots+a_{nj}\boldsymbol{v}_{n}
\]

indent换言之,如果有$\mathcal{V}=\mathcal{X}\oplus\mathcal{Y}$,且$\mathcal{X}$和$\mathcal{Y}$都是不变子空间,则$T$关于基底$\mathfrak{B}=\mathfrak{B}_\mathcal{X}\cup\mathfrak{B}_\mathcal{Y}$的表示矩阵具有如下分块形式:

\[
\Big(T\Big)_{\mathfrak{B}}=
\begin{pmatrix}
a_{11} & \cdots & a_{1m} & 0 & \cdots & 0 \\
\vdots & \ddots & \vdots & \vdots & \ddots & \vdots \\
a_{m1} & \cdots & a_{mm} & 0 & \cdots & 0 \\
0 & \cdots & 0 & a_{m+1,m+1} & \cdots & a_{m+1,n} \\
\vdots & \ddots & \vdots & \vdots & \ddots & \vdots \\
0 & \cdots & 0 & a_{n,m+1} & \cdots & a_{nn}
\end{pmatrix}=
\begin{pmatrix}
\Big(T_{/\mathcal{X}}\Big)_{\mathfrak{B}_{\mathcal{X}}} & 0 \\
0 & \Big(T_{/\mathcal{Y}}\Big)_{\mathfrak{B}_{\mathcal{Y}}}
\end{pmatrix}
\]

上述结论可推广至如下一般情况:如果向量空间$\mathcal{V}$为一组不变子空间直和,即:$\mathcal{V}=\mathcal{X}_1\oplus\cdots\oplus\mathcal{X}_m$,其中自空间$\mathcal{B}_i$的基底为$\mathfrak{B}_i$,那么$T$关于$\mathfrak{B}=\mathfrak{B}_1\cup\cdots\cup\mathfrak{B}_m$的表示矩阵为所有限定算子表示矩阵$\big(T_{\mathcal{X}_i}\big)_{\mathfrak{B}_i}$的直和,其形式为:

\[
\Big(T\Big)_{\mathfrak{B}}=
\begin{pmatrix}
\big(T_{/\mathcal{X}_1}\big)_{\mathfrak{B}_{1}} & ~ & ~ \\
~ & \ddots & ~ \\
~ & ~ & \big(T_{/\mathcal{X}_m}\big)_{\mathfrak{B}_{m}}
\end{pmatrix}=
\big(T_{/\mathcal{X}_1}\big)_{\mathfrak{B}_{1}}\oplus\cdots\oplus\big(T_{/\mathcal{X}_m}\big)_{\mathfrak{B}_{m}}
\]

\subsection*{限定算子表示矩阵的特殊形态}

事实上,不变子空间$\mathcal{X}_i$及其基底向量$\mathfrak{B}_i$之间的关系,决定了限定算子表示矩阵$\big(T_{/\mathcal{X}_i}\big)_{\mathfrak{B}_i}$的形态。以下讨论几种特殊形态下的限定算子表示矩阵所对应的矩阵形式。

\subsubsection*{对角矩阵}

设线性算子$T$定义于向量空间$\mathcal{V}$,且$\text{dim}\mathcal{V}=n$,令$\lambda_1,\cdots,\lambda_n$为$T$的特征值,对应的特征向量为$\boldsymbol{x}_1,\cdots,\boldsymbol{x}_n$,它们满足关系$T\boldsymbol{x}_i=\lambda_i\boldsymbol{x}_i$。此外,根据之前得到的讨论,有:$N(T-\lambda_iI)$是$T$的不变子空间。于是有:

\[
\begin{array}{cllll}
T\boldsymbol{x}_1= & \lambda_1\boldsymbol{x}_1 & & & \\
T\boldsymbol{x}_2= & & \lambda_2\boldsymbol{x}_2 & &\\
\vdots & & & \ddots &\\
T\boldsymbol{x}_n= & & & & \lambda_n\boldsymbol{x}_n\\
\end{array}
\]

indent那么,只要这$n$个特征向量[label]线性无关[/label],则$\text{span}\{\boldsymbol{x}_i\},i=1,\cdots,n$便是彼此不交集的$T$的不变子空间,于是向量空间$\mathcal{V}$可以分解为:

\[\mathcal{V}=\text{span}\{\boldsymbol{x}_1\}\oplus\cdots\oplus\text{span}\{\boldsymbol{x}_n\}\]

indent最后,取这$n$个线性独立的特征向量为基底$\mathfrak{B}=\{\boldsymbol{x}_1,\cdots,\boldsymbol{x}_n\}$,则线性算子$T$关于基底$\mathfrak{B}$的表示矩阵具有最简对角形式,且此时$T$是[label]可对角化[/label]的:

\[
\Big(T\Big)_{\mathfrak{B}}=
\begin{pmatrix}
\lambda_1 & & \\
& \ddots & \\
& & \lambda_n
\end{pmatrix}
\]

\subsubsection*{上三角化}

然而,并不是每一个线性算子$T$都是可对角化的,但它一定可以[label]上三角化[/label]。其原理是,必定存在一组向量空间$\mathcal{V}$的基底$\mathfrak{B}=\{\boldsymbol{v}_1,\cdots,\boldsymbol{v}_n\}$,使得:

\begin{align*}
T\boldsymbol{v}_1 &= a_{11}\boldsymbol{v}_1 \\
T\boldsymbol{v}_2 &= a_{12}\boldsymbol{v}_1+a_{22}\boldsymbol{v}_2 \\
\vdots& \\
T\boldsymbol{v}_n &= a_{1n}\boldsymbol{v}_1+a_{2n}\boldsymbol{v}_2+\cdots+a_{nn}\boldsymbol{v}_n
\end{align*}

indent因此,对$i=1,\cdots,n$,$\text{span}\{\boldsymbol{v}_1,\cdots,\boldsymbol{v}_i\}$都是不变子空间。那么,线性算子$T$关于基底$\mathfrak{B}$的表示矩阵具有如下上三角形式:

\[
\big(T\big)_{\mathfrak{B}}=
\begin{pmatrix}
a_{11} & a_{12} & \cdots & a_{1n} \\
0 & a_{22} & \cdots & a_{2n} \\
\vdots & \vdots & \ddots & \vdots \\
0 & 0 & \cdots & a_{nn}
\end{pmatrix}
\]

\subsubsection*{有理标准形式}

考察此前提到的一类特殊不变子空间:循环子空间。非零[label]种子向量[/label]$\boldsymbol{v}$在线性算子$T$的作用下所张成的循环子空间$\mathcal{X}_{\boldsymbol{v}}=\text{span}\{\boldsymbol{v},T\boldsymbol{v},T^2\boldsymbol{v},\cdots\}$是一个不变子空间。这里,取前$k$个向量作为$\mathcal{X}_{\boldsymbol{v}}$的基底$\mathfrak{B}_{\boldsymbol{v}}=\{\boldsymbol{v},T\boldsymbol{v},\cdots,T^{k-1}\boldsymbol{v}\}$,则有:

\[
a_0\boldsymbol{v}+a_1T\boldsymbol{v}+\cdots+a_{k-1}T^{k-1}\boldsymbol{v}+T^k\boldsymbol{v}=0
\]

indent那么,对$\mathfrak{B}_{\boldsymbol{v}}$中的每一个基底向量进行$T$的线性变换后可以得到:

\[
\begin{array}{lcrcrrrrr}
T(\boldsymbol{v}) & = & T\boldsymbol{v} & = & & 1(T\boldsymbol{v}) & & &\\
T(T\boldsymbol{v}) & = & T^2\boldsymbol{v} & = & & & 1(T^2\boldsymbol{v}) & &\\
& \vdots & & \vdots & & & & \vdots & \\
T(T^{k-1}\boldsymbol{v}) & = & T^k\boldsymbol{v} & = & -a_0\boldsymbol{v} & -a_1(T\boldsymbol{v}) & -a_2(T^2\boldsymbol{v}) & \cdots & -a_{k-1}(T^{k-1}\boldsymbol{v})
\end{array}
\]

indent则限定算子$T_{/\mathcal{X}_{\boldsymbol{v}}}$关于$\mathfrak{B}_{\boldsymbol{v}}$的表示矩阵为:

\begin{equation}
\label{eq:matcysub}
\Big(T_{/\mathcal{X}_{\boldsymbol{v}}}\Big)_{\mathfrak{B}_{\mathbf{v}}}=C=
\begin{pmatrix}
0 & 0 & \cdots & 0 & -a_0 \\
1 & 0 & \cdots & 0 & -a_1 \\
0 & 1 & \cdots & 0 & -a_2 \\
\vdots & \vdots & \ddots & \vdots & \vdots \\
0 & 0 & \cdots & 1 & -a_{k-1}
\end{pmatrix}
\end{equation}

[alert type="success"]式(\ref{eq:matcysub})同时也是多项式$p(t)=a_0+a_1t+\cdots+a_{k-1}t^{k-1}$的[label]相伴矩阵[/label]。

最后,给出[label]循环分解定理[/label],其描述为:B必定存在一组非零种子向量$\boldsymbol{v}_1,\cdots,\boldsymbol{v}_m$,使得向量空间$\mathcal{V}$可以分解为这些种子向量生成的B循环子空间的直和:

\[\mathcal{V}=\mathcal{X}_{\boldsymbol{v}_1}\oplus\cdots\oplus\mathcal{X}_{\boldsymbol{v}_m}\]

indent且其满足如下性质:循环子空间$\mathcal{X}_{\boldsymbol{v}_i}$有基底$\mathfrak{B}_{\boldsymbol{v}_i}=\{\boldsymbol{v}_i,T\boldsymbol{v}_i,\cdots,T^{k_i-1}\boldsymbol{v}_i\}$,其中$k_i$满足$\sum_{i=1}^mk_i=n$。换言之,$T$关于基底$\mathfrak{B}=\mathfrak{B}_1\cup\cdots\cup\mathfrak{B}_m$的表示矩阵便是所有相伴矩阵的直和:

\begin{equation}
\label{eq:rcf}
\Big(T\Big)_{\mathfrak{B}}=
\begin{pmatrix}
C_1 & & \\
& \ddots & \\
& & C_m
\end{pmatrix}
\end{equation}

indent其中,$C_i$为$k_i \times k_i$阶相伴矩阵,且$C_i$的最小多项式也称为$T$的[label]不变因子[/label](invariant factor),它整除$C_{i+1}$的最小多项式,$i=1,\cdots,m-1$。式(\ref{eq:rcf})也称为线性算子$T$的[label]有理标准形式[/label](rational canonical form)。













% H2, H3
\subsection*{dsfsdt}
%[latex mode=1]
%[+preamble]
%\usepackage[titlenotnumbered,lined,linesnumbered]{algorithm2e}
\let\oldnl\nl
\newcommand{\nonl}{\renewcommand{\nl}{\let\nl\oldnl}}
%[/preamble]
%\quicklatex{size=14}
\TitleOfAlgo{BLAS-3 Gaussian Elimination with Partial Pivoting}
\begin{algorithm}[H]
\For{$l\gets 1$ \KwTo $n-1$}{
$k\gets (l-1)b+1$\;
factorize $PA^{(l)}=LU$ using BLAS-2\;
store $L$ and $U$ in $A$\;
left multiply $A(1:n,k+b:n)$ by $P$\;
$A(k:k+b-1,k+b:n)\gets T^{-1}A(k:k+b-1,k+b:n)$\;
\nonl\Indp where $T$ is the unit-lower-trian of $A(k:k+b-1,k:k+b-1)$\;
\DontPrintSemicolon\Indm$A(k+b:n,k+b:n)\gets A(k+b:n,k+b:n)$\;
\PrintSemicolon\nonl\Indp$-A(k+b:n,k:k+b-1)A(k:k+b-1,k+b:n)$\;
}
\end{algorithm}
%[/latex]

% H4
% \subsubsection*{}

\end{document}
