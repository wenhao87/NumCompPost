\documentclass[UTF8,nofonts]{ctexart}

\setCJKmainfont[BoldFont=STHeiti,ItalicFont=STKaiti]{STKaiti}
\setCJKsansfont[BoldFont=STHeiti]{STXihei}
\setCJKmonofont{STFangsong}

\usepackage[letterpaper,left=1in,top=1in]{geometry}
\usepackage{multirow}
\usepackage{amsmath}
\usepackage{amsfonts}
\usepackage{amssymb}
\usepackage[titlenotnumbered,lined,linesnumbered]{algorithm2e}

\begin{document}

% Title
\section*{对称(正定)矩阵的直接三角分解}

% introduction
对于任意$n \times n$阶矩阵$A$,前文(LU分解的直接三角法)给出了三种直接三角分解法:[label]不选主元[/label]、[label]列主元[/label]和[lable]完全主元[/label]。本文考虑一类特殊矩阵——[label]对称矩阵[/label]和[label]对称正定矩阵[/label]的直接分解法。

\subsection*{对称矩阵的三角分解}

考虑一个$n \times n$阶[label]对称矩阵[/label]$A$,若$A$的[label]顺序主子式[/lable]都不为$0$,则$A$可以唯一分解(B不选主元)为:

\begin{equation}\label{eq:slu}A=LDL^T\end{equation}

indent 其中,$D$为[label]对角矩阵[/label],$L$为[label]单位下三角矩阵[/label]。理由如下:根据LU分解,有$A=LU,U=DU_0$,其中$D=\text{diag}(u_{11},u_{22},\cdots,u_{nn})$,$U_0$为单位上三角矩阵;因为$A$是对称矩阵,所以有$L(DU_0)=A=A^T=U_0^T(DL^T)$;根据LU分解的B唯一性,得到:$L=U_0^T$,则$A=LDL^T$。

\subsection*{对称正定矩阵的直接三角分解}

如果矩阵$A$为[label]对称正定矩阵[/label],则$A$可以唯一地分解为如下形式:

\begin{equation}\label{eq:splu}A=LL^T\end{equation}

indent 其中,$L$为[label]下三角矩阵[/label],且如果其主对角元素都为正,则这种分解B唯一(证略),也将其称之为[label]Cholesky[/label]分解。在Cholesky分解详解一文中,已经详细介绍了对称正定矩阵的Cholesky分解,以及其直接三角求法。对其进行如下转述,对于$1 \leq k \leq n$,有:

\[
l_{kk}=\sqrt{a_{kk}-\sum_{j=1}^{k-1}l_{kj}^2}
\]
\[l_{ik}=\frac{1}{l_{kk}}\left(a_{ik}-\sum_{j=1}^{k-1}l_{ij}l_{kj}\right),\quad i=k+1,k+2,\cdots,n
\]

Cholesky直接三角分解算法描述如下:

%[latex mode=1]
%[+preamble]
%\usepackage[titlenotnumbered,lined,linesnumbered]{algorithm2e}
\let\oldnl\nl
\newcommand{\nonl}{\renewcommand{\nl}{\let\nl\oldnl}}
%[/preamble]
\TitleOfAlgo{Direct Triangular Cholesky Factorization}
\begin{algorithm}[H]
\For{$k\gets 1$ \KwTo $n-1$}{
	$a_{kk}\gets \left(a_{kk}-\sum_{j=1}^{k-1}a^2_{kj}\right)^{1/2}$\;
	\For{$i\gets k+1$ \KwTo $n$}{
		$a_{ik}\gets \left(a_{ik}-\sum_{j=1}^{k-1}a_{ij}a_{kj}\right)/a_{kk}$\;
	}
}
%$k\gets (l-1)b+1$\;
%factorize $PA^{(l)}=LU$ using BLAS-2\;
%store $L$ and $U$ in $A$\;
%left multiply $A(1:n,k+b:n)$ by $P$\;
%$A(k:k+b-1,k+b:n)\gets T^{-1}A(k:k+b-1,k+b:n)$\;
%\nonl\Indp where $T$ is the unit-lower-trian of $A(k:k+b-1,k:k+b-1)$\;
%\DontPrintSemicolon\Indm$A(k+b:n,k+b:n)\gets A(k+b:n,k+b:n)$\;
%\PrintSemicolon\nonl\Indp$-A(k+b:n,k:k+b-1)A(k:k+b-1,k+b:n)$\;
%}
\end{algorithm}
%[/latex]

\subsubsection*{Cholesky分解的并行实现}

Cholesky分解的B并行算法实现类似于[label]Gauss消去法[/label]的并行实现,即:使用[label]负载均衡[/label]的[label]一维卷帘存储方式[/label](见BLAS-3的列主元Gauss消去法详解一文)。设共有$m$个进程,进程编号为$p_1,\cdots,p_{m}$,第$k$次Cholesky分解时,进程$p_{\{k\mod m\}}$计算$a_{kk}$,并将结果发送给其他进程,其他进程在收到$a_{kk}$后同时并行计算$a_{ik},i=k+1,\cdots,n$。重复此过程$n-1$次,即可得到$L$。

\subsubsection*{列方向的Cholesky分解}

可以将前面的Cholesky直接三角分解算法可以改写为如下等价的形式:

%[latex mode=1]
%[+preamble]
%\usepackage[titlenotnumbered,lined,linesnumbered]{algorithm2e}
%\let\oldnl\nl
%\newcommand{\nonl}{\renewcommand{\nl}{\let\nl\oldnl}}
%[/preamble]
\TitleOfAlgo{Direct Triangular Cholesky Factorization along Column Direction}
\begin{algorithm}[H]
\For{$j\gets 1$ \KwTo $n$}{
	\For{$k\gets 1$ \KwTo $j-1$}{
		$a_{ij}\gets a_{ij}-a_{jk}a_{ik},\quad i=j,j+1,\cdots,n$\;
	}
	$a_{jj}\gets\sqrt{a_{jj}},\quad a_{ij}\gets a_{ij}/a_{jj},\quad i=j+1,j+2,\cdots,n$\;
}
%$k\gets (l-1)b+1$\;
%factorize $PA^{(l)}=LU$ using BLAS-2\;
%store $L$ and $U$ in $A$\;
%left multiply $A(1:n,k+b:n)$ by $P$\;
%$A(k:k+b-1,k+b:n)\gets T^{-1}A(k:k+b-1,k+b:n)$\;
%\nonl\Indp where $T$ is the unit-lower-trian of $A(k:k+b-1,k:k+b-1)$\;
%\DontPrintSemicolon\Indm$A(k+b:n,k+b:n)\gets A(k+b:n,k+b:n)$\;
%\PrintSemicolon\nonl\Indp$-A(k+b:n,k:k+b-1)A(k:k+b-1,k+b:n)$\;
%}
\end{algorithm}
%[/latex]

这两种算法都称为[label]列方向Cholesky分解[/label],即:要计算第$j$列需要前面已经计算的列,因此也称为[label]“向左看”算法[/label]、[label]“需求驱动”算法[/label]、[label]“延迟更新”算法[/label]或[label]“扇入”算法[/label]。

\subsubsection*{子矩阵Cholesky分解}

与上面的列方向Cholesky分解算法或“延迟更新”算法相对应的,是[label]子矩阵Cholesky分解[/label],也称之为[label]“立即更新”算法[/label]、[label]“数据驱动”算法[/label]或[label]“扇出”算法[/label]。其不同之处在于,计算得到$L$的某一列后,该列需要用于更新$A$的后面其他列。子矩阵Cholesky分解算法描述如下:

%[latex mode=1]
%[+preamble]
%\usepackage[titlenotnumbered,lined,linesnumbered]{algorithm2e}
%\let\oldnl\nl
%\newcommand{\nonl}{\renewcommand{\nl}{\let\nl\oldnl}}
%[/preamble]
\TitleOfAlgo{Submatrix-based Direct Triangular Cholesky Factorization}
\begin{algorithm}[H]
\For{$j\gets 1$ \KwTo $n$}{
	$a_{kk}\gets\sqrt{a_{kk}},\quad a_{ik}\gets a_{ik}/a_{kk},\quad i=k+1,k+2,\cdots,n$\;
	\For{$j\gets k+1$ \KwTo $n$}{
		$a_{ij}\gets a_{ij}-a_{jk}a_{ik},\quad i=j,j+1,\cdots,n$\;
	}
}
%$k\gets (l-1)b+1$\;
%factorize $PA^{(l)}=LU$ using BLAS-2\;
%store $L$ and $U$ in $A$\;
%left multiply $A(1:n,k+b:n)$ by $P$\;
%$A(k:k+b-1,k+b:n)\gets T^{-1}A(k:k+b-1,k+b:n)$\;
%\nonl\Indp where $T$ is the unit-lower-trian of $A(k:k+b-1,k:k+b-1)$\;
%\DontPrintSemicolon\Indm$A(k+b:n,k+b:n)\gets A(k+b:n,k+b:n)$\;
%\PrintSemicolon\nonl\Indp$-A(k+b:n,k:k+b-1)A(k:k+b-1,k+b:n)$\;
%}
\end{algorithm}
%[/latex]

[alert type="info"]理论上也有[label]行方向Cholesky分解[/label],但由于其过程不容易实现,所以一般不采用。[/alert]

最后,对于对称正定矩阵,其LU分解不需要选主元,而如果$A$是$n \times n$阶[label]不定对称矩阵[/label]的话,假设其左上角二阶子矩阵$A_{11}$是不定的,如果使用不选主元的Gauss消去法或者直接三角分解的话,会导致很大的舍入误差。因此,对于不定的对称矩阵,一般需要结合列主元Gauss消去法和二阶块Gauss消去法来保证数值稳定性。

% H4
% \subsubsection*{}

\end{document}
